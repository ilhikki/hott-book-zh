\index{universal!property|(}%
By combining the path computation rules described in the preceding sections, we can show that various type forming operations satisfy the expected universal properties, interpreted in a homotopical way as equivalences.
For instance, given types $X,A,B$, we have a function
\index{type!product}%
\begin{equation}\label{eq:prod-ump-map}
  (X\to A\times B) \to (X\to A)\times (X\to B)
\end{equation}
defined by $f \mapsto (\proj1 \circ f, \proj2\circ f)$.

\begin{thm}\label{thm:prod-ump}
  \index{universal!property!of cartesian product}%
  \eqref{eq:prod-ump-map} is an equivalence.
\end{thm}
\begin{proof}
  We define the quasi-inverse by sending $(g,h)$ to $\lam{x}(g(x),h(x))$.
  (Technically, we have used the induction principle for the cartesian product $(X\to A)\times (X\to B)$, to reduce to the case of a pair.
  From now on we will often apply this principle without explicit mention.)

  Now given $f:X\to A\times B$, the round-trip composite yields the function
  \begin{equation}
    \lam{x} (\proj1(f(x)),\proj2(f(x))).\label{eq:prod-ump-rt1}
  \end{equation}
  By \cref{thm:path-prod}, for any $x:X$ we have $(\proj1(f(x)),\proj2(f(x))) = f(x)$.
  Thus, by function extensionality, the function~\eqref{eq:prod-ump-rt1} is equal to $f$.

  On the other hand, given $(g,h)$, the round-trip composite yields the pair $(\lam{x} g(x),\lam{x} h(x))$.
  By the uniqueness principle for functions, this is (judgmentally) equal to $(g,h)$.
\end{proof}

In fact, we also have a dependently typed version of this universal property.
Suppose given a type $X$ and type families $A,B:X\to \type$.
Then we have a function
\begin{equation}\label{eq:prod-umpd-map}
  \Parens{\prd{x:X} (A(x)\times B(x))} \to \Parens{\prd{x:X} A(x)} \times \Parens{\prd{x:X} B(x)}
\end{equation}
defined as before by $f \mapsto (\proj1 \circ f, \proj2\circ f)$.

\begin{thm}\label{thm:prod-umpd}
  \eqref{eq:prod-umpd-map} is an equivalence.
\end{thm}
\begin{proof}
  Left to the reader.
\end{proof}

Just as $\Sigma$-types are a generalization of cartesian products, they satisfy a generalized version of this universal property.
Jumping right to the dependently typed version, suppose we have a type $X$ and type families $A:X\to \type$ and $P:\prd{x:X} A(x)\to\type$.
Then we have a function
\index{type!dependent pair}%
\begin{equation}
  \label{eq:sigma-ump-map}
  \Parens{\prd{x:X}\dsm{a:A(x)} P(x,a)} \to
  \Parens{\sm{g:\prd{x:X} A(x)} \prd{x:X} P(x,g(x))}.
\end{equation}
Note that if we have $P(x,a) \defeq B(x)$ for some $B:X\to\type$, then~\eqref{eq:sigma-ump-map} reduces to~\eqref{eq:prod-umpd-map}.

\begin{thm}\label{thm:ttac}
  \index{universal!property!of dependent pair type}%
  \eqref{eq:sigma-ump-map} is an equivalence.
\end{thm}
\begin{proof}
  As before, we define a quasi-inverse to send $(g,h)$ to the function $\lam{x} (g(x),h(x))$.
  Now given $f:\prd{x:X} \sm{a:A(x)} P(x,a)$, the round-trip composite yields the function
  \begin{equation}
    \lam{x} (\proj1(f(x)),\proj2(f(x))).\label{eq:prod-ump-rt2}
  \end{equation}
  Now for any $x:X$, by \cref{thm:eta-sigma} (the uniqueness principle for $\Sigma$-types) we have
  %
  \begin{equation*}
    (\proj1(f(x)),\proj2(f(x))) = f(x).
  \end{equation*}
  %
  Thus, by function extensionality,~\eqref{eq:prod-ump-rt2} is equal to $f$.
  On the other hand, given $(g,h)$, the round-trip composite yields $(\lam {x} g(x),\lam{x} h(x))$, which is judgmentally equal to $(g,h)$ as before.
\end{proof}

\index{axiom!of choice!type-theoretic}
This is noteworthy because the propositions-as-types interpretation of~\eqref{eq:sigma-ump-map} is ``the axiom of choice''.
If we read $\Sigma$ as ``there exists'' and $\Pi$ (sometimes) as ``for all'', we can pronounce:
\begin{itemize}
\item $\prd{x:X} \sm{a:A(x)} P(x,a)$ as ``for all $x:X$ there exists an $a:A(x)$ such that $P(x,a)$'', and
\item $\sm{g:\prd{x:X} A(x)} \prd{x:X} P(x,g(x))$ as ``there exists a choice function $g:\prd{x:X} A(x)$ such that for all $x:X$ we have $P(x,g(x))$''.
\end{itemize}
Thus, \cref{thm:ttac} says that not only is the axiom of choice ``true'', its antecedent is actually equivalent to its conclusion.
(On the other hand, the classical\index{mathematics!classical} mathematician may find that~\eqref{eq:sigma-ump-map} does not carry the usual meaning of the axiom of choice, since we have already specified the values of $g$, and there are no choices left to be made.
We will return to this point in \cref{sec:axiom-choice}.)

The above universal property for pair types is for ``mapping in'', which is familiar from the category-theoretic notion of products.
However, pair types also have a universal property for ``mapping out'', which may look less familiar.
In the case of cartesian products, the non-dependent version simply expresses
the cartesian closure adjunction\index{adjoint!functor}:
\[ \eqvspaced{\big((A\times B) \to C\big)}{\big(A\to (B\to C)\big)}.\]
The dependent version of this is formulated for a type family $C:A\times B\to \type$:
\[ \eqvspaced{\Parens{\prd{w:A\times B} C(w)}}{\Parens{\prd{x:A}{y:B} C(x,y)}}. \]
Here the right-to-left function is simply the induction principle for $A\times B$, while the left-to-right is evaluation at a pair.
We leave it to the reader to prove that these are quasi-inverses.
There is also a version for $\Sigma$-types:
\begin{equation}
  \eqvspaced{\Parens{\prd{w:\sm{x:A} B(x)} C(w)}}{\Parens{\prd{x:A}{y:B(x)} C(x,y)}}.\label{eq:sigma-lump}
\end{equation}
Again, the right-to-left function is the induction principle.

Some other induction principles are also part of universal properties of this sort.
For instance, path induction is the right-to-left direction of an equivalence as follows:
\index{type!identity}%
\index{universal!property!of identity type}%
\begin{equation}
  \label{eq:path-lump}
  \eqvspaced{\Parens{\prd{x:A}{p:a=x} B(x,p)}}{B(a,\refl a)}
\end{equation}
for any $a:A$ and type family $B:\prd{x:A} (a=x) \to\type$.
However, inductive types with recursion, such as the natural numbers, have more complicated universal properties; see \cref{cha:induction}.

\index{type!limit}%
\index{type!colimit}%
\index{limit!of types}%
\index{colimit!of types}%
Since \cref{thm:prod-ump} expresses the usual universal property of a cartesian product (in an appropriate homotopy-theoretic sense), the categorically inclined reader may well wonder about other limits and colimits of types.
In \cref{ex:coprod-ump} we ask the reader to show that the coproduct type $A+B$ also has the expected universal property, and the nullary cases of $\unit$ (the terminal object) and $\emptyt$ (the initial object) are easy.
\index{type!empty}%
\index{type!unit}%
\indexsee{initial!type}{type, empty}%
\indexsee{terminal!type}{type, unit}%

\indexdef{pullback}%
For pullbacks, the expected explicit construction works: given $f:A\to C$ and $g:B\to C$, we define
\begin{equation}
  A\times_C B \defeq \sm{a:A}{b:B} (f(a)=g(b)).\label{eq:defn-pullback}
\end{equation}
In \cref{ex:pullback} we ask the reader to verify this.
Some more general homotopy limits can be constructed in a similar way, but for colimits we will need a new ingredient; see \cref{cha:hits}.

\index{universal!property|)}%