\index{function|(}%
\index{functoriality of functions in type theory@``functoriality'' of functions in type theory}%
Now we wish to establish that functions $f:A\to B$ behave functorially on paths.
In traditional type theory, this is equivalently the statement that functions respect equality.
\index{continuity of functions in type theory@``continuity'' of functions in type theory}%
Topologically, this corresponds to saying that every function is ``continuous'', i.e.\ preserves paths.

\begin{lem}\label{lem:map}
  Suppose that $f:A\to B$ is a function.
  Then for any $x,y:A$ there is an operation
  \begin{equation*}
    \apfunc f : (\id[A] x y) \to (\id[B] {f(x)} {f(y)}).
  \end{equation*}
  Moreover, for each $x:A$ we have $\apfunc{f}(\refl{x})\jdeq \refl{f(x)}$.
  \indexdef{application!of function to a path}%
  \indexdef{path!application of a function to}%
  \indexdef{function!application to a path of}%
  \indexdef{action!of a function on a path}%
\end{lem}

The notation $\apfunc f$ can be read either as the \underline{ap}plication of $f$ to a path, or as the \underline{a}ction on \underline{p}aths of $f$.

\begin{proof}[First proof]
  Let $D:\prd{x,y:A} (x=y) \to \type$ be the type family defined by
  \[D(x,y,p)\defeq (f(x)= f(y)).\]
  Then we have
  \begin{equation*}
    d\defeq\lam{x} \refl{f(x)}:\prd{x:A} D(x,x,\refl{x}).
  \end{equation*}
  By path induction, we obtain $\apfunc f : \prd{x,y:A} (x=y) \to (f(x)=f(y))$.
  The computation rule implies $\apfunc f({\refl{x}})\jdeq\refl{f(x)}$ for each $x:A$.
\end{proof}

\begin{proof}[Second proof]
  To define $\apfunc{f}(p)$ for all $p:x=y$, it suffices, by induction, to assume
  $p$ is $\refl{x}$.
  In this case, we may define $\apfunc f(p) \defeq \refl{f(x)}:f(x)= f(x)$.
\end{proof}

We will often write $\apfunc f (p)$ as simply $\ap f p$.
This is strictly speaking ambiguous, but generally no confusion arises.
It matches the common convention in category theory of using the same symbol for the application of a functor to objects and to morphisms.

We note that $\apfunc{}$ behaves functorially, in all the ways that one might expect.

\begin{lem}\label{lem:ap-functor}
  For functions $f:A\to B$ and $g:B\to C$ and paths $p:\id[A]xy$ and $q:\id[A]yz$, we have:
  \begin{enumerate}
  \item $\apfunc f(p\ct q) = \apfunc f(p) \ct \apfunc f(q)$.\label{item:apfunctor-ct}
  \item $\apfunc f(\opp p) = \opp{\apfunc f (p)}$.\label{item:apfunctor-opp}
  \item $\apfunc g (\apfunc f(p)) = \apfunc{g\circ f} (p)$.\label{item:apfunctor-compose}
  \item $\apfunc {\idfunc[A]} (p) = p$.
  \end{enumerate}
\end{lem}
\begin{proof}
  Left to the reader.
\end{proof}
\index{function|)}%

As was the case for the equalities in \cref{thm:omg}, those in \cref{lem:ap-functor} are themselves paths, which satisfy their own coherence laws (which can be proved in the same way), and so on.