%%%%%%%%%%%%%%%%%%%%%%%%%%%
\sectionNotes

The definition of identity types, with their induction principle, is due to Martin-L\"of \cite{Martin-Lof-1972}.
\index{intensional type theory}%
\index{extensional!type theory}%
\index{type theory!intensional}%
\index{type theory!extensional}%
\index{reflection rule}%
As mentioned in the notes to \cref{cha:typetheory}, our identity types are those that belong to \emph{intensional} type theory, rather than \emph{extensional} type theory.
In general, a notion of equality is said to be ``intensional'' if it distinguishes objects based on their particular definitions, and ``extensional'' if it does not distinguish between objects that have the same ``extension'' or ``observable behavior''.
In the terminology of Frege, an intensional equality compares \emph{sense}, while an extensional one compares only \emph{reference}.
We may also speak of one equality being ``more'' or ``less'' extensional than another, meaning that it takes account of fewer or more intensional aspects of objects, respectively.

\emph{Intensional} type theory is so named because its \emph{judgmental} equality, $x\jdeq y$, is a very intensional equality: it says essentially that $x$ and $y$ ``have the same definition'', after we expand the defining equations of functions.
By contrast, the propositional equality type $\id xy$ is more extensional, even in the axiom-free intensional type theory of \cref{cha:typetheory}: for instance, we can prove by induction that $n+m=m+n$ for all $m,n:\N$, but we cannot say that $n+m\jdeq m+n$ for all $m,n:\N$, since the \emph{definition} of addition treats its arguments asymmetrically.
We can make the identity type of intensional type theory even more extensional by adding axioms such as function extensionality (two functions are equal if they have the same behavior on all inputs, regardless of how they are defined) and univalence (which can be regarded as an extensionality property for the universe: two types are equal if they behave the same in all contexts).
The axioms of function extensionality, and univalence in the special case of mere propositions (``propositional extensionality''), appeared already in the first type theories of Russell and Church.

As mentioned before, \emph{extensional} type theory includes also a ``reflection rule'' saying that if $p:x=y$, then in fact $x\jdeq y$.
Thus extensional type theory is so named because it does \emph{not} admit any purely \emph{intensional} equality: the reflection rule forces the judgmental equality to coincide with the more extensional identity type.
Moreover, from the reflection rule one may deduce function extensionality (at least in the presence of a judgmental uniqueness principle for functions).
However, the reflection rule also implies that all the higher groupoid structure collapses (see \cref{ex:equality-reflection}), and hence is inconsistent with the univalence axiom (see \cref{thm:type-is-not-a-set}).
Therefore, regarding univalence as an extensionality property, one may say that intensional type theory permits identity types that are ``more extensional'' than extensional type theory does.

The proofs of symmetry (inversion) and transitivity (concatenation) for equalities are well-known in type theory.
The fact that these make each type into a 1-groupoid (up to homotopy) was exploited in~\cite{hs:gpd-typethy} to give the first ``homotopy'' style semantics for type theory.

The actual homotopical interpretation, with identity types as path spaces, and type families as fibrations, is due to \cite{AW}, who used the formalism of Quillen model categories.  An interpretation in (strict) $\infty$-groupoids\index{.infinity-groupoid@$\infty$-groupoid} was also given in the thesis \cite{mw:thesis}.
For a construction of \emph{all} the higher operations and coherences of an $\infty$-groupoid in type theory, see~\cite{pll:wkom-type} and~\cite{bg:type-wkom}.

\index{proof!assistant!Coq@\textsc{Coq}}%
Operations such as $\transfib{P}{p}{\blank}$ and $\apfunc{f}$, and one good notion of equivalence, were first studied extensively in type theory by Voevodsky, using the proof assistant \Coq.
Subsequently, many other equivalent definitions of equivalence have been found, which are compared in \cref{cha:equivalences}.

The ``computational'' interpretation of identity types, transport, and so on described in \cref{sec:computational} has been emphasized by~\cite{lh:canonicity}.
They also described a ``1-truncated'' type theory (see \cref{cha:hlevels}) in which these rules are judgmental equalities.
The possibility of extending this to the full untruncated theory is a subject of current research.

\index{function extensionality}%
The naive form of function extensionality which says that ``if two functions are pointwise equal, then they are equal'' is a common axiom in type theory, going all the way back to \cite{PM2}.
Some stronger forms of function extensionality were considered in~\cite{garner:depprod}.
The version we have used, which identifies the identity types of function types up to equivalence, was first studied by Voevodsky, who also proved that it is implied by the naive version (and by univalence; see \cref{sec:univalence-implies-funext}).

\index{univalence axiom}%
The univalence axiom is also due to Voevodsky.
It was originally motivated by semantic considerations in the simplicial set model; see~\cite{klv:ssetmodel}.
A similar axiom motivated by the groupoid model was proposed by Hofmann and Streicher~\cite{hs:gpd-typethy} under the name ``universe extensionality''.
It used quasi-inverses~\eqref{eq:qinvtype} rather than a good notion of ``equivalence'', and hence is correct (and equivalent to univalence) only for a universe of 1-types (see \cref{defn:1type}).

In the type theory we are using in this book, function extensionality and univalence have to be assumed as axioms, i.e.\ elements asserted to belong to some type but not constructed according to the rules for that type.
While serviceable, this has a few drawbacks.
For instance, type theory is formally better-behaved if we can base it entirely on rules rather than asserting axioms.
It is also sometimes inconvenient that the theorems of \crefrange{sec:compute-cartprod}{sec:compute-nat} are only propositional equalities (paths) or equivalences, since then we must explicitly mention whenever we pass back and forth across them.
One direction of current research in homotopy type theory is to describe a type system in which these rules are \emph{judgmental} equalities, solving both of these problems at once.
So far this has only been done in some simple cases, although preliminary results such as~\cite{lh:canonicity} are promising.
There are also other potential ways to introduce univalence and function extensionality into a type theory, such as having a sufficiently powerful notion of ``higher quotients'' or ``higher inductive-recursive types''.

The simple conclusions in \crefrange{sec:compute-coprod}{sec:compute-nat} such as ``$\inl$ and $\inr$ are injective and disjoint'' are well-known in type theory, and the construction of the function \encode is the usual way to prove them.
The more refined approach we have described, which characterizes the entire identity type of a positive type (up to equivalence), is a more recent development; see e.g.~\cite{ls:pi1s1}.

\index{axiom!of choice!type-theoretic}%
The type-theoretic axiom of choice~\eqref{eq:sigma-ump-map} was noticed in William Howard's original paper~\cite{howard:pat} on the propositions-as-types correspondence, and was studied further by Martin-L\"of with the introduction of his dependent type theory.  It is mentioned as a ``distributivity law'' in Bourbaki's set theory \cite{Bourbaki}.\index{Bourbaki}%

For a more comprehensive (and formalized) discussion of pullbacks and more general homotopy limits in homotopy type theory, see~\cite{AKL13}.
Limits\index{limit!of types} of diagrams over directed graphs\index{graph} are the easiest general sort of limit to formalize; the problem with diagrams over categories (or more generally $(\infty,1)$-categories)
\index{.infinity1-category@$(\infty,1)$-category}%
\indexsee{category!.infinity1-@$(\infty,1)$-}{$(\infty,1)$-category}%
is that in general, infinitely many coherence conditions are involved in the notion of (homotopy coherent) diagram.\index{diagram}
Resolving this problem is an important open question\index{open!problem} in homotopy type theory.