\sectionExercises

\begin{ex}\label{ex:basics:concat}
Show that the three obvious proofs of \cref{lem:concat} are pairwise equal.
\end{ex}

\begin{ex}\label{ex:eq-proofs-commute}
Show that the three equalities of proofs constructed in the previous exercise form a commutative triangle.
In other words, if the three definitions of concatenation are denoted by $(p \mathbin{\ct_1} q)$, $(p\mathbin{\ct_2} q)$, and $(p\mathbin{\ct_3} q)$, then the concatenated equality
\[(p\mathbin{\ct_1} q) = (p\mathbin{\ct_2} q) = (p\mathbin{\ct_3} q)\]
is equal to the equality $(p\mathbin{\ct_1} q) = (p\mathbin{\ct_3} q)$.
\end{ex}

\begin{ex}\label{ex:fourth-concat}
Give a fourth, different, proof of \cref{lem:concat}, and prove that it is equal to the others.
\end{ex}

\begin{ex}\label{ex:npaths}
Define, by induction on $n$, a general notion of \define{$n$-dimensional path}\index{path!n-@$n$-} in a type $A$, simultaneously with the type of boundaries for such paths.
\end{ex}

\begin{ex}\label{ex:ap-to-apd-equiv-apd-to-ap}
Prove that the functions~\eqref{eq:ap-to-apd} and~\eqref{eq:apd-to-ap} are inverse equivalences.
    % and that they take $\apfunc f(p)$ to $\apdfunc f (p)$ and vice versa. (that was \cref{thm:apd-const})
\end{ex}

\begin{ex}\label{ex:equiv-concat}
Prove that if $p:x=y$, then the function $(p\ct \blank):(y=z) \to (x=z)$ is an equivalence.
\end{ex}

\begin{ex}\label{ex:ap-sigma}
State and prove a generalization of \cref{thm:ap-prod} from cartesian products to $\Sigma$-types.
\end{ex}

\begin{ex}\label{ex:ap-coprod}
State and prove an analogue of \cref{thm:ap-prod} for coproducts.
\end{ex}

\begin{ex}\label{ex:coprod-ump}
\index{universal!property!of coproduct}%
Prove that coproducts have the expected universal property,
\[ \eqv{(A+B \to X)}{(A\to X)\times (B\to X)}. \]
Can you generalize this to an equivalence involving dependent functions?
\end{ex}

\begin{ex}\label{ex:sigma-assoc}
Prove that $\Sigma$-types are ``associative'',
\index{associativity!of Sigma-types@of $\Sigma$-types}%
in that for any $A:\UU$ and families $B:A\to\UU$ and $C:(\sm{x:A} B(x))\to\UU$, we have
\[\eqvspaced{\Parens{\sm{x:A}{y:B(x)} C(\pairr{x,y})}}{\Parens{\sm{p:\sm{x:A}B(x)} C(p)}}. \]
\end{ex}

\begin{ex}\label{ex:pullback}
A (homotopy) \define{commutative square}
\indexdef{commutative!square}%
\begin{equation*}
    \vcenter{\xymatrix{
        P\ar[r]^h\ar[d]_k &
        A\ar[d]^f\\
        B\ar[r]_g &
        C
    }}
\end{equation*}
consists of functions $f$, $g$, $h$, and $k$ as shown, together with a path $f \circ h= g \circ k$.
Note that this is exactly an element of the pullback $(P\to A) \times_{P\to C} (P\to B)$ as defined in~\eqref{eq:defn-pullback}.
A commutative square is called a (homotopy) \define{pullback square}
\indexdef{pullback}%
if for any $X$, the induced map
\[ (X\to P) \to (X\to A) \times_{(X\to C)} (X\to B) \]
is an equivalence.
Prove that the pullback $P \defeq A\times_C B$ defined in~\eqref{eq:defn-pullback} is the corner of a pullback square.
\end{ex}

\begin{ex}\label{ex:pullback-pasting}
Suppose given two commutative squares
\begin{equation*}
    \vcenter{\xymatrix{
        A\ar[r]\ar[d] &
        C\ar[r]\ar[d] &
        E\ar[d]\\
        B\ar[r] &
        D\ar[r] &
        F
    }}
\end{equation*}
and suppose that the right-hand square is a pullback square.
Prove that the left-hand square is a pullback square if and only if the outer rectangle is a pullback square.
\end{ex}

\begin{ex}\label{ex:eqvboolbool}
Show that $\eqv{(\eqv\bool\bool)}{\bool}$.
\end{ex}

\begin{ex}\label{ex:equality-reflection}
Suppose we add to type theory the \emph{equality reflection rule} which says that if there is an element $p:x=y$, then in fact $x\jdeq y$.
Prove that for any $p:x=x$ we have $p\jdeq \refl{x}$.
(This implies that every type is a \emph{set} in the sense to be introduced in \cref{sec:basics-sets}; see \cref{sec:hedberg}.)
\end{ex}

\begin{ex}\label{ex:strengthen-transport-is-ap}
Show that \cref{thm:transport-is-ap} can be strengthened to
\[\transfib{B}{p}{{\blank}} =_{B(x)\to B(y)} \idtoeqv(\apfunc{B}(p))\]
without using function extensionality.
(In this and other similar cases, the apparently weaker formulation has been chosen for readability and consistency.)
\end{ex}

\begin{ex}\label{ex:strong-from-weak-funext}
Suppose that rather than function extensionality (\cref{axiom:funext}), we suppose only the existence of an element
\[ \funext : \prd{A:\UU}{B:A\to \UU}{f,g:\prd{x:A} B(x)} (f\htpy g) \to (f=g) \]
(with no relationship to $\happly$ assumed).
Prove that in fact, this is sufficient to imply the whole function extensionality axiom (that $\happly$ is an equivalence).
This is due to Voevodsky; its proof is tricky and may require concepts from later chapters.
\end{ex}

\begin{ex}\label{ex:equiv-functor-types}\
\begin{enumerate}
    \item Show that if $\eqv{A}{A'}$  and $\eqv{B}{B'}$, then $\eqv{(A\times B)}{(A'\times B')}$.
    \item Give two proofs of this fact, one using univalence and one not using it, and show that the two proofs are equal.
    \item Formulate and prove analogous results for the other type formers: $\Sigma$, $\to$, $\Pi$, and $+$.
\end{enumerate}
\end{ex}

\begin{ex}\label{ex:dep-htpy-natural}
State and prove a version of \cref{lem:htpy-natural} for dependent functions.
\end{ex}