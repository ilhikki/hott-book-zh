\sectionNotes

恒等类型的定义, 以及它的归纳原理, 源自 Martin-L\"of \cite{Martin-Lof-1972}.
\index{内延的类型论}%
\index{外延的!类型论}%
\index{类型论!内延的}%
\index{反射规则}%
正如 \cref{cha:typetheory} 的备注所提及的, 我们的恒等类型属于\emph{内延的}类型论, 而不是\emph{外延的}类型论.
通常, 对于一个等同关系的概念, 如果基于对象的特殊的定义区分对象, 它就是 ``内延的'', 而如果不区分有相同 ``外延性'' 或者 ``可观察行为''的对象, 就是 ``外延的''.
在 Frege 的术语中, 内延的等同关系对比\emph{意义}, 而内延性的只对比\emph{引用}.
可以说一个等同关系比另外的等同关系有``更多'' 或 ``更少''的外延, 分别代表它关注更少或更多对象的内延.

\emph{内延的}类型论被这样命名是因为它的\emph{判断}等同, $x\jdeq y$, 是一个非常内延的等同关系: 它本质上是说, 在展开函数的定义等式后, $x$ 和 $y$ ``有一样的定义''.
通过对比, 命题的等同关系类型 $\id xy$ 更外延, 即使在\cref{cha:typetheory} 的无公理内延类型论: 例如, 对于所有 $m,n:\N$, 可以通过归纳证明 $n+m=m+n$, 但是不能说对于所有 $m,n:\N$ 有 $n+m\jdeq m+n$, 因为加法的\emph{定义}非对称地处理它的参数.
可以通过添加公理让内延的类型论的恒等类型更外延, 例如函数的外延性 (两个函数如果所有输入, 有相同的行为, 它们就是是相等的, 无论它们怎么定义) 和单值性 (可以当作全集的外延性质: 如果两个类型在所有语境行为一致, 那么它们相等).
函数的外延性公理, 和纯命题情况下的单值性 (``命题的外延性''), 已经出现在的 Russell 和 Church 的第一个类型论中.

正如之前提及的, \emph{外延的}类型论也包含了 ``反射规则'', 如果 $p:x=y$, 那么实际上 $x\jdeq y$.
因此外延的类型论如此命名的原因是, 它\emph{不}允许任何纯粹的\emph{内延的}等价: 反射规则强制判上的等同关系与更外延的恒等类型一致.
此外, 从反射规则可以推导出函数的外延性 (至少在函数的判断的唯一性原则存在的情况下).
然而, 反射规则也暗示了所有高阶群胚结构坍塌 (参见 \cref{ex:equality-reflection}), 因此与单值性公理矛盾 (参见 \cref{thm:type-is-not-a-set}).
于是, 单值性被视为外延性质, 也就是说内延的类型论允许恒等类型比外延的类型论 ``更外延''.

类型论中, 等同关系的对称性 (逆) 和传递性 (连接) 的证明是总所周知的.
在~\cite{hs:gpd-typethy}中, 利用些让每种类型成为一阶群胚 (到同伦) 的事实, 为类型论给出了第一 ``同伦'' 风格语义.

正真的同伦解释, 恒等类型作为路径空间, 类型族作为纤维化, 源自 \cite{AW}, 其中使用了 Quillen 模型范畴的形式主义.
(严格) $\infty$-维群胚\index{.infinity-groupoid@$\infty$-groupoid}中的一个解释同样在论文 \cite{mw:thesis} 给定.
对于\emph{所有}高阶运算和类型论中 $\infty$-维群胚的一致性的构造, 参见~\cite{pll:wkom-type} 和~\cite{bg:type-wkom}.

\index{证明!助手!Coq@\textsc{Coq}}%
例如 $\transfib{P}{p}{\blank}$ 和 $\apfunc{f}$ 的运算, 以及更好的等价关系的概念, 在类型论中, 首先是由 Voevodsky 广泛研究, 在证明助手 \Coq 中使用.
接着找到了很多等价关系的等价的定义, 它们会在 \cref{cha:equivalences} 中被比较.

恒等类型, 传输, 等等这些在\cref{sec:computational} 中讨论的``计算科学''的概念, 被~\cite{lh:canonicity} 强调.
它们还被描述为 ``一维截断'' 类型论 (参见 \cref{cha:hlevels}), 即它们的规则是判断等同的.
将它们扩展到完整的非截断理论的可能性是目前研究的一个课题.

\index{函数外延性}%
朴素地, 函数外延性说 ``如果函数逐点的相等, 那么它们相等'' 是类型论中常见的公理, 可以追溯到 \cite{PM2}.
在~\cite{garner:depprod}中, 考虑了一些更强的函数外延性的形式.
我们使用的版本, 让函数类型的恒等类型, 恒等于等价关系, 首先被 Voevodsky 安排, 他还证明了可以从普通的版本推导出来 (通过单值性; 参见 \cref{sec:univalence-implies-funext}).

\index{单值性公理}%
单值性公理同样源自于 Voevodsky.
它最初是为了研究单纯集合模型中语义; 参见~\cite{klv:ssetmodel}.
一个类似的公理叫做``全集外延性'', 由 Hofmann 和 Streicher~\cite{hs:gpd-typethy}提出, 为了研究群胚模型.
它使用准逆~\eqref{eq:qinvtype}而不是更好的 ``等价'', 因此仅对于一阶类型的全集生效 (并等价于单值) (参见 \cref{defn:1type}).

在本书使用的类型论中, 函数外延性和单值性被作为公理给出, 也就是说, 声明元素属于一些类型, 但是不通过这些类型的规则构造.
虽然够用, 但是仍有一些缺点.
例如, 如果能完全基于规则而不是声明公理, 类型论在形式上会更好.
不方便的是, \crefrange{sec:compute-cartprod}{sec:compute-nat} 中的公理只是命题的等同 (路径) 或者等价关系, 每当在它们之间来回穿过时, 都必须显式提及.
现在同伦类型论的一个研究方向是, 描述一个类型系统, 它的规则是\emph{判断的}等同, 将这些问题一次解决.
迄今为止, 这些只在一些简单的情况下做到了, 尽管诸如~\cite{lh:canonicity} 的初步结果是有希望的.
还有一些其它潜在的途径, 引入单值性和函数外延性到类型论中, 例如有一个足够强大的 ``higher quotients'' 或者 ``higher inductive-recursive types'' 概念.

\crefrange{sec:compute-coprod}{sec:compute-nat} 中简单的结论像 ``$\inl$ 和 $\inr$ 单射且不交'' 在类型论中众所周知, 而函数 \encode 的构造是证明它的常规方法.
描述过的更优雅的方法是, 塑造全部正类型的恒等类型 (直到等价), 是最近的发展; 参见例子~\cite{ls:pi1s1}.

\index{公理!选择!类型论的}%
类型论的选择~\eqref{eq:sigma-ump-map}公理, 在 William Howard 的关于命题即类型的对应关系原创论文~\cite{howard:pat} 中被关注, 并由 Martin-L\"of 随着引入他的依值类型论被进一步研究.
在 Bourbaki 的集合论 \cite{Bourbaki} 中, 作为``分配性法则'' 被提及.\index{Bourbaki}%

对于同伦类型论中, 更全面 (和形式的) 拉回和通用的同伦极限的讨论, 参见~\cite{AKL13}.
有向图\index{图}上关系图的极限\index{极限!类型的}是最容易形式化的一种常规极限; 范畴上的关系图 (或者更泛用的 $(\infty,1)$-维范畴) 的问题是
\index{.infinity1-维范畴@$(\infty,1)$-维范畴}%
\indexsee{范畴!.infinity1-@$(\infty,1)$-}{$(\infty,1)$-维范畴}%
一般来说, 在 (同伦一致) 关系图的概念中, 涉及到很多无穷的一致性判断条件.\index{关系图}
在同伦类型论中, 如何解决这个问题是一个重要的开放性问题\index{开放性!问题}.