\index{type!dependent pair|(}%
Let $A$ be a type and $P:A\to\type$ a type family.
Recall that the $\Sigma$-type, or dependent pair type, $\sm{x:A} P(x)$ is a generalization of the cartesian product type.
Thus, we expect its higher groupoid structure to also be a generalization of the previous section.
In particular, its paths should be pairs of paths, but it takes a little thought to give the correct types of these paths.

Suppose that we have a path $p:w=w'$ in $\sm{x:A}P(x)$.
Then we get $\ap{\proj{1}}{p}:\proj{1}(w)=\proj{1}(w')$.
However, we cannot directly ask whether $\proj{2}(w)$ is identical to $\proj{2}(w')$ since they don't have to be in the same type.
But we can transport\index{transport} $\proj{2}(w)$ along the path $\ap{\proj{1}}{p}$, and this does give us an element of the same type as $\proj{2}(w')$.
By path induction, we do in fact obtain a path $\trans{\ap{\proj{1}}{p}}{\proj{2}(w)}=\proj{2}(w')$.

Recall from the discussion preceding \cref{lem:mapdep} that
\narrowequation{
  \trans{\ap{\proj{1}}{p}}{\proj{2}(w)}=\proj{2}(w')
}
can be regarded as the type of paths from $\proj2(w)$ to $\proj2(w')$ which lie over the path $\ap{\proj1}{p}$ in $A$.
\index{fibration}%
\index{total!space}%
Thus, we are saying that a path $w=w'$ in the total space determines (and is determined by) a path $p:\proj1(w)=\proj1(w')$ in $A$ together with a path from $\proj2(w)$ to $\proj2(w')$ lying over $p$, which seems sensible.

\begin{rmk}
  Note that if we have $x:A$ and $u,v:P(x)$ such that $(x,u)=(x,v)$, it does not follow that $u=v$.
  All we can conclude is that there exists $p:x=x$ such that $\trans p u = v$.
  This is a well-known source of confusion for newcomers to type theory, but it makes sense from a topological viewpoint: the existence of a path $(x,u)=(x,v)$ in the total space of a fibration between two points that happen to lie in the same fiber does not imply the existence of a path $u=v$ lying entirely \emph{within} that fiber.
\end{rmk}

The next theorem states that we can also reverse this process.
Since it is a direct generalization of \cref{thm:path-prod}, we will be more concise.

\begin{thm}\label{thm:path-sigma}
Suppose that $P:A\to\type$ is a type family over a type $A$ and let $w,w':\sm{x:A}P(x)$. Then there is an equivalence
\begin{equation*}
\eqvspaced{(w=w')}{\dsm{p:\proj{1}(w)=\proj{1}(w')} \trans{p}{\proj{2}(w)}=\proj{2}(w')}.
\end{equation*}
\end{thm}

\begin{proof}
We define a function
\begin{equation*}
f : \prd{w,w':\sm{x:A}P(x)} (w=w') \to \dsm{p:\proj{1}(w)=\proj{1}(w')} \trans{p}{\proj{2}(w)}=\proj{2}(w')
\end{equation*}
by path induction, with
\begin{equation*}
f(w,w,\refl{w})\defeq(\refl{\proj{1}(w)},\refl{\proj{2}(w)}).
\end{equation*}
We want to show that $f$ is an equivalence.

In the reverse direction, we define
\begin{narrowmultline*}
  g : \prd{w,w':\sm{x:A}P(x)}
      \Parens{\sm{p:\proj{1}(w)=\proj{1}(w')}\trans{p}{\proj{2}(w)}=\proj{2}(w')}
      \to
      \narrowbreak
      (w=w')
\end{narrowmultline*}
by first inducting on $w$ and $w'$, which splits them into $(w_1,w_2)$ and
$(w_1',w_2')$ respectively, so it suffices to show
\begin{equation*}
\Parens{\sm{p:w_1 = w_1'}\trans{p}{w_2}=w_2'} \to ((w_1,w_2)=(w_1',w_2')).
\end{equation*}
Next, given a pair $\sm{p:w_1 = w_1'}\trans{p}{w_2}=w_2'$, we can
use $\Sigma$-induction to get $p : w_1 = w_1'$ and $q :
\trans{p}{w_2}=w_2'$.  Inducting on $p$, we have $q :
\trans{(\refl{w_1})}{w_2}=w_2'$, and it suffices to show
$(w_1,w_2)=(w_1,w_2')$.  But $\trans{(\refl{w_1})}{w_2} \jdeq w_2$, so
inducting on $q$ reduces the goal to
$(w_1,w_2)=(w_1,w_2)$, which we can prove with $\refl{(w_1,w_2)}$.

Next we show that $f(g(r))=r$ for all $w$, $w'$ and
$r$, where $r$ has type
\[\dsm{p:\proj{1}(w)=\proj{1}(w')} (\trans{p}{\proj{2}(w)}=\proj{2}(w')).\]
First, we break apart the pairs $w$, $w'$, and $r$ by pair induction, as in the
definition of $g$, and then use two path inductions to reduce both components
of $r$ to \refl{}.  Then it suffices to show that
$f (g(\refl{w_1},\refl{w_2})) = (\refl{w_1},\refl{w_2})$, which is true by definition.

Similarly, to show that $g(f(p))=p$ for all $w$, $w'$,
and $p : w = w'$, we can do path induction on $p$, and then pair induction to
split $w$, at which point it suffices to show that
$g(f (\refl{(w_1,w_2)})) = \refl{(w_1,w_2)}$, which is true by
definition.

Thus, $f$ has a quasi-inverse, and is therefore an equivalence.
\end{proof}

As we did in the case of cartesian products, we can deduce a propositional uniqueness principle as a special case.

\begin{cor}\label{thm:eta-sigma}
  \index{uniqueness!principle, propositional!for dependent pair types}%
  For $z:\sm{x:A} P(x)$, we have $z = (\proj1(z),\proj2(z))$.
\end{cor}
\begin{proof}
  We have $\refl{\proj1(z)} : \proj1(z) = \proj1(\proj1(z),\proj2(z))$, so by \cref{thm:path-sigma} it will suffice to exhibit a path $\trans{(\refl{\proj1(z)})}{\proj2(z)} = \proj2(\proj1(z),\proj2(z))$.
  But both sides are judgmentally equal to $\proj2(z)$.
\end{proof}

Like with binary cartesian products, we can think of
the backward direction of \cref{thm:path-sigma} as
an introduction form (\pairpath{}{}), the forward direction as
elimination forms (\projpath{1} and \projpath{2}), and the equivalence
as giving a propositional computation rule and uniqueness principle for these.

Note that the lifted path $\mathsf{lift}(u,p)$  of $p:x=y$ at $u:P(x)$ defined in \cref{thm:path-lifting} may be identified with the special case of the introduction form
\[\pairpath(p,\refl{\trans p u}):(x,u) = (y,\trans p u).\]
\index{transport!in dependent pair types}%
This appears in the statement of action of transport on $\Sigma$-types, which is also a generalization of the action for binary cartesian products:

\begin{thm}\label{transport-Sigma}
  Suppose we have type families
  %
  \begin{equation*}
    P:A\to\type
    \qquad\text{and}\qquad
    Q:\Parens{\sm{x:A} P(x)}\to\type.
  \end{equation*}
  %
  Then we can construct the type family over $A$ defined by
  \begin{equation*}
    x \mapsto \sm{u:P(x)} Q(x,u).
  \end{equation*}
  For any path $p:x=y$ and any $(u,z):\sm{u:P(x)} Q(x,u)$ we have
  \begin{equation*}
    \trans{p}{u,z}=\big(\trans{p}{u},\,\trans{\pairpath(p,\refl{\trans pu})}{z}\big).
  \end{equation*}
\end{thm}

\begin{proof}
Immediate by path induction.
\end{proof}

We leave it to the reader to state and prove a generalization of
\cref{thm:ap-prod} (see \cref{ex:ap-sigma}), and to characterize
the reflexivity, inverses, and composition of $\Sigma$-types
componentwise.

\index{type!dependent pair|)}%