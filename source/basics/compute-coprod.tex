\index{类型!余积|(}%
\index{编转码方法|(}%
迄今为止考虑过的类型的形式, 大部分都是\emph{负}类型.
\index{类型!负}\index{负!类型}%
\index{极性}%
直观地, 这意味这些类型确定于其消去规则上的行为: (依值)对子由其投影确定, 而(依值)函数由其值确定.
几乎所有负类型, 乃至其高阶结构, 可以被直接表征, 正如\crefrange{sec:compute-cartprod}{sec:compute-pi} 中所做的一样.
准确的说, 全集不是负类型, 不过其恒等类型表现类似: 有这个高阶结构的直接定性 (一价) 和描述.
当然, 恒等类型本身是一种特殊情况.

现在考虑\emph{正}类型形式的第一个例子.
\index{类型!正}\index{正!类型}%
非正式地说, 正类型由它那些构造器``体现'', 其泛性质\index{体现!通过正类型构造的} 由其消去规则表达.
(拓扑上讲, 正类型 ``映射出'' 泛性质, 而负类型 ``映射入'' 泛性质.)
因为通常是, 体现是不可计算的, 不是所有正类型都能直接表征其恒等类型.
不过, 很多特殊情况, 表征或者部分的表征是存在的, 可以从这个列子中看到这个通用的方法.

(技术上, 笛卡尔积和 $\Sigma$-类型同样也是正类型.
不过, 这些类型可以略有区别的表示为负类型, 它们的恒等类型不必像这样表征.)

考虑余积类型 $A+B$, ``体现''为单射 $\inl:A\to A+B$ 和 $\inr:B\to A+B$.
直观地, 期望 $A+B$ 存在不交的 $A$ 和 $B$ 的精确复制, 所以有
\begin{align}
{(\inl(a_1)=\inl(a_2))}
    &\eqvsym {(a_1=a_2)} \label{eq:inlinj}\\
    {(\inr(b_1)=\inr(b_2))}&\eqvsym {(b_1=b_2)}\\
    {(\inl(a)= \inr(b))} &\eqvsym {\emptyt}. \label{eq:inlrdj}
\end{align}
证明如下.
固定元素 $a_0:A$; 表征类型族
\begin{equation}
(x\mapsto (\inl(a_0)=x))
    : A+B \to \type.\label{eq:sumcodefam}
\end{equation}
对于任何 $b_0:B$ 类似的参数可以表征类似于 $x\mapsto (x = \inr(b_0))$ 的类型族.
它们一起表征了~\eqref{eq:inlinj}--\eqref{eq:inlrdj}的实现.

为了表征~\eqref{eq:sumcodefam}, 定义类型族 $\code:A+B\to\type$ 并展示 $\prd{x:A+B} (\eqv{(\inl(a_0)=x)}{\code(x)})$.
为了从它推断~\eqref{eq:inlinj}, 需要 $\code(\inl(a)) = (a_0=a)$, 而为了推断~\eqref{eq:inlrdj}, 需要 $\code (\inr(b)) = \emptyt$.
其实就是, 用 $A+B$ 的递归原则, 通过这两个等式来\emph{定义} $\code:A+B\to\type$:
\begin{align*}
    \code(\inl(a)) &\defeq (a_0=a),\\
    \code(\inr(b)) &\defeq \emptyt.
\end{align*}
这是证明技巧的非常简单的例子, 在同伦类型论中做同伦论时经常用到它;
例如 \cref{sec:pi1-s1-intro,sec:general-encode-decode}.
%
现在可以展示:

\begin{thm}
    \label{thm:path-coprod}
    对于所有 $x:A+B$ 有 $\eqv{(\inl(a_0)=x)}{\code(x)}$.
\end{thm}
\begin{proof}
    下面的证明的关键是, 所有的点 $x$ 一起证明, 在余积上使用消去原理.
    首先定义函数
    \[ \encode : \prd{x:A+B}{p:\inl(a_0)=x} \code(x) \]
    通过沿着 $p$ 运输自反:
    \[ \encode(x,p) \defeq \transfib{\code}{p}{\refl{a_0}}. \]
    注意 $\refl{a_0} : \code(\inl(a_0))$, 因为 $\code(\inl(a_0))\jdeq (a_0=a_0)$ 由 \code 定义.
    接下来, 定义函数
    \[ \decode : \prd{x:A+B}{c:\code(x)} (\inl(a_0)=x). \]
    为了定义 $\decode(x,c)$, 首先使用 $A+B$ 的消去原理, 从 $x$ 是 $\inl(a)$ 或 $\inr(b)$ 的情况分别得到.

    第一种情况, $x\jdeq \inl(a)$, 而 $\code(x)\jdeq (a_0=a)$, 所以 $c$ 是 $a_0$ 和 $a$ 之间的恒等.
    因此, $\apfunc{\inl}(c):(\inl(a_0)=\inl(a))$ 就是 $\decode(\inl(a),c)$ 的定义.

    第二种情况, $x\jdeq \inr(b)$, 而 $\code(x)\jdeq \emptyt$, 所以 $c$ 居留于空类型.
    因此, $\emptyt$ 的消去规则得到 $\decode(\inr(b),c)$ 的值.

    这完成了 \decode 的定义; 现在展示对于所有 $x$, $\encode(x,{\blank})$ 准逆于 $\decode(x,{\blank})$.
    一方面, 给定 $x:A+B$ 和 $p:\inl(a_0)=x$; 要展示
    \narrowequation{
        \decode(x,\encode(x,p)) = p.
    }
    而现在通过 (基础) 路径归纳, 足以认为 $x\jdeq\inl(a_0)$ 和 $p\jdeq \refl{\inl(a_0)}$:
    \begin{align*}
        \decode(x,\encode(x,p))
        &\jdeq \decode(\inl(a_0),\encode(\inl(a_0),\refl{\inl(a_0)}))\\
        &\jdeq \decode(\inl(a_0),\transfib{\code}{\refl{\inl(a_0)}}{\refl{a_0}})\\
        &\jdeq \decode(\inl(a_0),\refl{a_0})\\
        &\jdeq \apfunc{\inl}(\refl{a_0})\\
        &\jdeq \refl{\inl(a_0)}\\
        &\jdeq p.
    \end{align*}
    另一方面, 令 $x:A+B$ 和 $c:\code(x)$; 要展示 $\encode(x,\decode(x,c))=c$.
    再次从基于 $x$ 的情况推出.
    如果 $x\jdeq\inl(a)$, 那么 $c:a_0=a$ 并且 $\decode(x,c)\jdeq \apfunc{\inl}(c)$, 于是
    \begin{align}
        \encode(x,\decode(x,c))
        &\jdeq \transfib{\code}{\apfunc{\inl}(c)}{\refl{a_0}}
        \notag\\
        &= \transfib{a\mapsto (a_0=a)}{c}{\refl{a_0}}
        \tag{根据 \cref{thm:transport-compose}}\\
        &= \refl{a_0} \ct c
        \tag{根据 \cref{cor:transport-path-prepost}}\\
        &= c. \notag
    \end{align}
    最后, 如果 $x\jdeq \inr(b)$, 那么 $c:\emptyt$, 这样就满足了所有的要求.
\end{proof}

\noindent
当然, 如果固定 $b_0:B$ 而不是 $a_0:A$, 也有相应的定理.

特别的, \cref{thm:path-coprod} 暗示对于任何 $a : A$ 和 $b : B$ 有函数
%
\[ \encode(\inl(a), {\blank}) : (\inl(a_0)=\inl(a)) \to (a_0=a)\]
%
和
%
\[ \encode(\inr(b), {\blank}) : (\inl(a_0)=\inr(b)) \to \emptyt. \]
%
第二个函数说 ``$\inl(a_0)$ 不等于 $\inr(b)$'', 即 \inl 和 \inr 的图形是不交的.
第一个函数的传统的解读就是 $\inl$ 的单射性, 其中恒等类型被视为命题.
\cref{thm:path-coprod} 的完整同伦陈述给出了更多信息: 类型 $\inl(a_0)=\inl(a)$ 和 $a_0=a$ 实际上等价, $\inr(b_0)=\inr(b)$ 和 $b_0=b$ 亦如此.

\begin{rmk}\label{rmk:true-neq-false}
特殊地, 因此双元类型 $\bool$ 等价于 $\unit+\unit$, 有 $\bfalse\neq\btrue$.
\end{rmk}

这个证明展示了一个通用方法, 经常被用来描述路径空间.
为了塑造路径空间, 第一步是定义一个对照纤维化 ``$\code$'', 它提供了这个路径的更明确的描述.
有一些不同的方法来证明, 对照纤维化等价于这个路径(几个相同结果的不同证明在\cref{sec:pi1-s1-intro}).
这里使用\define{编解码方法}:
\indexdef{编解码方法}
关键在于定义 $\decode$ 并对纤维化的实例通用 (例如函数 $\prd{x:A+B} \code(x) \to (\inl(a_0)=x)$), 这样路径归纳可以用于分析 $\decode(x,\encode(x,p))$.

\index{传输!余积类型中}%
照常, 也可以塑造余积类型中传输的行为.
给定类型~$X$, 路径 $p:\id[X]{x_1}{x_2}$, 和类型族 $A,B:X\to\type$, 有
\begin{align*}
  \transfib{A+B}{p}{\inl(a)} &= \inl (\transfib{A}{p}{a}),\\
  \transfib{A+B}{p}{\inr(b)} &= \inr (\transfib{B}{p}{b}),
\end{align*}
照常, 上标 $A+B$ 滥用地代表类型族 $x\mapsto A(x)+B(x)$.
证明只需简单的路径归纳.

\index{encode-decode 方法|)}%
\index{类型!余积|)}%