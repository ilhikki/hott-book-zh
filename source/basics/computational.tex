在 \cref{cha:typetheory} 中, 介绍了很多途径来构造新类型: 笛卡尔积, 不交并, 依值乘积, 依值和, 等等.
在 \cref{sec:equality,sec:functors,sec:fibrations} 中, 看见了同伦类型论中\emph{所有}类型的表现如同空间或者高维群胚.
本章剩余部分的目标是, 弄明白 \cref{cha:typetheory} 中定义的特定类型的高维结构的表现.

事实证明对于很多类型 $A$, 等同类型 $\id[A]xy$ 可以被描述为, 也就是等价于, 每个用于构造 $A$ 的数据组成的表达式.
例如, 如果 $A$ 是一个笛卡尔积 $B\times C$, 而 $x\jdeq (b,c)$ 和 $y\jdeq(b',c')$, 那么有一个等价
\begin{equation}\label{eq:prodeqv}
  \eqv{\big((b,c)=(b',c')\big)}{\big((b=b')\times (c=c')\big)}.
\end{equation}
在更传统 的语言中, 只有在两个有序对子的组成部分都相等时它们相等(而等价~\eqref{eq:prodeqv} 的含义更多).
恒等类型的高阶结构也可以用这些等价符号表达;
例如, 连接两个对子间的等同相当于它的组成部分成对连接.

类似地, 一个类型族 $P:A\to\type$, 被纤维式地使用\cref{cha:typetheory} 中的类型构造规则所构造时, 运算 $\transfib{P}{p}{\blank}$ 可以被描述为, 也就是同伦于, 构造 $P$ 的数据上的对应的算子的表达式.
例如, 如果 $P(x) \jdeq B(x)\times C(x)$, 那么有
\[\transfib{P}{p}{(b,c)} = \big(\transfib{B}{p}{b},\transfib{C}{p}{c}\big).\]

最后, 类型构造规则也是函子, 如果一个函数 $f$ 由这个函子所构造, 那么运算 $\apfunc f$ 和 $\apdfunc f$ 可以在对应构造 $f$ 的数据的基础上计算.
例如, 如果 $g:B\to B'$ 和 $h:C\to C'$ 并定义 $f:B\times C \to B'\times C'$ 为 $f(b,c)\defeq (g(b),h(c))$, 然后对等价~\eqref{eq:prodeqv} 取模, 可以表示 $\apfunc f$ 为 ``$(\apfunc g,\apfunc h)$''.

接下来的一些小节 (\crefrange{sec:compute-cartprod}{sec:compute-nat}) 会专注于陈述和证明这类所有类型基础构造规则的理论, 每个基础类型构造是一个小节.
目前用的类型理论中有明显缺乏某些东西;
接下来的章中, 描述恒等类型, transport 等是\emph{判断}\index{判断等同}等同的, 会更方便和直观.
然而, 在 \cref{cha:typetheory} 提出的理论中, 恒等类型被其归纳原理对于所有类型统一定义, 所以无法 ``重新定义'' 它们, 让它在不同类型下成为不同的东西.
因此, 本章要讨论的特定类型的特征大多数是, 可能发现和证明的\emph{定理}.

实际上, \cref{cha:typetheory}的类型论不足以证明这两个类型的构造: $\Pi$-类型和宇宙.
因此, 会被迫引入公理到我们的类型论中, 来让这些``定理''为真.
类型论上, 一个\emph{公理} (比较~\cref{sec:axioms}) 是一个``原子性''元素, 它被声明是特定类型的居留元, 除了这些与它居留类型有关的, 没有任何类型管理它的行为.
\index{axiom!versus rules}%

\index{函数外延性}%
\indexsee{外延性, 函数的}{函数外延性}
\index{宇宙等价公理}%
$\Pi$-类型的公理(\cref{sec:compute-pi}) 是类型家所熟悉的: 被称为\emph{函数外延性}, 大致就是若两个函数是同伦的(在 \cref{sec:basics-equivalences} 的意义上), 它们就相等.
而宇宙的公理 (\cref{sec:compute-universe}), 是 Voevodsky 对同伦类型论的新贡献: 被称为\emph{宇宙等价公理}, 大致就是若两个类型是等价的(\cref{sec:basics-equivalences}), 它们就相等.
在导论中已经评论了这个公理; 在这本书中它扮演一个重要角色.%
\footnote{我们选择引入这些原则作为公理, 但还有潜在地其它方法, 来指定一个它们成立的类型论.
  参见本章的备注.}

值得注意的是, 不是\emph{所有}恒等类型可以通过在类型的构造上归纳来``确定''.
反例包含了大部分非平凡高维归纳类型 (参见 \cref{cha:hits,cha:homotopy}).
例如, 计算类型 $\Sn^n$ 的恒等类型 (参见 \cref{sec:circle}) 等价于球的计算高维同伦群, 代数拓扑学的一个深刻而重要的研究领域.