我们在 \cref{cha:typetheory}介绍了很多构造新类型的方式: 笛卡尔积, 不交并, 依值乘积, 依值和, 等等.
在 \cref{sec:equality,sec:functors,sec:fibrations}, 可以看到同伦类型论中\emph{所有}类型具备空间或者高维群胚的特征.
本章的目标是搞清楚 \cref{cha:typetheory} 中定义的特定类型的高维结构特征.

事实证明, 对于很多类型 $A$, 等同类型 $\id[A]xy$ 可以用构成 $A$ 的数据等价地表征.
例如, 如果 $A$ 是一个笛卡尔积 $B\times C$, 而 $x\jdeq (b,c)$ 和 $y\jdeq(b',c')$, 那么有等价
\begin{equation}\label{eq:prodeqv}
  \eqv{\big((b,c)=(b',c')\big)}{\big((b=b')\times (c=c')\big)}.
\end{equation}
更传统的语言中, 只有在两个序对的组成部分都相等时, 它们才相等(而等价~\eqref{eq:prodeqv} 有更多的含义).
这些等价关系的式子, 也能表达恒等类型的高阶结构;
例如, 连接两个序对间的等同相当于它的组成部分逐对地连接.

类似地, 类型族 $P:A\to\type$, 如果它由 \cref{cha:typetheory} 中的类型形成规则逐纤维地构造, 运算 $\transfib{P}{p}{\blank}$ 可以用进入 $P$ 的数据上对应的运算同伦地表征.
例如, 如果 $P(x) \jdeq B(x)\times C(x)$, 那么有
\[\transfib{P}{p}{(b,c)} = \big(\transfib{B}{p}{b},\transfib{C}{p}{c}\big).\]

最后, 类型形成规则也是函子性的, 如果用这些函子构造函数 $f$, 那么可以基于进入 $f$ 的数据, 得到运算 $\apfunc f$ 和 $\apdfunc f$.
例如, 如果 $g:B\to B'$ 和 $h:C\to C'$ 并将 $f:B\times C \to B'\times C'$ 定义为 $f(b,c)\defeq (g(b),h(c))$, 然后根据等价~\eqref{eq:prodeqv}, 可以将 $\apfunc f$ 表示为 ``$(\apfunc g,\apfunc h)$''.

接下来的小节 (\crefrange{sec:compute-cartprod}{sec:compute-nat}) 会专注于陈述和证明那些基础类型的形成规则, 每个小节对应一个基础类型形成器.
目前的类型论中, 明显地缺乏某些东西;
接下来的章节中, 恒等类型, 传输等\emph{判断}\index{判断等同}等同, 可以更方便和直观地描述.
然而, \cref{cha:typetheory} 提出的理论中, 恒等类型被其归纳原理对于所有类型统一定义, 所以无法 ``重新定义'' 它们, 让它在不同类型下成为不同的东西.
因此, 本章讨论的哪些特定类型大部分是, 可以发现和证明的\emph{定理}.

实际上, \cref{cha:typetheory}的类型论不足以证明这两个类型的形成器: $\Pi$-类型和全集.
因此, 我们被迫引入公理到我们的类型论中, 来让这些``定理''为真.
类型论上, 一个\emph{公理} (比较~\cref{sec:axioms}) 是一个``原子性''元素, 它被声明是特定类型的居留元, 没有任何规则管理它的行为, 除了这些它的居留类型.
\index{公理!与规则}%

\index{函数外延性}%
\indexsee{外延性, 函数的}{函数外延性}
\index{单值公理}%
$\Pi$-类型的公理(\cref{sec:compute-pi}) 是类型论学家熟悉的: \emph{函数外延性}, 粗略地说就是, 若两个函数在 \cref{sec:basics-equivalences} 的意义上是同伦的, 那么它们相等.
而全集的公理 (\cref{sec:compute-universe}), 是 Voevodsky 对同伦类型论的新贡献: 被称为\emph{单值公理}, 粗略地说就是, 若两个类型等价(\cref{sec:basics-equivalences}), 那么它们相等.
在导论中已经见过这个公理; 这本书中它将扮演一个重要角色.%
\footnote{我们选择引入这些原则作为公理, 但还有潜在地其它方法, 来指定一个它们成立的类型论.
  参见本章的备注.}

值得注意的是, 不是\emph{所有}恒等类型可以通过在类型的构造上归纳来``确定''.
反例包含了大部分非平凡高维归纳类型 (参见 \cref{cha:hits,cha:homotopy}).
例如, 计算类型 $\Sn^n$ 的恒等类型 (参见 \cref{sec:circle}) 等价于球的计算高维同伦群, 代数拓扑学的一个深刻而重要的研究领域.