同伦类型论核心新观点是类型可以看作同论论中的空间, 或者范畴论中的高维广群. 

\index{经典的!同伦理论|(}
\index{高阶范畴理论|(}
我们从简要介绍同伦论和高维范畴论之间的联系开始. 在经典同伦论, 空间 $X$ 是集合具备拓扑结构的点的汇集, \indexsee{空间!拓扑的}{拓扑空间}
\index{拓扑的!空间}
而点 $x$ 和点 $y$ 之间的路径代表连续的映射 $p : [0,1] \to X$, 其中 $p(0) = x$ 而 $p(1) = y$. \index{路径!拓扑的}
\index{拓扑的!路径}
这个函数可以被想象成每个``时刻'', 给定 $X$ 中一个点. 出于许多目的, 路径的严格等同 (即逐点相等的函数) 是一个很好的概念. 例如, 这样可以定义操作即路径串联 (若 $p$ 是一个 从 $x$ 到 $y$ 的路径, 而 $q$ 是 从 $y$ 到 $z$ 的路径, 那么串联 $p \ct q$ 是一个从 $x$ 到 $z$ 的路径) 和逆运算 ($\opp p$ 是一个从 $y$ 到 $x$ 的路径).
 这些运算之间存在自然的等式, 然而它们不满足严格的相等性: 例如, 路径 $p \ct \opp p$ (即从 $x$ 到 $y$, 然后沿着原路返回, 并伴随着时间从 $0$ 到 $1$) 与恒等路径 (即时刻都是 $x$) 不是严格相等的. 

补救措施是一种更松散的路径等同的概念, 也就是\emph{同伦}.
\index{同伦!拓扑的}
$f : X_1 \to X_2$ 和 $g : X_1\to X_2$ 这两个连续映射之间的同论是一个连续映射, 即 $H : X_1 \times [0, 1] \to X_2$, 它满足 $H(x, 0) = f (x)$ 和 $H(x, 1) = g(x)$.
 在 $p$ 和 $q$ 是两个从 $x$ 到 $y$ 的路径的特殊情况下, 同伦是一个连续映射 $H : [0,1] \times [0,1] \rightarrow X$, 其中对于所有 $s\in [0,1]$, 有 $H(s,0) = p(s)$ 而 $H(s,1) = q(s)$ . 这个情况下, 我们还要求对于所有 $t\in [0,1]$ 有 $H(0,t) = x$ 和 $H(1,t)=y$, 因此对于每个 $t$, 函数 $H(\blank,t)$ 仍是是一个从 $x$ 到 $y$ 的函数; 这种同论被称为\emph{端点保留}或者\emph{相对端点}. 在简单的情况下, $H$ 下的正方形 $[0,1]\times [0,1]$ 的图像可以被视为是 ``填充''  $p$ 和 $q$ 之间的 ``空间'', 尽管对于普通的 $X$ 实际上没有意义; 最好是把 $H$ 视为从端点固定的 $p$ 到 $q$ 的连续变形. 由于 $[0,1]\times [0,1]$ 是二维的, $H$ 也被称为\emph{路径之间的}二维\emph{路径}.\index{路径!2-} 

例如, 由于 $p \ct \opp p$ 沿着同样的路径来回, 你也许会想连续缩小 $p \ct \opp p$ 到恒等路径 --- 这样是不行的, 例如, 可能会卡在空间的洞上. 同伦是一个等价关系, 而且串联, 反转之类的操作都满足这个关系. 此外, 群上的点 $x_0$ (当环圈 $p$ 和 $q$ 之间有一个\emph{基础}同伦时, 它们相等, 即同论 $H$, 它额外满足对于所有 $t$ 有 $H(0,t) = H(1,t) = x_0$) 的环圈的同论等价被称为\emph{基本群}.\index{基本!群} 这个群是一个空间的\emph{代数不变量}, 可以用来研究两个空间是否\emph{同伦等价} (存在来回的连续映射, 它由恒等间的同论族层), 因为等价空间具有同构基本群. 

同论是它自己的二维的路径, 所以自然存在\emph{同伦间的}三维\emph{同论},\index{路径!3-} 然后是\emph{同伦间与同伦间之间的同伦}, 以及等等. 这种无限的层级, 点, 路径, 同伦, 同伦间的同伦, \ldots, 具有诸如基本群之类的代数运算, 是被称为 (弱) \emph{$\infty$-groupoid} 的代数结构的一个例子. $\infty$-groupoid\index{.infinity-groupoid@$\infty$-groupoid} 由对象, 对象间的\emph{态射}\indexdef{态射!在一个 .infinity-groupoid@在一个 $\infty$-groupoid}, 然后是\emph{态射间的态射}, 以及等等组成, 它具有一些复杂的代数结构; 一个等级 $k$ 的态射被称为一个 \define{$k$-morphism}\indexdef{k-morphism@$k$-morphism}. 每个等级的态射都有恒等, 复合, 和逆运算, 它们只有在升级到下一级的态射时才满足广群定律 (复合的结合率, 恒等是复合的单位元, 逆取消), 这个意义上它们是弱的, 弱的特性导致了进一步的结构. 例如, 因为态射复合交换律 $p \ct (q \ct r) = (p \ct q) \ct r$ 本身是高维态射, 所以需要额外的操作来关联各种结合律的证明: 各种再结合 $p \ct (q \ct (r \ct s))$ 到 $((p \ct q) \ct r) \ct s$ 的方法产生了麦克兰恩五边形\index{五边形, 麦克兰恩}. 这个弱的特征还产生了等级之间的非平凡相互作用. 

每个拓扑空间 $X$ 有一个\emph{基本$\infty$-groupoid} \index{.infinity-groupoid@$\infty$-groupoid!基本}
\index{基本!.infinity-groupoid@$\infty$-groupoid}
它的 $k$-mor\-ph\-isms 是 $X$ 中的 $k$-dimen\-sional 路径. $\infty$-group\-oid 的弱的特性直接相当于路径构成群只到同论, $(k+1)$-paths 充当 $k$-paths 之间的同伦. 此外, 空间作为 $\infty$-groupoid 的观点保持了足够的同伦论空间的特征: 基本的 $\infty$-groupoid 构造伴随\index{伴随!函子} $\infty$-groupoid 作为空间的几何\index{几何实现}实现, 这个伴随函子保持同伦论 (这被称为\emph{同伦猜想/定理}, \index{猜想!同伦}
\index{同伦!猜想}
因为它是猜想 还是定理依赖于你如何定义 $\infty$-groupoid). 例如, 可以轻松定义 $\infty$-groupoid 的基本群, 而且如果计算空间的 $\infty$-groupoid 的基本群, 它将和经典的基本空间基本群的定义一致. 因为这个对应关系, 同伦论和高维范畴论关系紧密. 

\index{经典的!同伦论|)}%
\index{高阶范畴论|)}%

\mentalpause

现在, 同伦类型论中每个类型可以被视为具有 $\infty$-groupoid 的结构. 回顾对于任何类型 $A$, 和任何 $x,y:A$, 都有恒等类型 $\id[A]{x}{y}$, 也可写作 $\idtype[A]{x}{y}$ 或者 $x=y$. 逻辑上可以把 $x=y$ 的元素作为 $x$ 和 $y$ 相等的证据, 或者看作 $x$ 与 $y$ 恒等. 此外, 类型论 (不同于一阶逻辑)  $\id[A]{x}{y}$ 的元素可以作为个体, 构成其他命题. 因此, 我们可以\emph{重申}恒等类型: 我们可以构造类型
 $\id[{(\id[A]{x}{y})}]{p}{q}$, 即两个恒等式 $p,q$ 之间的恒等式, 以及类型 $\id[{(\id[{(\id[A]{x}{y})}]{p}{q})}]{r}{s}$, 等等. 这种恒等类型的层级结构恰好相当于连续路径和空间之间的 (高阶) 同论, 或者$\infty$-groupoid.\index{.infinity-groupoid@$\infty$-groupoid} 


因此, 我们会经常把元素 $p : \id[A]{x}{y}$ 称为从 $x$ 到 $y$ 的 \define{路径}\index{路径}
我们称 $x$ 为它的\define{起点} \indexdef{路径的起点}
\indexdef{路径!的起点}
而 $y$ 是它的\define{终点}. \indexdef{路径的终点}
\indexdef{路径!的终点}
两个有同样的终点的路径 $p,q : \id[A]{x}{y}$ 被称为\define{并行}, \indexdef{并行路径}
\indexdef{路径!并行}
这种情况下元素 $r : \id[{(\id[A]{x}{y})}]{p}{q}$ 可以被视为同伦, 或者态射间的态射; 它经常被称为\define{2-path} \indexdef{path!2-}\indexsee{2-path}{path, 2-}%
或者\define{二维路径}. \index{维!路径的}%
\indexsee{二维路径}{路径, 2-}\indexsee{路径!二维的}{路径, 2-}%
类似的, $\id[{(\id[{(\id[A]{x}{y})}]{p}{q})}]{r}{s}$ 是二维并行路径之间的\define{3-dimensional 路径}类型,
\indexdef{path!3-}\indexsee{3-path}{path, 3-}\indexsee{3-dimensional 路径}{path, 3-}\indexsee{路径!3-dimensional}{path, 3-}%
依此类推. 类型 $A$ ``类似于集合'', 比如 \nat, 这些迭代的恒等类型将是无聊的 (参见 \cref{sec:basics-sets}), 不过在普通的情况下这可以构建非平凡同论类型. 

%% (Obviously, the
%% notation ``$\id[A]{x}{y}$'' has its limitations here.  The style
%% $\idtype[A]{x}{y}$ is only slightly better in iterations:
%% $\idtype[{\idtype[{\idtype[A]{x}{y}}]{p}{q}}]{r}{s}$.)

同伦类型论和经典同伦论有一个重要区别, 同伦类型论给定了空间的\emph{综合}\index{综合数学}%
\index{几何, 综合}%
\index{亚历山卓的欧基里得}%
描述, 在下面的意义上. 综合几何是欧几里风格的一种集合~\cite{Euclid}: 它开始于基本概念(点和线), 构造 (任意两点连接一条线), 和公理 (所有直角相等), 并按逻辑演绎结果. 这与解析几何相反\index{数学分析},%
其中点和线的概念使用 $\R^n$ 的笛卡尔座标系表示---线是点的集合---而基础的构造和公理都由此推出. 经典的同伦论是分析的(空间和路径由点构成), 同伦类型论是综合的: 点, 路径, 和路径之间的路径是基础的, 不可分的, 原始概念. 

此外, 同伦类型论令人惊奇的一个地方是, 所有的基础的构造和公理---所有的高阶广群结构---都自动从恒等类型的归纳原理衍生出来. 回顾 \cref{sec:identity-types} 中说如果 \begin{itemize}
\item 每个 $x,y:A$ 和每个 $p:\id[A]xy$ 有一个类型 $D(x,y,p)$, 和
\item 对于每个 $a:A$ 有一个元素 $d(a):D(a,a,\refl a)$,
\end{itemize}
那么 \begin{itemize}
\item 存在一个元素 $\indid{A}(D,d,x,y,p):D(x,y,p)$ 对于\emph{每}两个元素 $x,y:A$ 和 $p:\id[A]xy$, 于是 $\indid{A}(D,d,a,a,\refl a) \jdeq d(a)$.
\end{itemize}
换句话说, 给定依赖函数 \begin{align*}
D &:\prd{x,y:A} (\id{x}{y}) \to \type\\
d &:\prd{a:A} D(a,a,\refl{a})
\end{align*}
有依赖函数 \[\indid{A}(D,d):\prd{x,y:A}{p:\id{x}{y}} D(x,y,p)\]
满足 \begin{equation}\label{eq:Jconv}
\indid{A}(D,d,a,a,\refl{a})\jdeq d(a)
\end{equation}
对于每个 $a:A$. 通常, 每当应用归纳规则时, 要么不在乎这个被定义的特殊函数, 要么立即给它取一个不同的名字. 

非正式地, 恒等类型的归纳原理说明, 为了构造一个依赖于恒等类型 $p:\id[A]xy$ 居留元的对象 (或证明一个陈述) , 可以在这个特殊情况下进行构造 (或者证明), 即 $x$ 与 $y$ 相同 (判断上的) 而 $p$ 是自反元素 $\refl{x}:x=x$ (判断上的). 在非正式的情况下, 可以用短语 ``根据归纳, 可以假设\dots''. 这种``自反情况'' 的约减, 类似自然数的归纳证明中的, ``基础实例''和``归纳步骤'' 的约减, 以及不交并或析取的分类分析证明中, ``左实例'' 和 ``右实例''的约简.\index{归纳原理!恒等类型的} %


自然数归纳证明的上下文中, ``转换规则''~\eqref{eq:Jconv}不令人熟悉, 但是在递归定义的相关概念中有类似的概念. 如果一个序列\index{序列} $(a_n)_{n\in \mathbb{N}}$ 是通过给定 $a_0$ 和通过 $a_n$ 来指定 $a_{n+1}$ 来定义的, 那么 $0^{\mathrm{th}}$ 项就\emph{是}给定的结果, 而关联 $a_{n+1}$ 到 $a_n$ 的递推关系适用于所得序列. (这似乎很明显, 不值一提, 不过如果将递归的定义看作计算序列的值的算法\index{算法}, 这正是算法执行的过程.) 规则~\eqref{eq:Jconv} 类似于: 如果通过 $p:x=y$ 来定义一个对象, 那么当 $p$ 为 $\refl{x}:x=x$ 时, 就是它的值, 即 $f(\refl{x})$. 

这个归纳原理赋予每个类型$\infty$-groupoid\index{.infinity-groupoid@$\infty$-groupoid}结构, 而两个类型间的每个函数都具有其类群间的 $\infty$-functor\index{.infinity-functor@$\infty$-functor} 结构. 从数学的角度来看这很有趣, 因为给出了一个处理 $\infty$-groupoids 的新方法. 从类型论的角度来看它很有趣, 因为它揭示了它揭示了与每种类型和函数相关的新操作. 在本章的其余部分, 我们将开始探讨这种结构 