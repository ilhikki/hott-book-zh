\index{泛!性质|(}%
结合上节中的路径计算规则, 各种路径的形成运算满足预期的泛性质, 以同伦的方式, 被解释为等价.
例如, 给定类型 $X,A,B$, 函数
\index{type!product}%
\begin{equation}
    \label{eq:prod-ump-map}
    (X\to A\times B) \to (X\to A)\times (X\to B)
\end{equation}
定义为 $f \mapsto (\proj1 \circ f, \proj2\circ f)$.

\begin{thm}
    \label{thm:prod-ump}
    \index{泛!性质!笛卡尔积的}%
    \eqref{eq:prod-ump-map}是一个等价关系.
\end{thm}
\begin{proof}
    定义 quasi-inverse 为输入 $(g,h)$ 输出 $\lam{x}(g(x),h(x))$.
    (技术上, 使用了笛卡尔积 $(X\to A)\times (X\to B)$ 的归纳原理, 来化简为对子.
    现在开始会经常用到这个原理, 而不显式说明.)

    现在给定 $f:X\to A\times B$, 往返的组合产生了这个函数
    \begin{equation}
        \lam{x} (\proj1(f(x)),\proj2(f(x))).\label{eq:prod-ump-rt1}
    \end{equation}
    根据\cref{thm:path-prod}, 对于任何 $x:X$ 有 $(\proj1(f(x)),\proj2(f(x))) = f(x)$.
    因此, 根据函数的外延性, 函数~\eqref{eq:prod-ump-rt1}等于 $f$.

    另外一边, 给定 $(g,h)$, 往返的组合产生了这个对子 $(\lam{x} g(x),\lam{x} h(x))$.
    通过函数的唯一性定理, 它 (判断) 等于 $(g,h)$.
\end{proof}

实际上, 还有一个依值类型版本的泛性质.
给定一个类型 $X$ 和类型族 $A,B:X\to \type$.
然后有函数
\begin{equation}
    \label{eq:prod-umpd-map}
    \Parens{\prd{x:X} (A(x)\times B(x))} \to \Parens{\prd{x:X} A(x)} \times \Parens{\prd{x:X} B(x)}
\end{equation}
和之前一样, 定义为 $f \mapsto (\proj1 \circ f, \proj2\circ f)$.

\begin{thm}
    \label{thm:prod-umpd}
    \eqref{eq:prod-umpd-map} 是一个等价关系.
\end{thm}
\begin{proof}
    留给读者.
\end{proof}

就像 $\Sigma$-类型是笛卡尔积的推广, 它们满足这个泛性质的推广版本.
直接到依值类型的版本, 假设有类型 $X$ 和类型族 $A:X\to \type$ 与 $P:\prd{x:X} A(x)\to\type$.
那么有一个函数
\index{type!dependent pair}%
\begin{equation}
    \label{eq:sigma-ump-map}
    \Parens{\prd{x:X}\dsm{a:A(x)} P(x,a)} \to
    \Parens{\sm{g:\prd{x:X} A(x)} \prd{x:X} P(x,g(x))}.
\end{equation}
注意如果 $P(x,a) \defeq B(x)$ 其中 $B:X\to\type$, 那么~\eqref{eq:sigma-ump-map} 简化为~\eqref{eq:prod-umpd-map}.

\begin{thm}
    \label{thm:ttac}
    \index{泛!性质!依值对子类型}%
    \eqref{eq:sigma-ump-map} 是一个等价关系.
\end{thm}
\begin{proof}
    和之前一样, 定义一个 quasi-inverse, 输入 $(g,h)$ 并输出函数 $\lam{x} (g(x),h(x))$.
    现在给定 $f:\prd{x:X} \sm{a:A(x)} P(x,a)$, 往返的组合产生了这个函数
    \begin{equation}
        \lam{x} (\proj1(f(x)),\proj2(f(x))).\label{eq:prod-ump-rt2}
    \end{equation}
    现在对于任何 $x:X$, 推论\cref{thm:eta-sigma} ($\Sigma$-类型的唯一性原则) 有
    %
    \begin{equation*}
        (\proj1(f(x)),\proj2(f(x))) = f(x).
    \end{equation*}
    %
    因此, 根据函数外延性,~\eqref{eq:prod-ump-rt2} 等于 $f$.
    另一边, 给定 $(g,h)$, 往返的组合产生了 $(\lam {x} g(x),\lam{x} h(x))$, 也判断等同于 $(g,h)$.
\end{proof}

\index{公理!选择!类型论的}
值得注意的是, 因为命题即证明说明~\eqref{eq:sigma-ump-map} 是 ``选择公理''.
如果将 $\Sigma$ 看作 ``存在'' 而将 $\Pi$ (有时)看作 ``对于所有'', 可以宣布:
\begin{itemize}
    \item $\prd{x:X} \sm{a:A(x)} P(x,a)$ 是 ``对于所有 $x:X$ 存在 $a:A(x)$ 满足 $P(x,a)$'', 而且
    \item $\sm{g:\prd{x:X} A(x)} \prd{x:X} P(x,g(x))$ 是 ``存在选择函数 $g:\prd{x:X} A(x)$ 满足对于所有 $x:X$ 有 $P(x,g(x))$''.
\end{itemize}
因此, \cref{thm:ttac}说, 选择公理不仅是``对的'', 它的先行词实际上等同于它的结论.
(一方面, 古典的\index{数学家!古典的}数学会发现~\eqref{eq:sigma-ump-map}不具有选择公理的常规含义, 因为已经指定了 $g$ 的值, 并没有选择可言.
会在\cref{sec:axiom-choice}回到这一点.)

上面对子类型的泛性质针对 ``映射入'', 也就是范畴论中令人熟悉的积的概念.
然而, 对子类型还有针对 ``映射出'' 的泛性质, 而不那么熟悉.
笛卡尔积的情况下, 非依值版本简单表示了这个笛卡尔闭伴随\index{伴随!函子}:
\[ \eqvspaced{\big((A\times B) \to C\big)}{\big(A\to (B\to C)\big)}.\]
它的依值版本由类型族 $C:A\times B\to \type$ 表示:
\[ \eqvspaced{\Parens{\prd{w:A\times B} C(w)}}{\Parens{\prd{x:A}{y:B} C(x,y)}}. \]
这里从右到左的函数就是 $A\times B$ 的归纳原理, 而从左到右的是对子的求值.
留给读者证明它们是 quasi-inverses.
同样有 $\Sigma$-类型的版本:
\begin{equation}
    \eqvspaced{\Parens{\prd{w:\sm{x:A} B(x)} C(w)}}{\Parens{\prd{x:A}{y:B(x)} C(x,y)}}.\label{eq:sigma-lump}
\end{equation}
同样, 从右到左的函数是归纳原理.

一些其他归纳原理也是这种泛性质的一部分.
例如, 路径归纳是下面这个等价的从右到左:
\index{类型!恒等}%
\index{泛!性质!恒等类型的}%
\begin{equation}
    \label{eq:path-lump}
    \eqvspaced{\Parens{\prd{x:A}{p:a=x} B(x,p)}}{B(a,\refl a)}
\end{equation}
对于任何 $a:A$ 和类型族 $B:\prd{x:A} (a=x) \to\type$.
不过, 递归的归纳类型, 例如自然数, 有更复杂的泛性质; 参见 \cref{cha:induction}.

\index{类型!极限}%
\index{类型!余极限}%
\index{极限!类型的}%
\index{余极限!类型的}%
因为 \cref{thm:prod-ump} 表示了笛卡尔积的常规泛性质 (在恰当的同伦论的意义上), 范畴倾向的读者可能想知道其它类型的极限和余极限.
\cref{ex:coprod-ump} 询问读者余积类型 $A+B$ 预期的泛性质, 而 $\unit$ (终对象) 和 $\emptyt$ (始对象) 的零元情况是容易的.
\index{类型!空}%
\index{类型!单元}%
\indexsee{始!类型}{类型, 空}%
\indexsee{终!类型}{类型, 单元}%

\indexdef{拉回}%
对于拉回, 预期的显式构造成立: 给定 $f:A\to C$ 和 $g:B\to C$, 定义
\begin{equation}
    A\times_C B \defeq \sm{a:A}{b:B} (f(a)=g(b)).\label{eq:defn-pullback}
\end{equation}
在 \cref{ex:pullback} 请读者来验证它.
一些更泛用的同伦极限可以被简单的方法构造, 不过对于余极限, 需要新成分; 参见\cref{cha:hits}.

\index{泛!性质|)}%