\index{自然数|(}%
\index{编解码方法|(}%
自然数同样也是正类型, 可以使用编解码方法来塑造它的路径空间.
这种情况, 不是固定一个端点, 而是一次塑造两端的路径空间.
因此, 恒等的 codes 是类型族
\[\code:\N\to\N\to\type,\]
由两次 \N 上的递归定义, 如下:
\begin{align*}
  \code(0,0) &\defeq \unit\\
  \code(\suc(m),0) &\defeq \emptyt\\
  \code(0,\suc(n)) &\defeq \emptyt\\
  \code(\suc(m),\suc(n)) &\defeq \code(m,n).
\end{align*}
同样通过递归来定义依赖函数 $r:\prd{n:\N} \code(n,n)$, 其中
\begin{align*}
  r(0) &\defeq \ttt\\
  r(\suc(n)) &\defeq r(n).
\end{align*}

\begin{thm}\label{thm:path-nat}
  对于所有 $m,n:\N$ 有 $\eqv{(m=n)}{\code(m,n)}$.
\end{thm}
\begin{proof}
  定义
  \[ \encode : \prd{m,n:\N} (m=n) \to \code(m,n) \]
  通过传输 , $\encode(m,n,p) \defeq \transfib{\code(m,{\blank})}{p}{r(m)}$.
  并定义
  \[ \decode : \prd{m,n:\N} \code(m,n) \to (m=n) \]
  通过两次 $m,n$ 上的归纳.
  当 $m$ 和 $n$ 都是 $0$ 时, 需要函数 $\unit \to (0=0)$, 定义对于所有输入, 返回 $\refl{0}$.
  当 $m$ 是后继而 $n$ 是 $0$, 定义域 $\code(m,n)$ 为 \emptyt, 所以 \emptyt 的消元子即可, 反之亦然.
  而都为后继时, 可以定义 $\decode(\suc(m),\suc(n))$ 为组合
  %
  \begin{narrowmultline*}
    \code(\suc(m),\suc(n))\jdeq\code(m,n)
    \xrightarrow{\decode(m,n)} \narrowbreak
    (m=n)
    \xrightarrow{\apfunc{\suc}}
    (\suc(m)=\suc(n)).
  \end{narrowmultline*}
  %
  接下来展示, 对于所有 $m,n$, $\encode(m,n)$ 和 $\decode(m,n)$ 是 quasi-inverses.

  一方面, 从 $p:m=n$ 开始, 通过在 $p$ 上归纳得到
  \[\decode(n,n,\encode(n,n,\refl{n}))=\refl{n}.\]
  而 $\encode(n,n,\refl{n}) \jdeq r(n)$, 所以得到 $\decode(n,n,r(n)) =\refl{n}$.
  通过在 $n$ 上归纳来证明它.
  如果 $n\jdeq 0$, 那么 $\decode(0,0,r(0)) =\refl{0}$, 通过 \decode 的定义.
  而为后继的情况, 通过归纳法 $\decode(n,n,r(n)) = \refl{n}$, 可以得到 $\apfunc{\suc}(\refl{n}) \jdeq \refl{\suc(n)}$.

  另一变, 从 $c:\code(m,n)$ 开始, 在 $m$ 和 $n$ 上做两次归纳.
  如果都是 $0$, 那么 $\decode(0,0,c) \jdeq \refl{0}$, 而 $\encode(0,0,\refl{0})\jdeq r(0) \jdeq \ttt$.
  因此, 可以从 \cref{sec:compute-unit} 中的所有 $\unit$ 的居留元等于 \ttt 得到.
  如果 $m$ 是 $0$ 但 $n$ 是后继, 那么 $c:\emptyt$, 于是结束, 反之亦然.
  当两个都为后继的情况, 通过归纳法有
  \begin{multline*}
    \encode(\suc(m),\suc(n),\decode(\suc(m),\suc(n),c))\\
    \begin{aligned}
    &= \encode(\suc(m),\suc(n),\apfunc{\suc}(\decode(m,n,c)))\\
    &= \transfib{\code(\suc(m),{\blank})}{\apfunc{\suc}(\decode(m,n,c))}{r(\suc(m))}\\
    &= \transfib{\code(\suc(m),\suc({\blank}))}{\decode(m,n,c)}{r(\suc(m))}\\
    &= \transfib{\code(m,{\blank})}{\decode(m,n,c)}{r(m)}\\
    &= \encode(m,n,\decode(m,n,c))\\
    &= c.
  \end{aligned}
  \end{multline*}
\end{proof}

特殊地, 有
\begin{equation}\label{eq:zero-not-succ}
  \encode(\suc(m),0) : (\suc(m)=0) \to \emptyt
\end{equation}
说明 ``$0$ 不是任何自然数的后继''.
同样也有复合
\begin{narrowmultline}\label{eq:suc-injective}
  (\suc(m)=\suc(n))
  \xrightarrow{\encode} \narrowbreak
  \code(\suc(m),\suc(n))
  \jdeq \code(m,n) \xrightarrow{\decode} (m=n)
\end{narrowmultline}
展示了函数 $\suc$ 是单射.
\index{后继}%

在 \cref{cha:induction,cha:hits} 将学习更通用的正类型.
在 \cref{cha:homotopy} 可以看到, 用于塑造余积和 \nat 的恒等类型的技术, 还可用于计算球体的同伦群.

\index{encode-decode 方法|)}%
\index{自然数)}%