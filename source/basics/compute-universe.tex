\index{类型!全集|(}%
\index{等价关系|(}%
给定两个类型 $A$ 和 $B$, 可以将它们视作全集类型 \type 的元素, 因此可以形成恒等类型 $\id[\type]AB$.
正如导论中介提及的一样, 对于 $\id[\type]AB$ 和 $A$ 到 $B$ 的等价关系 $(\eqv AB)$, \emph{宇价}是它们的在 \cref{sec:basics-equivalences} 中描述的恒等关系.
下面通过这个典型函数来表现这个恒等关系.

\begin{lem}\label{thm:idtoeqv}
  对于类型 $A,B:\type$, 有一个典型的函数,
  \begin{equation}\label{eq:uidtoeqv}
    \idtoeqv : (\id[\type]AB) \to (\eqv A B),
  \end{equation}
  证明如下.
\end{lem}
\begin{proof}
  可以通过等同的归纳直接定义, 不过下列描述更方便.
  \index{恒等!函数}%
  \index{函数!恒等}%
  注意恒等函数 $\idfunc[\type]:\type\to\type$ 可以被视为以全集 \type 为定义域的类型族; 它为每个类型 $X:\type$ 指定类型 $X$ 自己.
  (当视为纤维化时, 全空间是``点类型''的类型 $\sm{A:\type}A$; 参见\cref{sec:object-classification}.)
  因此, 给定路径 $p:A =_\type B$, 有一个 transport\index{transport} 函数 $\transf{p}:A \to B$.
  我们主张 $\transf{p}$ 是一个等价.
  而通过归纳, 足以假定 $p$ 是 $\refl A$, 这种情况下 $\transf{p} \jdeq \idfunc[A]$, 即根据\cref{eg:idequiv}是一个等价关系.
  因此, 可以定义 $\idtoeqv(p)$ 为 $\transf{p}$ (以及上面它是等价关系的证明).
\end{proof}

我们想说 \idtoeqv 是一个等价关系.
然而, 正如 $\happly$ 之于函数类型, 在\cref{cha:typetheory} 中提出的基本的类型论无法证明它.
因此, 就像对函数外延性做的一样, 将这个性质作为公理阐述: Voevodsky 的\emph{宇价公理}.

\begin{axiom}[Univalence]\label{axiom:univalence}
  \indexdef{宇价公理}%
  \indexsee{公理!宇价}{宇价公理}%
  对于任何 $A,B:\type$, 函数~\eqref{eq:uidtoeqv} 是一个等价关系.
\end{axiom}

特殊地, 因此, 我们有
  \[
\eqv{(\id[\type]{A}{B})}{(\eqv A B)}.
\]

严格地说, 宇价公理是关于特殊全集类型 $\UU$ 的声明.
如果一个全集 $\UU$ 满足这个公理, 它是\define{宇价}的.
\indexdef{类型!全集!宇价的}%
\indexdef{宇价的全集}%
除非另外说明 (例如 \cref{sec:univalence-implies-funext} 中) 假定\emph{所有}全集是宇价的.

\begin{rmk}
  宇价公理的重点在于, 使用\cref{sec:basics-equivalences} 中描述``好''的 $\isequiv$ 来定义 $\eqv AB$ , 而不是 (例如) $\sm{f:A\to B} \qinv(f)$.
  参见 \cref{ex:qinv-univalence}.
\end{rmk}

特别地, 宇价意味着\emph{等价的类型可以是恒等的}.
就像在前一节一样, 将等价关系如下分解是有用的:
%
\symlabel{ua}
\begin{itemize}
  \item 一个构造规则 {(\id[\type]{A}{B})}, 记 ``宇价公理'' 为 $\ua$:
  \[
    \ua : ({\eqv A B}) \to (\id[\type]{A}{B}).
  \]
  \item 消去规则, 即 $\idtoeqv$,
  \[
    \idtoeqv \jdeq \transfibf{X \mapsto X} : (\id[\type]{A}{B}) \to (\eqv A B).
  \]
  \item 命题式的计算规则\index{计算规则!命题式的!宇价},
  \[
    \transfib{X \mapsto X}{\ua(f)}{x} = f(x).
  \]
  \item 命题式的唯一性原理: \index{唯一性!原理, 命题式的!宇价的}
  对于任何 $p : \id A B$,
  \[
    \id{p}{\ua(\transfibf{X \mapsto X}(p))}.
  \]
\end{itemize}
%
We can also identify the reflexivity, concatenation, and inverses of equalities in the universe with the corresponding operations on equivalences:
\begin{align*}
  \refl{A} &= \ua(\idfunc[A]) \\
  \ua(f) \ct \ua(g) &= \ua(g\circ f) \\
  \opp{\ua(f)} &= \ua(f^{-1}).
\end{align*}
这些式子中, 第一个根据 $\idfunc[A] = \idtoeqv(\refl{A})$, 通过 \idtoeqv 的定义得到, 以及 \ua 是 \idtoeqv 的逆.
对于第二个, 如果定义 $p \defeq \ua(f)$ 和 $q\defeq \ua(g)$, 然后有
\[ \ua(g\circ f) = \ua(\idtoeqv(q) \circ \idtoeqv(p)) = \ua(\idtoeqv(p\ct q)) = p\ct q\]
使用 \cref{thm:transport-concat} 和 $\idtoeqv$ 的定义.
第三个也类似.

以下观点是 \cref{thm:transport-compose} 的特殊情况, 在应用宇价公理时很有用.

\begin{lem}\label{thm:transport-is-ap}
  对于类型族 $B:A\to\type$ 和 $x,y:A$ 以及 $p:x=y$ 和 $u:B(x)$, 有
  \begin{align*}
    \transfib{B}{p}{u} &= \transfib{X\mapsto X}{\apfunc{B}(p)}{u}\\
    &= \idtoeqv(\apfunc{B}(p))(u).
  \end{align*}
\end{lem}

\index{等价|)}%
\index{类型!全集|)}%