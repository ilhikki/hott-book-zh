\index{类型!全集|(}%
\index{等价关系|(}%
给定两个类型 $A$ 和 $B$, 可以将它们视作全集类型 \type 的元素, 因此可以形成恒等类型 $\id[\type]AB$.
正如导论中提及的一样, \emph{一价}在 \cref{sec:basics-equivalences} 中描述的, 恒等 $\id[\type]AB$ 和 $A$ 到 $B$ 的等价关系 $(\eqv AB)$.
下面通过这个典型的函数来表现这个恒等关系.

\begin{lem}
    \label{thm:idtoeqv}
    对于类型 $A,B:\type$, 有一个经典的函数,
    \begin{equation}
        \label{eq:uidtoeqv}
        \idtoeqv : (\id[\type]AB) \to (\eqv A B),
    \end{equation}
    证明如下.
\end{lem}
\begin{proof}
    可以通过等同的归纳直接定义, 不过下列描述更方便.
    \index{恒等!函数}%
    \index{函数!恒等}%
    注意恒等函数 $\idfunc[\type]:\type\to\type$ 可以被视为类型族, 其中定义域是全集 \type;
    它为每个类型 $X:\type$ 返回类型 $X$ 自己.
    (当视它为纤维化时, 它的全空间是``点类型''的类型 $\sm{A:\type}A$; 参见\cref{sec:object-classification}.)
    因此, 给定路径 $p:A =_\type B$, 有一个传输\index{传输} 函数 $\transf{p}:A \to B$.
    我们主张 $\transf{p}$ 是一个等价关系.
    而通过归纳, 足以假定 $p$ 是 $\refl A$, 这种情况下 $\transf{p} \jdeq \idfunc[A]$, 即根据\cref{eg:idequiv}是一个等价关系.
    因此, 可以定义 $\idtoeqv(p)$ 为 $\transf{p}$ (以及上面它是等价关系的证明).
\end{proof}

我们想说 \idtoeqv 是一个等价关系.
然而, 正如 $\happly$ 之于函数类型, 在\cref{cha:typetheory} 中提出的基本的类型论无法证明它.
因此, 就像对函数外延性做的一样, 我们将这个性质作为公理阐述: Voevodsky 的\emph{一价公理}.

\begin{axiom}[Univalence]
    \label{axiom:univalence}
    \indexdef{一价公理}%
    \indexsee{公理!一价}{一价公理}%
    对于任何 $A,B:\type$, 函数~\eqref{eq:uidtoeqv} 是一个等价关系.
\end{axiom}

特殊地, 因此有
\[
    \eqv{(\id[\type]{A}{B})}{(\eqv A B)}.
\]

严格地说, 一价公理是关于特殊全集类型 $\UU$ 的声明.
如果一个全集 $\UU$ 满足这个公理, 那么它是\define{一价}的.
\indexdef{类型!全集!一价的}%
\indexdef{一价的全集}%
除非另外说明 (例如 \cref{sec:univalence-implies-funext} 中) 假定\emph{所有}全集是一价的.

\begin{rmk}
    一价公理的重点在于, 使用\cref{sec:basics-equivalences} 中描述``好''的 $\isequiv$ 来定义 $\eqv AB$ , 而不是 (例如) $\sm{f:A\to B} \qinv(f)$.
    参见 \cref{ex:qinv-univalence}.
\end{rmk}

特别地, 一价意味着\emph{等价的类型可以是恒等的}.
和上节一样, 将等价关系如下分解是有用的:
%
\symlabel{ua}
\begin{itemize}
    \item 一个构造规则 {(\id[\type]{A}{B})}, 记 ``一价公理'' 为 $\ua$:
    \[
        \ua : ({\eqv A B}) \to (\id[\type]{A}{B}).
    \]
    \item 消去规则, 即 $\idtoeqv$,
    \[
        \idtoeqv \jdeq \transfibf{X \mapsto X} : (\id[\type]{A}{B}) \to (\eqv A B).
    \]
    \item 命题式的计算规则\index{计算规则!命题式的!一价},
    \[
        \transfib{X \mapsto X}{\ua(f)}{x} = f(x).
    \]
    \item 命题式的唯一性原理: \index{唯一性!原理, 命题式的!一价的}
    对于任何 $p : \id A B$,
    \[
        \id{p}{\ua(\transfibf{X \mapsto X}(p))}.
    \]
\end{itemize}
%
对于全集的等同关系, 可以表示出它的自反性, 串联, 和逆, 通过等价关系上的对应运算:
\begin{align*}
    \refl{A} &= \ua(\idfunc[A]) \\
    \ua(f) \ct \ua(g) &= \ua(g\circ f) \\
    \opp{\ua(f)} &= \ua(f^{-1}).
\end{align*}
这些式子中, 第一个根据 $\idfunc[A] = \idtoeqv(\refl{A})$, 通过 \idtoeqv 的定义得到, 以及 \ua 是 \idtoeqv 的逆.
对于第二个, 如果定义 $p \defeq \ua(f)$ 和 $q\defeq \ua(g)$, 那么
\[ \ua(g\circ f) = \ua(\idtoeqv(q) \circ \idtoeqv(p)) = \ua(\idtoeqv(p\ct q)) = p\ct q\]
使用 \cref{thm:transport-concat} 和 $\idtoeqv$ 的定义.
第三个也类似.

以下观点是 \cref{thm:transport-compose} 的特殊情况, 在应用一价公理时很有用.

\begin{lem}
    \label{thm:transport-is-ap}
    对于类型族 $B:A\to\type$ 和 $x,y:A$ 以及 $p:x=y$ 和 $u:B(x)$, 有
    \begin{align*}
        \transfib{B}{p}{u} &= \transfib{X\mapsto X}{\apfunc{B}(p)}{u}\\
        &= \idtoeqv(\apfunc{B}(p))(u).
    \end{align*}
\end{lem}

\index{等价关系|)}%
\index{类型!全集|)}%