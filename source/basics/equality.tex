\index{类型!恒等|(}%
\index{路径|(}%
\index{.infinity-群胚@$\infty$-群胚!类型的结构|(}%
现在从归纳原理开始推导高阶群胚结构.
首先是等同的对称性, 即拓扑学语言中的``路径可逆''.

\begin{lem}
    \label{lem:opp}
    对于每个类型 $A$ 和每个 $x,y:A$ 有函数
    \begin{equation*}
    (x= y)
        \to(y= x)
    \end{equation*}
    记作 $p\mapsto \opp{p}$, 并对于每个 $x:A$ 有 $\opp{\refl{x}}\jdeq\refl{x}$.
    $\opp{p}$ 被称为是 $p$ 的\define{逆}.
    \indexdef{路径!逆}%
    \indexdef{逆!路径的}%
    \index{等同!的对称}%a
    \index{对称!等同的}%
\end{lem}

因为这是第一次使用 ``引理'' 或 ``定理'', 暂停来考虑它的含义.
回顾命题 (可以被证明的语句) 即类型, 而引理和定理是 (已经被证明的语句) 即 \emph{被居留的}类型.
因此, 引理或者定理的语句能转换为类型, 就像 \cref{sec:pat} 中那样, 其证明能转换为那个类型的居留元.
按照通用量词``对于所有''的理解, \cref{lem:opp} 对应的类型是
\[
    \prd{A:\UU}{x,y:A} (x= y)\to(y= x).
\]
\cref{lem:opp} 的证明会由构造这个类型元素组成, 推导判断 $f:\prd{A:\UU}{x,y:A} (x= y)\to(y= x)$ 的 $f$.
然后为它的元素 $f$, 引入符号 $\opp{(\blank)}$, 其中省略了参数 $A$, $x$, 和 $y$ 并从上下文中推断出.
(就像 \cref{sec:types-vs-sets} 中的评论, 第二个语句 ``对于每个 $x:A$ 有 $\opp{\refl{x}}\jdeq\refl{x}$'' 应该被视为单独的判断.)

\begin{proof}[第一个证明]
    假定 $A:\UU$,
    令 $D:\prd{x,y:A}(x= y) \to \type$ 为类型族, 定义为 $D(x,y,p)\defeq (y= x)$.
    换句话说, $D$ 是一个函数, 输入任何 $x,y:A$ 和 $p:x=y$ 返回类型 $y=x$.
    于是有元素
    \begin{equation*}
        d\defeq \lam{x} \refl{x}:\prd{x:A} D(x,x,\refl{x}).
    \end{equation*}
    因此, 恒等类型的归纳原理为每个 $p:(x= y)$ 给出元素
    \narrowequation{ \indid{A}(D,d,x,y,p): (y= x)}.
    现在可以将期望的函数 $\opp{(\blank)}$ 定义为 $\lam{p} \indid{A}(D,d,x,y,p)$, 即 $\opp{p} \defeq \indid{A}(D,d,x,y,p)$.
    这个变换规则~\eqref{eq:Jconv} 提供了所需的 $\opp{\refl{x}}\jdeq \refl{x}$.
\end{proof}

这种非常形式风格的证明, 在不熟悉恒等类型归纳规则时, 非常有帮助.
甚至更形式地, 可以说 \cref{lem:opp} 和它的证明由判断
\begin{narrowmultline*}
    \lam{A}{x}{y}{p} \indid{A}((\lam{x}{y}{p} (y=x)), (\lam{x} \refl{x}), x, y, p)
    \narrowbreak : \prd{A:\UU}{x,y:A} (x= y)\to(y= x)
\end{narrowmultline*}
(以及额外的等同判断)组成.
不过, 我们最终还是更乐于使用自然语言, 例如下面这个等价的证明.

\begin{proof}[第二个证明]
    要构造对于每个 $x,y:A$ 和 $p:x=y$, 有元素 $\opp{p}:y=x$.
    根据归纳, 只需在 $y$ 是 $x$ 且 $p$ 是 $\refl{x}$ 的情况下完成.
    而这个情况下, $p$ 的类型 $x=y$ 以及尝试构造的 $\opp{p}$ 的类型 $y=x$ 都是 $x=x$.
    因此, ``自反的情况'' 下, 可以定义 $\opp{\refl{x}}$ 为 $\refl{x}$.
    然后, 根据归纳原理得到的结果, 以及变换规则 $\opp{\refl{x}}\jdeq\refl{x}$, 恰好是给出的自反性的证明.
\end{proof}

之后的一些证明, 会同时使用两种风格, 来帮助读者习惯第二种.
接下来证明等同的转递性, 或者等价地 ``串联路径''.

\begin{lem}
    \label{lem:concat}
    对于每个类型 $A$ 和每个类型 $x,y,z:A$ 存在一个函数
    \begin{equation*}
    (x= y)
        \to (y= z)\to (x= z),
    \end{equation*}
    写作 $p \mapsto q \mapsto p\ct q$, 于是对于任意 $x:A$ 有 $\refl{x}\ct \refl{x}\jdeq \refl{x}$.
    $p\ct q$ 被称为\define{串联}或 $p$ 和 $q$ 的\define{复合}.
    \indexdef{路径!串联}%
    \indexdef{路径!复合}%
    \indexdef{路径的串联}%
    \indexdef{复合!路径的}%
    \index{等同!的传递性}%
    \index{传递性!等同的}%
\end{lem}

注意我们选择与函数复合相反的顺序来表示路径串联: 从 $p:x=y$ 和 $q:y=z$ 得到 $p\ct q : x=z$, 而从 $f:A\to B$ 和 $g:B\to C$ 得到 $g\circ f : A\to C$ (参见 \cref{ex:composition}).

\begin{proof}[第一个证明]
    所需函数类型是 $\prd{x,y,z:A} (x= y) \to (y= z)\to (x= z)$.
    我们将定义等价的类型 $\prd{x,y:A} (x= y) \to \prd{z:A} (y= z)\to (x= z)$ 来代替, 这允许我们应用两次路径归纳.
    令 $D:\prd{x,y:A} (x=y) \to \type$ 为类型族
    \begin{equation*}
        D(x,y,p)\defeq \prd{z:A}{q:y=z} (x=z).
    \end{equation*}
    注意 $D(x,x,\refl x) \jdeq \prd{z:A}{q:x=z} (x=z)$.
    因此, 为了应用恒等类型的归纳原理到这个 $D$, 需要一个类型函数
    \begin{equation}
        \label{eq:concatD}
        \prd{x:A} D(x,x,\refl{x})
    \end{equation}
    也就是说类型
    \[ \prd{x,z:A}{q:x=z} (x=z). \]
    现在令 $E:\prd{x,z:A}{q:x=z}\type$ 为类型族 $E(x,z,q)\defeq (x=z)$.
    注意 $E(x,x,\refl x) \jdeq (x=x)$.
    因此, 有此函数
    \begin{equation*}
        e(x) \defeq \refl{x} : E(x,x,\refl{x}).
    \end{equation*}
    通过恒等类型归纳原理应用到 $E$, 得到一个函数
    \begin{equation*}
        d : \prd{x,z:A}{q:x=z} E(x,z,q).
    \end{equation*}
    但是 $E(x,z,q)\jdeq (x=z)$, 所以 $d$ 的类型是~\eqref{eq:concatD}.
    因此, 我们可以使用该函数 $d$ 然后应用恒等类型的归纳原理到 $D$, 来得到所需类型的函数
    \begin{equation*}
        \prd{x,y:A} (x= y) \to \prd{z:A} (y= z)\to (x= z)
    \end{equation*}
    因此 $\prd{x,y,z:A} (y=z) \to (x=y) \to (x=z)$.
    这两个归纳原则的转换规则给了我们 $\refl{x}\ct \refl{x}\jdeq \refl{x}$ 对于任何 $x:A$.
\end{proof}

\begin{proof}[第二个证明]
    对于每个 $x,y,z:A$ 和每个 $p:x=y$ 和 $q:y=z$, an 元素 of $x=z$, 我们想构造.
    通过在 $p$ 上归纳, 假设 $y$ 为 $x$ 和 $p$ 是 $\refl{x}$ 即可.
    这个情况下, $q$ 的类型 $y=z$ 是 $x=z$.
    现在在 $q$ 上归纳, 假设 $z$ 是 $x$ 和 $q$ 是 $\refl{x}$ 即可.
    但是在这个情况下, $x=z$ 是 $x=x$, 而我们有 $\refl{x}:(x=x)$.
\end{proof}

读者可能会觉的我们对这个引理给了过于复杂的证明. 事实上, 在 $p$ 上归纳后就可以停止, 因为需要的就是同等 $x=z$, 我们已经有了这样的等同, 即 $q$.
为什么还在 $q$ 上的还做了另一个归纳?

答案就是, 像导论中描述的一样, 我们在做\emph{证明相关}的数学.
\index{数学!证明相关}%
证明一个引理时, 我们在定义元素的一个居留元, 这和证明过程中定义的\emph{特殊}元素有关, 而不仅仅是类型被这个元素居留(即引理的\emph{陈述}).
\cref{lem:concat} 有三个明显的证明: 可以在 $p$ 上归纳, 在 $q$ 上归纳, 或者一起归纳.
如果以三种不同的方式来证明, 那么将拥有同一类型的三个不同元素.
不难证明这三个元素是等同的 (参见 \cref{ex:basics:concat}), 但因为他们不是\emph{定义}等同, 因此可以偏爱其中一个理由.

在 \cref{lem:concat} 的情况下, 差异取决于计算规则.
如果使用在 $p$ 上单次归纳来证明引理, 那么可以得到形式为 $\refl{y} \ct q \jdeq q$ 的计算规则.
如果在 $q$ 上单次归纳来证明, 可以代替为 $p\ct\refl{y}\jdeq p$, 而归纳两次 (如我们所做的) 仅给出 $\refl{x}\ct\refl{x} \jdeq \refl{x}$.

\index{数学!形式}%
在形式数学时, 非对称计算规则有时很方便, 因为它们允许计算机自动简化更多事情.
不过, 在非形式的数学中, 甚至有争议的不形式的情况, 串联操作有时会令人困惑, 因为它不对称的行为而且必须记住哪变是``特殊''的一侧.
对称地对待两边有益于更健壮的证明;
这也是为什么那个证明这样做的原因. (不过, 诚然这是风格上的选择.)

下面的表格总结了目前所做的``等同'', ``同论'', ``高阶群胚"的观点.
\begin{center}
    \medskip
    \begin{tabular}{ccc}
        \toprule
        等同                 & 同伦   & $\infty$-Groupoid \\
        \midrule
        自反\index{等同!的自反}   & 常量路径 & 恒等态射              \\
        对称性\index{等同!对称性}  & 路径的逆 & 态射的逆              \\
        传递性\index{等同!的传递性} & 路径串联 & 态射复合              \\
        \bottomrule
    \end{tabular}
    \medskip
\end{center}

实际上, 传递性经常应用于通过一系列中间步骤证明等同. 我们会使用像 $a=b=c=d$ 这样的通用符号.
如果中间表达式很长, 或者想指定每个等同的见证, 可以写作
\begin{align*}
    a &= b &\text{(通过 $p$)}\\ &= c &\text{(通过 $q$)} \\ &= d &\text{(通过 $r$)}.
\end{align*}
在每种情况, 这个符号都表示元素 $(p\ct q)\ct r: (a=d)$ 的构造.
(它是左结合的, 尽管从 \cref{thm:omg}\ref{item:omg4} 来看区别不大.)
如果 $b$ 与 $c$ 是判断等同的, 那么可以写作
\begin{align*}
    a &= b &\text{(通过 $p$)}\\ &\jdeq c \\ &= d &\text{(通过 $r$)}
\end{align*}
来表示 $p\ct r : (a=d)$ 的构造. 同样遵循普通数学惯例, 不需要显示为符号 (``通过 $p$'' 和 ``通过 $r$'') 显式地提供见证;
而是允许简单提及最重要的 (或者不明显的) 构造见证的要素.
例如, 如果``引理 A'' 指出对于所有 $x$ 和 $y$ 我们有 $f(x)=g(y)$, 那么我们可以写 ``通过引理 A'' 作为 $f(a) = g(b)$ 的理由, 相信读者可以演绎应用引理 A 到 $x\defeq a$ 和$y\defeq b$.
我们也可能会完全忽略理由, 如果我们相信读者有能力猜到它.

现在, 因为证明相关的原因, 我们无法在证明等同的``对称性''与``传递性''后就停下来: 我们需要知道等式上的\emph{操作}是表现良好的.
(在集合论中这个问题是不可见的, 因为对称性和传递性是等同的纯\emph{形式}, 而不是路径上的结构.)
从同伦论的观点来看, 串联和逆只是高阶群胚的``第一级'' --- 我们同样需要这些操作的一致性\index{一致性}定律, 以及在更高维度的类似运算.
例如, 我们知道串联是\emph{结合}的, 而逆提供了串联的\emph{逆}.

\begin{lem}
    \label{thm:omg}%[The $\omega$-groupoid structure of types]
    \index{结合律!路径串联的}%
    \index{单元!路径串联法则}%
    提供 $A:\type$, $x,y,z,w:A$, $p:x= y$, $q:y = z$ 和 $r:z=w$.
    于是有以下:
    \begin{enumerate}
        \item $p= p\ct \refl{y}$ and $p = \refl{x} \ct p$.\label{item:omg1}
        \item $\opp{p}\ct p= \refl{y}$ and $p\ct \opp{p}= \refl{x}$.\label{item:omg2}
        \item $\opp{(\opp{p})}= p$.\label{item:omg3}
        \item $p\ct (q\ct r)= (p\ct q)\ct r$.\label{item:omg4}
    \end{enumerate}
\end{lem}

备注, 特别的, \ref{item:omg1}--\ref{item:omg4} 是命题等同本身, 存在于恒等类型\emph{的}恒等类型中, 就像 $p=_{x=y}q$ 对于 $p,q:x=y$.
拓扑上, 他们是 \emph{路径的路径}, 即同伦.
事实上, 当串联路径 $p$ 和被反转的路径 $\opp p$ 时, 不会确实得到一个常量路径 (相当于在类型论中的等同 $\refl{}$) --- 而是同伦, 或者路径, 它从 $p\ct\opp p$ 到常量路径.

\begin{proof}[\cref{thm:omg} 的证明]
    所有的证明使用等同的归纳原则.
    \begin{enumerate}
        \item \emph{第一个证明:} 令 $D:\prd{x,y:A} (x=y) \to \type$ 为类型族, 并以此给定
        \begin{equation*}
            D(x,y,p)\defeq (p= p\ct \refl{y}).
        \end{equation*}
        然后 $D(x,x,\refl{x})$ 是 $\refl{x}=\refl{x}\ct\refl{x}$.
        因为 $\refl{x}\ct\refl{x}\jdeq\refl{x}$, 所以 $D(x,x,\refl{x})\jdeq (\refl{x}=\refl{x})$.
        因此, 有函数
        \begin{equation*}
            d\defeq\lam{x} \refl{\refl{x}}:\prd{x:A} D(x,x,\refl{x}).
        \end{equation*}
        现在恒等类型的归纳原理对于每个 $p:x= y$ 给出一个元素 $\indid{A}(D,d,x,y,p):(p= p\ct\refl{y})$.
        另一个等式同样被证明了.
        \mentalpause

        \noindent
        \emph{第二个:} 通过对 $p$ 归纳, 足以假定 $y$ 是 $x$ 和 $p$ 是 $\refl x$.
        在这种情况下, 有 $\refl{x}\ct\refl{x}\jdeq\refl{x}$.
        \item \emph{第一个证明:} 令 $D:\prd{x,y:A} (x=y) \to \type$ 为类型族, 并以此给定
        \begin{equation*}
            D(x,y,p)\defeq (\opp{p}\ct p= \refl{y}).
        \end{equation*}
        然后 $D(x,x,\refl{x})$ 是 $\opp{\refl{x}}\ct\refl{x}=\refl{x}$.
        因为 $\opp{\refl{x}}\jdeq\refl{x}$ 和 $\refl{x}\ct\refl{x}\jdeq\refl{x}$, 得到 $D(x,x,\refl{x})\jdeq (\refl{x}=\refl{x})$.
        因此发现这个函数
        \begin{equation*}
            d\defeq\lam{x} \refl{\refl{x}}:\prd{x:A} D(x,x,\refl{x}).
        \end{equation*}
        现在路径归纳对于任何 $p:x= y$ in $A$ 给定一个元素 $\indid{A}(D,d,x,y,p):\opp{p}\ct p=\refl{y}$.
        另一个等同也类似.

        \mentalpause

        \noindent \emph{第二个证明:} 通过归纳, 足以假定 $p$ 是 $\refl x$.
        在这种情况下, 有 $\opp{p} \ct p \jdeq \opp{\refl x} \ct \refl x \jdeq \refl x$.

        \item \emph{第一个证明:} 令 $D:\prd{x,y:A} (x=y) \to \type$ 为类型族, 并以此给定
        \begin{equation*}
            D(x,y,p)\defeq (\opp{\opp{p}}= p).
        \end{equation*}
        然后 $D(x,x,\refl{x})$ 是这个类型 $(\opp{\opp{\refl x}}=\refl{x})$.
        不过因为 $\opp{\refl{x}}\jdeq \refl{x}$ for each $x:A$, 有 $\opp{\opp{\refl{x}}}\jdeq \opp{\refl{x}} \jdeq\refl{x}$, 因此 $D(x,x,\refl{x})\jdeq(\refl{x}=\refl{x})$.
        因此发现这个函数
        \begin{equation*}
            d\defeq\lam{x} \refl{\refl{x}}:\prd{x:A} D(x,x,\refl{x}).
        \end{equation*}
        现在路径归纳对每个 $p:x= y$ 给定一个元素 $\indid{A}(D,d,x,y,p):\opp{\opp{p}}= p$.

        \mentalpause

        \noindent \emph{第二个证明:} 通过归纳, 足以假定 $p$ 是 $\refl x$.
        在这种情况下, 有 $\opp{\opp{p}}\jdeq \opp{\opp{\refl x}} \jdeq \refl x$.

        \item \emph{第一个证明:} 令 $D_1:\prd{x,y:A} (x=y) \to \type$ 为类型族, 并以此给定
        \begin{equation*}
            D_1(x,y,p)\defeq\prd{z,w:A}{q:y= z}{r:z= w} \big(p\ct (q\ct r)= (p\ct q)\ct r\big).
        \end{equation*}
        然后 $D_1(x,x,\refl{x})$ 就是
        \begin{equation*}
            \prd{z,w:A}{q:x= z}{r:z= w} \big(\refl{x}\ct(q\ct r)= (\refl{x}\ct q)\ct r\big).
        \end{equation*}
        为了构造此类型的元素, 令 $D_2:\prd{x,z:A} (x=z) \to \type$ 为类型族
        \begin{equation*}
            D_2 (x,z,q) \defeq \prd{w:A}{r:z=w} \big(\refl{x}\ct(q\ct r)= (\refl{x}\ct q)\ct r\big).
        \end{equation*}
        然后 $D_2(x,x,\refl{x})$ 是
        \begin{equation*}
            \prd{w:A}{r:x=w} \big(\refl{x}\ct(\refl{x}\ct r)= (\refl{x}\ct \refl{x})\ct r\big).
        \end{equation*}
        为了构造\emph{这个}类型的元素, 令 $D_3:\prd{x,w:A} (x=w) \to \type$ 为类型族
        \begin{equation*}
            D_3(x,w,r) \defeq \big(\refl{x}\ct(\refl{x}\ct r)= (\refl{x}\ct \refl{x})\ct r\big).
        \end{equation*}
        然后 $D_3(x,x,\refl{x})$ 就是
        \begin{equation*}
            \big(\refl{x}\ct(\refl{x}\ct \refl{x})= (\refl{x}\ct \refl{x})\ct \refl{x}\big)
        \end{equation*}
        它与类型 $(\refl{x} = \refl{x})$ 定义等同, 因此被 $\refl{\refl{x}}$ 所居留.
        应用三次路径归纳规则, 得到全部所需类型一个元素.

        \mentalpause

        \noindent \emph{第二个证明:} 通过归纳, 足够假设 $p$, $q$, 和 $r$ 都是 $\refl x$.
        在这中情况下, 有
        \begin{align*}
            p\ct (q\ct r)
            &\jdeq \refl{x}\ct(\refl{x}\ct \refl{x})\\
            &\jdeq \refl{x}\\
            &\jdeq (\refl{x}\ct \refl x)\ct \refl x\\
            &\jdeq (p\ct q)\ct r.
        \end{align*}
        因此, 我们有 $\refl{\refl{x}}$ 居留于这个类型. \qedhere
    \end{enumerate}
\end{proof}

\begin{rmk}
    有另外一种定义这种高阶路径的方法.
    例如, 在 \cref{thm:omg}\ref{item:omg4} 可以只在一个或者两个路径上归纳归纳路径, 而不是所有三个.
    每种都可以缠身\emph{定义}不同的证明, 但是它们彼此相等.
    同样, 可以通过回纳证明任意两个特定的证明之间的等同性, 简化问题所有的路径到自反性, 然后就观察到两个证明都简化自己为自反.
\end{rmk}

因为\cref{thm:omg}\ref{item:omg4}, 我们经常经用 $p\ct q\ct r$ 表示 $(p\ct q)\ct r$, 以及用 $p\ct q\ct r \ct s$ 表示 $((p\ct q)\ct r)\ct s$ 等等.
我们选择做结合作为定义, 但是并没有更本的区别.
我们相信读者可以添加\cref{thm:omg}\ref{item:omg4} 的实例, 根据需要关联这样表达式.

我们还没有真正的在高阶群胚结构上做: 路径~\ref{item:omg1}--\ref{item:omg4} 还必须满足其高阶一致性\index{一致性}法则, 他们本身就是高阶路径, \index{结合律!路径串联的!的一致性}%
\index{globular operad}%
\index{operad}%
\index{groupoid!higher}%
以及等等``一直到无穷大'' (这种用法更精确, 例如 globular operad 的概念).
不过, 大部分情况没有必要显式形成整个无限维度.
同伦类型论的一个优点是所有这样的结构只需从恒等类型的归纳性质即可\emph{证明}, 所以显式的多少可以根据需要来.

特别的, 这本书内我们不需要涉及复杂的组合数学来提出精确的概念, ``所有高阶等级的一致结构''.
除普通路径外, 我们还会使用路径的路径 (即类型 $p =_{x=_A y} q$ 的元素), 也就是之前的\emph{2-paths}\index{path!2-}或者\emph{二维路径}, 和路径的路径的路径 (即元素类型的 $r = _{p =_{x=_A y} q} s$), 我们称其为 \emph{3-paths}\index{path!3-}或者\emph{三维路径}.
也可以定义一般化的\emph{$n$维路径}概念\indexdef{路径!n-@$n$-}%
\indexsee{n-path@$n$-path}{path, $n$-}%
\indexsee{n-dimensional path@$n$-dimensional path}{path, $n$-}%
\indexsee{path!n-dimensional@$n$-dimensional}{path, $n$-}%
(参见\cref{ex:npaths}), 不过我们不需要它.

不过, 我们使用高阶路径的一种特别重要的简单情况, 也就是当起点和终点相同时.
在集合论中, 命题 $a=a$ 完全没有意义, 不过在同伦论, 从一个点到它自己的路径被称为\emph{环圈}\index{环圈}, 而它带来了很多有趣的高阶结构.
因此, 给定一个类型 $A$ 和一个点 $a:A$, 我们定义它的\define{环圈空间} \index{环圈空间}%
$\Omega(A,a)$ 为类型 $\id[A]{a}{a}$. 有时简写为 $\Omega A$ 如果点 $a$ 在上下文中是明确的.

因为任意两个 $\Omega A$ 的元素都是具有相同起点和终点的路径, 可以被串联;
因此我们有操作 $\Omega A\times \Omega A\to \Omega A$. 推而广之, $A$ 的高阶群胚结构给了 $\Omega A$ 类似的一个``高阶群''结构.

$A$ 的环圈空间\emph{的}环圈空间\index{环圈空间!迭代的}\index{迭代的环圈空间}也是有用的, 它就是 $a$ 的恒等环圈上, 二维环圈的空间.
写做 $\Omega^2(A,a)$ 和表示在类型论中的类型 $\id[({\id[A]{a}{a}})]{\refl{a}}{\refl{a}}$.
$\Omega^2(A,a)$ 作为一个环圈空间是一个``高阶群'', 而它还有一些额外的结构, 因为它的元素是一维环圈之间的二维的环圈.

\begin{thm}[Eckmann--Hilton]
    \label{thm:EckmannHilton}
    第二个环圈空间上的复合运算符
%
    \begin{equation*}
        \Omega^2(A)\times \Omega^2(A)\to \Omega^2(A)
    \end{equation*}
    对于 $\alpha, \beta:\Omega^2(A)$, 满足交换律: $\alpha\ct\beta = \beta\ct\alpha$.
    \index{Eckmann–Hilton argument}%
\end{thm}

\begin{proof}
    首先, 观察到 $1$-loops 的复合 $\Omega A\times \Omega A\to \Omega A$ 得到一个运算符
    \[
        \star : \Omega^2(A)\times \Omega^2(A)\to \Omega^2(A)
    \]
    如下: 考虑到元素 $a, b, c : A$ 以及一维二维路径, %

    \begin{align*}
        p &: a = b, &r &: b = c \\
        q &: a = b, &s &: b = c \\
        \alpha &: p = q, &\beta &: r = s
    \end{align*}
%
    如下图所示(其中箭头代表路径). % Changed this to xymatrix in the name of having uniform source code,

% maybe the original using xy looked better (I think it was too big).
% It is commented out below in case you want to reinstate it.
    \[
        \xymatrix@+5em{
                {a} \rtwocell<10>^p_q{\alpha}
            &
                {b} \rtwocell<10>^r_s{\beta}
            &
                {c}
        }
    \]
    分别复合上方和下方的一维路径, 我们得到两个路径 $p\ct r,\ q\ct s : a = c$, 它们都是``水平复合'' %

    \begin{equation*}
        \alpha\hct\beta : p\ct r = q\ct s
    \end{equation*}
%
    它们之间, 有如下定义. 首先, 通过路径归纳在 $r$ 上定义 $\alpha \rightwhisker r : p\ct r = q\ct r$, 有
    \[
        \alpha \rightwhisker \refl{b} \jdeq \opp{\mathsf{ru}_p} \ct \alpha \ct \mathsf{ru}_q
    \]
    其中 $\mathsf{ru}_p : p = p \ct \refl{b}$ 是 \cref{thm:omg}\ref{item:omg1} 中的右单元法则.
    类似地可以通过 $\alpha$ 上的归纳定义 $\rightwhisker$, 和其它的所有路径, 得到不同的判断等同, 不过在 $r$ 上归纳定义更简单.
    类似的, 通过归纳 $q$ 定义 $q\leftwhisker \beta : q\ct r = q\ct s$, 于是
    \[ \refl{b} \leftwhisker \beta \jdeq \opp{\mathsf{lu}_r} \ct \beta \ct \mathsf{lu}_s \]
    其中 $\mathsf{lu}_r$ 表示做单元法则.
    操作符 $\leftwhisker$ 和 $\rightwhisker$ 被称为\define{触须}\indexdef{触须}.
    然后, 因为 $\alpha \rightwhisker r$ 和 $q\leftwhisker \beta$ 是二维路径, 我们可以定义\define{水平复合} \indexdef{水平复合!路径的}%
    \indexdef{复合!路径的!水平的}%
    通过:
    \[
        \alpha\hct\beta\ \defeq\ (\alpha\rightwhisker r) \ct (q\leftwhisker \beta).
    \]
    现在令 $a \jdeq b \jdeq c$, 于是所有一维路径 $p$, $q$, $r$, 和 $s$ 都是元素 $\Omega(A,a)$, 此外假设 $p\jdeq q \jdeq r \jdeq s\jdeq \refl{a}$, 于是 $\alpha:\refl{a} = \refl{a}$ 和 $\beta:\refl{a} = \refl{a}$ 都能以相同方式复合.
    这种情况, 我们有
    \begin{align*}
        \alpha\hct\beta
        &\jdeq (\alpha\rightwhisker\refl{a}) \ct (\refl{a}\leftwhisker \beta)\\
        &= \opp{\mathsf{ru}_{\refl{a}}} \ct \alpha \ct \mathsf{ru}_{\refl{a}} \ct \opp{\mathsf{lu}_{\refl a}} \ct \beta \ct \mathsf{lu}_{\refl{a}}\\
        &\jdeq \opp{\refl{\refl{a}}} \ct \alpha \ct \refl{\refl{a}} \ct \opp{\refl{\refl a}} \ct \beta \ct \refl{\refl{a}}\\
        &= \alpha \ct \beta.
    \end{align*}
    (回顾 $\mathsf{ru}_{\refl{a}} \jdeq \mathsf{lu}_{\refl{a}} \jdeq \refl{\refl{a}}$, 通过路径归纳的计算规则.)
    一方面, 我们类似地定义另一复合, 通过
    \[
        \alpha\hct'\beta\ \defeq\ (p\leftwhisker \beta)\ct (\alpha\rightwhisker s)
    \]
    我们类似地了解到
    \[
        \alpha\hct'\beta = \beta\ct\alpha.
    \]
    \index{交换法}%
    但是, 普通的, 这两种定义水平复合满足了, $\alpha\hct\beta = \alpha\hct'\beta$, 可以看见通过在 $\alpha$ 和 $\beta$ 上归纳从而在剩下两个一维路径上, 缩小到自反.
    因此 我们有
    \[\alpha \ct \beta = \alpha\hct\beta = \alpha\hct'\beta = \beta\ct\alpha.
    \qedhere
    \]
\end{proof}

上述的事实, 被称为 \emph{Eckmann–Hilton 观点}, 来自于经典的同伦理论, 而且的确在后面的\cref{cha:homotopy}中使用它来表明高阶同伦群的类型始终为阿贝尔\index{群!阿贝尔}群.
触须和水平复合操作定义, 同样也是 $\infty$-groupoid 结构类型的的一部分.
它们满足自身的法则(直到高阶同伦), 比如
\[
    \alpha \rightwhisker (p\ct q) = (\alpha \rightwhisker p) \rightwhisker q
\]
和等等.
从现在开始, 我们相信读者可以在需要时应用路径归纳来定义进一步的操作并校验其属性.

这个例子表明了, 高阶路径类型代数比每个等级上的群结构更加复杂;
每个等级的相互作用产生了进一步的操作和法则, 例如在同伦论中环圈空间迭代的研究中.
的确, 在经典的同伦理论, 我们可以做下面的通用的定义:

\begin{defn}
    \label{def:pointedtype}
    一个\define{点类型}
    \indexsee{点!类型}{类型, 点}%
    \indexdef{类型!点}%
    $(A,a)$ 是一个类型 $A:\type$, 伴随一个点 $a:A$, 被称为\define{单位元}.
    \indexdef{单位元}%
    我们写作 $\pointed{\type} \defeq \sm{A:\type} A$ 表示为宇宙 $\type$ 中的点类型.
\end{defn}

\begin{defn}
    \label{def:loopspace}
    给定一个点类型 $(A,a)$, 我们定义 $(A,a)$ 的\define{环圈空间}
    \indexdef{环圈空间}%
    为下面的点类型:
    \[\Omega(A,a)\defeq ((\id[A]aa),\refl a).\]
    它的一个元素被称为一个 $a$ 上的\define{环圈}\indexdef{环圈}.
    对于 $n:\N$, 点类型 $(A,a)$ 的 \define{$n$-fold 迭代环圈空间} $\Omega^{n}(A,a)$
    \indexdef{环圈空间!迭代}%
    \indexsee{环圈空间!n-fold@$n$-fold}{环圈空间, 迭代}%
    是通过递归来定义:
    \begin{align*}
        \Omega^0(A,a)&\defeq(A,a)\\
        \Omega^{n+1}(A,a)&\defeq\Omega^n(\Omega(A,a)).
    \end{align*}
    它的元素被称为 $a$ 上的\define{$n$-环圈}
    \indexdef{环圈!n-@$n$-}%
    \indexsee{n-loop@$n$-环圈}{环圈, $n$-}%
    或者\define{$n$ 维环圈}.
    \indexsee{环圈!n 维@$n$ 维}{环圈, $n$ }%
    \indexsee{n 维环圈@$n$ 维环圈}{环圈, $n$}%

\end{defn}

我们会在\cref{cha:hlevels,cha:hits,cha:homotopy}返回到迭代环圈空间.
\index{.infinity-groupoid@$\infty$-groupoid!结构的类型|)}%
\index{类型!恒等|)}
\index{路径|)}%