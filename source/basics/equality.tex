\index{类型!恒等|(}%
\index{路径|(}%
\index{.infinity-广群@$\infty$-广群!类型的结构|(}%
现在从归纳原理开始推导高阶广群结构.
首先是等同的对称性, 即拓扑学语言中的``路径可逆''.

\begin{lem}
    \label{lem:opp}
    对于每个类型 $A$ 和每个 $x,y:A$ 有函数
    \begin{equation*}
    (x= y)
        \to(y= x)
    \end{equation*}
    记作 $p\mapsto \opp{p}$, 并对于每个 $x:A$ 有 $\opp{\refl{x}}\jdeq\refl{x}$.
    $\opp{p}$ 被称为是 $p$ 的\define{逆}.
    \indexdef{路径!逆}%
    \indexdef{逆!路径的}%
    \index{等同!的对称}%a
    \index{对称!等同的}%
\end{lem}

因为这是第一次使用 ``引理'' 或 ``定理'', 暂停来考虑它的含义.
回顾命题 (可以被证明的语句) 即类型, 而引理和定理是 (已经被证明的语句) 即 \emph{被居留的}类型.
因此, 引理或者定理的语句能转换为类型, 就像 \cref{sec:pat} 中那样, 其证明能转换为那个类型的居留元.
按照通用量词``对于所有''的理解, \cref{lem:opp} 对应的类型是
\[
    \prd{A:\UU}{x,y:A} (x= y)\to(y= x).
\]
\cref{lem:opp} 的证明会由构造这个类型元素组成, 推导判断 $f:\prd{A:\UU}{x,y:A} (x= y)\to(y= x)$ 的 $f$.
然后为它的元素 $f$, 引入符号 $\opp{(\blank)}$, 其中省略了参数 $A$, $x$, 和 $y$ 并从上下文中推断出.
(就像 \cref{sec:types-vs-sets} 中的评论, 第二个语句 ``对于每个 $x:A$ 有 $\opp{\refl{x}}\jdeq\refl{x}$'' 应该被视为单独的判断.)

\begin{proof}[第一个证明]
    假定 $A:\UU$,
    令 $D:\prd{x,y:A}(x= y) \to \type$ 为类型族, 定义为 $D(x,y,p)\defeq (y= x)$.
    换句话说, $D$ 是一个函数, 输入任何 $x,y:A$ 和 $p:x=y$ 返回类型 $y=x$.
    于是有元素
    \begin{equation*}
        d\defeq \lam{x} \refl{x}:\prd{x:A} D(x,x,\refl{x}).
    \end{equation*}
    因此, 恒等类型的归纳原理为每个 $p:(x= y)$ 给出元素
    \narrowequation{ \indid{A}(D,d,x,y,p): (y= x)}.
    现在可以将期望的函数 $\opp{(\blank)}$ 定义为 $\lam{p} \indid{A}(D,d,x,y,p)$, 即 $\opp{p} \defeq \indid{A}(D,d,x,y,p)$.
    这个变换规则~\eqref{eq:Jconv} 提供了所需的 $\opp{\refl{x}}\jdeq \refl{x}$.
\end{proof}

这种非常形式风格的证明, 在不熟悉恒等类型归纳规则时, 非常有帮助.
甚至更形式地, 可以说 \cref{lem:opp} 和它的证明由判断
\begin{narrowmultline*}
    \lam{A}{x}{y}{p} \indid{A}((\lam{x}{y}{p} (y=x)), (\lam{x} \refl{x}), x, y, p)
    \narrowbreak : \prd{A:\UU}{x,y:A} (x= y)\to(y= x)
\end{narrowmultline*}
(以及额外的等同判断)组成.
不过, 我们最终还是更乐于使用自然语言, 例如下面这个等价的证明.

\begin{proof}[第二个证明]
    要构造对于每个 $x,y:A$ 和 $p:x=y$, 有元素 $\opp{p}:y=x$.
    根据归纳, 只需在 $y$ 是 $x$ 且 $p$ 是 $\refl{x}$ 的情况下完成.
    而这个情况下, $p$ 的类型 $x=y$ 以及尝试构造的 $\opp{p}$ 的类型 $y=x$ 都是 $x=x$.
    因此, ``自反的情况'' 下, 可以定义 $\opp{\refl{x}}$ 为 $\refl{x}$.
    然后, 根据归纳原理得到的结果, 以及变换规则 $\opp{\refl{x}}\jdeq\refl{x}$, 恰好是给出的自反性的证明.
\end{proof}

之后的一些证明, 会同时使用两种风格, 来帮助读者习惯第二种.
接下来证明等同的转递性, 或者等价地 ``串联路径''.

\begin{lem}
    \label{lem:concat}
    对于每个类型 $A$ 和每个 $x,y,z:A$ 存在函数
    \begin{equation*}
    (x= y)
        \to (y= z)\to (x= z),
    \end{equation*}
    写作 $p \mapsto q \mapsto p\ct q$, 并对于任意 $x:A$ 有 $\refl{x}\ct \refl{x}\jdeq \refl{x}$.
    $p\ct q$ 被称为\define{串联}或 $p$ 和 $q$ 的\define{复合}.
    \indexdef{路径!串联}%
    \indexdef{路径!复合}%
    \indexdef{路径的串联}%
    \indexdef{复合!路径的}%
    \index{等同!的传递性}%
    \index{传递性!等同的}%
\end{lem}

注意路径串联的顺序与函数复合相反: 从 $p:x=y$ 和 $q:y=z$ 得到 $p\ct q : x=z$, 而从 $f:A\to B$ 和 $g:B\to C$ 得到 $g\circ f : A\to C$ (参见 \cref{ex:composition}).

\begin{proof}[第一个证明]
    期望的函数具有类型 $\prd{x,y,z:A} (x= y) \to (y= z)\to (x= z)$.
    作为替代, 定义一个函数, 具有等价的类型 $\prd{x,y:A} (x= y) \to \prd{z:A} (y= z)\to (x= z)$, 这允许应用两次路径归纳.
    令 $D:\prd{x,y:A} (x=y) \to \type$ 为类型族
    \begin{equation*}
        D(x,y,p)\defeq \prd{z:A}{q:y=z} (x=z).
    \end{equation*}
    注意 $D(x,x,\refl x) \jdeq \prd{z:A}{q:x=z} (x=z)$.
    因此, 为了应用恒等类型的归纳原理于 $D$, 需要函数, 类型为
    \begin{equation}
        \label{eq:concatD}
        \prd{x:A} D(x,x,\refl{x})
    \end{equation}
    也就是类型
    \[ \prd{x,z:A}{q:x=z} (x=z). \]
    现在令 $E:\prd{x,z:A}{q:x=z}\type$ 为类型族 $E(x,z,q)\defeq (x=z)$.
    注意 $E(x,x,\refl x) \jdeq (x=x)$.
    因此, 有函数
    \begin{equation*}
        e(x) \defeq \refl{x} : E(x,x,\refl{x}).
    \end{equation*}
    通过应用恒等类型的归纳原理于 $E$, 得到一个函数
    \begin{equation*}
        d : \prd{x,z:A}{q:x=z} E(x,z,q).
    \end{equation*}
    但是 $E(x,z,q)\jdeq (x=z)$, 所以 $d$ 的类型是~\eqref{eq:concatD}.
    因此, 可以使用函数 $d$, 然后应用恒等类型的归纳原理于 $D$, 来得到期望的函数, 类型为
    \begin{equation*}
        \prd{x,y:A} (x= y) \to \prd{z:A} (y= z)\to (x= z)
    \end{equation*}
    因此 $\prd{x,y,z:A} (y=z) \to (x=y) \to (x=z)$.
    这两个归纳原则的变换规则对于任何 $x:A$, 满足 $\refl{x}\ct \refl{x}\jdeq \refl{x}$ .
\end{proof}

\begin{proof}[第二个证明]
    对于每个 $x,y,z:A$ 与每个 $p:x=y$ 和 $q:y=z$, 要构造 $x=z$ 的元素.
    通过在 $p$ 上归纳, 可以假设 $y$ 为 $x$ 和 $p$ 是 $\refl{x}$.
    这个情况下, $q$ 的类型 $y=z$ 是 $x=z$.
    现在, 在 $q$ 上归纳, 可以假设 $z$ 是 $x$ 和 $q$ 是 $\refl{x}$.
    而这个情况下, $x=z$ 是 $x=x$, 从而有 $\refl{x}:(x=x)$.
\end{proof}

读者可能会觉的给了这个引理过于复杂的证明. 实际上, 在 $p$ 上归纳后就可以停止, 因为需要的就是同等 $x=z$, 我们已经有了这样的等同, 即 $q$.
为什么还在 $q$ 上的还做了另一个归纳?

答案就是, 像导论中描述的一样, 我们在做\emph{证明相关}的数学.
\index{数学!证明相关}%
证明引理时, 我们做的是定义某些类型的居留元, 它和证明过程中定义的\emph{特殊}元素紧密相关, 不仅是类型被元素居留(即引理的\emph{语句}).
\cref{lem:concat} 有三个明显的证明: 可以在 $p$ 上归纳, 在 $q$ 上归纳, 或者都归纳.
如果以三种不同的方式来证明, 那么将拥有同一类型的三个不同元素.
不难证明这三个元素相等同 (参见 \cref{ex:basics:concat}), 但因为它们不是\emph{定义}等同的, 有理由选择其中一个而不是其它.

\cref{lem:concat} 的情况下, 差异取决于计算规则.
如果使用在 $p$ 上单次归纳来证明引理, 最终可以得到形式 $\refl{y} \ct q \jdeq q$ 的计算规则.
如果在 $q$ 上单次归纳来证明, 那就是 $p\ct\refl{y}\jdeq p$, 而归纳两次 (正如所做的) 仅给出 $\refl{x}\ct\refl{x} \jdeq \refl{x}$.

\index{数学!形式}%
做形式数学时, 不对称的计算规则有时很方便, 因为它们允许计算机自动简化更多事情.
不过, 非形式的数学中, 甚至不形式的情况, 非对称的串联运算会令人困惑, 因为它不对称的行为而且必须记住哪边是``特殊''的一侧.
对称地对待两边有益于更健壮的证明;
这也是为什么这个证明如此做的原因. (诚然, 这也是风格上的选择.)

下面的表格总结了目前所做的``等同'', ``同论'', ``高阶广群"的观点.
\begin{center}
    \medskip
    \begin{tabular}{ccc}
        \toprule
        等同                 & 同伦    & $\infty$-广群 \\
        \midrule
        自反\index{等同!的自反}   & 常量路径  & 恒等态射        \\
        对称性\index{等同!对称性}  & 路径的逆  & 逆态射         \\
        传递性\index{等同!的传递性} & 路径的串联 & 态射的复合       \\
        \bottomrule
    \end{tabular}
    \medskip
\end{center}

实际上, 传递性经常应用于通过一系列中间步骤证明等同.
可以会使用常规符号, 诸如 $a=b=c=d$.
如果中间表达式很长, 或者要指定每个等同的见证, 可以写成
\begin{align*}
    a &= b &\text{(根据 $p$)}\\ &= c &\text{(根据 $q$)} \\ &= d &\text{(根据 $r$)}.
\end{align*}
总之, 这个符号表示元素 $(p\ct q)\ct r: (a=d)$ 的构造.
(它是左结合的, 尽管从 \cref{thm:omg}\ref{item:omg4} 来看区别不大.)
如果 $b$ 判断等同于 $c$, 那么可以写成
\begin{align*}
    a &= b &\text{(根据 $p$)}\\ &\jdeq c \\ &= d &\text{(根据 $r$)}
\end{align*}
来表示 $p\ct r : (a=d)$ 的构造.
同样遵循常规数学惯例, 不需要为符号 (``通过 $p$'' 和 ``通过 $r$'') 显式地提供见证;
而是允许只提及最重要的 (或者不明显的) 构造见证的部分.
例如, 如果``引理 A'' 指出对于所有 $x$ 和 $y$ 有 $f(x)=g(y)$, 那么可以将 ``根据引理 A'' 作为 $f(a) = g(b)$ 的理由, 相信读者可以推断出, 应用了引理 A 于 $x\defeq a$ 和 $y\defeq b$.
也可能会完全忽略理由, 如果相信读者有能力猜出它.

现在, 因为 proof-relevance, 不能止步于证明等同的``对称性''和``传递性'': 必须知道等同上的\emph{运算}是表现良好的.
(集合论没有这个问题, 因为对称性和传递性纯粹是等同的\emph{性质}, 而不是路径上的结构.)
从同伦论的观点来看, 串联和逆只是高阶广群的``第一级'' --- 同样需要这些运算的一致性\index{一致性}法则, 以及在更高维度的类似的运算.
例如, 需要知道串联满足\emph{结合律}, 而逆提供了\emph{逆}串联.

\begin{lem}
    \label{thm:omg}%[The $\omega$-groupoid structure of types]
    \index{结合律!路径串联的}%
    \index{单元!路径串联法则}%
    提供 $A:\type$, $x,y,z,w:A$, $p:x= y$, $q:y = z$ 和 $r:z=w$.
    于是有以下:
    \begin{enumerate}
        \item $p= p\ct \refl{y}$ and $p = \refl{x} \ct p$.\label{item:omg1}
        \item $\opp{p}\ct p= \refl{y}$ and $p\ct \opp{p}= \refl{x}$.\label{item:omg2}
        \item $\opp{(\opp{p})}= p$.\label{item:omg3}
        \item $p\ct (q\ct r)= (p\ct q)\ct r$.\label{item:omg4}
    \end{enumerate}
\end{lem}

注意, 特别地, \ref{item:omg1}--\ref{item:omg4} 本身就是命题等同, 居留于恒等类型\emph{的}恒等类型, 就像 $p=_{x=y}q$ 之于 $p,q:x=y$.
拓扑学上, 它们是\emph{路径的路径}, 即同伦.
拓扑学中, 串联路径 $p$ 和被逆路径 $\opp p$ 时, 不会字面地得到常量路径 (对应于于在类型论中的等同 $\refl{}$) --- 而是从 $p\ct\opp p$ 到常量路径的同伦或者高维路径.

\begin{proof}[\cref{thm:omg} 的证明]
    所有的证明使用等同的归纳原理.
    \begin{enumerate}
        \item \emph{第一个证明:} 令 $D:\prd{x,y:A} (x=y) \to \type$ 为类型族, 并如此给定
        \begin{equation*}
            D(x,y,p)\defeq (p= p\ct \refl{y}).
        \end{equation*}
        那么 $D(x,x,\refl{x})$ 就是 $\refl{x}=\refl{x}\ct\refl{x}$.
        因为 $\refl{x}\ct\refl{x}\jdeq\refl{x}$, 所以 $D(x,x,\refl{x})\jdeq (\refl{x}=\refl{x})$.
        于是, 有函数
        \begin{equation*}
            d\defeq\lam{x} \refl{\refl{x}}:\prd{x:A} D(x,x,\refl{x}).
        \end{equation*}
        现在恒等类型的归纳原理为每个 $p:x= y$ 给出了一个元素 $\indid{A}(D,d,x,y,p):(p= p\ct\refl{y})$.
        另一个等同的证明也类似.
        \mentalpause

        \noindent
        \emph{第二个:} 通过对 $p$ 归纳, 足以认为 $y$ 就是 $x$ 而 $p$ 就是 $\refl x$.
        这种情况下, 有 $\refl{x}\ct\refl{x}\jdeq\refl{x}$.
        \item \emph{第一个证明:} 令 $D:\prd{x,y:A} (x=y) \to \type$ 为类型族, 并如此给定
        \begin{equation*}
            D(x,y,p)\defeq (\opp{p}\ct p= \refl{y}).
        \end{equation*}
        那么 $D(x,x,\refl{x})$ 就是 $\opp{\refl{x}}\ct\refl{x}=\refl{x}$.
        因为 $\opp{\refl{x}}\jdeq\refl{x}$ 和 $\refl{x}\ct\refl{x}\jdeq\refl{x}$, 所以得到 $D(x,x,\refl{x})\jdeq (\refl{x}=\refl{x})$.
        由此发现这个函数
        \begin{equation*}
            d\defeq\lam{x} \refl{\refl{x}}:\prd{x:A} D(x,x,\refl{x}).
        \end{equation*}
        现在路径归纳对 $A$ 中每个 $p:x= y$ 给定一个元素 $\indid{A}(D,d,x,y,p):\opp{p}\ct p=\refl{y}$.
        另一个等同也类似.

        \mentalpause

        \noindent \emph{第二个证明:} 通过归纳, 足以认为 $p$ 是 $\refl x$.
        在这个情况下, 有 $\opp{p} \ct p \jdeq \opp{\refl x} \ct \refl x \jdeq \refl x$.

        \item \emph{第一个证明:} 令 $D:\prd{x,y:A} (x=y) \to \type$ 为类型族, 并如此给定
        \begin{equation*}
            D(x,y,p)\defeq (\opp{\opp{p}}= p).
        \end{equation*}
        那么 $D(x,x,\refl{x})$ 就是类型 $(\opp{\opp{\refl x}}=\refl{x})$.
        而对于每个 $x:A$ 有 $\opp{\refl{x}}\jdeq \refl{x}$, 所以有 $\opp{\opp{\refl{x}}}\jdeq \opp{\refl{x}} \jdeq\refl{x}$, 因此 $D(x,x,\refl{x})\jdeq(\refl{x}=\refl{x})$.
        由此发现这个函数
        \begin{equation*}
            d\defeq\lam{x} \refl{\refl{x}}:\prd{x:A} D(x,x,\refl{x}).
        \end{equation*}
        现在路径归纳对每个 $p:x= y$ 能给定元素 $\indid{A}(D,d,x,y,p):\opp{\opp{p}}= p$.

        \mentalpause

        \noindent \emph{第二个证明:} 通过归纳, 足以认为 $p$ 就是 $\refl x$.
        这种情况下, 有 $\opp{\opp{p}}\jdeq \opp{\opp{\refl x}} \jdeq \refl x$.

        \item \emph{第一个证明:} 令 $D_1:\prd{x,y:A} (x=y) \to \type$ 为类型族, 并如此给定
        \begin{equation*}
            D_1(x,y,p)\defeq\prd{z,w:A}{q:y= z}{r:z= w} \big(p\ct (q\ct r)= (p\ct q)\ct r\big).
        \end{equation*}
        那么 $D_1(x,x,\refl{x})$ 就是
        \begin{equation*}
            \prd{z,w:A}{q:x= z}{r:z= w} \big(\refl{x}\ct(q\ct r)= (\refl{x}\ct q)\ct r\big).
        \end{equation*}
        为了构造该类型的元素, 令 $D_2:\prd{x,z:A} (x=z) \to \type$ 为类型族
        \begin{equation*}
            D_2 (x,z,q) \defeq \prd{w:A}{r:z=w} \big(\refl{x}\ct(q\ct r)= (\refl{x}\ct q)\ct r\big).
        \end{equation*}
        那么 $D_2(x,x,\refl{x})$ 就是
        \begin{equation*}
            \prd{w:A}{r:x=w} \big(\refl{x}\ct(\refl{x}\ct r)= (\refl{x}\ct \refl{x})\ct r\big).
        \end{equation*}
        为了构造\emph{这个}类型的元素, 令 $D_3:\prd{x,w:A} (x=w) \to \type$ 为类型族
        \begin{equation*}
            D_3(x,w,r) \defeq \big(\refl{x}\ct(\refl{x}\ct r)= (\refl{x}\ct \refl{x})\ct r\big).
        \end{equation*}
        那么 $D_3(x,x,\refl{x})$ 就是
        \begin{equation*}
            \big(\refl{x}\ct(\refl{x}\ct \refl{x})= (\refl{x}\ct \refl{x})\ct \refl{x}\big)
        \end{equation*}
        它定义等同于类型 $(\refl{x} = \refl{x})$, 于是 $\refl{\refl{x}}$ 所居留.
        应用三次路径归纳规则, 得到期望类型的元素.

        \mentalpause

        \noindent \emph{第二个证明:} 通过归纳, 足够认为 $p$, $q$, 和 $r$ 都是 $\refl x$.
        这个情况下, 有
        \begin{align*}
            p\ct (q\ct r)
            &\jdeq \refl{x}\ct(\refl{x}\ct \refl{x})\\
            &\jdeq \refl{x}\\
            &\jdeq (\refl{x}\ct \refl x)\ct \refl x\\
            &\jdeq (p\ct q)\ct r.
        \end{align*}
        因此, 居留于这个类型的有 $\refl{\refl{x}}$.
        \item \qedhere
    \end{enumerate}
\end{proof}

\begin{rmk}
    有定义这些高维路径的其它方法.
    例如, \cref{thm:omg}\ref{item:omg4} 中可以只在一个或两个路径上归纳路径, 而不是全部三个.
    每种情况都可以制造一个\emph{定义上}有区别的证明, 但是彼此相等同.
    任意两个证明之间的等同, 同样可以用归纳证明, 简化问题中所有路径到自反, 于是每个证明本身就是自反.
\end{rmk}

按照 \cref{thm:omg}\ref{item:omg4} 的观点, 经常用 $p\ct q\ct r$ 表示 $(p\ct q)\ct r$, 而用 $p\ct q\ct r \ct s$ 表示 $((p\ct q)\ct r)\ct s$, 等等.
选择左结合作为定义, 但实际上没有区别.
相信读者可以添加\cref{thm:omg}\ref{item:omg4} 的实例, 必要时重新关联这样的表达式.

还没有真正地完成高阶广群结构: 路径~\ref{item:omg1}--\ref{item:omg4} 还必须满足它们的高阶一致性\index{一致性}法则, 后者本身也是高维路径,
\index{结合律!路径串联的!的一致性}%
\index{globular operad}%
\index{operad}%
\index{广群!高阶}%
等等``一直到无穷大'' (这种用法更精确, 例如 globular operad 的概念).
不过, 大部分情况没有必要显式地写出整个无限维结构.
同伦类型论的一个优点就是, 所有这样的结构, 恒等类型的归纳性质即可\emph{证明}, 所以显式的程度可以根据需要来确定.

特别地, 这本书中不需要涉及复杂的组合数学来提出精确的概念, 例如 ``所有高阶等级的一致结构''.
除普通路径外, 还会使用路径的路径 (即类型 $p =_{x=_A y} q$ 的元素), 之前记作\emph{二阶路径}或者\emph{二维路径}, 和路径的路径的路径 (即元素类型的 $r = _{p =_{x=_A y} q} s$), 即\emph{三阶路径}或\emph{三维路径}.
还可以定义一个通用的\emph{$n$维路径}概念
\indexsee{n-维路径@$n$- 路径}{路径, $n$-}%
\indexsee{path!n-维@$n$-维}{路径, $n$-}%
(参见\cref{ex:npaths}), 不过没必要.

不过, 高阶路径有一种特别重要的简单情况, 起点和终点相同时.
集合论中, 命题 $a=a$ 完全没有意义, 不过同伦论中, 从一个点到它自己的路径被称为\emph{环圈}\index{环圈}, 具有很多有趣的高阶结构.
于是, 给定类型 $A$ 和点 $a:A$, 定义它的\define{环圈空间}
\index{环圈空间}%
$\Omega(A,a)$ 为类型 $\id[A]{a}{a}$.
如果点 $a$ 在语境中是明确的, 简写为 $\Omega A$.

因为任意两个 $\Omega A$ 的元素都是具有相同起点和终点的路径, 可以被串联;
因此有运算 $\Omega A\times \Omega A\to \Omega A$.
推而广之, $A$ 的高阶广群结构给予 $\Omega A$ 类似的``高阶群''结构.

$A$ 的环圈空间\emph{的}环圈空间%
\index{环圈空间!迭代的}%
\index{迭代的环圈空间}%
也是有意义的, 即 $a$ 的恒等环圈上, 二维环圈的空间.
写作 $\Omega^2(A,a)$, 并在类型论中的, 以类型 $\id[({\id[A]{a}{a}})]{\refl{a}}{\refl{a}}$ 表示.
作为环圈空间, $\Omega^2(A,a)$ 是一个``高阶群'', 还有一些额外的结构, 因为它的元素是一维环圈之间的二维的环圈.

\begin{thm}[Eckmann--Hilton]
    \label{thm:EckmannHilton}
    二阶环圈空间上的复合运算符
%
    \begin{equation*}
        \Omega^2(A)\times \Omega^2(A)\to \Omega^2(A)
    \end{equation*}
    对于任何 $\alpha, \beta:\Omega^2(A)$, 满足交换律: $\alpha\ct\beta = \beta\ct\alpha$.
    \index{Eckmann–Hilton argument}%
\end{thm}

\begin{proof}
    首先, 观察 $1$-阶环路的复合 $\Omega A\times \Omega A\to \Omega A$ 引出运算
    \[
        \star : \Omega^2(A)\times \Omega^2(A)\to \Omega^2(A)
    \]
    如下: 考虑到元素 $a, b, c : A$ 以及一阶和二阶路径, %

    \begin{align*}
        p &: a = b, &r &: b = c \\
        q &: a = b, &s &: b = c \\
        \alpha &: p = q, &\beta &: r = s
    \end{align*}
%
    如下图所示(其中箭头代表路径). % Changed this to xymatrix in the name of having uniform source code,

% maybe the original using xy looked better (I think it was too big).
% It is commented out below in case you want to reinstate it.
    \[
        \xymatrix@+5em{
                {a} \rtwocell<10>^p_q{\alpha}
            &
                {b} \rtwocell<10>^r_s{\beta}
            &
                {c}
        }
    \]
    分别地复合上方和下方的一阶路径, 得到两个路径 $p\ct r,\ q\ct s : a = c$, 然后有它们之间的``水平复合'' %

    \begin{equation*}
        \alpha\hct\beta : p\ct r = q\ct s
    \end{equation*}
%
    定义如下.
    首先, 通过路径归纳在 $r$ 上定义 $\alpha \rightwhisker r : p\ct r = q\ct r$, 于是
    \[
        \alpha \rightwhisker \refl{b} \jdeq \opp{\mathsf{ru}_p} \ct \alpha \ct \mathsf{ru}_q
    \]
    其中 $\mathsf{ru}_p : p = p \ct \refl{b}$ 是 \cref{thm:omg}\ref{item:omg1} 中的右单元法则.
    定义 $\rightwhisker$ 可以类似地在 $\alpha$ 上归纳, 或者在所有看到的路径上, 得到的结果判断不等同, 不过在 $r$ 上归纳定义更简单.
    类似地, 通过在 $q$ 上做归纳, 定义 $q\leftwhisker \beta : q\ct r = q\ct s$, 于是
    \[ \refl{b} \leftwhisker \beta \jdeq \opp{\mathsf{lu}_r} \ct \beta \ct \mathsf{lu}_s \]
    其中 $\mathsf{lu}_r$ 表示左单元法则.
    运算 $\leftwhisker$ 和 $\rightwhisker$ 被称为\define{whiskering}\indexdef{whiskering}.
    然后, 因为 $\alpha \rightwhisker r$ 和 $q\leftwhisker \beta$ 是可复合的二阶路径, 所以可以将\define{水平复合}%
    \indexdef{水平复合!路径的}%
    \indexdef{复合!路径的!水平的}%
    定义为:
    \[
        \alpha\hct\beta\ \defeq\ (\alpha\rightwhisker r) \ct (q\leftwhisker \beta).
    \]
    现在让 $a \jdeq b \jdeq c$, 于是所有一阶路径 $p$, $q$, $r$, 和 $s$ 都是  $\Omega(A,a)$ 的元素, 进一步假设 $p\jdeq q \jdeq r \jdeq s\jdeq \refl{a}$, 那么 $\alpha:\refl{a} = \refl{a}$ 和 $\beta:\refl{a} = \refl{a}$ 的两个顺序都可复合.
    这种情况, 有
    \begin{align*}
        \alpha\hct\beta
        &\jdeq (\alpha\rightwhisker\refl{a}) \ct (\refl{a}\leftwhisker \beta)\\
        &= \opp{\mathsf{ru}_{\refl{a}}} \ct \alpha \ct \mathsf{ru}_{\refl{a}} \ct \opp{\mathsf{lu}_{\refl a}} \ct \beta \ct \mathsf{lu}_{\refl{a}}\\
        &\jdeq \opp{\refl{\refl{a}}} \ct \alpha \ct \refl{\refl{a}} \ct \opp{\refl{\refl a}} \ct \beta \ct \refl{\refl{a}}\\
        &= \alpha \ct \beta.
    \end{align*}
    (回顾 $\mathsf{ru}_{\refl{a}} \jdeq \mathsf{lu}_{\refl{a}} \jdeq \refl{\refl{a}}$, 通过路径归纳的计算规则.)
    另一边, 类似地定义另一个水平复合, 通过
    \[
        \alpha\hct'\beta\ \defeq\ (p\leftwhisker \beta)\ct (\alpha\rightwhisker s)
    \]
    类似地得到
    \[
        \alpha\hct'\beta = \beta\ct\alpha.
    \]
    \index{交换法}%
    但是, 常规地, 这两种定义水平复合允许, $\alpha\hct\beta = \alpha\hct'\beta$ 在 $\alpha$ 和 $\beta$ 上归纳, 从而在剩余两个一阶路径上, 化简到自反.
    因此 有
    \[\alpha \ct \beta = \alpha\hct\beta = \alpha\hct'\beta = \beta\ct\alpha.
    \qedhere
    \]
\end{proof}

上述事实, 被称为 \emph{Eckmann–Hilton 论点}, 源于经典同伦论, 而且后面的\cref{cha:homotopy}中实际上使用它来表明高阶同伦群的类型始终为阿贝尔\index{群!阿贝尔}群.
whiskering 和水平复合操作定义, 同样也是类型的 $\infty$-阶广群结构的一部分.
它们满足自身的法则(直到高阶同伦), 比如
\[
    \alpha \rightwhisker (p\ct q) = (\alpha \rightwhisker p) \rightwhisker q
\]
等等.
从现在开始, 相信读者可以在需要时, 应用路径归纳来定义更多的运算并校验其性质.

这个例子表明了, 高阶路径类型的代数比每个等级上的广群类群结构更加复杂;
每个等级的相互作用产生了更多的运算和法则, 例如在同伦论中环圈空间迭代的研究中.
实际上, 在经典同伦论, 可以如下推广定义:

\begin{defn}
    \label{def:pointedtype}
    \define{点类型}
    \indexsee{点!类型}{类型, 点}%
    \indexdef{类型!点}%
    $(A,a)$ 是类型 $A:\type$, 和一个\define{基点} $a:A$.
    \indexdef{基点}%
    将 $\pointed{\type} \defeq \sm{A:\type} A$ 记作全集 $\type$ 中的点类型.
\end{defn}

\begin{defn}
    \label{def:loopspace}
    给定一个点类型 $(A,a)$, 可定义 $(A,a)$ 的\define{环圈空间}%
    \indexdef{环圈空间}%
    为下面的点类型:
    \[\Omega(A,a)\defeq ((\id[A]aa),\refl a).\]
    它的元素被称为 $a$ 上的\define{环圈}\indexdef{环圈}.
    对于 $n:\N$, 点类型 $(A,a)$ 的 \define{$n$-倍迭代环圈空间} $\Omega^{n}(A,a)$
    \indexdef{环圈空间!迭代}%
    \indexsee{环圈空间!n-倍@$n$-倍}{环圈空间, 迭代}%
    被递归地定义为:
    \begin{align*}
        \Omega^0(A,a)&\defeq(A,a)\\
        \Omega^{n+1}(A,a)&\defeq\Omega^n(\Omega(A,a)).
    \end{align*}
    它的元素被称为 $a$ 上的\define{$n$-环圈}
    \indexdef{环圈!n-@$n$-}%
    \indexsee{n-loop@$n$-环圈}{环圈, $n$-}%
    或者\define{$n$ 维环圈}.
    \indexsee{环圈!n 维@$n$ 维}{环圈, $n$ }%
    \indexsee{n 维环圈@$n$ 维环圈}{环圈, $n$}%

\end{defn}

会在\cref{cha:hlevels,cha:hits,cha:homotopy}回到迭代环圈空间.
\index{.infinity-groupoid@$\infty$-groupoid!结构的类型|)}%
\index{类型!恒等|)}
\index{路径|)}%