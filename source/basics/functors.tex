\index{函数|(}%
\index{类型论中的函子性函@类型论中的``函子性''}%
现在希望明确函数 $f:A\to B$ 在路径上的具备函子性.
在传统的类型论, 这等价于函数遵守等同关系.
\index{类型论中函数的连续性@``连续性''类型论中函数的连续性}%
拓扑学上, 这相当于说每个函数都是``连续的'', 即保持路径.

\begin{lem}
    \label{lem:map}
    设 $f:A\to B$ 是一个函数.
    那么对于任意 $x,y:A$ 都有一个运算
    \begin{equation*}
        \apfunc f : (\id[A] x y) \to (\id[B] {f(x)} {f(y)}).
    \end{equation*}
    此外, 对于所有 $x:A$ 有 $\apfunc{f}(\refl{x})\jdeq \refl{f(x)}$.
    \indexdef{应用!函数到路径的}%
    \indexdef{路径!函数到其的应用}%
    \indexdef{函数!应用到路径的}%
    \indexdef{行为!函数在路径的}%
\end{lem}

符号 $\apfunc f$ 可以看作是 ``the \underline{ap}plication of $f$ to a path'', 或者是 ``the \underline{a}ction on \underline{p}aths of $f$''.

\begin{proof}[第一个证明]
    令 $D:\prd{x,y:A} (x=y) \to \type$ 为类型族, 定义如下
    \[D(x,y,p)\defeq (f(x)= f(y)).\]
    然后有
    \begin{equation*}
        d\defeq\lam{x} \refl{f(x)}:\prd{x:A} D(x,x,\refl{x}).
    \end{equation*}
    通过路径归纳, 得到 $\apfunc f : \prd{x,y:A} (x=y) \to (f(x)=f(y))$.
    这个计算规则意味对于每个 $x:A$ 有 $\apfunc f({\refl{x}})\jdeq\refl{f(x)}$.
\end{proof}

\begin{proof}[第二个证明]
    对于所有 $p:x=y$, 为了定义 $\apfunc{f}(p)$, 只需通过归纳假设 $p$ 是 $\refl{x}$.
    这种情况下, 可以定义 $\apfunc f(p) \defeq \refl{f(x)}:f(x)= f(x)$.
\end{proof}

$\apfunc f (p)$ 经常被简写为 $\ap f p$.
严格来说, 这是含糊的, 不过通常不会引起困惑.
这符合范畴论中的约定, 使用与函子到对象和态射的应用相同的符号.

注意 $\apfunc{}$ 满足期望的的函子性.

\begin{lem}
    \label{lem:ap-functor}
    对于函数 $f:A\to B$, $g:B\to C$, 路径 $p:\id[A]xy$ 和 $q:\id[A]yz$, 有:
    \begin{enumerate}
        \item $\apfunc f(p\ct q) = \apfunc f(p) \ct \apfunc f(q)$.\label{item:apfunctor-ct}
        \item $\apfunc f(\opp p) = \opp{\apfunc f (p)}$.\label{item:apfunctor-opp}
        \item $\apfunc g (\apfunc f(p)) = \apfunc{g\circ f} (p)$.\label{item:apfunctor-compose}
        \item $\apfunc {\idfunc[A]} (p) = p$.
    \end{enumerate}
\end{lem}
\begin{proof}
    留给读者.
\end{proof}
\index{函数|)}%

就像 \cref{thm:omg} 中的等同一样, \cref{lem:ap-functor}是它们自己的路径, 满足自己的一致性等原则(可以用相同的方式证明).