\sectionExercises

\begin{ex}
    \label{ex:basics:concat}
    逐对展示\cref{lem:concat}的三个显然的证明相等.
\end{ex}

\begin{ex}
    \label{ex:eq-proofs-commute}
    展示上一个练习中构建的三个等式构成一个可交换三角形.
    换句话说, 如果三个串接的定义被记作 $(p \mathbin{\ct_1} q)$, $(p\mathbin{\ct_2} q)$, 和 $(p\mathbin{\ct_3} q)$, 那么串接的等同关系
    \[(p\mathbin{\ct_1} q) = (p\mathbin{\ct_2} q) = (p\mathbin{\ct_3} q)\]
    等于等同关系 $(p\mathbin{\ct_1} q) = (p\mathbin{\ct_3} q)$.
\end{ex}

\begin{ex}
    \label{ex:fourth-concat}
    给出\cref{lem:concat}的第四个不同的证明, 并证明它等同于其它的.
\end{ex}

\begin{ex}
    \label{ex:npaths}
    通过在 $n$ 上归纳, 在类型 $A$ 中, 定义一个通用的 \define{$n$-维路径}\index{路径!n-@$n$-} 的概念, 同时定义这种路径的边界类型.
\end{ex}

\begin{ex}
    \label{ex:ap-to-apd-equiv-apd-to-ap}
    证明函数~\eqref{eq:ap-to-apd} 逆等价于~\eqref{eq:apd-to-ap}.
    % and that they take $\apfunc f(p)$ to $\apdfunc f (p)$ and vice versa. (that was \cref{thm:apd-const})
\end{ex}

\begin{ex}
    \label{ex:equiv-concat}
    证明如果 $p:x=y$, 那么函数 $(p\ct \blank):(y=z) \to (x=z)$ 是等价关系.
\end{ex}

\begin{ex}
    \label{ex:ap-sigma}
    陈述并证明, 将\cref{thm:ap-prod} 从笛卡尔积推广到 $\Sigma$-类型.
\end{ex}

\begin{ex}
    \label{ex:ap-coprod}
    陈述并证明, 对于余积, \cref{thm:ap-prod} 的类似.
\end{ex}

\begin{ex}
    \label{ex:coprod-ump}
    \index{泛!性质!余积的}%
    证明余积有期望的泛性质,
    \[ \eqv{(A+B \to X)}{(A\to X)\times (B\to X)}. \]
    你能推广它到涉及依值函数的一个等价关系吗?
\end{ex}

\begin{ex}
    \label{ex:sigma-assoc}
    证明 $\Sigma$-类型符合 ``结合律'',
    \index{结合律!Sigma-类型的@$\Sigma$-类型的}%
    也就是对于任何 $A:\UU$ 和族 $B:A\to\UU$ 以及 $C:(\sm{x:A} B(x))\to\UU$, 有
    \[\eqvspaced{\Parens{\sm{x:A}{y:B(x)} C(\pairr{x,y})}}{\Parens{\sm{p:\sm{x:A}B(x)} C(p)}}. \]
\end{ex}

\begin{ex}
    \label{ex:pullback}
    一个 (同伦的) \define{交换方形}
    \indexdef{交换!方形}%
    \begin{equation*}
        \vcenter{\xymatrix{
            P\ar[r]^h\ar[d]_k &
            A\ar[d]^f\\
            B\ar[r]_g &
            C
        }}
    \end{equation*}
    存在函数 $f$, $g$, $h$, 以及 $k$ 如图所示, 其中路径 $f \circ h= g \circ k$.
    注意这恰好是是定义在~\eqref{eq:defn-pullback}的拉回 $(P\to A) \times_{P\to C} (P\to B)$ 的一个元素.
    交换方形被称为 (同伦) \define{拉回方形}
    \indexdef{拉回}%
    如果对于任何 $X$, 相关映射
    \[ (X\to P) \to (X\to A) \times_{(X\to C)} (X\to B) \]
    是等价关系.
    证明定义在~\eqref{eq:defn-pullback}中的拉回 $P \defeq A\times_C B$  是拉回方形的一个角.
\end{ex}

\begin{ex}
    \label{ex:pullback-pasting}
    假设给定两个交换方形
    \begin{equation*}
        \vcenter{\xymatrix{
            A\ar[r]\ar[d] &
            C\ar[r]\ar[d] &
            E\ar[d]\\
            B\ar[r] &
            D\ar[r] &
            F
        }}
    \end{equation*}
    并假设右边的方形是拉回方形.
    证明左边的方形是拉回方形, 当且仅当外边的长方形是一个拉回方形.
\end{ex}

\begin{ex}
    \label{ex:eqvboolbool}
    展示 $\eqv{(\eqv\bool\bool)}{\bool}$.
\end{ex}

\begin{ex}
    \label{ex:equality-reflection}
    假设添加 \emph{等同关系自反规则} 到类型论, 也就是说如果有一元素 $p:x=y$, 那么 $x\jdeq y$.
    证明对于任何 $p:x=x$ 有 $p\jdeq \refl{x}$.
    (这暗示了所有类型, 在 \cref{sec:basics-sets} 中介绍的意义中, 都是 \emph{集合}; 参见 \cref{sec:hedberg}.)
\end{ex}

\begin{ex}
    \label{ex:strengthen-transport-is-ap}
    展示 \cref{thm:transport-is-ap} 可以增强为
    \[\transfib{B}{p}{{\blank}} =_{B(x)\to B(y)} \idtoeqv(\apfunc{B}(p))\]
    不需要使用函数外延性.
    (为了可读性和一致性, 在它和类似的情况之间, 选择了看起来弱的表述方式.)
\end{ex}

\begin{ex}
    \label{ex:strong-from-weak-funext}
    假设没有函数外延性 (\cref{axiom:funext}), 而是只假设存在一个元素
    \[ \funext : \prd{A:\UU}{B:A\to \UU}{f,g:\prd{x:A} B(x)} (f\htpy g) \to (f=g) \]
    (假定和 $\happly$ 没有关系).
    证明实际上, 这意味整个函数外延性公理 (即 $\happly$ 是一个等价关系).
    这源自 Voevodsky; 它的证明是复杂的而且需要接下来的章中的概念.
\end{ex}

\begin{ex}
    \label{ex:equiv-functor-types}\
    \begin{enumerate}
        \item 展示如果 $\eqv{A}{A'}$  而且 $\eqv{B}{B'}$, 那么 $\eqv{(A\times B)}{(A'\times B')}$.
        \item 给定两个这个事实的证明, 一个使用一价性而另一个不使用, 然后展示这两个证明相等.
        \item 用公式表示并证明其他类型形成器的类似结果: $\Sigma$, $\to$, $\Pi$, 以及 $+$.
    \end{enumerate}
\end{ex}

\begin{ex}
    \label{ex:dep-htpy-natural}
    陈述并证明依值函数版本的 \cref{lem:htpy-natural}.
\end{ex}