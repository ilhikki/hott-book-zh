We now consider one example to illustrate the interaction between the groupoid structure on a type and the type
formers.  In the introduction we remarked that one of the
advantages of univalence is that two isomorphic things are interchangeable,
in the sense that every property or construction involving one also
applies to the other.  Common ``abuses of notation''\index{abuse!of notation} become formally
true.  Univalence itself says that equivalent types are equal, and
therefore interchangeable, which includes e.g.\  the common practice of identifying isomorphic sets.  Moreover, when we define other
mathematical objects as sets, or even general types, equipped with structure or properties, we
can derive the correct notion of equality for them from univalence.  We will illustrate this
point with a significant example in \cref{cha:category-theory}, where we
define the basic notions of category theory in such a way that equality
of categories is equivalence, equality of functors is natural
isomorphism, etc. See in particular \cref{sec:sip}.
 In this section, we describe a very simple example, coming from algebra.

For simplicity, we use \emph{semigroups} as our example, where a
semigroup is a type equipped with an associative ``multiplication''
operation.  The same ideas apply to other algebraic structures, such as
monoids, groups, and rings.
Recall from \cref{sec:sigma-types,sec:pat} that the definition of a kind of mathematical structure should be interpreted as defining the type of such structures as a certain iterated $\Sigma$-type.
In the case of semigroups this yields the following.

\begin{defn}
Given a type $A$, the type \semigroupstr{A} of \define{semigroup structures}
\indexdef{semigroup!structure}%
\index{structure!semigroup}%
\index{associativity!of semigroup operation}%
with carrier\index{carrier} $A$ is defined by
\[
\semigroupstr{A} \defeq \sm{m:A \to A \to A} \prd{x,y,z:A} m(x,m(y,z)) = m(m(x,y),z).
\]
%
A \define{semigroup}
\indexdef{semigroup}%
is a type together with such a structure:
%
\[
\semigroup \defeq \sm{A:\type} \semigroupstr A
\]
\end{defn}

\noindent
In the next two sections, we describe two ways in which univalence makes
it easier to work with such semigroups.

\subsection{Lifting equivalences}

\index{lifting!equivalences}%
When working loosely, one might say that a bijection between sets $A$
and $B$ ``obviously'' induces an isomorphism between semigroup
structures on $A$ and semigroup structures on $B$.  With univalence,
this is indeed obvious, because given an equivalence between types $A$
and $B$, we can automatically derive a semigroup structure on $B$ from
one on $A$, and moreover show that this derivation is an equivalence of
semigroup structures.  The reason is that \semigroupstrsym\ is a family
of types, and therefore has an action on paths between types given by
$\mathsf{transport}$:
\[
\transfibf{\semigroupstrsym}{(\ua(e))} : \semigroupstr{A} \to \semigroupstr{B}.
\]
Moreover, this map is an equivalence, because
$\transfibf{C}(\alpha)$ is always an equivalence with inverse
$\transfibf{C}{(\opp \alpha)}$, see \cref{thm:transport-concat,thm:omg}.

While the univalence axiom\index{univalence axiom} ensures that this map exists, we need to use
facts about $\mathsf{transport}$ proven in the preceding sections to
calculate what it actually does. Let $(m,a)$ be a semigroup structure on
$A$, and we investigate the induced semigroup structure on $B$ given by
\[
\transfib{\semigroupstrsym}{\ua(e)}{(m,a)}.
\]
First, because
\semigroupstr{X} is defined to be a $\Sigma$-type, by
\cref{transport-Sigma},
\[
\transfib{\semigroupstrsym}{\ua(e)}{(m,a)} = (m',a')
\]
where $m'$ is an induced multiplication operation on $B$
\begin{flalign*}
& m' : B \to B \to B \\
& m'(b_1,b_2) \defeq \transfib{X \mapsto (X \to X \to X)}{\ua(e)}{m}(b_1,b_2)
\end{flalign*}
and $a'$ an induced proof that $m'$ is associative.
We have, again by \cref{transport-Sigma},
\begin{equation}\label{eq:transport-semigroup-step1}
  \begin{aligned}
{}&a' :     \mathsf{Assoc}(B,m')\\
  &a' \defeq \transfib{(X,m) \mapsto \mathsf{Assoc}(X,m)}{(\pairpath(\ua(e),\refl{m'}))}{a},
  \end{aligned}
\end{equation}
where $\mathsf{Assoc}(X,m)$ is the type $\prd{x,y,z:X} m(x,m(y,z)) = m(m(x,y),z)$.
By function extensionality, it suffices to investigate the behavior of $m'$ when
applied to arguments $b_1,b_2 : B$. By applying
\eqref{eq:transport-arrow} twice, we have that $m'(b_1,b_2)$ is equal to
%
\begin{narrowmultline*}
  \transfibf{X \mapsto X}\big(
      \ua(e), \narrowbreak
      m(\transfib{X \mapsto X}{\opp{\ua(e)}}{b_1},
        \transfib{X \mapsto X}{\opp{\ua(e)}}{b_2}
       )
   \big).
\end{narrowmultline*}
%
Then, because $\ua$ is quasi-inverse to $\transfibf{X\mapsto X}$, this is equal to
\[
e(m(\opp{e}(b_1), \opp{e}(b_2))).
\]
Thus, given two elements of $B$, the induced multiplication $m'$
sends them to $A$ using the equivalence $e$, multiplies them in $A$, and
then brings the result back to $B$ by $e$, just as one would expect.

Moreover, though we do not show the proof, one can calculate that the
induced proof that $m'$ is associative (see \eqref{eq:transport-semigroup-step1})
is equal to a function sending
$b_1,b_2,b_3 : B$ to a path given by the following steps:
\begin{equation}
  \label{eq:transport-semigroup-assoc}
  \begin{aligned}
    m'(m'(b_1,b_2),b_3)
    &= e(m(\opp{e}(m'(b_1,b_2)),\opp{e}(b_3))) \\
    &= e(m(\opp{e}(e(m(\opp{e}(b_1),\opp{e}(b_2)))),\opp{e}(b_3))) \\
    &= e(m(m(\opp{e}(b_1),\opp{e}(b_2)),\opp{e}(b_3))) \\
    &= e(m(\opp{e}(b_1),m(\opp{e}(b_2),\opp{e}(b_3)))) \\
    &= e(m(\opp{e}(b_1),\opp{e}(e(m(\opp{e}(b_2),\opp{e}(b_3)))))) \\
    &= e(m(\opp{e}(b_1),\opp{e}(m'(b_2,b_3)))) \\
    &= m'(b_1,m'(b_2,b_3)).
\end{aligned}
\end{equation}
These steps use the proof $a$ that $m$ is associative and the inverse
laws for~$e$.  From an algebra perspective, it may seem strange to
investigate the identity of a proof that an operation is associative,
but this makes sense if we think of $A$ and $B$ as general spaces, with
non-trivial homotopies between paths.  In \cref{cha:logic}, we will
introduce the notion of a \emph{set}, which is a type with only trivial
homotopies, and if we consider semigroup structures on sets, then any
two such associativity proofs are automatically equal.

\subsection{Equality of semigroups}
\label{sec:equality-semigroups}

Using the equations for path spaces discussed in the previous sections,
we can investigate when two semigroups are equal. Given semigroups
$(A,m,a)$ and $(B,m',a')$, by \cref{thm:path-sigma}, the type of paths
\narrowequation{
  (A,m,a) =_\semigroup (B,m',a')
}
is equal to the type of pairs
\begin{align*}
p_1 &: A =_{\type} B \qquad\text{and}\\
p_2 &: \transfib{\semigroupstrsym}{p_1}{(m,a)} = {(m',a')}.
\end{align*}
By univalence, $p_1$ is $\ua(e)$ for some equivalence $e$. By
\cref{thm:path-sigma}, function extensionality, and the above analysis of
transport in the type family $\semigroupstrsym$, $p_2$ is equivalent to a pair
of proofs, the first of which shows that
\begin{equation*} \label{eq:equality-semigroup-mult}
\prd{y_1,y_2:B} e(m(\opp{e}(y_1), \opp{e}(y_2))) = m'(y_1,y_2)
\end{equation*}
and the second of which shows that $a'$ is equal to the induced
associativity proof constructed from $a$ in
\eqref{eq:transport-semigroup-assoc}.  But by cancellation of inverses
\eqref{eq:equality-semigroup-mult} is equivalent to
\[
\prd{x_1,x_2:A} e(m(x_1, x_2)) = m'(e(x_1),e(x_2)).
\]
This says that $e$ commutes with the binary operation, in the sense
that it takes multiplication in $A$ (i.e.\ $m$) to multiplication in $B$
(i.e.\ $m'$).  A similar rearrangement is possible for the equation relating
$a$ and $a'$.  Thus, an equality of semigroups consists exactly of an
equivalence on the carrier types that commutes with the semigroup
structure.

For general types, the proof of associativity is thought of as part of
the structure of a semigroup.  However, if we restrict to set-like types
(again, see \cref{cha:logic}), the
equation relating $a$ and $a'$ is trivially true.  Moreover, in this
case, an equivalence between sets is exactly a bijection.  Thus, we have
arrived at a standard definition of a \emph{semigroup isomorphism}:\index{isomorphism!semigroup} a
bijection on the carrier sets that preserves the multiplication
operation.  It is also possible to use the category-theoretic definition
of isomorphism, by defining a \emph{semigroup homomorphism}\index{homomorphism!semigroup} to be a map
that preserves the multiplication, and arrive at the conclusion that equality of
semigroups is the same as two mutually inverse homomorphisms; but we
will not show the details here; see \cref{sec:sip}.

The conclusion is that, thanks to univalence, semigroups are equal
precisely when they are isomorphic as algebraic structures. As we will see in \cref{sec:sip}, the
conclusion applies more generally: in homotopy type theory, all constructions of
mathematical structures automatically respect isomorphisms, without any
tedious proofs or abuse of notation.