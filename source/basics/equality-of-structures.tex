现在考虑一个例子, 来介绍类型上的广群结构和类型形成器的相互关系.
在导论中评论过, 单值的优势之一是, 两个同构的事物是可交换的, 从某种意义上来说, 涉及到其中之一的性质或者构造, 都适用于另一个.
通用的 ``符号滥用''\index{滥用!符号的}形式上为真.
单值本身说, 等价的类型是相等的, 因此也是可交换的, 例如恒等同构集合的常规实践.
进一步说, 其它数学对象例如集合, 乃至具有一定结构或者性质的一般类型, 从单值可以准取推导它们的等同符号.
\cref{cha:category-theory}中有一个重要的例子来介绍它, 定义范畴论的基础概念时, 用到了范畴的等同关系是等价关系, 函子的等同关系是自然同构, 等等.
细节参见 \cref{sec:sip}.
本节会描述一个源自代数的非常简单的例子.

为简单起见, 使用\emph{半群}作为例子, 半群是一个类型, 以及满足结合律的 ``乘法'' 运算.
同样的思路适用于其他代数结构, 比如幺半群, 群, 和环.
回顾\cref{sec:sigma-types,sec:pat}, 一种数学结构的定义, 应该被理解为该结构作为某个迭代的 $\Sigma$-类型.
半群的结果如下.

\begin{defn}
    给定类型 $A$, 以 $A$ 为载子\index{载子}的\define{半群结构}的类型 \semigroupstr{A}
    \indexdef{半群!结构}%
    \index{结构!半群}%
    \index{结合律!半群的运算}%
    被定义为
    \[
        \semigroupstr{A} \defeq \sm{m:A \to A \to A} \prd{x,y,z:A} m(x,m(y,z)) = m(m(x,y),z).
    \]
    \define{半群}
    \indexdef{半群}%
    是一个类型, 具有这样的结构:
    \[
        \semigroup \defeq \sm{A:\type} \semigroupstr A
    \]
\end{defn}

\noindent
在接下来两小节中, 会介绍单值让处理此类半群更简单的两种方法.

\subsection{等价关系提升}

\index{提升!等价关系}%
不精准地说, 集合 $A$ 与 $B$ 之间的双射 ``显然'' 蕴含了 $A$ 的半群结构同构于 $B$ 的半群结构.
根据单值, 确实明显, 因为给定类型 $A$ 和 $B$ 之间的等价, 可以从 $A$ 上的半群结构, 自动推导出 $B$ 上的半群结构, 而且这个结果是半群结构的等价关系.
因为 \semigroupstrsym\ 是类型族, 可以 $\mathsf{transport}$ 类型之间的路径:
\[
    \transfibf{\semigroupstrsym}{(\ua(e))} : \semigroupstr{A} \to \semigroupstr{B}.
\]
并且, 这个映射是一个等价, 因为 $\transfibf{C}(\alpha)$ 与逆 $\transfibf{C}{(\opp \alpha)}$ 总是等价的, 参见 \cref{thm:transport-concat,thm:omg}.

尽管\index{单值公理}确保了这个映射存在, 还是要使用之前的节中的中证明关于 $\mathsf{transport}$ 的事实, 来计算它实际做了什么.
令 $(m,a)$ 为 $A$ 上的半群结构, 并研究引导的 $B$ 上的半群结构
\[
    \transfib{\semigroupstrsym}{\ua(e)}{(m,a)}.
\]
首先, 因为 \semigroupstr{X} 被定义为 $\Sigma$-类型, 通过\cref{transport-Sigma},
\[
    \transfib{\semigroupstrsym}{\ua(e)}{(m,a)} = (m',a')
\]
其中 $m'$ 是 $B$ 上的乘法运算,
\begin{flalign*}
    & m' : B \to B \to B \\
    & m'(b_1,b_2) \defeq \transfib{X \mapsto (X \to X \to X)}{\ua(e)}{m}(b_1,b_2)
\end{flalign*}
而 $a'$ 是 $m'$ 符合结合律的证明.
再次通过 \cref{transport-Sigma},
\begin{equation}
    \label{eq:transport-semigroup-step1}
    \begin{aligned}
    {}
        &a' :     \mathsf{Assoc}(B,m')\\
        &a' \defeq \transfib{(X,m) \mapsto \mathsf{Assoc}(X,m)}{(\pairpath(\ua(e),\refl{m'}))}{a},
    \end{aligned}
\end{equation}
其中 $\mathsf{Assoc}(X,m)$ 是类型 $\prd{x,y,z:X} m(x,m(y,z)) = m(m(x,y),z)$.
根据函数外延性, 可以研究应用参数 $b_1,b_2 : B$ 时, $m'$ 的行为.
通过应用两次 \eqref{eq:transport-arrow}, 有 $m'(b_1,b_2)$ 等于
%
\begin{narrowmultline*}
    \transfibf{X \mapsto X}\big(
    \ua(e), \narrowbreak
    m(\transfib{X \mapsto X}{\opp{\ua(e)}}{b_1},
    \transfib{X \mapsto X}{\opp{\ua(e)}}{b_2}
    )
    \big).
\end{narrowmultline*}
%
然后, 根据 $\ua$ 准逆于 $\transfibf{X\mapsto X}$, 它等于
\[
    e(m(\opp{e}(b_1), \opp{e}(b_2))).
\]
因此, 给定两个 $B$ 的元素, 对应的乘法 $m'$ 使用等价关系 $e$, 使它们变成 $A$, 在 $A$ 中相乘, 并通过 $e$ 将结果变成 $B$, 和期待的一样.

此外, 尽管没有展示 $m'$ 符合结合律
(参见 \eqref{eq:transport-semigroup-step1})
的证明, 但它等于一个函数, 这个函数将 $b_1,b_2,b_3 : B$ 变成一个路径, 按照这些步骤:
\begin{equation}
    \label{eq:transport-semigroup-assoc}
    \begin{aligned}
        m'(m'(b_1,b_2),b_3)
        &= e(m(\opp{e}(m'(b_1,b_2)),\opp{e}(b_3))) \\
        &= e(m(\opp{e}(e(m(\opp{e}(b_1),\opp{e}(b_2)))),\opp{e}(b_3))) \\
        &= e(m(m(\opp{e}(b_1),\opp{e}(b_2)),\opp{e}(b_3))) \\
        &= e(m(\opp{e}(b_1),m(\opp{e}(b_2),\opp{e}(b_3)))) \\
        &= e(m(\opp{e}(b_1),\opp{e}(e(m(\opp{e}(b_2),\opp{e}(b_3)))))) \\
        &= e(m(\opp{e}(b_1),\opp{e}(m'(b_2,b_3)))) \\
        &= m'(b_1,m'(b_2,b_3)).
    \end{aligned}
\end{equation}
在这些步骤中, $a$ 是 $m$ 符合结合律的证明, 并使用了 $e$ 的逆法则.
从代数的角度, 对于运算交换律的证明, 研究它们的恒等看起来很奇怪,
不过这样看是有意义的, 如果将 $A$ 和 $B$ 看作是常规的空间, 路径之间有非平凡的同伦.
在 \cref{cha:logic}, 会介绍\emph{集合}的概念, 它是同伦中的一个平凡的类型,
而如果考虑的是集合的半群结构, 那么任何结合律的证明, 都相等.

\subsection{半群的等同关系}
\label{sec:equality-semigroups}

使用上小节中, 路径空间的等式, 可以研究两个半群何时相等.
给定半群 $(A,m,a)$ 和 $(B,m',a')$, 根据 \cref{thm:path-sigma}, 路径
\narrowequation{
    (A,m,a) =_\semigroup (B,m',a')
}
的类型等于序对
\begin{align*}
    p_1 &: A =_{\type} B \qquad\text{和}\\
    p_2 &: \transfib{\semigroupstrsym}{p_1}{(m,a)} = {(m',a')}
\end{align*}的类型.
根据单值, 对于等价关系 $e$, $p_1$ 就是 $\ua(e)$.
根据 \cref{thm:path-sigma}, 函数外延性, 以及之前关于类型族 $\semigroupstrsym$ 中的传输的分析, $p_2$ 等价于证明一个的序对, 序对的第一个部分是
\begin{equation}
    \label{eq:equality-semigroup-mult}
    \prd{y_1,y_2:B} e(m(\opp{e}(y_1), \opp{e}(y_2))) = m'(y_1,y_2)
\end{equation}
而第二个部分就是, $a'$ 等于 \eqref{eq:transport-semigroup-assoc} 中从 $a$ 构造出的结合律的证明.
但是通过取消逆运算, \eqref{eq:equality-semigroup-mult}等于
\[
    \prd{x_1,x_2:A} e(m(x_1, x_2)) = m'(e(x_1),e(x_2)).
\]
也就是说 $e$ 将 $A$ 中的乘法(即 $m$) 变成 $B$ 中的乘法(即 $m'$), 在这个意义上, 它与这个二元运算是易对的.
关于 $a$ 和 $a'$ 的方程也可以类似地重新排列.
因此, 半群的等式完全由, 与半群结构互换的载体的等价关系组成.

对于通用类型, 结合率的证明被认为是半群的结构的一部分.
然而, 如果限制为类似于集合的类型 (再次参见 \cref{cha:logic}), 关于 $a$ 和 $a'$ 的等式没有价值.
此外, 在这个例子中, 集合之间的等价关系正好是一个双射.
因此, 得到\emph{半群同构}:\index{同构!半群}的标准定义, 保持乘法运算的载体集上的双射.
也可以使用同构的范畴论定义, 通过定义\emph{半群同态}\index{同态!半群}定义为保留乘法的运算, 并得出结论, 半群的等同关系就是两个互逆的同态;
不过这里不展示细节;
参见 \cref{sec:sip}.

感谢单值, 结论就是, 当半群作为代数结构而同构时, 它们正好相等.
在\cref{sec:sip} 中, 结论更通用: 同伦类型论中, 数学结构的所有构造天然的遵循同构关系, 不需要任何冗长的证明或者滥用符号.