现在考虑一个例子, 来介绍类型上的广群结构和类型形成器的相互关系.
在导论中评论过, 宇价的优势之一是, 两个同构的事物是可交换的, 从某种意义上来说, 涉及到其中之一的性质或者构造, 都适用于另一个.
通用的 ``符号滥用''\index{滥用!符号的}形式上为真.
宇价本身说, 等价的类型是相等的, 因此也是可交换的, 例如恒等同构集合的常规实践.
进一步说, 其它数学对象例如集合, 乃至具有一定结构或者性质的一般类型, 从宇价可以准取推导它们的等同符号.
\cref{cha:category-theory}中有一个重要的例子来介绍它, 定义了范畴论的基础概念, 其中用到了范畴的等同关系是等价关系, 函子的等同关系是自然同构, 等等.
细节参见 \cref{sec:sip}.
本节会描述一个源自代数的非常简单的例子.

为简单起见, 使用\emph{半群}作为例子, 半群是一个类型, 与一个满足结合律的 ``乘法'' 运算.
同样的思路适用于其他代数结构, 比如幺半群, 群, 和环.
回顾\cref{sec:sigma-types,sec:pat}, 一种数学结构的定义, 应该被理解为该结构作为某个迭代的 $\Sigma$-类型.
半群的结果如下.

\begin{defn}
    给定类型 $A$, 以 $A$ 为载子\index{载子}的\define{半群结构}的类型 \semigroupstr{A}
    \indexdef{半群!结构}%
    \index{结构!半群}%
    \index{结合律!半群的运算}%
    被定义为
    \[
        \semigroupstr{A} \defeq \sm{m:A \to A \to A} \prd{x,y,z:A} m(x,m(y,z)) = m(m(x,y),z).
    \]
    \define{半群}
    \indexdef{半群}%
    是一个类型, 具有这样的结构:
    \[
        \semigroup \defeq \sm{A:\type} \semigroupstr A
    \]
\end{defn}

\noindent
在接下来两小节中, 会介绍宇价让处理此类半群更简单的两种方法.

\subsection{等价关系提升}

\index{提升!等价关系}%
松散地说, 集合 $A$ 与 $B$ 之间的双射 ``显然'' 引导了一个 $A$ 的半群结构与 $B$ 的半群结构的之间的同构.
根据宇价, 确实明显, 因为给定类型 $A$ 和 $B$ 之间的等价, 可以从 $A$ 上的半群结构, 自动推导出 $B$ 上的半群结构, 而且这个推导过程是半群结构的等价关系.
因为 \semigroupstrsym\ 是类型族, 并可以 $\mathsf{transport}$ 类型之间的路径:
\[
    \transfibf{\semigroupstrsym}{(\ua(e))} : \semigroupstr{A} \to \semigroupstr{B}.
\]
并且, 这个映射是一个等价关系, 因为 $\transfibf{C}(\alpha)$ 与逆 $\transfibf{C}{(\opp \alpha)}$ 总是等价的, 参见 \cref{thm:transport-concat,thm:omg}.

尽管\index{宇价公理}确保了这个映射存在, 还是要使用之前的节中的中证明关于 $\mathsf{transport}$ 的事实, 来计算它实际做了什么.
令 $(m,a)$ 为 $A$ 上的半群结构, 并研究引导的 $B$ 上的半群结构
\[
    \transfib{\semigroupstrsym}{\ua(e)}{(m,a)}.
\]
首先, 因为 \semigroupstr{X} 被定义为 $\Sigma$-类型, 通过\cref{transport-Sigma},
\[
    \transfib{\semigroupstrsym}{\ua(e)}{(m,a)} = (m',a')
\]
其中 $m'$ 是 $B$ 上的乘法运算,
\begin{flalign*}
    & m' : B \to B \to B \\
    & m'(b_1,b_2) \defeq \transfib{X \mapsto (X \to X \to X)}{\ua(e)}{m}(b_1,b_2)
\end{flalign*}
而 $a'$ 是 $m'$ 符合结合律的证明.
再次通过 \cref{transport-Sigma},
\begin{equation}
    \label{eq:transport-semigroup-step1}
    \begin{aligned}
    {}
        &a' :     \mathsf{Assoc}(B,m')\\
        &a' \defeq \transfib{(X,m) \mapsto \mathsf{Assoc}(X,m)}{(\pairpath(\ua(e),\refl{m'}))}{a},
    \end{aligned}
\end{equation}
其中 $\mathsf{Assoc}(X,m)$ 是类型 $\prd{x,y,z:X} m(x,m(y,z)) = m(m(x,y),z)$.
根据函数外延性, 足以研究应用参数 $b_1,b_2 : B$ 时, $m'$ 的行为.
通过应用两次 \eqref{eq:transport-arrow}, 有 $m'(b_1,b_2)$ 等于
%
\begin{narrowmultline*}
    \transfibf{X \mapsto X}\big(
    \ua(e), \narrowbreak
    m(\transfib{X \mapsto X}{\opp{\ua(e)}}{b_1},
    \transfib{X \mapsto X}{\opp{\ua(e)}}{b_2}
    )
    \big).
\end{narrowmultline*}
%
然后, 根据 $\ua$ quasi-inverse 于 $\transfibf{X\mapsto X}$, 它等于
\[
    e(m(\opp{e}(b_1), \opp{e}(b_2))).
\]
因此, 给定两个 $B$ 的元素, 对应的乘法 $m'$ 使用等价关系 $e$, 使它们变成 $A$, 在 $A$ 中想乘, 并通过 $e$ 将结果变成 $B$, 和期待的一样.

此外, 尽管没有展示, 但 $m'$ 符合结合律
(参见 \eqref{eq:transport-semigroup-step1})
的证明等于一个函数, 这个函数将 $b_1,b_2,b_3 : B$ 变成一个路径, 并由下面的步骤所给定:
\begin{equation}
    \label{eq:transport-semigroup-assoc}
    \begin{aligned}
        m'(m'(b_1,b_2),b_3)
        &= e(m(\opp{e}(m'(b_1,b_2)),\opp{e}(b_3))) \\
        &= e(m(\opp{e}(e(m(\opp{e}(b_1),\opp{e}(b_2)))),\opp{e}(b_3))) \\
        &= e(m(m(\opp{e}(b_1),\opp{e}(b_2)),\opp{e}(b_3))) \\
        &= e(m(\opp{e}(b_1),m(\opp{e}(b_2),\opp{e}(b_3)))) \\
        &= e(m(\opp{e}(b_1),\opp{e}(e(m(\opp{e}(b_2),\opp{e}(b_3)))))) \\
        &= e(m(\opp{e}(b_1),\opp{e}(m'(b_2,b_3)))) \\
        &= m'(b_1,m'(b_2,b_3)).
    \end{aligned}
\end{equation}
在这些步骤中, $a$ 是 $m$ 符合结合律的证明, 并使用了 $e$ 的逆法则.
从代数的角度, 运算符合交换律的证明, 研究它的恒等看起来很奇怪,
不过这样看是有意义的, $A$ 和 $B$ 是常规的空间, 和路径之间有意义的同伦关系.
在 \cref{cha:logic}, 会介绍\emph{集合}的概念, 它是同伦关系不重要的类型,
而如果考虑集合的半群结构, 那么任何结合律的证明, 都相等.

\subsection{Equality of semigroups}
\label{sec:equality-semigroups}

Using the equations for path spaces discussed in the previous sections,
we can investigate when two semigroups are equal. Given semigroups
$(A,m,a)$ and $(B,m',a')$, by \cref{thm:path-sigma}, the type of paths
\narrowequation{
  (A,m,a) =_\semigroup (B,m',a')
}
is equal to the type of pairs
\begin{align*}
p_1 &: A =_{\type} B \qquad\text{and}\\
p_2 &: \transfib{\semigroupstrsym}{p_1}{(m,a)} = {(m',a')}.
\end{align*}
By univalence, $p_1$ is $\ua(e)$ for some equivalence $e$. By
\cref{thm:path-sigma}, function extensionality, and the above analysis of
transport in the type family $\semigroupstrsym$, $p_2$ is equivalent to a pair
of proofs, the first of which shows that
\begin{equation*} \label{eq:equality-semigroup-mult}
\prd{y_1,y_2:B} e(m(\opp{e}(y_1), \opp{e}(y_2))) = m'(y_1,y_2)
\end{equation*}
and the second of which shows that $a'$ is equal to the induced
associativity proof constructed from $a$ in
\eqref{eq:transport-semigroup-assoc}.  But by cancellation of inverses
\eqref{eq:equality-semigroup-mult} is equivalent to
\[
\prd{x_1,x_2:A} e(m(x_1, x_2)) = m'(e(x_1),e(x_2)).
\]
This says that $e$ commutes with the binary operation, in the sense
that it takes multiplication in $A$ (i.e.\ $m$) to multiplication in $B$
(i.e.\ $m'$).  A similar rearrangement is possible for the equation relating
$a$ and $a'$.  Thus, an equality of semigroups consists exactly of an
equivalence on the carrier types that commutes with the semigroup
structure.

For general types, the proof of associativity is thought of as part of
the structure of a semigroup.  However, if we restrict to set-like types
(again, see \cref{cha:logic}), the
equation relating $a$ and $a'$ is trivially true.  Moreover, in this
case, an equivalence between sets is exactly a bijection.  Thus, we have
arrived at a standard definition of a \emph{semigroup isomorphism}:\index{isomorphism!semigroup} a
bijection on the carrier sets that preserves the multiplication
operation.  It is also possible to use the category-theoretic definition
of isomorphism, by defining a \emph{semigroup homomorphism}\index{homomorphism!semigroup} to be a map
that preserves the multiplication, and arrive at the conclusion that equality of
semigroups is the same as two mutually inverse homomorphisms; but we
will not show the details here; see \cref{sec:sip}.

The conclusion is that, thanks to univalence, semigroups are equal
precisely when they are isomorphic as algebraic structures. As we will see in \cref{sec:sip}, the
conclusion applies more generally: in homotopy type theory, all constructions of
mathematical structures automatically respect isomorphisms, without any
tedious proofs or abuse of notation.