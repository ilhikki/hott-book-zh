\index{类型!恒等|(}%
就像类型 \id[A]{a}{a'} 表征为同构, 其中每个 $A$ 都有单独的 ``定义'', 路径 $p,q : \id[A]{a}{a'}$ 之间的路径 \id[{\id[A]{a}{a'}}]{p}{q} 却不能轻易定性.
然而, 通用版本的定理, 扩展了恒等类型, 例如它们遵循等价关系的事实.

\begin{thm}
    \label{thm:paths-respects-equiv}
    如果 $f : A \to B$ 是一个等价关系, 那么对于所有 $a,a':A$,
    \[\apfunc{f} : (\id[A]{a}{a'}) \to (\id[B]{f(a)}{f(a')})\]
    也是一个等价关系.
\end{thm}
\begin{proof}
    令  $f$ 是 $\opp f$ 的准逆, 以及同伦关系
    %
    \begin{equation*}
        \alpha:\prd{b:B} (f(\opp f(b))=b)
        \qquad\text{和}\qquad
        \beta:\prd{a:A} (\opp f(f(a)) = a).
    \end{equation*}
    %
    $\apfunc{f}$ 的准逆本质就是
    \[
        \apfunc{\opp f} : (\id{f(a)}{f(a')}) \to (\id{\opp f(f(a))}{\opp f(f(a'))}).
    \]
    而为了从 $\apfunc{\opp f}(q)$ 得到 $\id[A]{a}{a'}$ 的元素, 必须在两边连接路径 $\opp{\beta_a}$ 和 $\beta_{a'}$.
    为了展示它, 给出 $\apfunc{f}$ 的准逆, 一方面必须展示对于所有 $p:\id[A]{a}{a'}$ 有
    \[
        \opp{\beta_a} \ct \apfunc{\opp f}(\apfunc{f}(p)) \ct \beta_{a'} = p.
    \]
    从 $\apfunc{}$ 的函子性和同伦关系的自然性能得到它, \cref{lem:ap-functor,lem:htpy-natural}.
    另一方面, 必须展示对于所有 $q:\id[B]{f(a)}{f(a')}$ 有
    \[
        \apfunc{f}\big( \opp{\beta_a} \ct \apfunc{\opp f}(q) \ct \beta_{a'} \big) = q.
    \]
    它的证明有一些复杂, 不过每一步都还是 \cref{lem:ap-functor,lem:htpy-natural} 的应用(或者简单逆路径消除):
    \begin{align*}
        \apfunc{f}\big( \narrowamp\opp{\beta_a} \ct \apfunc{\opp f}(q) \ct \beta_{a'} \big) \narrowbreak
        &= \opp{\alpha_{f(a)}} \ct {\alpha_{f(a)}} \ct
        \apfunc{f}\big( \opp{\beta_a} \ct \apfunc{\opp f}(q) \ct \beta_{a'} \big)
        \ct \opp{\alpha_{f(a')}} \ct {\alpha_{f(a')}}\\
        &= \opp{\alpha_{f(a)}} \ct
        \apfunc f \big(\apfunc{\opp f}\big(\apfunc{f}\big( \opp{\beta_a} \ct \apfunc{\opp f}(q) \ct \beta_{a'} \big)\big)\big)
        \ct {\alpha_{f(a')}}\\
        &= \opp{\alpha_{f(a)}} \ct
        \apfunc f \big(\beta_a \ct \opp{\beta_a} \ct \apfunc{\opp f}(q) \ct \beta_{a'} \ct \opp{\beta_{a'}} \big)
        \ct {\alpha_{f(a')}}\\
        &= \opp{\alpha_{f(a)}} \ct
        \apfunc f (\apfunc{\opp f}(q))
        \ct {\alpha_{f(a')}}\\
        &= q.\qedhere
    \end{align*}
\end{proof}

因此, 如果对于类型 $A$ 有一个 $\id[A]{a}{a'}$ 的充分的表征, 类型 $\id[{\id[A]{a}{a'}}]{p}{q}$ 也是确定的.
例如:
\begin{itemize}
    \item 路径 $p = q$, 其中 $p,q : \id[A \times B]{w}{w'}$, 它们等价于路径的对子
    \[\id[{\id[A]{\proj{1} w}{\proj{1} w'}}]{\projpath{1}{p}}{\projpath{1}{q}}
    \quad\text{和}\quad
    \id[{\id[B]{\proj{2} w}{\proj{2} w'}}]{\projpath{2}{p}}{\projpath{2}{q}}.
    \]
    \item 路径 $p = q$, 其中 $p,q : \id[\prd{x:A} B(x)]{f}{g}$, 等价于同伦关系
    \[\prd{x:A} (\id[f(x)=g(x)] {\happly(p)(x)}{\happly(q)(x)}).\]
\end{itemize}

\index{传输!恒等类型中}%
接下来考虑路径的族中的传输, 即 $C:A\to\type$ 中的传输, 其中 $C(x)$ 是恒等类型.
最简单的情况, $C(x)$ 是 $A$ 中自身路径的类型, 允许固定其中一个端点.

\begin{lem}
    \label{cor:transport-path-prepost}
    对于任何 $A$ 和 $a:A$, 以及 $p:x_1=x_2$, 有
    %
    \begin{align*}
        \transfib{x \mapsto (\id{a}{x})} {p} {q} &= q \ct p
        & &\text{其中 $q:a=x_1$,}\\
        \transfib{x \mapsto (\id{x}{a})} {p} {q} &= \opp {p} \ct q
        & &\text{其中 $q:x_1=a$,}\\
        \transfib{x \mapsto (\id{x}{x})} {p} {q} &= \opp{p} \ct q \ct p
        & &\text{其中 $q:x_1=x_1$.}
    \end{align*}
\end{lem}
\begin{proof}
    在 $p$ 上路径归纳, 根据组合的单元法则.
\end{proof}

换句话说, 传输 ${x \mapsto \id{c}{x}}$ 的结果, 是后连接, 而传输 ${x \mapsto \id{x}{c}}$ 的结果是逆变的前连接.
它们就像范畴论中同伦函子的协变 $\hom(c, {\blank})$ 和逆变 $\hom({\blank},c)$ 的函子行为一样令人熟悉.

类似地, 下面可以证明更泛用的 \cref{cor:transport-path-prepost}, 和\cref{thm:transport-compose} 有关.

\begin{thm}
    \label{thm:transport-path}
    对于 $f,g:A\to B$, 以及 $p : \id[A]{a}{a'}$ 和 $q : \id[B]{f(a)}{g(a)}$, 有
    \begin{equation*}
        \id[f(a') = g(a')]{\transfib{x \mapsto \id[B]{f(x)}{g(x)}}{p}{q}}
            {\opp{(\apfunc{f}{p})} \ct q \ct \apfunc{g}{p}}.
    \end{equation*}
\end{thm}

因为 $\apfunc{(x \mapsto x)}$ 是恒等函数, 而 $\apfunc{(x \mapsto c)}$ (其中 $c$ 是常数) 是 $p \mapsto\refl{c}$, \cref{cor:transport-path-prepost} 是一个特殊情况.
更通用的版本是, $B$ 是基于 $A$ 的类型族:

\begin{thm}
    \label{thm:transport-path2}
    对于 $B : A \to \type$ 和 $f,g : \prd{x:A} B(x)$, 以及 $p : \id[A]{a}{a'}$ 和 $q : \id[B(a)]{f(a)}{g(a)}$.
    有
    \begin{equation*}
        \transfib{x \mapsto \id[B(x)]{f(x)}{g(x)}}{p}{q} =
        \opp{(\apdfunc{f}(p))} \ct \apfunc{(\transfibf{B}{p})}(q) \ct \apdfunc{g}(p).
    \end{equation*}
\end{thm}

最后, 就像在\cref{sec:compute-pi} 中, 对于恒等类型族, 还有一个等价的依值路径的表征方法.
\index{路径!依值!恒等路径中}

\begin{thm}
    \label{thm:dpath-path}
    对于 $p:\id[A]a{a'}$ 其中 $q:a=a$ 和 $r:a'=a'$, 有
    \[ \eqvspaced{ \big(\transfib{x\mapsto (x=x)}{p}{q} = r \big) }{ \big( q \ct p = p \ct r \big). } \]
\end{thm}
\begin{proof}
    在 $p$ 上路径归纳, 根据组合单元等同 $q\ct 1 = q$ 和 $r = 1\ct r$ 的事实, 它是一个等价关系.
\end{proof}

还有一些涉及函数应用的更通用的等价关系, 类似于 \cref{thm:transport-path,thm:transport-path2}.

\index{类型!恒等|)}%