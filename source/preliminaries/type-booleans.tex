\indexsee{布尔!类型}{布尔类型}%
\index{类型!布尔|(defstyle}%
布尔类型 $\bool:\UU$ 有两个居留元素 $\bfalse,\btrue : \bool$.
显然可以用余积\index{类型!余积}和单元\index{类型!单元}类型$\unit + \unit$来构造这个类型.
% this one results in a warning message just because it's on the same page as the previous entry
% for {type!coproduct}, so it's not our fault
%
不过由于它被频繁地使用, 这里给出显式的规则.
实际可以观察到, 有另一种方式从 $\Sigma$-类型 和 $\bool$ 得到二元余积.

\index{递归原理!布尔类型的}
为了得到函数 $f : \bool \to C$, 需要 $c_0,c_1 : C$ 并加入定义等同
\begin{align*}
    f(\bfalse) &\defeq c_0, \\
    f(\btrue) &\defeq c_1.
\end{align*}
递归子相当于函数式编程中的 if-then-else 结构:
\symlabel{defn:recursor-bool}%
\[ \rec{\bool} : \prd{C:\UU} C \to C \to \bool \to C \]
与定义等式
\begin{align*}
    \rec{\bool}(C,c_0,c_1,\bfalse) &\defeq c_0, \\
    \rec{\bool}(C,c_0,c_1,\btrue) &\defeq c_1.
\end{align*}

\index{归纳原理!布尔类型的}
给定 $C : \bool \to \UU$, 为了得到一个依赖函数 $f : \prd{x:\bool}C(x)$, 需要 $c_0:C(\bfalse)$ 和 $c_1 : C(\btrue)$, 于是可以给出定义等式
\begin{align*}
    f(\bfalse) &\defeq c_0, \\
    f(\btrue) &\defeq c_1.
\end{align*}
将其封装为归纳原理
\symlabel{defn:induction-bool}%
\[ \ind{\bool} : \dprd{C:\bool \to \UU}  C(\bfalse) \to C(\btrue)
\to \tprd{x:\bool} C(x) \]
与定义等式
\begin{align*}
    \ind{\bool}(C,c_0,c_1,\bfalse) &\defeq c_0, \\
    \ind{\bool}(C,c_0,c_1,\btrue) &\defeq c_1.
\end{align*}

举个例子, 使用归纳原理可以演绎出, 所有 $\bool$ 的元素必须是 $\btrue$ 或者 $\bfalse$.
和以前一样, 为了陈述它, 需要使用还没有被引入的等同类型, 使用所有东西等于它自己这个事实, 即 $\refl{x}:x=x$.
也就是, 构造一个元素, 它的类型是
\begin{equation}
    \label{thm:allbool-trueorfalse}
    \prd{x:\bool}(x=\bfalse)+(x=\btrue),
\end{equation}
即一个函数接收参数 $x:\bool$ 返回 $x=\bfalse$ 或 $x=\btrue$.
使用 \bool 的归纳原理, 其中 $C(x) \defeq (x=\bfalse)+(x=\btrue)$;
两个输入为 $\inl(\refl{\bfalse}) : C(\bfalse)$ 和 $\inr(\refl{\btrue}):C(\btrue)$.
换句话说,\eqref{thm:allbool-trueorfalse} 元素是
\[ \ind{\bool}\big(\lam{x}(x=\bfalse)+(x=\btrue),\, \inl(\refl{\bfalse}),\, \inr(\refl{\btrue})\big). \]

回忆 $\Sigma$-types 可以被当作不交和, 而余积可以被视为二元的不交并.
自然地希望用具有两个元素的类型 \bool 来构造二元不交并 $A+B$.
为此需要一个类型族 $P:\bool\to\type$ 比如 $P(\bfalse)\jdeq A$ 和 $P(\btrue)\jdeq B$.
实际上, 可以从 $\bool$ 的递归原理严格得出这个类型族.
\index{类型!族}%
(在通过归纳和递归定义\emph{类型族}的方法里, 使用了宇宙类型 $\UU$ 是它自己的类型这个事实, 这是类型论一个微妙而重要的地方.)
于是, 有定义
\index{类型!余积}%
\[ A + B \defeq \sm{x:\bool} \rec{\bool}(\UU,A,B,x) \]
与
\begin{align*}
    \inl(a) &\defeq \tup{\bfalse}{a}, \\
    \inr(b) &\defeq \tup{\btrue}{b}.
\end{align*}
习题: 从这个定义, 派生余积类型的归纳原理.
(参见 \cref{ex:sum-via-bool,sec:appetizer-univalence}.)

同样的想法可以应用到乘积和 $\Pi$-类型: 可以定义
\[ A \times B \defeq \prd{x:\bool}\rec{\bool}(\UU,A,B,x). \]
对 \bool 做归纳来构造对子:
\[ \tup{a}{b} \defeq \ind{\bool}(\rec{\bool}(\UU,A,B),a,b) \]
于是投影是直接的应用
\begin{align*}
    \fst(p) &\defeq p(\bfalse), \\
    \snd(p) &\defeq p(\btrue).
\end{align*}
用归纳原理定义二元乘积的方式稍微有一点复杂, 而且需要会在 \cref{sec:compute-pi} 引入的函数的外延性.
此外, 也在不再和以往一样, 在这里给出定义等同;
参见 \cref{ex:prod-via-bool}.
当一个类型编码成另一个类型时, 会出现这个循环的问题;
会在 \cref{sec:htpy-inductive} 返回这个问题.

偶尔会把 $\bool$ 的元素 $\bfalse$ 和 $\btrue$ 分别称为 ``false'' 和 ``true''.
注意和经典的\index{数学!经典的}数学不一样, 不使用 $\bool$ 的元素当作真值%
\index{值!真}%
或者命题.
(作为替代让命题等同于类型;
参见 \cref{sec:pat}.)
特别地, 类型 $A \to \bool$ 不是 $A$ 的幂集\index{幂集}; 只代表$A$ 的子集的``可判定性'' (参见 \cref{cha:logic}).
\index{可判定性!子集}%

\index{类型!布尔类型的|)}%