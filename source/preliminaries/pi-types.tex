\index{类型!依值函数|(defstyle}%
\index{函数!依值|(defstyle}%
\indexsee{依值!函数}{函数, 依值}%
\indexsee{类型!Pi-@$\Pi$-}{类型, 依值函数}%
\indexsee{Pi-类型@$\Pi$-类型}{类型, 依值函数}%
在类型论中, 常使用更通用的一种函数, 它被称为\define{$\Pi$-类型}或者\define{依值函数类型}.
$\Pi$-的元素是函数, 被称为\define{依值函数}, 达到域的类型, 可以依赖函数所应用的定义域的元素, 随之变化.
之所以使用 ``$\Pi$-类型'' 这个名字, 是因为这个类型也可以被视为给定类型的笛卡尔积.

给定类型 $A:\UU$ 和族 $B:A \to \UU$, 可以构造依值函数的类型 $\prd{x:A}B(x) : \UU$.
有许多可选的记号, 例如
\[
    \tprd{x:A} B(x) \qquad \dprd{x:A}B(x) \qquad \lprd{x:A} B(x).
\]
如果 $B$ 是一个常量族, 那么这个依值乘积类型是普通的函数类型:
\[
    \tprd{x:A} B \jdeq (A \to B).
\]
实际上, 所有 $\Pi$-类型的构造相都当于普通函数类型的构造的推广.

\indexdef{定义!对于函数,直接的}%
可以通过显式的定义来引入依值函数: 要定义 $\prd{x:A}B(x)$, 其中 $f$ 是要被定义的依值函数, 需要一个表达式$\Phi : B(x)$, 这个表达式可以包含变量 $x:A$,
\index{变量}%
然后写作
\[
    f(x) \defeq \Phi \qquad \mbox{对于 $x:A$}.
\]
或者, 可以使用 \define{$\lambda$-抽象}%
\index{lambda 抽象@$\lambda$-抽象|defstyle}%
\begin{equation}
    \label{eq:lambda-abstraction}
    \lamu{x:A} \Phi \ :\ \prd{x:A} B(x).
\end{equation}
\indexdef{应用!依值函数的}%
\indexdef{函数!依值!应用}%
与非依值函数中一样, 可以\define{应用}依值函数 $f : \prd{x:A}B(x)$ 于参数 $a:A$ 得到元素 $f(a):B(a)$.
等同也和普通的函数一样, 例如有计算规则,
\index{计算规则!对于依值函数类型}%
即给定 $a:A$ 有 $f(a) \jdeq \Phi'$ 和 $(\lamu{x:A} \Phi)(a) \jdeq \Phi'$, 其中 $\Phi'$ 是通过替换 $\Phi$ 中所有 $x$ 为 $a$ 的结果(一如既往地避免变量捕获).
对于任何 $f:\prd{x:A} B(x)$, 类似地, 有唯一性原理 $f\jdeq (\lam{x} f(x))$.
\index{唯一性!原理!依值函数类型的}%

举个例子, 回顾\cref{sec:universes}中, 类型族 $\Fin:\nat\to\UU$, 它的值是标准有限集, 元素 $0_n,1_n,\dots,(n-1)_n : \Fin(n)$.
于是有一个依值函数 $\fmax : \prd{n:\nat} \Fin(n+1)$ 它返回每个非空有限集的``最大的''元素, $\fmax(n) \defeq n_{n+1}$.
\index{有限!集合, 族的}%
和 $\Fin$ 本身一样, 现在还不能定义 $\fmax$, 但马上就可以了;
参见 \cref{ex:fin}.

现在, 可以定义依值函数类型的另一个重要的优点, 给定的全集上\define{多态}函数.
\indexdef{函数!多态}%
\indexdef{多态函数}%
多态函数从其参数获取一个类型, 并影响这个类型的元素(或者构造其它类型).
\symlabel{idfunc}%
\indexdef{函数!态恒}%
\indexdef{态恒!函数}%
举个例子, 多态恒等函数 $\idfunc : \prd{A:\UU} A \to A$, 定义为 $\idfunc{} \defeq \lam{A:\type}{x:A} x$.
(类似 $\lambda$-抽象, 如果不显界定, $\Pi$ 的自然的作用域\index{作用域}包括了表达式的剩余部分;
因此 $\idfunc : \prd{A:\UU} A \to A$ 意味 $\idfunc : \prd{A:\UU} (A \to A)$.
这种习惯在数学中不常见, 但在类型论中常见.)

有时把一些依值函数的参数作为下标.
例如, 可以通过 $\idfunc[A](x) \defeq x$ 等价地定义多态恒等函数.
此外, 如果一个参数可以从上下文中推断, 可以完全忽略它.
例如, 如果 $a:A$, 那么 $\idfunc(a)$ 的写法没有歧义, 因为 $\idfunc$ 必须表示为 $\idfunc[A]$ 才能应用于 $a$.

多态函数的另一个不那么简单的例子是``交换''运算, 它交换一个(柯里化后的)函数的参数的顺序:
\[
    \mathsf{swap} : \prd{A:\UU}{B:\UU}{C:\UU} (A\to B\to C) \to (B\to A \to C).
\]
可以这样定义
\[
    \mathsf{swap}(A,B,C,g) \defeq \lam{b}{a} g(a)(b).
\]
也可以等价地把类型参数写为下标:
\[
    \mathsf{swap}_{A,B,C}(g)(b,a) \defeq g(a,b).
\]

注意与普通函数一样, 我们使用柯里化来定义多个参数的依值函数 (就像 $\mathsf{swap}$).
不过, 在依值情况下, 定义域的第二个类型可能会依赖于第一个第一个元素, 而到达域可能会依赖所有的.
也就是, 给定 $A:\UU$ 和类型族 $B : A \to \UU$ 和 $C : \prd{x:A}B(x) \to \UU$, 可以构造出两个参数的函数 $\prd{x:A}{y : B(x)} C(x,y)$.
当 $B$ 是常量而且等同于 $A$ 的情况下, 可以合并符号然后写为 $\prd{x,y:A}$;
例如, $\mathsf{swap}$ 的类型也可以写为
\[
    \mathsf{swap} : \prd{A,B,C:\UU} (A\to B\to C) \to (B\to A \to C).
\]
最后, 给定 $f:\prd{x:A}{y : B(x)} C(x,y)$ 和参数 $a:A$ 和 $b:B(a)$, 有 $f(a)(b) : C(a,b)$, 和以前一样, 记作 $f(a,b) : C(a,b)$.

\index{类型!依值函数|)}%
\index{函数!依值|)}%