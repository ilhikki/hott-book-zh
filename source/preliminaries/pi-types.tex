\index{类型!依值函数|(defstyle}%
\index{函数!依值|(defstyle}%
\indexsee{依值!函数}{函数, 依值}%
\indexsee{类型!Pi-@$\Pi$-}{类型, 依值函数}%
\indexsee{Pi-类型@$\Pi$-类型}{类型, 依值函数}%
在类型论中, 我们经常使用一种更通用的函数, 它被称为\define{$\Pi$-类型}或者\define{依值函数类型}.
$\Pi$-类型的元素是一个函数, 它被称为\define{依值函数}, 它的达到域的类型可以根据定义域所接受的元素而变化.
之所以使用 ``$\Pi$-类型'' 这个名字, 是因为这个类型也可以被视为给定类型的笛卡尔积.

给定一个类型 $A:\UU$ 和一个族 $B:A \to \UU$, 可以构造这个依值函数的类型 $\prd{x:A}B(x) : \UU$.
有一些其它的符号来表达这个类型, 比如
\[
    \tprd{x:A} B(x) \qquad \dprd{x:A}B(x) \qquad \lprd{x:A} B(x).
\]
如果 $B$ 是一个常量族, 那么这个依值乘积类型是普通的函数类型:
\[
    \tprd{x:A} B \jdeq (A \to B).
\]
实际上, 所有 $\Pi$-类型的构造相当于普通函数类型的构造的扩展.

\indexdef{定义!对于函数,直接的}%
可以通过显式的定义来引入依值函数: 为了定义 $\prd{x:A}B(x)$, 其中 $f$ 是要被定义的依值函数, 需要一个表达式$\Phi : B(x)$, 这个表达式可以包含变量 $x:A$,
\index{变量}%
然后写作
\[
    f(x) \defeq \Phi \qquad \mbox{对于 $x:A$}.
\]
或者, 可以使用 \define{$\lambda$-抽象}%
\index{lambda 抽象@$\lambda$-抽象|defstyle}%
\begin{equation}
    \label{eq:lambda-abstraction}
    \lamu{x:A} \Phi \ :\ \prd{x:A} B(x).
\end{equation}
\indexdef{应用!之于函数类型}%
\indexdef{函数!依值!应用}%
与非依值函数中一样, \define{应用}一个依值函数 $f : \prd{x:A}B(x)$ 到参数 $a:A$ 来获得元素 $f(a):B(a)$.
等同性也和普通的函数一样, 例如有计算规则,
\index{计算规则!对于依值函数类型}%
即给定 $a:A$ 有 $f(a) \jdeq \Phi'$ 和 $(\lamu{x:A} \Phi)(a) \jdeq \Phi'$, 这里的 $\Phi' $ 是通过替换 $\Phi$ 中所有 $x$ 为 $a$ 的结果(和往常一样避免变量捕获).
对于任何 $f:\prd{x:A} B(x)$ 有类似的唯一性原理 $f\jdeq (\lam{x} f(x))$.
\index{唯一性!原理!对于依值函数类型}%

举个例子, 回顾\cref{sec:universes}, 有一个类型族 $\Fin:\nat\to\UU$, 值是标准有限集, 它的元素是 $0_n,1_n,\dots,(n-1)_n : \Fin(n)$.
然后有一个依值函数 $\fmax : \prd{n:\nat} \Fin(n+1)$ 它返回每个非空有限集的``最大的''元素, $\fmax(n) \defeq n_{n+1}$.
\index{有限!集合, 族的}%
和 $\Fin$ 本身一样, 现在还不能定义 $\fmax$, 但马上就可以了;
参见 \cref{ex:fin}.

现在可以定义依值函数类型的另一个重要的方面, 即给定的宇宙的\define{多态性}.
\indexdef{函数!多态性}%
\indexdef{多态函数}%
多态函数是一个函数, 根据其中一个参数获得一个类型, 然后将其作为其它的元素的类型(或者构造其它类型).
\symlabel{idfunc}%
\indexdef{函数!态恒}%
\indexdef{态恒!函数}%
举个例子, 多态恒等函数 $\idfunc : \prd{A:\UU} A \to A$, 定义为 $\idfunc{} \defeq \lam{A:\type}{x:A} x$.
(类似于 $\lambda$-抽象, 如果不显界定, $\Pi$ 的自然的作用域\index{作用域}包括了表达式的剩余部分;
因此 $\idfunc : \prd{A:\UU} A \to A$ 意味着 $\idfunc : \prd{A:\UU} (A \to A)$.
这在数学上不常见, 在类型论中则相反.)

有时把一些依值函数的参数用下标表示.
例如, 可以通过 $\idfunc[A](x) \defeq x$ 来等价地定义多态恒等函数.
此外, 如果一个参数可以从上下文中推断, 可以完全忽略它.
例如, 如果 $a:A$, 那么 $\idfunc(a)$ 的写法没有歧义, 因为 $\idfunc$ 必须表示为 $\idfunc[A]$ 才能应用到 $a$.

另外一个不那么简单的例子是多态函数``交换''运算, 它交换一个(柯里化后的)函数的参数的顺序:
\[
    \mathsf{swap} : \prd{A:\UU}{B:\UU}{C:\UU} (A\to B\to C) \to (B\to A \to C).
\]
可以这样定义
\[
    \mathsf{swap}(A,B,C,g) \defeq \lam{b}{a} g(a)(b).
\]
也可以等价地把类型参数写为下标:
\[
    \mathsf{swap}_{A,B,C}(g)(b,a) \defeq g(a,b).
\]

注意于普通函数一样, 使用柯里化来定义有一些参数的依值函数 (例如 $\mathsf{swap}$).
不过, 在有依赖的情况下, 定义域的第二个类型可能会依赖于第一个第一个元素, 而到达域可能会依赖所有的.
也就是, 给定 $A:\UU$ 和类型族 $B : A \to \UU$ 和 $C : \prd{x:A}B(x) \to \UU$, 可以构造出两个参数的函数 $\prd{x:A}{y : B(x)} C(x,y)$.
当 $B$ 是静态的而且和 $A$ 相同的情况下, 可以合并符号然后写为 $\prd{x,y:A}$;
例如, $\mathsf{swap}$ 的类型也可以写为
\[
    \mathsf{swap} : \prd{A,B,C:\UU} (A\to B\to C) \to (B\to A \to C).
\]
最后, 给定 $f:\prd{x:A}{y : B(x)} C(x,y)$ 和参数 $a:A$ 和 $b:B(a)$, 有 $f(a)(b) : C(a,b)$, 和以前一样, 记作 $f(a,b) : C(a,b)$.

\index{类型!依值函数|)}%
\index{函数!依值|)}%