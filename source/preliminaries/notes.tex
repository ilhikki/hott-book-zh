\sectionNotes

这里提到的类型论是 Martin-L\"{o}f 的直觉类型论的一个版本~\cite{Martin-Lof-1972,Martin-Lof-1973,Martin-Lof-1979,martin-lof:bibliopolis},
直觉类型论基于并受到\cite{beeson}, Heyting~\cite{heyting1966intuitionism}, Scott~\cite{scott70},
de Bruijn~\cite{deBruijn-1973}, Howard~\cite{howard:pat}, Tait~\cite{Tait-1966,Tait-1968},
和 Lawvere~\cite{lawvere:adjinfound}\index{Lawvere}的基础工作影响.
\index{proof!assistant}%
三个 Martin-L\"{o}f 类型伦的主要变体的计算实现的是 \NuPRL \cite{constable+86nuprl-book}, \Coq~\cite{Coq}, 和 \Agda \cite{norell2007towards} 的基础.
这里给出的理论和这些形式有一些区别, 其中是同伦解释的关键, 而另一些是技术上的便利设施或者暂时没有纳入同伦场景.

\index{类型论!内延的}%
\index{类型论!外延的}%
\index{内延的类型论}%
\index{外延!类型论}%
显然, 这里给出的类型论是从\emph{内延}版本的 Martin-L\"{o}f 类型论~\cite{Martin-Lof-1973} 派生的, 而不是\emph{外延}版本~\cite{Martin-Lof-1979}.
而外延的理论中判断和命题等同没有区分, 内延的理论视判断等同为纯的定义性, 并认为同伦解释的核心恒等类型有广泛的, 证明相关的解释.
从同伦的角度看, 外延类型论将其自己界定为同伦的离散集合 (参见 \cref{sec:basics-sets}), 而内延类型论认为类型带有高维结构.
\NuPRL 系统~\cite{constable+86nuprl-book} 是外延的, 而 \Coq~\cite{Coq} 和 \Agda~\cite{norell2007towards} 都是内延的.
在内延类型论中, 有一些变体, 它们恒等证明的结构有区别.
最开明的理解, 这里依靠的, 允许一个等于的\emph{证明相关}理解,而受限的变体主张例如\emph{恒等证明的唯一性(UIP)}~\cite{Streicher93}的限制,
\indexsee{UIP}{恒等证明的唯一性}%
\index{唯一性!恒等证明的}%
让任何两个等于的证明为命题等同的, 而且\emph{K 公理}~\cite{Streicher93},
\index{公理!Streicher 的 K 公理}
认为等于的唯一证明就是自反(想到于判断等同).
这些额外的条件在 \Coq 和 \Agda\ 系统中是可选的.

%(In the terminology of \cref{cha:hlevels} such a type theory is about $0$-truncated types.)

另一个变体之一的内延理论的观点是判断等同的强度, 尤其是关心函数类型的对象.
这里包含了唯一性原理\index{唯一性!原理} ($\eta$-变换) $f \jdeq \lam{x} f(x)$, 作为一个判断的等同的原则.
例如这样使用这个原理, 在\cref{sec:univalence-implies-funext}, 展示了宇价蕴含了命题函数外延性.
有时在其他类型上考虑唯一性原理.
例如, 笛卡尔积的 $A\times B$唯一性原理\index{唯一性!原理!乘积类型} 是在 \cref{sec:finite-product-types} 构造的命题等同 $\uniq{A\times B}$ 的一个判断版本, 它说明 $u \jdeq (\proj1(u),\proj2(u))$.
它和依值类型的对应版本可以被合理选择(我们没有选择), 不过无法包含所有这样的规则, 因为对应的等同类型的唯一性原理会平凡化所有高阶同伦结构.
所以我们\emph{被强迫}排除它, 然后问题编程了在哪里划清界限.
关于归纳类型, 将在~\cref{sec:htpy-inductive}进一步讨论.

函数的(命题)等同被认为是\emph{外延的}对我们的目标很重要 (与字面上的使用不同!).
这不是这章得到的规则;
它会在 \cref{axiom:funext} 中被表达.
\index{函数外延性}%
这个决定对我们的目的很重要, 因为它令函数的相等性符合数学中的预期.
尽管把\cref{axiom:funext}作为公理包含, 它也可以从宇价公理和函数的唯一性原理index{唯一性!原理!函数类型} (参见\cref{sec:univalence-implies-funext}), 也能从间点类型得到 (参见 \cref{thm:interval-funext}).

关于归纳类型例如乘积, $\Sigma$-类型, 余积, 自然数等等 (参见 \cref{cha:induction}),  关于递归和归纳的形式有额外的选择.
\index{模式匹配}%
我们已经把\emph{递归原理}作为基础, 而且\emph{模式匹配}由它派生, 不过其中一个可以派生另一个; 参见\cref{cha:rules}.
通常后者也允许\emph{深}模式匹配; 参见~\cite{Coquand92Pattern}.
我们的选择有一些原因.
原因之一是可以语义明确地自然得到归纳原理.
另一个原因是指定(深)模式匹配的可行的种类有一些复杂;
例如 \Agda 的
\index{证明!助手!Agda@\textsc{Agda}}%
模式匹配默认可以证明 K 公理,
\index{公理!Streicher 的 K 公理}%
尽管标记 \texttt{--without-K} 可以阻止它~\cite{CDP14}.
最后, 深模式匹配不是良好理解\emph{高阶}归纳类型 (参见 \cref{cha:hits}).
因此, 我们只像\cref{sec:pattern-matching}讨论的那样用模式匹配, 直接等价于归纳类型的应用.

\index{证明!助手!Coq@\textsc{Coq}}%
和 \Coq 的类型伦不一样, 我们没有包含命题的原始类型.
而是像\cref{sec:pat}讨论的那样, 我们接受\emph{命题即类型 (PAT)}原理, 恒等命题与类型.
这被 de Bruijn~\cite{deBruijn-1973}, Howard~\cite{howard:pat}, Tait~\cite{Tait-1968}, 和 Martin-L\"{o}f~\cite{Martin-Lof-1972} 所建议.
(关于我们的决定, 更充分的解释在 \cref{subsec:pat?,subsec:hprops}.)

不过我们做的是, 包含宇宙的全累积层次, 所以类型形式和等同命题成为一个宇宙的成员和等同命题的实例.
为了方便, 视宇宙的对象为类型, 而不是类型的代码;
在 \cite{martin-lof:bibliopolis} 的术语中, 这意味使用了 ``Russell-style 宇宙'' 而不是 ``Tarski-style 宇宙''.
\index{类型!宇宙!Tarski-style}%
\index{类型!宇宙!Russell-style}%
另一种方法是 Tarski-style 宇宙, 显式约束\index{要求, 宇宙提升}函数需要一个宇宙的元素 $A:\UU$ 返回一个 $\mathsf{El}(A)$ 的类型, 而且这个约束在非形式的情况是可以忽略的.

我们对待宇宙层级的方法也是累积的, 也就是当 $j\geq i$ 时 $\UU_i$ 中的类型也在 $\UU_j$ 中.
实现累积形式有不同的方法: 最简单的就是包含一个规则, $A:\UU_i$ 则 $A:\UU_j$.
不过, 它有这个令人烦恼的后果, 就是对于类型族 $B:A\to \UU_i$ 无法得到 $B:A\to\UU_j$, 尽管可以得到 $\lam{a} B(a) : A\to\UU_j$.
解决这个问题的一个更复杂的方法是引入一个判断的子类型关系 $<:$ 生成于 $\UU_i<:\UU_j$, 不过这样类型论变得更复杂以至于难以学习.
另一个选择是包含显示的约束函数 $\uparrow : \UU_i \to \UU_j$, 在非形式的情况是可以忽略.

没有必要通过自然数和线性排序为宇宙编号.
对于一些目的, 只假设每个宇宙是更大宇宙的元素更合适, 伴随一个``有向''特性 that 每两个宇宙是都存在于一些更大的宇宙中.
有很多其他可能的变体, 例如一个宇宙 ``$\UU_{\omega}$'' 它包含了所有 $\UU_i$ (甚至更高的 ``大基数'' 类型宇宙), 过着通过引入层级到类型族 $\lam{i} \UU_i$.
在 \Agda 中实际使用了后者.

Martin-L\"{o}f~\cite{Martin-Lof-1972} 形式化了恒等类型的路径归纳原理.
Martin-L\"of 类型论设定的基础路径归纳原理来自于 Paulin-Mohring \cite{Moh93};
它能看作是\NuPRL~\cite[Section~8.1]{constable+86nuprl-book}中假设命题的``逐点函数''\index{逐点!函数的}的概念的内延的推广.
Martin-L\"of 的规则包含 Paulin-Mohring 的规则, 通过 Streicher 使用 K 公理 (参见~\cref{sec:hedberg})证明, 而 Altenkirch 和 Goguen 的证明在 \cref{sec:identity-types}, 而最终被 Hofmann 证明并没有使用宇宙 (在 \cref{ex:pm-to-ml}); 参见~\cite[\S1.3 and Addendum]{Streicher93}.
