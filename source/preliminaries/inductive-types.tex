\indexsee{类型!自然数的}{自然数}%
\index{自然数|(defstyle}%
\indexsee{数!自然}{自然数}%
迄今为止, 我们有一些规则, 来根据抽象运算构造新类型, 而为了做具体数学, 还需要一些具体类型, 例如数字.
最基础的就是自然数的类型 $\nat : \UU$;
有了自然数, 就可以构造整数, 有理数, 实数, 等等 (参见\cref{cha:real-numbers}).

$\nat$ 的元素使用 $0 : \nat$\indexdef{零}, 和后继\indexdef{后继}符号 $\suc : \nat \to \nat$ 构造.
书写自然数时, 接受十进制的符号 $1 \defeq \suc(0)$, $2 \defeq \suc(1)$, $3 \defeq \suc(2)$, \dots.

自然数的重要性在于, 可以用递归定义函数和用归纳来做证明---现在 ``递归'' 和 ``归纳''这两个词有了令人熟悉的含义.
\index{递归原理!自然数的}%
为了通过递归构造一个定义域为自然数的非依值函数 $f : \nat \to C$, 提供一个起点 $c_0 : C$ 和 ``下一步'' 函数 $c_s : \nat \to C \to C$ 即可.
于是得到 $f$, 通过定义等式
\index{计算规则!for 自然数}%
\begin{align*}
    f(0) &\defeq c_0, \\
    f(\suc(n)) &\defeq c_s(n,f(n)).
\end{align*}
我们说 $f$ 是通过\define{原始递归}定义的.
\indexdef{原始!递归}%
\indexdef{递归!原始}%

举个例子, 如何在自然数上定义一个函数, 返回参数的双倍.
这个例子中, 有 $C\defeq \nat$.
首先提供 $\dbl(0)$ 的值, 很简单, 即 $c_0 \defeq 0$.
然后, 为了计算关于自然数 $n$ 的 $\dbl(\suc(n))$ 的值 ,首先要计算 $\dbl(n)$ 的值, 然后使用两次后继运算符.
这又重复\index{重复}被 $c_s(n,y) \defeq \suc(\suc(y))$ 所捕获.
注意 $c_s$ 的第二个参数 $y$ 代表\emph{递归调用}\index{递归调用} $\dbl(n)$ 的结果.

通过原始递归定义 $\dbl:\nat\to\nat$, 于是, 得到这个定义等式:
\begin{align*}
    \dbl(0) &\defeq 0\\
    \dbl(\suc(n)) &\defeq \suc(\suc(\dbl(n))).
\end{align*}
它的确具有正确的计算行为: 例如, 有
\begin{align*}
    \dbl(2) &\jdeq \dbl(\suc(\suc(0)))\\
    &\jdeq c_s(\suc(0), \dbl(\suc(0))) \\
    &\jdeq \suc(\suc(\dbl(\suc(0)))) \\
    &\jdeq \suc(\suc(c_s(0,\dbl(0)))) \\
    &\jdeq \suc(\suc(\suc(\suc(\dbl(0))))) \\
    &\jdeq \suc(\suc(\suc(\suc(c_0)))) \\
    &\jdeq \suc(\suc(\suc(\suc(0))))\\
    &\jdeq 4.
\end{align*}
也可以通过原始递归定义多元函数, 通过柯里化并让 $C$ 是一个函数类型.
\indexdef{加法!自然数的}%
例如, 定义加法 $\add : \nat \to \nat \to \nat$ 其中 $C \defeq \nat \to \nat$ 和 ``起点'' 与 ``下一步'' 为:
\begin{align*}
    c_0 &: \nat \to \nat \\
    c_0 (n) &\defeq n \\
    c_s &: \nat \to (\nat \to \nat) \to (\nat \to \nat) \\
    c_s(m,g)(n) &\defeq \suc(g(n)).
\end{align*}
于是得到 $\add : \nat \to \nat \to \nat$ 满足这个定义等同
\begin{align*}
    \add(0,n) &\jdeq n \\
    \add(\suc(m),n) &\jdeq \suc(\add(m,n)).
\end{align*}
按照习惯, $\add(m,n)$ 写作 $m+n$.
请读者校验 $2+2\jdeq 4$.
% ex: define multiplication and exponentiation.


和以前一样, 可以包装这个原始递归的原理为递归器:
\[\rec{\nat} : \dprd{C:\UU} C \to (\nat \to C \to C) \to \nat \to C \]
通过定义等式
\symlabel{defn:recursor-nat}%
\begin{align*}
    \rec{\nat}(C,c_0,c_s,0) &\defeq c_0, \\
    \rec{\nat}(C,c_0,c_s,\suc(n)) &\defeq c_s(n,\rec{\nat}(C,c_0,c_s,n)).
\end{align*}
%ex derive rec from it
可以这样使用 $\rec{\nat}$ 表示 $\dbl$ 和 $\add$:
\begin{align}
    \dbl &\defeq \rec\nat\big(\nat,\, 0,\, \lamu{n:\nat}{y:\nat} \suc(\suc(y))\big) \label{eq:dbl-as-rec}\\
    \add &\defeq \rec{\nat}\big(\nat \to \nat,\, \lamu{n:\nat} n,\, \lamu{n:\nat}{g:\nat \to \nat}{m :\nat} \suc(g(m))\big).
\end{align}
当然, 所有只用原始递归原理定义的函数都是\emph{可计算的}.
(存在高阶函数类型 --- 例如, 其它函数作为参数的函数 --- 意味可以定义比原始递归能力更多的东西;
参见例如~\cref{ex:ackermann}.)
这在构造主义数学中是合理的;
\index{数学!构造主义}%
在\cref{sec:intuitionism,sec:axiom-choice} 会看到, 如何扩展类型论来可以定义更通用的的数学的函数.

\index{归纳原理!自然数的}
现在把同样的方法作用于其它类型, 推广原始递归到依值函数得到\emph{归纳原理}.
从而, 假设给定族 $C : \nat \to \UU$, 元素 $c_0 : C(0)$, 和函数 $c_s : \prd{n:\nat} C(n) \to C(\suc(n))$;
可以构造 $f : \prd{n:\nat} C(n)$ 通过定义等式:\index{计算规则!for 自然数}
\begin{align*}
    f(0) &\defeq c_0, \\
    f(\suc(n)) &\defeq c_s(n,f(n)).
\end{align*}
也可以把它包装到单独的函数中 \symlabel{defn:induction-nat}%
\[\ind{\nat} : \dprd{C:\nat\to \UU} C(0) \to \Parens{\tprd{n : \nat} C(n) \to C(\suc(n))} \to \tprd{n : \nat} C(n) \]
通过这个定义等式
\begin{align*}
    \ind{\nat}(C,c_0,c_s,0) &\defeq c_0, \\
    \ind{\nat}(C,c_0,c_s,\suc(n)) &\defeq c_s(n,\ind{\nat}(C,c_0,c_s,n)).
\end{align*}
这里终于看到了, 与归纳证明这个经典概念的联系.
回顾在类型论中, 用类型表示命题, 而且通过提供对应类型的居留元来证明命题.
特别地, 一个自然数的\emph{性质}通过类型族 $P:\nat\to\type$ 来表现.
按照这个观点, 上面的归纳原理说, 如果可以证明 $P(0)$, 且若 $P(n)$ 则 $P(\suc(n))$, 那么对于所有 $n$ 有 $P(n)$.
也就是说, 这就是自然数的归纳证明法.

\index{结合律!加法的!自然数的}
举个例子, 如何证明 $+$ 的结合率.
(实际上不会写出这种风格的证明, 不过它是一个好例子, 方便理解在类型论中的归纳的表现形式.)
为了推导出
\[\assoc : \prd{i,j,k:\nat} \id{i + (j + k)}{(i + j) + k}, \]
提供
\[ \assoc_0 : \prd{j,k:\nat} \id{0 + (j + k)}{(0+ j) + k} \]
和
\begin{narrowmultline*}
    \assoc_s : \prd{i:\nat} \left(\prd{j,k:\nat} \id{i + (j + k)}{(i + j) + k}\right)
    \narrowbreak
    \to \prd{j,k:\nat} \id{\suc(i) + (j + k)}{(\suc(i) + j) + k}
\end{narrowmultline*}
即可.
为了推导 $\assoc_0$, 回顾 $0+n \jdeq n$, 而 $0 + (j + k) \jdeq j+k \jdeq (0+ j) + k$.
因此只需令
\[ \assoc_0(j,k) \defeq \refl{j+k}. \]
对于 $\assoc_s$, 回顾 $+$ 的定义有 $\suc(m)+n \jdeq \suc(m+n)$, 于是
\begin{align*}
    \suc(i) + (j + k) &\jdeq \suc(i+(j+k)) \qquad\text{和}\\
    (\suc(i)+j)+k &\jdeq \suc((i+j)+k).
\end{align*}
因此, $\assoc_s$ 返回的类型等价于 $\id{\suc(i+(j+k))}{\suc((i+j)+k)}$.
但是它的输入(``归纳猜想'') \index{猜想!归纳}%
\index{归纳!猜想}%
是 $\id{i+(j+k)}{(i+j)+k}$, 所以只需调用相同自然数的后继也相同这个事实.
(会在 \cref{lem:map} 证明这个明显的事实, 通过使用等同类型的归纳原理.)
调用这个后来的事实 $\apfunc{\suc} :  (\id[\nat]{m}{n}) \to (\id[\nat]{\suc(m)}{\suc(n)})$, 所以可以定义
\[
    \assoc_s(i,h,j,k) \defeq \apfunc{\suc}( %\prd{m,n:\nat}%n+(j+k),(n+j)+k,
    h(j,k)).
\]
将它们一起代入到 $\ind{\nat}$, 得到结合律的一个证明.

\index{自然数|)}%