给定 $A,B:\UU$, 引入它们的\define{余积}类型 $A+B:\UU$.
\indexsee{余积}{类型, 余积}%
\index{类型!余积|(defstyle}%
\indexsee{不交!和}{类型, 余积}%
\indexsee{不交!并}{类型, 余积}%
\indexsee{并!不交}{类型, 余积}%
\indexsee{并!不交}{类型, 余积}%
它相当于集合论中的\emph{不交并}, 有时我们也使用这个名字.
和函数与乘积一样, 余积必须是类型论中的基础结构, 因为之前没有给出 ``类型的并'' 的概念.
也引入零元的版本: \define{空类型 $\emptyt:\UU$}.
\indexsee{无效!余积}{类型, 空}%
\indexsee{空类型}{类型, 空}%
\index{类型!空|(defstyle}%

有两种构造 $A+B$ 的元素的方法, $\inl(a) : A+B$ 对于 $a:A$, 以及 $\inr(b):A+B$ 对于 $b:B$.
($\inl$ 和 $\inr$ 是``left injection'' 和 ``right injection'' 的缩写.)
无法构造空类型的元素.

\index{递归原理!余积的}
为了构造非依值函数 $f : A+B \to C$, 需要函数 $g_0 : A \to C$ 和 $g_1 : B \to C$.
于是 $f$ 被定义为
\begin{align*}
    f(\inl(a)) &\defeq g_0(a), \\
    f(\inr(b)) &\defeq g_1(b).
\end{align*}
也就是说, 函数 $f$ 通过\define{情况分析}定义.
\indexdef{情况分析}%
和以前一样, 可以推导出递归器:
\symlabel{defn:recursor-plus}%
\[ \rec{A+B} : \dprd{C:\UU}(A \to C) \to (B\to C) \to A+B \to C\]
通过定义等式
\begin{align*}
    \rec{A+B}(C,g_0,g_1,\inl(a)) &\defeq g_0(a), \\
    \rec{A+B}(C,g_0,g_1,\inr(b)) &\defeq g_1(b).
\end{align*}

\index{递归原理!空类型的}
总是可以构造一个函数 $f : \emptyt \to C$ 而不需要给定任何定义等式, 因为 \emptyt 没有任何元素来定义 $f$.
因此, $\emptyt$ 的递归器是
\symlabel{defn:recursor-emptyt}%
\[\rec{\emptyt} : \tprd{C:\UU} \emptyt \to C,\]
即一个经典的函数, 输入空类型返回任何其它类型.
逻辑上, 它相当于爆炸原理.
\index{爆炸原理}%

\index{归纳原理!余积的}
为了构造一个定义域类型为余积的依值函数 $f:\prd{x:A+B}C(x)$, 假设给定族 $C: (A + B) \to \UU$, 然后需要
\begin{align*}
    g_0 &: \prd{a:A} C(\inl(a)), \\
    g_1 &: \prd{b:B} C(\inr(b)).
\end{align*}
通过这个定义等式得到 $f$ :\index{计算规则!余积类型}
\begin{align*}
    f(\inl(a)) &\defeq g_0(a), \\
    f(\inr(b)) &\defeq g_1(b).
\end{align*}
把它打包为封装余积的归纳定理:
\symlabel{defn:induction-plus}%
\begin{narrowmultline*}
    \ind{A+B} :
    \dprd{C: (A + B) \to \UU}
    \Parens{\tprd{a:A} C(\inl(a))} \to \narrowbreak
    \Parens{\tprd{b:B} C(\inr(b))} \to \tprd{x:A+B}C(x).
\end{narrowmultline*}
和以前一样, 递归器是族 $C$ 为常量时的特殊情况.

\index{归纳原理!空类型的}
空类型的归纳原理
\symlabel{defn:induction-emptyt}%
\[ \ind{\emptyt} : \prd{C:\emptyt \to \UU}{z:\emptyt} C(z) \]
可以定义一个琐屑的定义域类型是空类型的依值函数.
% In the presence of $\eta$-equality it is derivable
% from the recursor.
% ex

\index{类型!余积|)}%
\index{类型!空|)}%