\index{类型!依值对子|(defstyle}%
\indexsee{类型!依值和}{类型, 依值对子}%
\indexsee{类型!Sigma-@$\Sigma$-}{类型, 依值对子}%
\indexsee{Sigma-type@$\Sigma$-type}{类型, 依值对子}%
\indexsee{和!依值}{类型, 依值对子}%

类似于将函数类型 (\cref{sec:function-types}) 泛化为依值函数类型 (\cref{sec:pi-types}), 乘积类型 (\cref{sec:finite-product-types}) 也可以被泛化, 来让对子的第二个类型, 依赖第一个类型.
这被称为\define{依值对子类型}, 或者\define{$\Sigma$-类型}, 因为在集合论中, 它相当于给定类型的不交和.

给定类型 $A:\UU$ 和族 $B : A \to \UU$, 依值对子类型可以写做 $\sm{x:A} B(x) : \UU$.
或者
\[
    \tsm{x:A} B(x) \hspace{2cm} \dsm{x:A}B(x) \hspace{2cm} \lsm{x:A} B(x).
\]
与 $\lambda$-抽象和 $\Pi$ 一样, 如果没有显式界定 $\Sigma$ 的作用域\index{作用域}, 那么它的作用域包括了表达式的剩余部分, 例如 $\sm{x:A} B(x) \to C$ 相当于 $\sm{x:A} (B(x) \to C)$.

\symlabel{defn:dependent-pair}%
\indexdef{对子!依值}%
构造依值对子类型的元素也是通过对子构造的: 给定 $a:A$ 和 $b:B(a)$ 有 $\tup{a}{b} : \sm{x:A} B(x)$.
如果 $B$ 是常量, 那么依值对子类型就是有序的笛卡尔积类型:
\[
    \Parens{\sm{x:A} B} \jdeq (A \times B).
\]
所有 $\Sigma$-类型可以当作是乘积类型中, 直接用依值函数代替非依值函数的结果.

例如, 递归原理\index{递归原理!依值对子类型的}%
表示, 为了定义一个以 $\Sigma$-类型为参数的函数 $f : (\sm{x:A} B(x)) \to C$, 只要给定函数  $g : \prd{x:A} B(x) \to C$, 然后就可以通过这个定义等式
\[
    f(\tup{a}{b}) \defeq g(a)(b)
\]
定义 $f$.
\indexdef{投影!依值对子类型的}%
例如, 可以派生 $\Sigma$-类型的第一投影:
\symlabel{defn:dependent-proj1}%
\begin{equation*}
    \fst : \Parens{\sm{x : A}B(x)} \to A
\end{equation*}
通过这个定义等式
\begin{equation*}
    \fst(\tup{a}{b}) \defeq a.
\end{equation*}
不过, 因为
\narrowequation{
    (a,b):\sm{x:A} B(x)
}
的第二个组成的类型是 $B(a)$, 第二投影必须是\emph{依值}函数, 它的类型含有第一投影函数:
\symlabel{defn:dependent-proj2}%
\[
    \snd : \prd{p:\sm{x : A}B(x)}B(\fst(p)).
\]
因此需要 $\Sigma$-类型\emph{归纳}原理\index{归纳原理!依值对子类型的}(``依值消去子'')
即为了构造一个接收 $\Sigma$-类型, 返回一个族的依值函数 $C : (\sm{x:A} B(x)) \to \UU$, 需要一个函数
\[
    g : \prd{a:A}{b:B(a)} C(\tup{a}{b}).
\]
然后派生一个函数
\[
    f : \prd{p : \sm{x:A}B(x)} C(p)
\]
和定义等式\index{计算规则!依值对子类型的}
\[
    f(\tup{a}{b}) \defeq g(a)(b).
\]
把这个函数应用到 $C(p)\defeq B(\fst(p))$, 可以定义
\narrowequation{\snd : \prd{p:\sm{x : A}B(x)}B(\fst(p))}
以这个明显的等式
\[
    \snd(\tup{a}{b}) \defeq b.
\]
注意 $B (\fst(\tup{a}{b})) \jdeq B(a)$, 对 $\fst$ 使用定义等式 , 结果确实是 $b : B(a)$, 说明它是正确的.

可以包装 $\Sigma$ 的递归和归纳原理到它的递归子:
\symlabel{defn:recursor-sm}%
\[
    \rec{\sm{x:A}B(x)} : \dprd{C:\UU}\Parens{\tprd{x:A} B(x) \to C} \to
    \Parens{\tsm{x:A}B(x)} \to C
\]
与定义等式
\[
    \rec{\sm{x:A}B(x)}(C,g,\tup{a}{b}) \defeq g(a)(b)
\]
和归纳符号: \symlabel{defn:induction-sm}%
\begin{narrowmultline*}
    \ind{\sm{x:A}B(x)} : \narrowbreak
    \dprd{C:(\sm{x:A} B(x)) \to \UU}
    \Parens{\tprd{a:A}{b:B(a)} C(\tup{a}{b})}
    \to \dprd{p : \sm{x:A}B(x)} C(p)
\end{narrowmultline*}
与定义等式
\[
    \ind{\sm{x:A}B(x)}(C,g,\tup{a}{b}) \defeq g(a)(b).
\]
和之前一样, 递归子是归纳子在 $C$ 为常量的特殊情况.

进一步的例子, 考虑到下面的原理, 其中 $A$ 和 $B$ 是类型, 而 $R:A\to B\to \UU$:
\[
    \ac : \Parens{\tprd{x:A} \tsm{y :B} R(x,y)} \to
    \Parens{\tsm{f:A\to B} \tprd{x:A} R(x,f(x))}.
\]
可以把 $R$ 看成 ``证明''$A$ 和 $B$ 之间的``相关关系'' \index{数学!证明相关}, 而 $R(a,b)$ 一个类型, 它的元素是 $a:A$ 和 $b:B$ 的相关性的见证.
然后 $\ac$ 直观地说明了如果有一个依值函数 $g$ 把所有的 $a:A$ 映射为依值对子 $(b,r)$, 而且 $b:B$ 且 $r:R(a,b)$, 那么有一个函数 $f:A\to B$ 和一个依值函数, 把$a:A$ 映射为 $R(a,f(a))$.
直觉上, 只需把 $g$ 的值, 分别表示它们的组成.
的确, 只要使用投影, 就能定义:
\[
    \ac(g) \defeq \Parens{\lamu{x:A} \fst(g(x)),\, \lamu{x:A} \snd(g(x))}.
\]
为了校验它是良型的, 注意如果 $g:\prd{x:A} \sm{y :B} R(x,y)$, 则有
\begin{align*}
    \lamu{x:A} \fst(g(x)) &: A \to B, \\
    \lamu{x:A} \snd(g(x)) &: \tprd{x:A} R(x,\fst(g(x))).
\end{align*}
此外, 类型族 $\lamu{f:A\to B} \tprd{x:A} R(x,f(x))$ 是 \ac 的达到域的成分, 类型 $\prd{x:A} R(x,\fst(g(x)))$ 是应用这个类型族到 $\lamu{x:A} \fst(g(x))$ 的结果:
\[
    \tprd{x:A} R(x,\fst(g(x))) \jdeq
    \Parens{\lamu{f:A\to B} \tprd{x:A} R(x,f(x))}\big(\lamu{x:A} \fst(g(x))\big).
\]
因此, 可以得到所需的
\[
    \Parens{\lamu{x:A} \fst(g(x)),\, \lamu{x:A} \snd(g(x))} : \tsm{f:A\to B} \tprd{x:A} R(x,f(x)).
\]


如果把 $\Pi$ 当作 ``对于所有'' 而把 $\Sigma$ 当作 ``存在'', 那么函数 $\ac$ 的类型表示: \emph{若对于所有 $x:A$ 存在 $y:B$ 满足 $R(x,y)$, 那么存在一个函数 $f : A \to B$ 满足对于所有 $x:A$ 有 $R(x,f(x))$}.
这听起来像是选择公理的一个版本, 函数 \ac 被传统地称为\define{类型论的选择公理}, 而就像刚才看见的, 它可以直接从类型论的规则中证明, 而不是作为公理.
\index{公理!选择的!类型论的}%
不过, 注意实际没有包含选择, 因为早已选择已经被假定: 只需要把它分解成两个函数: 一个代表选择而另一个代表正确性.
在 \cref{sec:axiom-choice} 会给出另一个更接近于使用习惯的 ``选择公理'' 的形式.

经常使用依值对子类型定义一些数学结构的类型, 其结构通常由一些相关的片段组成.
举个简单的例子, 把\define{原群}\indexdef{原群}定义成一个类型, 这个类型 $A$ 伴随一个二元运算符 $m:A\to A\to A$.
短语``伴随''\index{伴随}的含义是 (和同义的 ``具有''\index{具有})的意思是``原群''是一个 \emph{对子} $(A,m)$, 它包含了类型 $A:\UU$和一个操作符 $m:A\to A\to A$.
因为对子的第二个部分 $m$ 的类型 $A\to A\to A$ 依赖于第一个部分 $A$, 所以这样的对子属于依值对子类型.
因此, 定义``原群是一个类型 $A$ 伴随一个二元操作符 $m:A\to A\to A$'' 应该被当作 \emph{原群的类型}是
\[
    \mathsf{Magma} \defeq \sm{A:\UU} (A\to A\to A).
\]
给定一个原群, 通过第一投影 $\proj1$ 抽取基础类型 (它的``载体''\index{载体}), 通过第二投影 $\proj2$ 抽取它的运算符.
当然, 为了构建超过两个数据的类型, 需要迭代的对子类型, 可能只需要部分依赖.
例如具有单位元的原群 (具有一个单位元 $e:A$ 的原群 $(A,m)$) 的类型是
\[
    \mathsf{PointedMagma} \defeq \sm{A:\UU} (A\to A\to A) \times A.
\]
通常也像把公理推广到这样的结构上, 例如把单位元的原群变为一个幺半群或是一个群.
可以通过使用 $\Sigma$-types 来实现;
参见 \cref{sec:pat}.

在本书的剩余部分, 有时会使用这样直接的定义, 但相信读者可以把它从自然语言翻译成 $\Sigma$-类型.
通常也会遵循通常的数学惯例,即对此类结构及其载体使用相同的写法 (相当于在符号中隐含适当的投影函数): 比如讨论一个原群, $A$ 和它的操作符 $m:A\to A\to A$.

注意 $\mathsf{PointedMagma}$ 的标准元素, 是用 $(A,(m,e))$ 中构造的, 这论的 $A:\UU$, $m:A\to A\to A$, 而 $e:A$.
因为 $\Sigma$-类型出现频率很高, 使用有序三元对, 四元对和多元对, 来表示于右边嵌套的两元(依值)对子.
也就是说, $(x,y,z) \defeq (x,(y,z))$, $(x,y,z,w)\defeq (x,(y,(z,w)))$, 等等.

\index{类型!依值对子|)}%