到现在为止, 还在非形式地使用表达式``$A$ 是一个类型'', 现在通过引进\define{宇宙}来更严格地表达它.
\index{类型!宇宙|(defstyle}%
\indexsee{宇宙}{类型, 宇宙}%
宇宙是一个类型, 它的元素也是类型.
在朴素的集合论中, 希望有一个宇宙 $\UU_\infty$, 包含所有类型, 也包含它自己(也就是 $\UU_\infty : \UU_\infty$).
不过, 在集合论中这是不健全的, 例如, 可以由它演绎出所有类型, 包括代表命题是错误 (参见 \cref{sec:coproduct-types}) 的空类型也有居留元.
例如, 把集合作为树, 我们可以直接编写出罗素悖论\index{悖论}\cite{coquand:paradox}.

%%%  or alternatively, in order to avoid the use of
% inductive types to define trees, we can follow Girard \cite{girard:paradox} and encode the Burali-Forti paradox,
% which shows that the collection of all ordinals cannot be an ordinal.
为了避免这个悖论, 引入我们引入了宇宙的等级结构
\indexsee{等级结构!宇宙}{类型, 宇宙}
\[ \UU_0 : \UU_1 : \UU_2 : \cdots \]
每个宇宙 $\UU_i$ 都是下一个等级的宇宙 $\UU_{i+1}$ 的元素.
此外, 我们假设我们的宇宙是\define{累积的}, \indexdef{类型!宇宙!累积} \indexdef{累积!宇宙}, 也就是说 $i^{\mathrm{th}}$ 宇宙的所有元素也同样是 $(i+1)^{\mathrm{st}}$ 宇宙的元素, 例如如果 $A:\UU_i$ 那么 $A:\UU_{i+1}$.
这样是方便的, 但是有一些稍微不令人愉快的结果就是元素不再是唯一的类型, 并且在我们不需要关注的方式上有一些棘手;
参见 Notes.

%%%
当 $A$ 被认为是一个类型时, 意味着它居留于一些 $\UU_i$ 中.
通常避免显式声明等级
\indexdef{宇宙等级}%
\indexsee{等级}{宇宙的等级 或者 $n$-类型}%
\indexsee{类型!宇宙!等级}{宇宙 等级}%
$i$, 而是假定它的等级已经被一致的方式分配;
因此可以忽略它的等级记作 $A:\UU$.
按照这样的方式甚至可以写出 $\UU:\UU$, 它被看作 $\UU_i:\UU_{i+1}$, 隐式地包含了索引.
这种编写宇宙的风格被称为\define{类型歧义}.\indexdef{类型歧义}
它是方便的, 同时有一点危险, 因为允许写看起来合法的证明, 会出现自指的悖论.
如果有参数是否合法的疑问, 检查的方法是, 尝试为所有出现的宇宙分配一致的等级.
当一些宇宙 \UU 被假定, 我们可以把属于 \UU 的类型称为\define{小类型}.
\indexdef{小!类型}%
\indexdef{类型!小}%


%%%%%%
为了给类型的集合建模, 给定类型 $A$, 可以使用函数 $B : A \to \UU$, 它的到达域是一个宇宙.
这些函数被称为
\define{类型族}%
(也叫\emph{依赖类型});
\indexsee{族!类型}{类型, 族}%
\indexdef{类型!族}%
\indexsee{类型!依赖}{类型, 族}%
\indexsee{依赖!类型}{类型, 族}%
在集合论中, 它们相当于集合中的集合族.


%%%%
\symlabel{fin}
一个类型族的例子是, 有限集的族 $\Fin : \nat \to \UU$, 这里的 $\Fin(n)$ 是一个类型, 它恰好有 $n$ 个元素.
(还无法\emph{定义}族 $\Fin$ --- 的确, 还没有介绍它的定义域 $\nat$ --- 但是马上会介绍; 参见 \cref{ex:fin}.)
可以通过 $0_n,1_n,\dots,(n-1)_n$ 表示 $\Fin(n)$, 它的下标强调了如果 $n$ 和 $m$ 不同,  $\Fin(n)$ 的元素与 $\Fin(m)$ 的元素也不同, 而且所有的元素和普通的自然数都不同 (在 \cref{sec:inductive-types} 会引入自然数).
\index{有限!集合, 族}%


%%
一个相对不重要(但也很重要)的类型族的例子是, 类型 $B:\UU$ 上的\define{常量}类型族\indexdef{常量!类型族} \indexdef{类型!的族!常量}, 即常量函数 $(\lam{x:A} B):A\to\UU$.

%%
作为一个\emph{非}示例, 我们这个版本的类型论中, 没有类型族 ``$\lam{i:\nat} \UU_i$''.
实际上, 没有足够大的宇宙可以作为它的到达域.
此外, 甚至没有定义类型论中的自然数 \nat (在 \cref{sec:inductive-types} 后面会引入它), 来标识宇宙 $\UU_i$ 的下标 $i$.

\index{类型!宇宙|)}