\index{类型!函数|(defstyle}%
\indexsee{函数类型}{类型, 函数}%
给定类型 $A$ 和 $B$, 可以构造\define{函数}的类型 $A \to B$,
\index{函数|(defstyle}%
\indexsee{映射}{函数}%
\indexsee{映射}{函数}%
其定义域是 $A$, 到达域是 $B$.
函数有时也被称为\define{映射}.\index{定义域!函数}\index{到达域, 函数}\index{函数!定义域}\index{函数!到达域}\index{函数的关系}
和集合论不同, 函数没有被定义成泛函关系, 而是类型论中的原始概念.
我们通过规定可以对函数所做的事情, 来解释函数类型, 例如: 如何构造函数, 和函数的的等同.

给定函数 $f : A \to B$ 和一个定义域的元素 $a : A$, 可以\define{应用}%
\indexdef{应用!函数}%
\indexdef{函数!应用}%
\indexsee{赋值}{应用, 函数}%
这个函数, 来得到到达域 $B$ 中的一个元素, 记作 $f(a)$, 又被称为 $f$ 在 $a$ 上的\define{值}.
\indexdef{值!函数}%
类型论中, 通常省略圆括号\index{圆括号}, 比如将 $f(a)$ 简化为 $f\,a$, 我们有时也会这样.

但是如何构造 $A \to B$ 的元素?
有两个等价的方法: 直接定义, 或者使用 $\lambda$-抽象.
通过定义引入函数\indexdef{定义!函数, 直接}就是通过给定名称, 比如说 $f$, 然后给出一个等式
\begin{equation}
    \label{eq:expldef} f(x) \defeq \Phi
\end{equation}
来定义函数 $f : A \to B$, 其中 $x$ 是一个变量\index{变量}%
而 $\Phi$ 是一个可能用到 $x$ 的表达式.
为了确保它的正确性, 必须检查假设 $x:A$ 时, 有 $\Phi : B$.

现在可以将 $\Phi$ 中的变量 $x$ 替换为 $a$ 来计算 $f(a)$.
例如, 思考函数 $f : \nat \to \nat$, 被定义为 $f(x) \defeq x+x$.(如何定义 $\nat$ 和 $+$ 的方法参见 \cref{sec:inductive-types}.)
然后 $f(2)$ 判断地等同于 $2+2$.

如果不想介绍函数的名称, 可以使用\define{$\lambda$-抽象}.
\index{lambda 抽象@$\lambda$-抽象|defstyle}%
\indexsee{函数!lambda 抽象@$\lambda$-抽象}{$\lambda$-抽象}%
\indexsee{抽象!lambda-@$\lambda$-}{$\lambda$-抽象}%
给定类型为 $B$ 的表达式 $\Phi$, 和之前一样, 也许会使用到 $x:A$, $\lam{x:A} \Phi$ 表示\eqref{eq:expldef}定义的一样的函数.
因此, 有
\[ (\lamt{x:A}\Phi) : A \to B. \]
对于上一个段落中的例子, 有类型判断
\[ (\lam{x:\nat}x+x) : \nat \to \nat. \]
另一个例子, 对于任意两个类型 $A$ 和 $B$ 与任意的一个元素 $y:B$, 可以构造\define{常量函数}\indexdef{常量!函数}\indexdef{函数!常量}
\[ (\lam{x:A} y): A\to B. \]

我们经常地忽略 $\lambda$-抽象中 $x$ 的类型写作 $\lam{x}\Phi$, 因为从判断 $\lam x \Phi$ 的类型为 $A\to B$, 可以推导 $x:A$.
按照惯例, 变量绑定 ``$\lam{x}$'' 的``作用域'' \indexdef{变量!作用域的} \indexdef{作用域}是表达式的剩余部分, 除非用圆括号界定\index{圆括号}.
例如, $\lam{x} x+x$ 应该被解析成 $\lam{x} (x+x)$, 而不是 $(\lam{x}x)+x$ (这个例子中, 会出现异常类型).

另一个等价的符号是\symlabel{mapsto}
\[ (x \mapsto \Phi) : A \to B. \]
\symlabel{blank}有时也使用空白``$\blank$''代替表达式 $\Phi$ 中的变量, 来表示隐式的 $\lambda$-抽象.
例如, $g(x,\blank)$ 是 $\lam{y} g(x,y)$ 的另一种写法.

现在 $\lambda$-抽象是一个函数, 所以可以将其应用于参数 $a:A$.
然后有如下\define{计算规则}\indexdef{计算规则!函数类型}%
\footnote{这种等同的使用通常被称为 \define{$\beta$-变换} \indexsee{beta-变换@$\beta $-变换}{$\beta$-化简} \indexsee{变换!beta@$\beta $-}{$\beta$-化简} 或 \define{$\beta$-化简}. \index{beta-化简@$\beta $-化简|footstyle} \indexsee{化简!beta@$\beta $-}{$\beta$-化简} },
即一个定义等同:
\[(\lamu{x:A}\Phi)(a) \jdeq \Phi'\]
这里的 $\Phi'$ 是出现在表达式 $\Phi$ 的所有 $x$ 被替换成 $a$ 的结果.
继续上面的例子, 有
\[ (\lamu{x:\nat}x+x)(2) \jdeq 2+2. \]
注意, 对于任何函数 $f:A\to B$, 都可以构造 $\lambda$-抽象函数 $\lam{x} f(x)$.
因为它被定义为``应用 $f$ 于其参数的函数'', 所以认为它定义地等同于 $f$:
\footnote{这种等同的使用通常被称为 \define{$\eta$-变换} \indexsee{eta-变换@$\eta $-变换}{$\eta$-扩展} \indexsee{变换!eta@$\eta $-}{$\eta$-扩展} 或 \define{$\eta$-扩展. \index{eta-扩展@$\eta $-扩展|footstyle}%
\indexsee{扩展, eta-@扩展, $\eta $-}{$\eta$-扩展} }}%
\[ f \jdeq (\lam{x} f(x)). \]
这个等同就是\define{函数类型的唯一性原则}\indexdef{唯一性!原则!函数类型的}, 因为它表示 $f$ 唯一地被它的值确定.

通过显性参数来定义函数的方法可以被 $\lambda$-抽象简化, 例如 $f: A\to B$ 的定义: \[ f(x) \defeq \Phi \] 即 \[ f \defeq \lamu{x:A}\Phi.\]

在作涉及变量的计算时, 必须小心地把变量替换成包含变量的表达式, 因为要保持表达式的绑定结构.
所谓\emph{绑定结构}\indexdef{绑定结构}是一种无形的连接, 它由变量引入和使用之间的绑定符号产生, 例如 $\lambda$, $\Pi$ 和 $\Sigma$ (很快就能看到).
例如, 思考 $f : \nat \to (\nat \to \nat)$, 它被定义为
\[ f(x) \defeq \lamu{y:\nat} x + y. \]
假设 $y : \nat$, 那么 $f(y)$ 是什么?
直接把定义 $f(x)$ 的表达式 ``$\lam{y}x+y$'' 中, 所有 $x$ 替换为 $y$, 从而得到 $\lamu{y:\nat} y + y$ 是错误的, 因为这意味 $y$ 被\define{捕获}了.
\indexdef{捕获, 变量}%
\indexdef{变量!捕获}%
之前, 被替换\index{替换物}的 $y$ 代表假设的元素, 但现在代表 $\lambda$-抽象的参数.
因此直接的替代会破坏绑定结构, 执行语义不可靠的计算.

那么在这个例子中, $f(y)$ 到底\emph{是}什么?
注意这个绑定(或者``虚拟'')变量,
\indexdef{变量!绑定}%
\indexdef{变量!虚拟}%
\indexsee{绑定变量}{变量, 绑定}%
\indexsee{虚拟变量}{变量, 绑定}%
例如表达式 $\lamu{y:\nat} x + y$ 中的 $y$, 仅具有局部意义, 并可以在保持绑定结构的情况下用其它的任何变量替代.
实际上, $\lamu{y:\nat} x + y$ 判断地等同于
\footnote{这种等同的使用通常被称为 \define{$\alpha$-变换. \indexfoot{alpha-变换@$\alpha $-变换}\indexsee{变换!alpha@$\alpha$-}{$\alpha$-变换} }}
$\lamu{z:\nat} x + z$.
因此, $f(y)$ 判断地等同于 $\lamu{z:\nat} y + z$, 这就是答案.
(可以使用任何与 $y$ 不同的变量代替 $z$, 获得同样的结果)

当然, 所有数学家都应该熟悉它:
和实际应用一样, 如果 $f(x) \defeq \int_1^2 \frac{dt}{x-t}$, 那么 $f(t)$ 不是 $\int_1^2 \frac{dt}{t-t}$ 而是 $\int_1^2 \frac{ds}{t-s}$.
$\lambda$-抽象和在积分中绑定变量的方法是完全相同的.

现在已经看到如何以一个变量定义来函数.
之后会介绍一种使用笛卡尔积的方法, 通过多个变量定义函数;
一个接收两个参数 $A$ 和 $B$ 返回结果 $C$ 的函数, 它的类型是$f : A \times B \to C$.
不过, 这里有另一种方法来避免乘积类型, 即\define{柯里化}%
\indexdef{柯里化}%
\indexdef{函数!柯里化}%
(由数学家 Haskell Curry 发明).
\index{程序}%

%%%
柯里化就是把有 $a:A$ 和 $b:B$ 两个输入的函数表示为一个函数, 这个函数接收\emph{一个}输入 $a:A$ 返回\emph{另一个函数}, 返回的函数接收第二个输入, 并返回结果.
也就是说, 二元函数被认为是迭代的函数类型 $f : A \to (B \to C)$.
也可以不带圆括号\index{圆括号}, 记作 $f : A \to B \to C$, 默认是右结合\index{结合性!函数类型}的.
给定 $a : A$ 和 $b : B$, 可以应用 $f$ 于 $a$, 然后应用返回的结果于 $b$, 得到 $f(a)(b) : C$.
为了避免大量的括号, 即使不是乘积类型, 也允许把 $f(a)(b)$ 写作 $f(a,b)$.
当完全省略函数的参数周围的括号时, $(f\,a)\,b$ 记作 $f\,a\,b$, 默认是左结合的, 这样 $f$ 就以正确的顺序被应用于它的参数.

有显式参数定义的符号, 也能扩展到这种情况: 可以定义被命名的函数 $f : A \to B \to C$, 通过给出等式
\[ f(x,y) \defeq \Phi\]
假定 $x:A$ 和 $y:B$, 那么 $\Phi:C$.
这样使用 $\lambda$-抽象\index{lambda 抽象@$\lambda$-抽象}相当于
\[ f \defeq \lamu{x:A}{y:B} \Phi, \]
也可以写作
\[ f \defeq x \mapsto y \mapsto \Phi. \]
还可以通过写入多个空白来隐式地抽象多个变量, 例如
\ $g(\blank,\blank)$ 意味着 $\lam{x}{y} g(x,y)$.
三个或者更多参数的函数柯里化, 只需在刚刚描述的方法上直接扩展.

\index{类型!函数|)} \index{函数|)} 