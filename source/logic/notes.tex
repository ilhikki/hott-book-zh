Voevodsky 最先观察到, 类型论中可以定义集合, 纯命题, 和高阶可缩类型, 而且所有的高阶同伦都会被自动处理, 就像\cref{sec:basics-sets,subsec:hprops,sec:contractibility}中一样.
事实上, 他通过归纳定义了 $n$-类型全部层级, 参见 \cref{cha:hlevels}.\index{n-type@$n$-type!类型论中可定义的}%

\cref{thm:not-dneg,thm:not-lem} 本质上依赖于 Hedberg 的经典理论,
\index{Hedberg 的理论}%
\index{理论!Hedberg 的}%
我们在 \cref{sec:hedberg} 中证明了这个理论.
Mart\'\i n Escard\'o 在 \Agda 的邮件列表中发现到, 命题即类型形式中的 \LEM{} 和单值性之间存在矛盾.
这件事在 \cref{thm:not-dneg} 给出了证明, 它是由 Thierry Coquand 完成的.

命题截断在 1983 年被 Constable~\cite{Con85} 引入到 \index{证明!助手!\NuPRL}, 作为 ``子集'' 和 ``商'' 类型.
本书中的 ``命题截断'' 在 \NuPRL 类型论~\cite{constable+86nuprl-book} 中叫做 ``挤压''.
在~\cite{ab:bracket-types}中, 给出了外延类型论中, 能直接描述命题截断的规则.
Voevodsky 构造了同伦类型论中内延的版本, 使用了非直谓\index{非直谓!截断}的量词, 然后 Lumsdaine 使用高阶归纳类型 (参见 \cref{sec:hittruncations}).

\index{命题的!重调大小}%
Voevodsky~\cite{Universe-poly} 提出了 \cref{subsec:prop-subsets} 中讨论的重调大小规则 .
\index{公理!可约性}\index{重调大小}%
这显然与著名的 Russell 在他和 Whitehead 的 \emph{Principia Mathematica}~\cite{PM2} 书中的\emph{可约性公理}有关.\index{Russell, Bertrand}

副词 ``纯'' 用于指代未截断逻辑时, 用了单子模式在编程语言中用于建模的效果的用法;
参见 \cref{sec:modalities} 和 \cref{cha:hlevels} 的备注.

类型论中有很多方式来处理逻辑关系.
例如, 除了在 \cref{sec:pat} 描述的命题即类型逻辑, 以及 \cref{subsec:logic-hprop} 中仅使用纯命题的替代方案外, 还可以引入一种独立的命题 ``种类'', 它的行为与类型类似, 但和类型不完全相同.
这种方法被用于逻辑扩展类型理论~\cite{aczel2002collection}, 如拓扑斯\index{拓扑斯}及相关范畴的内部语言(例如~\cite{jacobs1999categorical,elephant}), 以及证明助手 \Coq 也有体现.\index{证明!助手!Coq@\textsc{Coq}}
这种方法更为一般化, 但也较为局限.
例如, 在 \Coq~\cite{Spiwack} 的集值范畴, 逻辑扩展类型理论~\cite{aczel2002collection}, 以及最小类型理论y~\cite{maietti2005toward}中, 唯一选择原理 (\cref{sec:unique-choice}) 不成立.
\index{集值}
因此, 单值公理让我们的类型理论更像是一个拓扑斯的内部逻辑;
参见 \cref{cha:set-math}.\index{拓扑斯}

Martin-L\"of~\cite{martin2006100} 提供了一段关于选择公理历史的讨论.
当然, 构造性和直觉主义数学有着悠久且复杂的历史, 在此不作深入探讨;
参见~\cite{TroelstraI,TroelstraII}.
