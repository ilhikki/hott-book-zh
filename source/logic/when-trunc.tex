\index{逻辑!纯命题|(}%
\index{纯命题|(}%
\index{逻辑!截断}%

初看起来, $+$ 和 $\Sigma$ 的截断的版本, 比非截断的版本更接近``或''和``存在''在非正式数学中的含义.
当然, 截断版本更接近一阶逻辑\index{一阶!逻辑}中的``或''和``存在'' 的\emph{准确}含义, 因为一阶逻辑中并不试图记住命题为真的任何证据, 而一阶逻辑是形式集合论的基础.
然而, 令人惊讶的是, \emph{非正式}的数学的实践中, 往往使用非截断版本.

\index{素数}%
例如, 考虑这个陈述 ``素数要么是 $2$ 要么是奇数''.
数学家会在毫无顾忌地使用这一事实, 不仅是证明关于素数的\emph{定理}, 还用来\emph{构造}素数, 在素数是 $2$ 的情况下做一件事, 而在素数是奇数时做另一件事.
构造的结果不仅是某个陈述是对的, 还是一段可能依赖素数奇偶性的数据信息.
因此, 从类型论的视角, 这种构造自然地用余积类型 ``$(p=2)+(p\text{ 是奇数})$'' 表示, 而不是它的命题截断.

诚然, 这不是一个理想的例子, 因为 ``$p=2$'' 和 ``$p$ 是奇数'' 互不相交, 所以 $(p=2)+(p\text{ 是奇数})$ 实际上已经是纯命题, 因此它等于它的截断 (参见 \cref{ex:disjoint-or}).
更令人信服的例子来自于存在两次.
常见的做法是, 先证明一个形如 ``存在 $x$ 满足 \dots'' 的定理, 然后在后面引用它 ``该 $x$ 在定理 Y 中构建'' (注意定冠词的使用).
此外, 当进一步推导 $x$ 的性质时, 可能会使用 ``由证明 Y 时构造的 $x$'' 这样的短语.

一个非常常见的例子是 ``$A$ 同构于 $B$'', 严格意义上这仅代表存在\emph{某些} $A$ 和 $B$ 之间的同构.
但几乎在证明这样的陈述时, 总会展示一个特定的同构, 或是证明已知的映射是同构, 而且给出的这个同态往往很重要.

接受集合论训练的数学家通常对这种``语言滥用''感到一丝愧疚.\index{滥用!语言}
我们可能试图为此道歉, 从最终稿中删除这些表达, 或者用 ``典范'' 这样模糊的词来替代.
% 比如 Show that if $\UU_{i+1}$ satisfies \LEM{}, then the canonical inclusion $\prop_{\UU_i} \to \prop_{\UU_{i+1}}$ is an equivalence.
这个问题在形式化的集合论中更加严重, 因为技术上根本无法 ``构建'' 对象 --- 只能证明某个具有特定属性的对象存在.
因此, 类型论中的未截断逻辑能够更直接地反映非正式数学中的一些常见做法, 而这些做法在集合论的重构中往往被模糊掉了.
(这类似于单值公理如何使识别同构对象的常见做法合法化, 尽管这种做法缺乏正式的证明.)

一方面, 某些时候截断是必要的.
我们已经在 \LEM{} 和 \choice{} 的陈述中看到了这一点;
一些其他的例子会在本书的稍后出现.
因此, 我们面对一个问题: 当编写非正式类型论时, ``或'' 和 ``存在'' 这两个词应该意味什么 (以及 ``有'' 和 ``我们有'' 等常见同义词)?

达成普遍共识好像不太可能.
根据数学分类, 也许一种约定可能比另一种更有用 --- 也有可能无关紧要.
这种情况下, 数学论文开头的备注可能就将其使用的语言约定告知读者.
然而, 尽管选择了总体约定, 其他种类的逻辑也可能偶尔出现, 所以需要一种方法来引用它.
更通俗一些说, 可以考虑把命题截断替换为, 表现类似的其他的操作, 例如双重否定 $A\mapsto \neg\neg A$, 或者 \cref{cha:hlevels} 中出现的 $n$-截断.
作为一种表达的实验, 在接下来的内容中, 偶尔使用\emph{副词}\index{副词}来表示诸如类型截断 ``模态'' 的应用.

例如, 如果非截断逻辑是默认约定, 那么会使用副词\define{仅}%
\indexdef{仅}%
来表示截断命题.
因此短句
\begin{center}
    ``仅存在 $x:A$ 满足 $P(x)$''
\end{center}
表示类型 $\brck{\sm{x:A} P(x)}$.
类似地, 可以说类型 $A$ \define{仅被居留}%
\indexdef{仅!被居留}%
\indexdef{被居留类型!仅}%
来代表命题截断 $\brck A$ 被居留 (例如我们有该类型的未命名的元素).
注意这里对副词``仅''的\emph{定义}\index{定义!副词} 是为了在我们的非正式数学语言中使用, 就像我们定义有特定数学含义的词一样, 比如名词\index{名词} ``群'' 和 ``环'', 形容词\index{形容词} ``正规'' 和 ``正态''.
我们并没没有说字典里的 ``仅'' 的定义指的是命题截断;
选择这个词的意义只是提醒数学读者, 纯命题的包含``仅''一个真值的信息, 而没有其他(``纯命题'' 的英语是 ``mere proposition'', 而副词的英语是 ``merely'').

另一方面, 如果被截断逻辑是默认约定, 会使用副词\define{确实}%
\indexdef{确实}%
或 \define{构建性地} 来表示没有截断, 所以
\begin{center}
    ``确实存在 $x:A$ 满足 $P(x)$''
\end{center}
表示类型 $\sm{x:A} P(x)$.
即使在默认的情况下, 也可以使用 ``确实'' 来强调缺少截断.

本书会继续使用非截断逻辑作为默认约定, 有这些原因.
\begin{enumerate}[label=(\arabic*)]
    \item 我们希望鼓励新手去尝试它, 而不是仅仅因为更熟悉而坚持使用截断逻辑.
    \item 在类型论中使用截断逻辑作为默认会面临与集合论基础相同种类的 ``语言滥用''\index{滥用!语言}问题, 而非截断逻辑不会.
    例如, 定义 ``$\eqv A B$'' 作为 $A$ 和 $B$ 之间的等价类型, 而不是它的命题截断, 这代表证明定理 ``$\eqv A B$'' 字面上是构造一个具体的等价.
    这个具体的等价可以在之后引用
    \item 希望强调 ``纯命题'' 这个概念不是类型论的一部分.
    正如 \cref{cha:hlevels} 中所见的, 纯命题只是无限阶梯的第二层, 而且有许多其他模态根本不在这条阶梯上.
    \item 许多本来是纯命题的经典陈述, 在同伦类型论中, 不再是纯命题了.
    当然, 其中最重要的就是等同.
    \item 一方面, 同伦类型论最有趣的地方是, 很多类型居然\emph{自动地}就是纯命题, 或者只需稍加修改就能称为这样, 而不需要任何截断.
    (参见 \cref{thm:isprop-isprop,cha:equivalences,cha:hlevels,cha:category-theory,cha:set-math}.)
    因此, 尽管这些类型不存在超出命题为真的任何数据, 也可以用它来构造非截断的对象, 因为不需要使用命题截断的归纳原则.
    如果命题截断对于所有的语句都是默认, 会跟臃肿地表达这些有用的事实.
    \item 最后, 截断在本书的大部分数学中并不是很有用, 因此在发生时明确标记它们会更简单.
\end{enumerate}

\index{纯命题|)}%
\index{逻辑!纯命题|)}%
