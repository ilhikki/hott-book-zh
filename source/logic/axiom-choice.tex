\index{公理!选择|(defstyle}%
\index{否定|(}%
现在可以在同伦类型论中正式表述选择公理.
给定一个类型 $X$ 和类型族
%
\begin{equation*}
    A:X\to\type
    \qquad\text{和}\qquad
    P:\prd{x:X} A(x)\to\type,
\end{equation*}
%
以及
\begin{itemize}
    \item $X$ 是一个集合,
    \item 对于所有 $x:X$, $A(x)$ 是一个集合
    \item 对于所有 $x:X$ 和 $a:A(x)$, $P(x,a)$ 是一个纯命题.
\end{itemize}
\define{选择公理}
$\choice{}$ 可以表示为,
\begin{equation}
    \label{eq:ac}
    \Parens{\prd{x:X} \Brck{\sm{a:A(x)} P(x,a)}}
    \to
    \Brck{\sm{g:\prd{x:X} A(x)} \prd{x:X} P(x,g(x))}.
\end{equation}
当然, 这对~\eqref{eq:english-ac} 的直接翻译,  可以将 ``存在 $x:A$ 满足 $B(x)$'' 视作 $\brck{\sm{x:A}B(x)}$, 于是可以将这个命题用熟悉的符号表示
\begin{narrowmultline*}
    \textstyle
    \Big(\fall{x:X}\exis{a:A(x)} P(x,a)\Big)
    \Rightarrow \narrowbreak
    \Big(\exis{g : \prd{x:X} A(x)} \fall{x : X} P(x,g(x))\Big).
\end{narrowmultline*}
%
特别地, 注意命题截断出现了两次.
定义域中出现的截断代表着, 对于每个 $x$ 存在一些 $a:A(x)$ 满足 $P(x,a)$, 但是它们的值没有被任何方式选择或指定.
值域中出现的截断代表着, 可以推断出存在一些函数 $g$, 但是这些函数无法被确定或指定.

实际上, 因为 \cref{thm:ttac}, 这个公理可以被简化表示.

\begin{lem}
    \label{thm:ac-epis-split}
    选择公理~\eqref{eq:ac} 等价于这个语句: 对于任意集合 $X$ 和任意 $Y:X\to\type$ 使得每个 $Y(x)$ 都是集合, 有
    \begin{equation}
        \Parens{\prd{x:X} \Brck{Y(x)}}
        \to
        \Brck{\prd{x:X} Y(x)}.\label{eq:epis-split}
    \end{equation}
\end{lem}

这对应于经典\index{数学的!经典的}选择公理的等价形式, 即 ``非空集合族的笛卡尔积是非空的''.

\begin{proof}
    通过 \cref{thm:ttac}, ~\eqref{eq:ac} 的值域等价于
    \[\Brck{\prd{x:X} \sm{a:A(x)} P(x,a)}.\]
    因此,~\eqref{eq:ac} 等价于~\eqref{eq:epis-split} 的实例, 其中 \narrowequation{Y(x) \defeq \sm{a:A(x)} P(x,a).}
    (根据 \cref{thm:isset-prod,thm:prop-set}, 这是一个集合.)
    反过来,~\eqref{eq:epis-split} 等价于~\eqref{eq:ac} 的实例, 其中 $A(x)\defeq Y(x)$ 而 $P(x,a)\defeq\unit$.
    因此, 两边都逻辑地等价.
    因为它们都是纯命题, 根据 \cref{lem:equiv-iff-hprop} 它们是等价的类型.
\end{proof}

类似于 \LEM{}, ~\eqref{eq:ac} 和~\eqref{eq:epis-split} 的等价形式不是我们的基本类型论中的推论, 但是可以作为公理假设, 且具备一致性.

\begin{rmk}
    很容易证明~\eqref{eq:epis-split} 的右边可以推出左边.
    因为它们都是纯命题, 根据 \cref{lem:equiv-iff-hprop} 选择公理也等价于要求这个等价关系
    \[ \eqv{\Parens{\prd{x:X} \Brck{Y(x)}}}{\Brck{\prd{x:X} Y(x)}} \]
    这说明了一个常见的陷阱: 尽管依赖函数类型保持纯命题 (\cref{thm:isprop-forall}), 但它们与截断不是交换的: $\brck{\prd{x:A} P(x)}$ 通常不等价于 $\prd{x:A} \brck{P(x)}$.
    选择公理, 如果假定了它, 代表\emph{对于集合}是成立的; 正如我们将在下文看到的, 一般是不成立的.
\end{rmk}

对于选择公理的限制, 可以在一定程度上放宽为任意类型.
例如, 在 在~\eqref{eq:ac} 中可以放宽 $A$ 和 $P$, 或放宽 ~\eqref{eq:epis-split} 中的 $Y$, 为任意的类型族;
这会导致一个看似更强的语句, 但同样是一致的.
也可以用\cref{cha:hlevels} 中, 更一般的 $n$-截断替换命题截断, 得到一系列 $\choice n$, 插值于在~\eqref{eq:ac}, 简称 \choice{} (或强调为 $\choice{-1}$ ), 和 \cref{thm:ttac}, 又叫做 $\choice\infty$ 之间.
参见 \cref{ex:acnm,ex:acconn}.
不过, 需要注意的是我们无法放宽 $X$ 是一个集合的要求.

\begin{lem}
    \label{thm:no-higher-ac}
    存在类型 $X$ 和族 $Y:X\to \type$ 满足每个 $Y(x)$ 都是集合, 但是~\eqref{eq:epis-split} 不成立.
\end{lem}
\begin{proof}
    定义 $X\defeq \sm{A:\type} \brck{\bool = A}$, 并令 $x_0 \defeq (\bool, \bproj{\refl{\bool}}) : X$.
    然后通过 $\Sigma$-类型的路径的恒等性, 事实上 $\brck{A=\bool}$ 是纯命题, 以及单值性, 对于任意 $(A,p),(B,q):X$ 有 $\eqv{(\id[X]{(A,p)}{(B,q)})}{(\eqv AB)}$.
    特别的, $\eqv{(\id[X]{x_0}{x_0})}{(\eqv \bool\bool)}$, 所以正如 \cref{thm:type-is-not-a-set}, $X$ 不是集合.

    另一边, 如果 $(A,p):X$, 那么 $A$ 是一个集合;
    可以这样得到: 对截断 $p:\brck{\bool=A}$ 归纳以及 $\bool$ 是一个集合.
    因为 $\eqv A B$ 是一个集合无论 $A$ 和 $B$ 是不是, 这代表 $\id[X]{x_1}{x_2}$ 对于任意 $x_1,x_2:X$ 是集合, 即 $X$ 是 1-类型.
    特别的, 如果通过 $Y(x) \defeq (x_0=x)$ 定义 $Y:X\to\UU$, 那么每个 $Y(x)$ 都是一个集合.

    现在通过定义, 对于任何 $(A,p):X$ 有 $\brck{\bool=A}$, 因此 $\brck{x_0 = (A,p)}$.
    于是, 有 $\prd{x:X} \brck{Y(x)}$.
    如果~\eqref{eq:epis-split} 保留这样的 $X$ 和 $Y$, 那么同样有 $\brck{\prd{x:X} Y(x)}$.
    因为我们正在尝试推导出矛盾 ($\emptyt$), 是一个纯命题, 假设 $\prd{x:X} Y(x)$, 即 $\prd{x:X} (x_0=x)$.
    而 $X$ 是一个纯命题, 因此也是一个集合, 所以这是一个矛盾.
\end{proof}

\index{否定|)}%
\index{公理!选择|)}%
