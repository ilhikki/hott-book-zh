正式或非正式地说, 类型论是规则的汇集, 这些规则用来操作类型和类型的元素.
不过当非正式的讨论数学时, 普遍会用那些熟悉的词语, 特别是逻辑连词比如 ``而且'' 和 ``或'', 以及逻辑量词比如 ``对于所有'' 和 ``存在''.
相比集合论, 类型论提供了超过一种方法来将自然语言作为类型上的运算.
为了解决潜在的歧义, 需要设置局部或者全局的约定, 或者在非形式的数学中引入新的注解, 或者两个都要.
其中一些要求已经实现了, 因为类型论可以更好的分析逻辑, 我们可以更准确地表示数学, 而且 ``语言的滥用'' 比集合论更少.
本章我们会讲解这些相关的缺陷, 并证明我们的选择是正确的.
