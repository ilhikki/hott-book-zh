Type theory, formal or informal, is a collection of rules for manipulating types and their elements.
But when writing mathematics informally in natural language, we generally use familiar words, particularly logical connectives such as ``and'' and ``or'', and logical quantifiers such as ``for all'' and ``there exists''.
In contrast to set theory, type theory offers us more than one way to regard these English phrases as operations on types.
This potential ambiguity needs to be resolved, by setting out local or global conventions, by introducing new annotations to informal mathematics, or both.
This requires some getting used to, but is offset by the fact that because type theory permits this finer analysis of logic, we can represent mathematics more faithfully, with fewer ``abuses of language'' than in set-theoretic foundations.
In this chapter we will explain the issues involved, and justify the choices we have made.
