\index{逻辑!纯命题的|(}%
\index{纯命题(defstyle}%
\indexsee{命题!纯}{纯命题}%
我们已经看到命题作为类型的逻辑的优点和缺点.
它们有这样的共性: 当类型视为命题, 它们存在更多的信息, 不止是对和错, 它们上的 ``逻辑'' 构造都遵循这些额外信息.
这表明我们可以得到更常规的逻辑, 通过只关注\emph{不}包含任何比真值外信息的类型, 并单纯视其为逻辑命题.

比如类型 $A$ 为 ``真'' 当其被居留时, 而 ``假'' 如果它的居留元可以得到矛盾 (即 $\neg A \jdeq (A\to\emptyt)$ 被居留).
\index{被居留的类型}%
为了得到更加传统的逻辑, 要避免将这样的类型视为逻辑命题, 给出它们的元素比知道这个类型被居留提供了更多的信息.
例如, 如果给出 \bool 的元素, 那么会收到根多的信息, 相比 \bool 存在元素这个单纯的事实.
确实, 得到了确切的\emph{一比特}
\index{比特}%
的额外信息: 知道了给出的是 \bool 的\emph{哪个}元素.
相反的, 如果给出 \unit 的元素, 那么不会得到更多的信息, 相比 \unit 存在元素这个单纯的事实, 因为任何两个 \unit 的元素彼此相等.
这启发了这个定义.

\begin{defn}\label{defn:isprop}
  类型 $P$ 是一个\define{纯命题}
  如果对于所有 $x,y:P$ 有 $x=y$.
\end{defn}

主义因为我们还是在类型论\emph{中}做数学, 这个定义\emph{在}类型论\emph{中}, 意味着一个类型 --- 或是一个类型族.
特殊的, 对于任意 $P:\type$, 类型 $\isprop(P)$ 被定义为
\[ \isprop(P) \defeq \prd{x,y:P} (x=y). \]
因此, 假定 ``$P$ 是纯命题'' 意味着展示 $\isprop(P)$ 的一个居留元, 即通过路径连接任意两个 $P$ 的元素的依值函数.
该函数的连续性/自然性暗示了不仅 $P$ 的任意两个元素都相等, 而且 $P$ 也不存在高阶同伦.

\begin{lem}\label{thm:inhabprop-eqvunit}
  如果 $P$ 是纯命题而且 $x_0:P$, 那么 $\eqv P \unit$.
\end{lem}
\begin{proof}
  定义 $f:P\to\unit$ 为 $f(x)\defeq \ttt$, 而 $g:\unit\to P$ 为 $g(u)\defeq x_0$.
  然后使用下面的引理, 其中根据 \cref{thm:path-unit}, \unit 是纯命题.
\end{proof}

\begin{lem}\label{lem:equiv-iff-hprop}
  如果 $P$ 和 $Q$ 是纯命题而 $P\to Q$ 且 $Q\to P$, 那么 $\eqv P Q$.
\end{lem}
\begin{proof}
  提供 $f:P\to Q$ 和 $g:Q\to P$.
  对于任何 $x:P$, 有 $g(f(x))=x$ 因为 $P$ 是纯命题.
  类似地, 对于任何 $y:Q$ 有 $f(g(y))=y$ 因为 $Q$ 是纯命题; 因此 $f$ 准逆于 $g$.
\end{proof}

也就是正如 \cref{sec:pat} 中约定的一样, 如果两个纯命题之间逻辑上等价, 那么它们等价.

同伦论中, 同伦等价于 \unit 的空间是\emph{可缩的}.
因此, 任何被居留的纯命题都是可缩的 (参见 \cref{sec:contractibility}).
另外, 不被居留的类型 \emptyt 也是 (空洞的) 纯命题.
至少在经典\index{数学!经典的}数学中, 只有这两种可能.

纯命题也被称为\emph{子终止对象} (范畴学上), \emph{单元素子集} (集合论上), 或者\emph{h-propositions}.
\indexsee{对象!子终结}{纯命题}%
\indexsee{子终结对象}{纯命题}%
\indexsee{子终结}{纯命题}%
\indexsee{h-proposition}{纯命题}
按照 \cref{sec:basics-sets} 中的讨论, 应该把它们称为 \emph{$(-1)$-类型}; 在 \cref{cha:hlevels} 会返回到它.
形容词 ``纯'' 强调尽管可以将任何类型视为命题 (通过给出居留元来证明它), 作为纯命题的类型, 只有将其视为命题, 否则没有意义: 它为真的见证不包含额外信息.

注意类型 $A$ 是一个集合当且仅当对于所有 $x,y:A$, 恒等类型 $\id[A]xy$ 是纯命题.
另一方面, 通过复制和简化 \cref{thm:isset-is1type} 的证明, 有:

\begin{lem}\label{thm:prop-set}
  每个纯命题都是集合.
\end{lem}
\begin{proof}
  假设 $f:\isprop(A)$; 因此对于所有 $x,y:A$ 有 $f(x,y):x=y$.
  固定 $x:A$ 并定义 $g(y)\defeq f(x,y)$.
  然后对于任何 $y,z:A$ 且 $p:y=z$ 有 $\apd g p : \trans{p}{g(y)}={g(z)}$.
  于是通过 \cref{cor:transport-path-prepost}, 有 $g(y)\ct p = g(z)$, 也就是说 $p=\opp{g(y)}\ct g(z)$.
  因此, 对于所有 $p,q:x=y$, 有 $p = \opp{g(x)}\ct g(y) = q$.
\end{proof}

特别地, 这意味着:

\begin{lem}\label{thm:isprop-isprop}\label{thm:isprop-isset}
  对于任何类型 $A$, 类型 $\isprop(A)$ 和 $\isset(A)$ 都是纯命题.
\end{lem}
\begin{proof}
  假设 $f,g:\isprop(A)$.
  通过函数外延性, 为了展示 $f=g$ 只需展示 $f(x,y)=g(x,y)$ 对于任何 $x,y:A$.
  不过 $f(x,y)$ 和 $g(x,y)$ 都是 $A$ 中的路径, 而它们相等, 因为根据 $f$ 或 $g$, $A$ 是纯命题, 因此根据 \cref{thm:prop-set} 它是集合.
  类似地, 假设 $f,g:\isset(A)$, 也就是说对于所有 $a,b:A$ 和 $p,q:a=b$, 有 $f(a,b,p,q):p=q$ 和 $g(a,b,p,q):p=q$.
  因为 $A$ 是集合 (根据 $f$ 或 $g$), 所以也是 1-阶类型, 于是有 $f(a,b,p,q)=g(a,b,p,q)$;
  因此根据函数外延性 $f=g$.
\end{proof}

之前有另一个例子: \cref{sec:basics-equivalences} 中的判断~\ref{item:be3} 假定对于任何函数 $f$, 类型 $\isequiv (f)$ 应该是纯命题.

\index{逻辑!,纯命题的|)}%
\index{纯命题)}%
