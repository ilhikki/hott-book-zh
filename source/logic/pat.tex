\index{命题!作为类型|(}%
\index{逻辑!构建主义与经典的|(}%
\index{anger|(}%
直到现在, 都在遵循 \cref{sec:pat} 中``命题作为类型''的简单哲学, 也就是自然语言的 ``存在 $x:A$ 满足 $P(x)$'' 可以理解为对应的类型 $\sm{x:A} P(x)$, 而证明一个语句被视为判断某些特殊的元素居留于这个类型.
不过, 也有一些经典数学家不熟悉的 ``逻辑''.
例如, \cref{thm:ttac} 中的这个语句
\index{公理!选择!类型论风格}%
\begin{equation}
    \label{eq:english-ac}
    \parbox{\textwidth-2cm}{``如果对于所有 $x:X$ 存在一个 $a:A(x)$ 满足 $P(x,a)$, 那么存在函数 $g:\prd{x:X} A(x)$ 满足对于所有 $x:X$ 有 $P(x,g(x))$'',}
\end{equation}
这看起来像是经典的\index{数学!经典的}\emph{选择公理}, 这种情况下永远是对的.
这是一个常用的重点, 命题作为类型的逻辑, 不过这也说明了它和经典的逻辑演绎的显著区别, 后者的选择公理不在逻辑上成立, 而是额外的 ``公理''.

另一方面, 还会展示经典的\emph{双重否定}以及\emph{排中律}与单值公理不相容.
\index{单值公理}%

\begin{thm}
    \label{thm:not-dneg}
    \index{双重否定, 律}%
    这是错误的: 对于所有 $A:\UU$ 都有 $\neg(\neg A) \to A$.
\end{thm}
\begin{proof}
    回顾 $\neg A \jdeq (A\to\emptyt)$.
    可以将运算符 $\neg$ 看作 ``\dots 是错误的''.
    因此, 为了证明这个语句, 根据 $f:\prd{A:\UU} (\neg\neg A \to A)$ 构建 \emptyt 的元素既可.

    下面这个证明的理念是观察 $f$, 与类型论中任何的函数一样, 都是 ``连续的''.
    \index{类型论中函数的连续性@类型论中函数的``连续性''}%
    \index{类型论中函数的函子性@类型论中函数的``函子性''}%
    根据单值, 这意味 $f$ \emph{天然}地遵循类型的等价.
    由此以及一个 fixed-point-free autoequivalence\index{automorphism!fixed-point-free}, 可以提取一个结论.

    定义等价 $e:\eqv\bool\bool$, 其中 $e(\btrue)\defeq\bfalse$ 而 $e(\bfalse)\defeq\btrue$, 就像 \cref{thm:type-is-not-a-set}.
    令 $p:\bool=\bool$ 为 $e$ 由单值对应的路径, 即 $p\defeq \ua(e)$.
    然后有 $f(\bool) : \neg\neg\bool \to\bool$ 和
    \[\apd f p : \transfib{A\mapsto (\neg\neg A \to A)}{p}{f(\bool)} = f(\bool).\]
    因此, 对于任何 $u:\neg\neg\bool$, 有
    \[\happly(\apd f p,u) : \transfib{A\mapsto (\neg\neg A \to A)}{p}{f(\bool)}(u) = f(\bool)(u).\]

    现在通过~\eqref{eq:transport-arrow}, 沿着 $p$ 在类型族 ${A\mapsto (\neg\neg A \to A)}$ 中运输 $f(\bool):\neg\neg\bool\to\bool$, 它等于一个函数, 这个函数沿着 $\opp p$ 在类型族 $A\mapsto \neg\neg A$ 中运输, 应用 $f(\bool)$ 后, 沿着 $p$ 在类型族 $A\mapsto A$ 运输:
    \begin{narrowmultline*}
        \transfib{A\mapsto (\neg\neg A \to A)}{p}{f(\bool)}(u) =
        \narrowbreak
        \transfib{A\mapsto A}{p}{f(\bool) (\transfib{A\mapsto \neg\neg
        A}{\opp{p}}{u})}.
    \end{narrowmultline*}
    %
    而任意两个点 $u,v:\neg\neg\bool$ 按照函数外延性都相等, 应为对于任何 $x:\neg\bool$ 有 $u(x):\emptyt$ 因此可以得到任意结论, 比如 $u(x)=v(x)$.
    因此, 有 $\transfib{A\mapsto \neg\neg A}{\opp{p}}{u} = u$, 陨石从 $\happly(\apd f p,u)$ 得到一个等同
    \[ \transfib{A\mapsto A}{p}{f(\bool)(u)} = f(\bool)(u).\]
    最后, 就像在 \cref{sec:compute-universe} 中讨论的, 沿着路径 $p\jdeq \ua(e)$ 运输类型族 $A\mapsto A$ 的结果等价于应用这个等价 $e$;
    因此有
    \begin{equation}
        e(f(\bool)(u)) = f(\bool)(u).\label{eq:fpaut}
    \end{equation}
    %
    也可以证明
    \begin{equation}
        \prd{x:\bool} \neg(e(x)=x).\label{eq:fpfaut}
    \end{equation}
    这可以通过在 $x$ 上分析情况得到: 每个情况都直接代入 $e$ 的定义而实际上 $\bfalse\neq\btrue$ (\cref{rmk:true-neq-false}).
    因此, 应用~\eqref{eq:fpfaut} 于 $f(\bool)(u)$ 和~\eqref{eq:fpaut}, 可以得到 $\emptyt$ 的元素.
\end{proof}

\begin{rmk}
    \index{选择运算符}%
    \indexsee{运算符!选择}{选择运算符}%
    实际上, 这意味着没有 Hilbert 风格的 ``选择运算符'', 即选择每个非空类型的一个元素.
    关键在于没有\emph{天然的}这种运算, 在单值公理下, 作用于类型的所有函数必须是天然地遵循等价性.
\end{rmk}

\begin{rmk}
    不过, 对于任何 $A$ 还是有 $\neg\neg\neg A \to \neg A$; 参见 \cref{ex:neg-ldn}.
\end{rmk}

\begin{cor}
    \label{thm:not-lem}
    \index{排中}%
    这是错误的: 对于所有 $A:\UU$ 有 $A+(\neg A)$.
\end{cor}
\begin{proof}
    给定 $g:\prd{A:\UU} (A+(\neg A))$.
    只要能得到 $\prd{A:\UU} (\neg\neg A \to A)$, 就能应用 \cref{thm:not-dneg}.
    因此, 给定 $A:\UU$ 和 $u:\neg\neg A$;
    要构造 $A$ 的元素.

    现在 $g(A):A+(\neg A)$, 于是根据模式匹配, 只有 $g(A)\jdeq \inl(a)$ 其中 $a:A$, 或者 $g(A)\jdeq \inr(w)$ 其中 $w:\neg A$ 两种情况.
    第一种情况, 有 $a:A$, 而第二种情况有 $u(w):\emptyt$ 于是可以得到任何想要的 (比如 $A$).
    因此, 每种情况都能得到所需的 $A$ 的元素.
\end{proof}

因此, 如果像假定单值公理 (当然, 我们会这样), 仍然可以选择(所需的)经典\index{数学!经典的}定理, 不能使用未改动的命题作为类型原则来诠释 \emph{所有} 非形式的数学语句到类型轮中, 因为排中律可能是错的.
不过, 我们也不想完全地放弃命题作为类型, 因为它有很多好的性质 (比如简单, 可构造, 可计算).
接下来会讨论一个命题作为类型的改进, 可以解决这个问题;
在 \cref{subsec:when-trunc} 会回到什么时候使用什么逻辑的问题.
\index{anger|)}%
\index{命题!作为类型|)}%
\index{逻辑!构造主义与经典的|)}%
