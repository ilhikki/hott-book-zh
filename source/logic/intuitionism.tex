\index{逻辑!构造性与经典|(}%
\index{否定|(}%
掌握了纯命题的概念, 现在可以给同伦类型论中给出\define{排中律}的正确公式:
\indexdef{排中}%
\indexsee{公理!排中}{排中}%
\indexsee{法则!排中律}{排中法则}%
\begin{equation}
    \label{eq:lem}
    \LEM{}\;\defeq\;
    \prd{A:\UU} \Big(\isprop(A) \to (A + \neg A)\Big).
\end{equation}
类似地, \define{双重否定法则}%
\indexdef{双重否定, 法则}%
\indexdef{公理!双重否定}%
\indexdef{法则!双重否定}%
是
\begin{equation}
    \label{eq:ldn}
    % \mathsf{DN}\;\defeq\;
    \prd{A:\UU} \Big(\isprop(A) \to (\neg\neg A \to A)\Big).
\end{equation}
很容易发现它们两者之间等价---参见 \cref{ex:lem-ldn}---所以从现在开始统称为 \LEM{}.

公式 \LEM{} 避免了\cref{thm:not-dneg,thm:not-lem}的``悖论'', 因为 \bool 不是一个纯命题.
为了将其从命题作为类型公式区分它, 重新命名为:
\symlabel{lem-infty}
\begin{equation*}
    \LEM\infty \defeq \prd{A:\UU} (A + \neg A).
\end{equation*}
强调, 正确的版本~\eqref{eq:lem}
被记作 $\LEM{-1}$;
参见 \cref{ex:lemnm}.
尽管 $\LEM{}$ 不是在 \cref{cha:typetheory}讨论的基本类型论的推论, 它也可以一致地作为一个公理被假设 (不像它的 $\infty$-版本).
例如, 在 \cref{sec:wellorderings} 中会假设它.

不过, 意外的是不使用 \LEM{}, 我们能走多远.
很常见的是, 简单地重新将定义和定理重新表示可以避免使用排中律.
尽管需要一些时间来适应, 通常是值得的, 从而得到更优雅和通用的证明.
导论中讨论了它的优势.

例如, 经典\index{数学!经典}数学中, 不必要的双重否定被频繁使用.
一个非常简单的例子是假设普通集合 $A$ ``非空'', 字面意思就是, $A$ \emph{无法}\emph{没有}元素.
其实就是 $A$  存在至少一个元素, 通过移除双重否懂, 让语句不再依赖 \LEM{}.
回顾之前说 $A$ \emph{被居留}%
\index{被居留类型}%
当假定 $A$ 是命题 (即构造 $A$ 的元素, 通常无需命名).
因此, 通常转换经典的证明为构造主义逻辑, 替换 ``非空'' 为 ``被巨留'' (尽管有时必须替换为 ``被纯居留''; 参见 \cref{subsec:prop-trunc}).

类似地, 经典数学里通常不会根据矛盾做非必要的证明.
\index{证明!根据矛盾}%
当然, 经典的通过矛盾证明的方法是使用双重否定: 假设 $\neg A$ 可以推导出一个矛盾, 就能推导出 $\neg \neg A$, 并通过双重否定得到 $A$.
不过, 从 $\neg A$ 派生的矛盾通常可以改成直接产生 $A$ 的证明, 不需要使用 \LEM{}.

值得注意的是, 如果是为了证明 \emph{否定}\index{否定}, 那么 ``根据矛盾证明'' 不会调用 \LEM{}.
实际上, 因为 $\neg A$ 被定义为类型 $A\to\emptyt$, 根据这个定义, 证明 $\neg A$ 就是假设 $A$ 成立, 然后证明矛盾 (\emptyt).
类似地, 双重否定法则对于否定命题也成立: $\neg\neg\neg A \to \neg A$.
通过实践, 可以学习如何更小心的区别否定命题和非否定命题, 并注意何时该包含 \LEM{} 而何时不该.

因此, 和表面相反, ``可构造性'' 的数学 通常不会包含不会抛弃重要的定理, 而是找到最好的方式来这个定义, 让这个重要的定理构造式的证明.
也就是, 可以在初次研究一个主题是使用 \LEM{}, 不过一旦有更好的解释, 就希望能改进这个证明来避免这个公理.
% For instance, the theory of ordinal numbers, which classically makes heavy use of \LEM{}, works quite well constructively once we choose the correct definition of ``ordinal''; see \cref{sec:ordinals}.
这种观点在\emph{同伦}类型论中更加明显, 其中强大的单值和高阶归纳类型让我们能够构造式地处理很多传统上需要经典推理的问题\index{数学!经典的}.
在 \cref{part:mathematics} 中有一些例子.
% For instance, none of the ``synthetic'' homotopy theory we will develop in \cref{cha:homotopy} requires \LEM{} --- despite the fact that classical homotopy theory (formulated using topological spaces or simplicial sets) makes heavy use of them (as well as the axiom of choice).

同样值得注意的是, 构造主义数学中, 排中律对于\emph{某些}命题是成立的.
它们在传统上被命名为\emph{可判定}.

\begin{defn}
    \label{defn:decidable-equality}
    \mbox{}
    \begin{enumerate}
        \item 类型 $A$ 被称为 \define{可判定}
        \indexdef{可判定!类型}%
        \indexdef{类型!可判定}%
        如果 $A+\neg A$.
        \item 类似地, 类型族 $B:A\to \type$ \define{可判定}
        \indexdef{可判定!类型族}%
        \indexdef{类型!族!可判定}%
        如果 \narrowequation{\prd{a:A} (B(a)+\neg B(a)).} \label{item:decidable-equality2}
        \item 特别地, $A$ \define{可判定等同}
        \indexdef{可判定!等同}%
        \indexsee{等同!可判定}{可判定等同}%
        如果 \narrowequation{\prd{a,b:A} ((a=b) + \neg(a=b)).}
    \end{enumerate}
\end{defn}

因此, $\LEM{}$ 本质就是所有的纯命题可判定, 以及所有纯命题的类型族.
特别地, $\LEM{}$ 暗示了所有集合 (从 \cref{sec:basics-sets} 的意义上说) 都可判定等同.
在这个意义上, 具备可判定等同是非常强大的; 参见 \cref{thm:hedberg}.

\index{拒绝|)}%
\index{逻辑!构造主义和经典的|)}%
