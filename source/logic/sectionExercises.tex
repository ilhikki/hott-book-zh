\begin{ex}
    \label{ex:equiv-functor-set}
    证明 $\eqv A B$ 且 $A$ 是集合, 那么 $B$ 也是集合.
\end{ex}

\begin{ex}
    \label{ex:isset-coprod}
    证明如果 $A$ 和 $B$ 都是集合, 那么 $A+B$ 也是集合.
\end{ex}

\begin{ex}
    \label{ex:isset-sigma}
    证明如果 $A$ 是集合, $B:A\to \type$ 是类型族, 而且对于所有 $x:A$, $B(x)$ 是集合, 那么 $\sm{x:A} B(x)$ 是集合.
\end{ex}

\begin{ex}
    \label{ex:prop-endocontr}
    证明 $A$ 是纯命题, 当且仅当 $A\to A$ 可缩.
\end{ex}

\begin{ex}
    \label{ex:prop-inhabcontr}
    证明 $\eqv{\isprop(A)}{(A\to\iscontr(A))}$.
\end{ex}

\begin{ex}
    \label{ex:lem-mereprop}
    证明如果 $A$ 是纯命题, 那么 $A+(\neg A)$ 也是纯命题.
    因此, 在 \eqref{eq:lem} 中不需要插入命题截断.
\end{ex}

\begin{ex}
    \label{ex:disjoint-or}
    更概括地, 证明 $A$ 和 $B$ 是纯命题而且 $\neg(A\times B)$, 那么 $A+B$ 也是纯命题.
\end{ex}

% \begin{ex}\label{ex:hprop-iff-equiv}
%   Show that if $A$ and $B$ are mere propositions such that $A\to B$ and $B\to A$, then $\eqv A B$.
% \end{ex}

% \begin{ex}\label{ex:isprop-isprop}
%   Show that for any type $A$, the types $\isprop(A)$ and $\isset(A)$ are mere propositions.
% \end{ex}

% \begin{ex}\label{ex:prop-eqvtrunc}
%   Show that if $A$ is already a mere proposition, then $\eqv A{\brck{A}}$.
% \end{ex}

\begin{ex}
    \label{ex:brck-qinv}
    假定一些类型 $\isequiv(f)$ 满足 \cref{sec:basics-equivalences} 中的条件 \ref{item:be1}--\ref{item:be3}, 证明类型 $\brck{\qinv(f)}$ 同样的条件而且等价于 $\isequiv(f)$.
\end{ex}

\begin{ex}
    \label{ex:lem-impl-prop-equiv-bool}
    证明如果 \LEM{} 成立, 那么类型 $\prop \defeq \sm{A:\type} \isprop(A)$ 等价于 \bool.
\end{ex}

\begin{ex}
    \label{ex:lem-impred}
    证明 $\UU_{i+1}$ 满足 \LEM{}, 典范包含 $\prop_{\UU_i} \to \prop_{\UU_{i+1}}$ 是一个等价.
\end{ex}

\begin{ex}
    \label{ex:not-brck-A-impl-A}
    证明并非对于所有类型 $A:\type$ 有 $\brck{A} \to A$.
    (然而, 纯在特定的类型有 $\brck{A}\to A$.
    \cref{ex:brck-qinv} 表明 $\qinv(f)$ 就是这样的类型.)
\end{ex}

\begin{ex}
    \label{ex:lem-impl-simple-ac}
    \index{公理!选择}%
    注意如果 \LEM{} 成立, 那么对于所有 $A:\type$ 有 $\bbrck{(\brck A \to A)}$.
    (从选择公理很容易得到这个性质, 没有 \LEM{} 的情况可能失败;
    参见 \cite{krausgeneralizations}.)
\end{ex}

\begin{ex}
    \label{ex:naive-lem-impl-ac}
    在 \cref{thm:not-lem} 下面的 \LEM{} 的朴素形式和单值是不一致的:
    \[ \prd{A:\type} (A+(\neg A)) \]
    没有单值的情况, 这个公理是一致的.
    然而, 证明这个公理暗含了 \eqref{eq:ac}.
\end{ex}

\begin{ex}
    \label{ex:lem-brck}
    假定 \LEM{} 成立, 证明双重否定 $\neg \neg A$ 有和命题截断 $\brck A$ 相同的递归原则, 但是使用的是命题计算规则而不是判断计算规则.
    换句话说, 假设 \LEM{} 成立, 证明如果 $B$ 是纯命题而且有 $f:A\to B$, 那么有一个诱导的 $g:\neg\neg A \to B$ 满足 $g(\bproj a) = f(a)$ 对于所有 $a:A$.
    推导 (假设 \LEM{} 成立) $\eqv{\neg\neg A}{\brck{A}}$.
    因此, 在 \LEM{} 成立的情况, 命题截断可以被定义, 而不是作为一个单独的类型构造器.
\end{ex}

\begin{ex}
    \label{ex:impred-brck}
    \index{命题的!重调大小}%
    证明如果假设命题重调, 就像 \cref{subsec:prop-subsets} 中的一样, 那么类型
    \[\prd{P:\prop} \Parens{(A\to P)\to P}\]
    有相同的归纳命题 $\brck A$, \emph{以及}相同的判断计算规则.
    因此, 也可以这样定义命题截断.
\end{ex}

\begin{ex}
    \label{ex:lem-impl-dn-commutes}
    假设 \LEM{} 成立, 证明双重否定交换于集合上的纯命题的全称量词.
    也就是说, 证明如果 $X$ 是集合而且每个 $Y(x)$ 是纯命题, 那么 \LEM{} 意味
    \begin{equation}
        \eqv{\Parens{\prd{x:X} \neg\neg Y(x)}}{\Parens{\neg\neg \prd{x:X} Y(x)}}.\label{eq:dnshift}
    \end{equation}
    观察到如果假设每个 $Y(x)$ 是集合, 那么 \eqref{eq:dnshift} 等价于选择公理 \eqref{eq:epis-split}.
\end{ex}

\begin{ex}
    \label{ex:prop-trunc-ind}
    \index{induction principle!for truncation}%
    证明 \cref{subsec:prop-trunc} 给定的命题截断规则可以推断出下面的归纳原理:
    对于任何类型族 $B:\brck A \to \type$, 其中 $B(x)$ 是纯命题, 如果对于每个 $a:A$ 有 $B(\bproj a)$, 那么对于每个 $x:\brck A$ 有 $B(x)$.
\end{ex}

\begin{ex}
    \label{ex:lem-ldn}
    证明排中律\eqref{eq:lem}和双重否定\eqref{eq:ldn}是逻辑等价的.
\end{ex}

\begin{ex}
    \label{ex:decidable-choice}
    假设 $P:\nat\to\type$ 是纯命题的一个可判定的族 (参见 \cref{defn:decidable-equality}\ref{item:decidable-equality2}).
    证明
    \[ \Brck{\sm{n:\nat} P(n)} \;\to\; \sm{n:\nat}P(n).\]
\end{ex}

\begin{ex}
    \label{ex:omit-contr2}
    证明 \cref{thm:omit-contr}\ref{item:omitcontr2}: 如果 $A$ 是可错的, 而且中心是 $a$, 那么 $\sm{x:A} P(x)$ 等价于 $P(a)$.
\end{ex}

\begin{ex}
    \label{ex:isprop-equiv-equiv-bracket}
    证明 $\isprop(P) \simeq (P \simeq \brck P)$.
\end{ex}

\begin{ex}
    \label{ex:finite-choice}
    在经典集合论中, 选择公理的有限版本是一个定理. 证明选择公理 \eqref{eq:ac} 成立, 当 $X$ 是有限类型 $\Fin(n)$ (定义于 \cref{ex:fin}).
\end{ex}

\begin{ex}
    \label{ex:decidable-choice-strong}
    如果 $P:\nat\to\type$ 是可判定的族, 证明 \cref{ex:decidable-choice} 的结论为真.
\end{ex}

% Local Variables:
% TeX-master: "hott-online"
% End: