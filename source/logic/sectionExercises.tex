\begin{ex}\label{ex:equiv-functor-set}
  Prove that if $\eqv A B$ and $A$ is a set, then so is $B$.
\end{ex}

\begin{ex}\label{ex:isset-coprod}
  Prove that if $A$ and $B$ are sets, then so is $A+B$.
\end{ex}

\begin{ex}\label{ex:isset-sigma}
  Prove that if $A$ is a set and $B:A\to \type$ is a type family such that $B(x)$ is a set for all $x:A$, then $\sm{x:A} B(x)$ is a set.
\end{ex}

\begin{ex}\label{ex:prop-endocontr}
  Show that $A$ is a mere proposition if and only if $A\to A$ is contractible.
\end{ex}

\begin{ex}\label{ex:prop-inhabcontr}
  Show that $\eqv{\isprop(A)}{(A\to\iscontr(A))}$.
\end{ex}

\begin{ex}\label{ex:lem-mereprop}
  Show that if $A$ is a mere proposition, then so is $A+(\neg A)$.
  Thus, there is no need to insert a propositional truncation in~\eqref{eq:lem}.
\end{ex}

\begin{ex}\label{ex:disjoint-or}
  More generally, show that if $A$ and $B$ are mere propositions and $\neg(A\times B)$, then $A+B$ is also a mere proposition.
\end{ex}

% \begin{ex}\label{ex:hprop-iff-equiv}
%   Show that if $A$ and $B$ are mere propositions such that $A\to B$ and $B\to A$, then $\eqv A B$.
% \end{ex}

% \begin{ex}\label{ex:isprop-isprop}
%   Show that for any type $A$, the types $\isprop(A)$ and $\isset(A)$ are mere propositions.
% \end{ex}

% \begin{ex}\label{ex:prop-eqvtrunc}
%   Show that if $A$ is already a mere proposition, then $\eqv A{\brck{A}}$.
% \end{ex}

\begin{ex}\label{ex:brck-qinv}
  Assuming that some type $\isequiv(f)$ satisfies conditions~\ref{item:be1}--\ref{item:be3} of \cref{sec:basics-equivalences}, show that the type $\brck{\qinv(f)}$ satisfies the same conditions and is equivalent to $\isequiv(f)$.
\end{ex}

\begin{ex}\label{ex:lem-impl-prop-equiv-bool}
  Show that if \LEM{} holds, then the type $\prop \defeq \sm{A:\type} \isprop(A)$ is equivalent to \bool.
\end{ex}

\begin{ex}\label{ex:lem-impred}
  Show that if $\UU_{i+1}$ satisfies \LEM{}, then the canonical inclusion $\prop_{\UU_i} \to \prop_{\UU_{i+1}}$ is an equivalence.
\end{ex}

\begin{ex}\label{ex:not-brck-A-impl-A}
  Show that it is not the case that for all $A:\type$ we have $\brck{A} \to A$.
  (However, there can be particular types for which $\brck{A}\to A$.
  \cref{ex:brck-qinv} implies that $\qinv(f)$ is such.)
\end{ex}

\begin{ex}\label{ex:lem-impl-simple-ac}
  \index{axiom!of choice}%
  Show that if \LEM{} holds, then for all $A:\type$ we have $\bbrck{(\brck A \to A)}$.
  (This property is a very simple form of the axiom of choice, which can fail in the absence of \LEM{}; see~\cite{krausgeneralizations}.)
\end{ex}

\begin{ex}\label{ex:naive-lem-impl-ac}
  We showed in \cref{thm:not-lem} that the following naive form of \LEM{} is inconsistent with univalence:
  \[ \prd{A:\type} (A+(\neg A)) \]
  In the absence of univalence, this axiom is consistent.
  However, show that it implies the axiom of choice~\eqref{eq:ac}.
\end{ex}

\begin{ex}\label{ex:lem-brck}
  Show that assuming \LEM{}, the double negation $\neg \neg A$ has the same recursion principle as the propositional truncation $\brck A$ but with a propositional computation rule rather than a judgmental one.
  In other words, prove that assuming \LEM{}, if $B$ is a mere proposition and we have $f:A\to B$, then there is an induced $g:\neg\neg A \to B$ such that $g(\bproj a) = f(a)$ for all $a:A$.
  Deduce that (assuming \LEM{}) we have $\eqv{\neg\neg A}{\brck{A}}$.
  Thus, under \LEM{}, the propositional truncation can be defined rather than taken as a separate type former.
\end{ex}

\begin{ex}\label{ex:impred-brck}
  \index{propositional!resizing}%
  Show that if we assume propositional resizing as in \cref{subsec:prop-subsets}, then the type
  \[\prd{P:\prop} \Parens{(A\to P)\to P}\]
  has the same recursion principle as $\brck A$, \emph{with} the same judgmental computation rule.
  Thus, we can also define the propositional truncation in this case.
\end{ex}

\begin{ex}\label{ex:lem-impl-dn-commutes}
  Assuming \LEM{}, show that double negation commutes with universal quantification of mere propositions over sets.
  That is, show that if $X$ is a set and each $Y(x)$ is a mere proposition, then \LEM{} implies
  \begin{equation}
    \eqv{\Parens{\prd{x:X} \neg\neg Y(x)}}{\Parens{\neg\neg \prd{x:X} Y(x)}}.\label{eq:dnshift}
  \end{equation}
  Observe that if we assume instead that each $Y(x)$ is a set, then~\eqref{eq:dnshift} becomes equivalent to the axiom of choice~\eqref{eq:epis-split}.
\end{ex}

\begin{ex}\label{ex:prop-trunc-ind}
  \index{induction principle!for truncation}%
  Show that the rules for the propositional truncation given in \cref{subsec:prop-trunc} are sufficient to imply the following induction principle: for any type family $B:\brck A \to \type$ such that each $B(x)$ is a mere proposition, if for every $a:A$ we have $B(\bproj a)$, then for every $x:\brck A$ we have $B(x)$.
\end{ex}

\begin{ex}\label{ex:lem-ldn}
  Show that the law of excluded middle~\eqref{eq:lem} and the law of double negation~\eqref{eq:ldn} are logically equivalent.
\end{ex}

\begin{ex}\label{ex:decidable-choice}
  Suppose $P:\nat\to\type$ is a decidable family (see \cref{defn:decidable-equality}\ref{item:decidable-equality2}) of mere propositions.
  Prove that
  \[ \Brck{\sm{n:\nat} P(n)} \;\to\; \sm{n:\nat}P(n).\]
\end{ex}

\begin{ex}\label{ex:omit-contr2}
  Prove \cref{thm:omit-contr}\ref{item:omitcontr2}: if $A$ is contractible with center $a$, then $\sm{x:A} P(x)$ is equivalent to $P(a)$.
\end{ex}

\begin{ex}\label{ex:isprop-equiv-equiv-bracket}
  Prove that $\isprop(P) \simeq (P \simeq \brck P)$.
\end{ex}

\begin{ex}\label{ex:finite-choice}
  As in classical set theory, the finite version of the axiom of choice is a theorem.  Prove that the axiom of choice \eqref{eq:ac} holds when $X$ is a finite type $\Fin(n)$ (as defined in \cref{ex:fin}).
\end{ex}

\begin{ex}\label{ex:decidable-choice-strong}
  Show that the conclusion of \cref{ex:decidable-choice} is true if $P:\nat\to\type$ is any decidable family.
\end{ex}

% Local Variables:
% TeX-master: "hott-online"
% End: