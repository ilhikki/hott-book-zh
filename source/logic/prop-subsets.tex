\index{纯命题(}%

作为纯命题的另一个有用的例子, 我们来讨论子集合 (并推广到子类型).
假设 $P:A\to\type$ 是类型族, 每个类型 $P(x)$ 被视为一个命题.
然后 $P$ 本身就是 $A$ 上的一个\emph{断言}, 或者 $A$ 的元素的一个 \emph{性质} .

集合论中, 只要在集合 $A$ 上有断言 $P$ 时, 就可以构造子集合 $\setof{x\in A | P(x)}$.
正如 \cref{sec:pat} 中提起的一样, 类型论中对应的就是 $\Sigma$-类型 $\sm{x:A} P(x)$.
$\sm{x:A} P(x)$ 的居留元显然就是 $(x,p)$, 其中 $x:A$ 而 $p$ 是 $P(x)$ 的一个证明.
不过, 对于更通用的 $P$, 元素 $a:A$ 可以可以对应 $\sm{x:A} P(x)$ 的多个元素, 如果命题 $P(a)$ 有超过一个证明.
这与常规的子集合直觉相反.
不过 $P$ 是\emph{纯}命题, 这将不会发生.

\begin{lem}
    \label{thm:path-subset}
    假设 $P:A\to\type$ 是类型族, 那么对于所有 $x:A$, $P(x)$ 是纯命题.
    如果 $u,v:\sm{x:A} P(x)$ 满足 $\proj1(u) = \proj1(v)$, 那么 $u=v$.
\end{lem}
\begin{proof}
    假设 $p:\proj1(u) = \proj1(v)$.
    根据 \cref{thm:path-sigma}, 为了得到 $u=v$, 展示 $\trans{p}{\proj2(u)} = \proj2(v)$ 即可.
    而 $\trans{p}{\proj2(u)}$ 和 $\proj2(v)$ 都是 $P(\proj1(v))$ 的元素, 而它是纯命题; 因此它们相等.
\end{proof}

例如, 回顾 \cref{sec:basics-equivalences}, 定义了
\[(\eqv A B) \;\defeq\; \sm{f:A\to B} \isequiv (f),\]
其中每个类型 $\isequiv (f)$ 都被假设为了纯命题.
因此如果两个等价内部的函数都相等, 那么这两个等价彼此也相等.

\label{defn:setof}%
现在开始, 如果 $P:A\to \type$ 是纯命题的类型族 (即每个 $P(x)$ 都是纯命题), 可以将
%
\begin{equation}
    \label{eq:subset}
    \setof{x:A | P(x)}
\end{equation}
%
用另外的符号 $\sm{x:A} P(x)$ 代替.
(技术上没有理由限制对 $P$ 使用这个符号, 不过这种无意识的做法可能会带来困扰.)
如果 $A$ 是集合, 可以将\eqref{eq:subset}称为 $A$ 的\define{子集合};
\indexdef{子集合}%
\indexdef{类型!子集合}%
作为推广可以把 $A$ 称为\define{子类型}.
\indexdef{子类型}%
也可以将 $P$ 称为 $A$ 的\emph{子集合}或\emph{子类型};
实际上,这更正确,因为孤立地看,类型~\eqref{eq:subset}不记得它和 $A$ 的关系.

\symlabel{membership}
给定 $P$ 和 $a:A$, 可以写作 $a\in P$ 或 $a\in \setof{x:A | P(x)}$ 来表示纯命题 $P(a)$.
如果它成立, 那可以说 $a$ 是 $P$ 的\define{成员}.
\symlabel{subset}
类似地, 如果 $\setof{x:A | Q(x)}$ 是 $A$ 的另一个集合, 且 $\prd{x:A}(P(x)\rightarrow Q(x))$, 那么可以说 $P$ \define{包含于}%
\indexdef{包含关系!子集合}%
$Q$ 中, 并写作 $P\subseteq Q$.

进一步地, 可以在全集 \UU 中, 定义集合与纯命题的 ``子全集'':
\symlabel{setU}\symlabel{propU}
\begin{align*}
    \setU &\defeq \setof{A:\UU | \isset(A) },\\
    \propU &\defeq \setof{A:\UU | \isprop(A) }.
\end{align*}
$\setU$ 的元素是类型 $A:\UU$ 以及证明 $s:\isset(A)$, 而 $\propU$ 也类似.
\cref{thm:path-subset} 中暗示了 $\id[\setU]{(A,s)}{(B,t)}$ 等价于 $\id[\UU]AB$ (因此有 $\eqv AB$).
因此, 我们会经常滥用记号, 用 $A:\setU$ 代表 $(A,s):\setU$.
有时也会省略下标 \UU, 如果不需要具体说明问题中的全集.

回顾对于任意两个全集 $\UU_i$ 和 $\UU_{i+1}$, 如果 $A:\UU_i$ 那么 $A:\UU_{i+1}$.
因此, 对于任意 $(A,s):\set_{\UU_i}$ 同样有 $(A,s):\set_{\UU_{i+1}}$, 而 $\prop_{\UU_i}$ 也类似, 得到一个自然的映射
\begin{align}
    \set_{\UU_i} &\to \set_{\UU_{i+1}}\label{eq:set-up},\\
    \prop_{\UU_i} &\to \prop_{\UU_{i+1}}.\label{eq:prop-up}
\end{align}
该映射~\eqref{eq:set-up}不是等价的, 否则可以产生类似于 Cantorian 集合论的自指的悖论\index{悖论}.
不过, 尽管~\eqref{eq:prop-up} 在我们提出的类型论中暂时还不是天然的等价, 不过它是一致的.
也就是说, 可以考虑在类型论中添加这样的公理.

\begin{axiom}[命题缩放]
    \indexsee{公理!命题缩放}{命题缩放}%
    \indexdef{命题!缩放}%
    \indexsee{缩放!命题}{命题缩放}%
    映射 $\prop_{\UU_i} \to \prop_{\UU_{i+1}}$ 是等价.
\end{axiom}

这个公理被称为\define{命题缩放},
因为它意味着全集 $\UU_{i+1}$ 的任何纯命题可以 ``缩小'' 到等价的更小的全集 $\UU_i$.
它自动地成立, 前提是 $\UU_{i+1}$ 满足 \LEM{} (参见 \cref{ex:lem-impred}).
我们不会在所有场景假设这个公理成立, 尽管在某些地方会将它作为明确的假设.
这是纯命题\emph{非直谓}的构造, 通过避免使用它, 类型论可以保留\emph{直谓性}.
\indexsee{非直谓!纯命题的}{直谓的缩放}%
\indexdef{数学!直谓性}%

实践中, 我们最想要的是一个不同的陈述: 考虑某全集 \UU 包含类型 ``classifies all mere propositions''.
换句话说, 我们想要一个类型 $\Omega:\UU$, 以及以 $\Omega$-为索引的纯命题类型族, 其中包含所有纯命题的等价类.
这个陈述遵循之前的命题缩放, 如果 $\UU$ 不是最小的全集 $\UU_0$, 因为可以定义 $\Omega\defeq \prop_{\UU_0}$.

非直谓性的一个用处是定义幂集合.
可以自然地定义 $A$ 的\define{幂集合}\indexdef{幂集合}为 $A\to\propU$;
不过如果缺少非直谓性, 这个定义依赖 (甚至是等价于) 宇宙的算则 \UU.
不过根据命题缩放, 可以定义幂集合为
\symlabel{powerset}%
\[ \power A \defeq (A\to\Omega),\]
这样就不依赖 $\UU$.
参见 \cref{subsec:piw}.

