\index{type!contractible|(defstyle}%
\index{contractible!type|(defstyle}%

In \cref{thm:inhabprop-eqvunit} we observed that a mere proposition which is inhabited must be equivalent to $\unit$,
\index{type!unit}%
and it is not hard to see that the converse also holds.
A type with this property is called \emph{contractible}.
Another equivalent definition of contractibility, which is also sometimes convenient, is the following.

\begin{defn}\label{defn:contractible}
  A type $A$ is \define{contractible},
  or a \define{singleton},
  \indexdef{type!singleton}%
  \indexsee{singleton type}{type, singleton}%
  if there is $a:A$, called the \define{center of contraction},
  \indexdef{center!of contraction}%
  such that $a=x$ for all $x:A$.
  We denote the specified path $a=x$ by $\contr_x$.
\end{defn}

In other words, the type $\iscontr(A)$ is defined to be
\[ \iscontr(A) \defeq \sm{a:A} \prd{x:A}(a=x). \]
Note that under the usual propositions-as-types reading, we can pronounce $\iscontr(A)$ as ``$A$ contains exactly one element'', or more precisely ``$A$ contains an element, and every element of $A$ is equal to that element''.

\begin{rmk}
  We can also pronounce $\iscontr(A)$ more topologically as ``there is a point $a:A$ such that for all $x:A$ there exists a path from $a$ to $x$''.
  Note that to a classical ear, this sounds like a definition of \emph{connectedness} rather than contractibility.
  \index{continuity of functions in type theory@``continuity'' of functions in type theory}%
  \index{functoriality of functions in type theory@``functoriality'' of functions in type theory}%
  The point is that the meaning of ``there exists'' in this sentence is a continuous/natural one.

  A better way to express connectedness would be $\sm{a:A}\prd{x:A} \brck{a=x}$.
  This is indeed correct if $A$ is assumed to be pointed --- see the remark after \cref{thm:connected-pointed} --- but in general a type can be connected without being pointed.
  In \cref{sec:connectivity} we will define connectedness as the $n=0$ case of a general notion of $n$-connectedness, and in \cref{ex:connectivity-inductively} the reader is asked to show that this definition is equivalent to having both $\brck{A}$ and $\prd{x,y:A} \brck{x=y}$.
\end{rmk}

\begin{lem}\label{thm:contr-unit}
  For a type $A$, the following are logically equivalent.
  \begin{enumerate}
  \item $A$ is contractible in the sense of \cref{defn:contractible}.\label{item:contr}
  \item $A$ is a mere proposition, and there is a point $a:A$.\label{item:contr-inhabited-prop}
  \item $A$ is equivalent to \unit.\label{item:contr-eqv-unit}
  \end{enumerate}
\end{lem}
\begin{proof}
  If $A$ is contractible, then it certainly has a point $a:A$ (the center of contraction), while for any $x,y:A$ we have $x=a=y$; thus $A$ is a mere proposition.
  Conversely, if we have $a:A$ and $A$ is a mere proposition, then for any $x:A$ we have $x=a$; thus $A$ is contractible.
  And we showed~\ref{item:contr-inhabited-prop}$\Rightarrow$\ref{item:contr-eqv-unit} in \cref{thm:inhabprop-eqvunit}, while the converse follows since \unit easily has property~\ref{item:contr-inhabited-prop}.
\end{proof}

\begin{lem}\label{thm:isprop-iscontr}
  For any type $A$, the type $\iscontr(A)$ is a mere proposition.
\end{lem}
\begin{proof}
  Suppose given $c,c':\iscontr(A)$.
  We may assume $c\jdeq(a,p)$ and $c'\jdeq(a',p')$ for $a,a':A$ and $p:\prd{x:A} (a=x)$ and $p':\prd{x:A} (a'=x)$.
  By the characterization of paths in $\Sigma$-types, to show $c=c'$ it suffices to exhibit $q:a=a'$ such that $\trans{q}{p}=p'$.
  %
  We choose $q\defeq p(a')$.
  Now since $A$ is contractible (by $c$ or $c'$), by \cref{thm:contr-unit} it is a mere proposition.
  Hence, by \cref{thm:prop-set,thm:isprop-forall}, so is $\prd{x:A}(a'=x)$; thus $\trans{q}{p}=p'$ is automatic.
\end{proof}

\begin{cor}\label{thm:contr-contr}
  If $A$ is contractible, then so is $\iscontr(A)$.
\end{cor}
\begin{proof}
  By \cref{thm:isprop-iscontr} and \cref{thm:contr-unit}\ref{item:contr-inhabited-prop}.
\end{proof}

Like mere propositions, contractible types are preserved by many type constructors.
For instance, we have:

\begin{lem}\label{thm:contr-forall}
  If $P:A\to\type$ is a type family such that each $P(a)$ is contractible, then $\prd{x:A} P(x)$ is contractible.
\end{lem}
\begin{proof}
  By \cref{thm:isprop-forall}, $\prd{x:A} P(x)$ is a mere proposition since each $P(x)$ is.
  But it also has an element, namely the function sending each $x:A$ to the center of contraction of $P(x)$.
  Thus by \cref{thm:contr-unit}\ref{item:contr-inhabited-prop}, $\prd{x:A} P(x)$ is contractible.
\end{proof}

\index{function extensionality}%
(In fact, the statement of \cref{thm:contr-forall} is equivalent to the function extensionality axiom.
See~\cref{sec:univalence-implies-funext}.)

Of course, if $A$ is equivalent to $B$ and $A$ is contractible, then so is $B$.
More generally, it suffices for $B$ to be a \emph{retract} of $A$.
By definition, a \define{retraction}
\indexdef{retraction}%
\indexdef{function!retraction}%
is a function $r : A \to B$ such that there exists a function $s : B \to A$, called its \define{section},
\indexdef{section}%
\indexdef{function!section}%
and a homotopy $\epsilon:\prd{y:B} (r(s(y))=y)$; then we say that $B$ is a \define{retract}%
\indexdef{retract!of a type}
of $A$.

\begin{lem}\label{thm:retract-contr}
  If $B$ is a retract of $A$, and $A$ is contractible, then so is $B$.
\end{lem}
\begin{proof}
  Let $a_0 : A$ be the center of contraction.
  We claim that $b_0 \defeq r(a_0) : B$ is a center of contraction for $B$.
  Let $b : B$; we need a path $b = b_0$.
  But we have $\epsilon_b : r(s(b)) = b$ and $\contr_{s(b)} : s(b) = a_0$, so by composition
  \[ \opp{\epsilon_b} \ct \ap{r}{\contr_{s(b)}} : b = r(a_0) \jdeq b_0. \qedhere\]
\end{proof}

Contractible types may not seem very interesting, since they are all equivalent to \unit.
One reason the notion is useful is that sometimes a collection of individually nontrivial data will collectively form a contractible type.
An important example is the space of paths with one free endpoint.
As we will see in \cref{sec:identity-systems}, this fact essentially
encapsulates the based path induction principle for identity types.


\begin{lem}\label{thm:contr-paths}
  For any $A$ and any $a:A$, the type $\sm{x:A} (a=x)$ is contractible.
\end{lem}
\begin{proof}
  We choose as center the point $(a,\refl a)$.
  Now suppose $(x,p):\sm{x:A}(a=x)$; we must show $(a,\refl a) = (x,p)$.
  By the characterization of paths in $\Sigma$-types, it suffices to exhibit $q:a=x$ such that $\trans{q}{\refl a} = p$.
  But we can take $q\defeq p$, in which case $\trans{q}{\refl a} = p$ follows from the characterization of transport in path types.
\end{proof}

When this happens, it can allow us to simplify a complicated construction up to equivalence, using the informal principle that contractible data can be freely ignored.
This principle consists of many lemmas, most of which we leave to the reader; the following is an example.

\begin{lem}\label{thm:omit-contr}
  Let $P:A\to\type$ be a type family.
  \begin{enumerate}
  \item If each $P(x)$ is contractible, then $\sm{x:A} P(x)$ is equivalent to $A$.\label{item:omitcontr1}
  \item If $A$ is contractible with center $a$, then $\sm{x:A} P(x)$ is equivalent to $P(a)$.\label{item:omitcontr2}
  \end{enumerate}
\end{lem}
\begin{proof}
  In the situation of~\ref{item:omitcontr1}, we show that $\proj1:\sm{x:A} P(x) \to A$ is an equivalence.
  For quasi-inverse we define $g(x)\defeq (x,c_x)$ where $c_x$ is the center of $P(x)$.
  The composite $\proj1 \circ g$ is obviously $\idfunc[A]$, whereas the opposite composite is homotopic to the identity by using the contractions of each $P(x)$.

  We leave the proof of~\ref{item:omitcontr2} to the reader (see \cref{ex:omit-contr2}).
\end{proof}

Another reason contractible types are interesting is that they extend the ladder of $n$-types mentioned in \cref{sec:basics-sets} downwards one more step.

\begin{lem}\label{thm:prop-minusonetype}
  A type $A$ is a mere proposition if and only if for all $x,y:A$, the type $\id[A]xy$ is contractible.
\end{lem}
\begin{proof}
  For ``if'', we simply observe that any contractible type is inhabited.
  For ``only if'', we observed in \cref{subsec:hprops} that every mere proposition is a set, so that each type $\id[A]xy$ is a mere proposition.
  But it is also inhabited (since $A$ is a mere proposition), and hence by \cref{thm:contr-unit}\ref{item:contr-inhabited-prop} it is contractible.
\end{proof}

Thus, contractible types may also be called \define{$(-2)$-types}.
They are the bottom rung of the ladder of $n$-types, and will be the base case of the recursive definition of $n$-types in \cref{cha:hlevels}.

\index{type!contractible|)}%
\index{contractible!type|)}%
