\index{类型!可缩的|(defstyle}%
\index{可缩的!类型|(defstyle}%

在 \cref{thm:inhabprop-eqvunit} 中可以观察到, 有居留元的纯命题必须等价于 $\unit$,
\index{type!unit}%
不难看到, 反过来也一样.
类型的这种性质叫做 \emph{可缩性}.
收缩性的另一个等效定义有时也很方便, 如下所示.

\begin{defn}
    \label{defn:contractible}
    对于类型 $A$, 如果有 $a:A$, ,
    \indexdef{中心!收缩}%
    对于所有 $x:A$ 满足 $a=x$.
    那么类型 $A$ 是 \define{可收缩的},
    或者 \define{单例},
    \indexdef{类型!单例}%
    \indexsee{单例类型}{类型, 单例}%
    $x:A$ 是它的 \define{收缩中心}.
    这个特定的路径 $a=x$ 记作 $\contr_x$.
\end{defn}

换句话说, 类型 $\iscontr(A)$ 的定义是
\[ \iscontr(A) \defeq \sm{a:A} \prd{x:A}(a=x). \]
注意从命题作为类型的视角来看, 可以将 $\iscontr(A)$ 读作 ``$A$ 确实存在一个元素'', 或更准确地 ``$A$ 存在一个元素, 而且所有 $A$ 的元素都彼此相等''.

\begin{rmk}
    也可以把 $\iscontr(A)$ 读作拓扑学上的 ``存在 $a:A$ 满足对于所有 $x:A$ 存在从 $a$ 到 $x$ 的路径''.
    对于熟悉经典数学概念的人来说, 这个定义听起来像是\emph{连通性}而不是可缩性.
    \index{类型论中的函数的连续性@类型论中的函数的``连续性''}%
    \index{类型论中的函数的函子性@类型论中的函数的``函子性''}%
    重点在于, 这句话中的 ``存在'' 是连续或者自然的.

    连通性可以更好地表示为 $\sm{a:A}\prd{x:A} \brck{a=x}$.
    当 $A$ 可以被视为是点时, 这是正确的 --- 参见 \cref{thm:connected-pointed} --- 但是对于更一般的类型, 没有点也可以是连通的.
    在 \cref{sec:connectivity} 将连通性定义为一般意义上$n$-连同的 $n=0$ 的情况, 而在 \cref{ex:connectivity-inductively} 中, 读者被要求证明这个定义等价于 $\brck{A}$ 且 $\prd{x,y:A} \brck{x=y}$.
\end{rmk}

\begin{lem}
    \label{thm:contr-unit}
    对于类型 $A$, 以下是逻辑上等价的.
    \begin{enumerate}
        \item $A$ 于 \cref{defn:contractible} 意义上是可缩的.\label{item:contr}
        \item $A$ 是纯命题,而且存在点 $a:A$.\label{item:contr-inhabited-prop}
        \item $A$ 等价于 \unit.\label{item:contr-eqv-unit}
    \end{enumerate}
\end{lem}
\begin{proof}
    如果 $A$ 是可缩的, 那么必然存在点 $a:A$ (即收缩的中心), 于是对于任意 $x,y:A$ 有 $x=a=y$; 因此 $A$ 是纯命题.
    相反, 如果有 $a:A$ 而且 $A$ 是纯命题, 那么多余任何 $x:A$ 有 $x=a$; 因此 $A$ 是可缩的.
    然后在 \cref{thm:inhabprop-eqvunit} 中有 ~\ref{item:contr-inhabited-prop}$\Rightarrow$\ref{item:contr-eqv-unit}, 而反方向易证 \unit 有性质~\ref{item:contr-inhabited-prop}.
\end{proof}

\begin{lem}
    \label{thm:isprop-iscontr}
    对于任何类型 $A$, 类型 $\iscontr(A)$ 是纯命题.
\end{lem}
\begin{proof}
    给定 $c,c':\iscontr(A)$.
    假设 $c\jdeq(a,p)$ 和 $c'\jdeq(a',p')$ 对于 $a,a':A$ 和 $p:\prd{x:A} (a=x)$ 和 $p':\prd{x:A} (a'=x)$.
    通过 $\Sigma$-类型中的路径中的特征, 证明 $c=c'$ 只需展示 $q:a=a'$ 满足 $\trans{q}{p}=p'$.
    %
    选择 $q\defeq p(a')$.
    现在因为 $A$ 是可缩的 (根据 $c$ 或 $c'$), 因为 \cref{thm:contr-unit} 是纯命题.
    因此, 通过 \cref{thm:prop-set,thm:isprop-forall}, 所以 $\prd{x:A}(a'=x)$; 因此 $\trans{q}{p}=p'$.
\end{proof}

\begin{cor}
    \label{thm:contr-contr}
    如果 $A$ 可缩, 那么 $\iscontr(A)$ 也是.
\end{cor}
\begin{proof}
    根据 \cref{thm:isprop-iscontr} 和 \cref{thm:contr-unit}\ref{item:contr-inhabited-prop}.
\end{proof}

就像纯命题, 可缩类型被很多类型构造器保留.
例如, 有:

\begin{lem}
    \label{thm:contr-forall}
    如果 $P:A\to\type$ 是类型族而且每个 $P(a)$ 都是可缩的, 那么 $\prd{x:A} P(x)$ 是可缩的.
\end{lem}
\begin{proof}
    根据 \cref{thm:isprop-forall}, 因为每个 $P(x)$ 都是纯命题, 所以 $\prd{x:A} P(x)$ 是纯命题.
    但是也有元素, 即函数发送每个 $x:A$ 到 $P(x)$ 收缩的中心.
    因此根据 \cref{thm:contr-unit}\ref{item:contr-inhabited-prop}, $\prd{x:A} P(x)$ 是可缩的.
\end{proof}

\index{函数外延性}%
(实际上, 语句 \cref{thm:contr-forall} 等价于函数外延性公理.
参见~\cref{sec:univalence-implies-funext}.)

当然, 如果 $A$ 等价于 $B$ 而且 $A$ 可缩, 那么 $B$ 也如此.
跟一般地, 它满足 $B$ 是 $A$ \emph{retract}.
根据定义, \define{收缩映射}
\indexdef{收缩映射}%
\indexdef{函数!收缩映射}%
是函数 $r : A \to B$ 满足存在 $s : B \to A$, 也就是 \define{截面},
\indexdef{截面}%
\indexdef{函数!截面}%
和同伦 $\epsilon:\prd{y:B} (r(s(y))=y)$; 那么 $B$ 是 $A$ 的 \define{收缩映射}%
\indexdef{收缩映射!类型}.

\begin{lem}
    \label{thm:retract-contr}
    如果 $B$ 是 $A$ 的收缩映射, 而且 $A$ 是可缩的, 那么 $B$ 也如此.
\end{lem}
\begin{proof}
    令 $a_0 : A$ 为收缩的中心.
    我们声称 $b_0 \defeq r(a_0) : B$ 是 $B$ 的收缩中心.
    令 $b : B$; 需要路径 $b = b_0$.
    而 $\epsilon_b : r(s(b)) = b$ 且 $\contr_{s(b)} : s(b) = a_0$, 所以通过组合
    \[ \opp{\epsilon_b} \ct \ap{r}{\contr_{s(b)}} : b = r(a_0) \jdeq b_0. \qedhere\]
\end{proof}

可缩类型看起来不是很有意思, 因为它们都等价于 \unit.
它们有用的一个原因是, 有时一组单个的平凡数据可以共同的构成一个可缩类型.
一个重要的例子是一些空间上的路径, 它们具有一个自由端点.
在 \cref{sec:identity-systems} 会看到, 这个事实本质上概括了恒等类型的基础路径归纳原理.


\begin{lem}
    \label{thm:contr-paths}
    对于任何 $A$ 和任何 $a:A$, 类型 $\sm{x:A} (a=x)$ 是可缩的.
\end{lem}
\begin{proof}
    选择点 $(a,\refl a)$ 作为中心.
    现在假设 $(x,p):\sm{x:A}(a=x)$;
    需要证明 $(a,\refl a) = (x,p)$.
    根据 $\Sigma$-类型中的路径的特征, 足以得到 $q:a=x$ 满足 $\trans{q}{\refl a} = p$.
    不过可以获取 $q\defeq p$, 这种情况下 $\trans{q}{\refl a} = p$ 从路径类型中的传输特征.
\end{proof}

这种情况发生时, 允许我们化简复杂的构造到等价, 非正式定理中可收缩的数据可以自由地省略.
这个定理包含很多引理, 大部分留给读者; 一下是一些例子.

\begin{lem}
    \label{thm:omit-contr}
    令 $P:A\to\type$ 为类型族.
    \begin{enumerate}
        \item 如果每个 $P(x)$ 是可缩的, 那么 $\sm{x:A} P(x)$ 等价于 $A$.\label{item:omitcontr1}
        \item 如果 $A$ 是可缩的, 而且中心是 $a$, 那么 $\sm{x:A} P(x)$ 等价于 $P(a)$.\label{item:omitcontr2}
    \end{enumerate}
\end{lem}
\begin{proof}
    在~\ref{item:omitcontr1}中, 我们证明了 $\proj1:\sm{x:A} P(x) \to A$ 是一个等价关系.
    对于准逆我们定义 $g(x)\defeq (x,c_x)$ 其中 $c_x$ 是 $P(x)$ 的中心.
    组合 $\proj1 \circ g$ 明显地 $\idfunc[A]$, 然而对应的组成同伦于很等, 通过使用每个 $P(x)$ 的可缩性.

    我们在 ~\ref{item:omitcontr2} 会将这个证明留给读者(参见 \cref{ex:omit-contr2}).
\end{proof}

另一个可缩类型有意思的理由是, 它们\cref{sec:basics-sets}中提到的 $n$-类型的阶梯向下移动了一步.

\begin{lem}
    \label{thm:prop-minusonetype}
    类型 $A$ 是纯命题当且仅当对于所有 $x,y:A$, 类型 $\id[A]xy$ 可缩.
\end{lem}
\begin{proof}
    对于 ``当'', 容易观察到任意类型都满足.
    对于 ``仅当'', 观察在 \cref{subsec:hprops} 中每个纯命题是集合, 所以每个类型 $\id[A]xy$ 是纯命题.
    但它也未被居留 (因为 $A$ 是纯命题), 因此通过 \cref{thm:contr-unit}\ref{item:contr-inhabited-prop} 它可缩.
\end{proof}

因此, 可缩类型也被叫做 \define{$(-2)$-类型}.
它们是 $n$-类型阶梯的底部, 而且在 \cref{cha:hlevels} 中会使递归定义 $n$-类型的基础情况.

\index{类型!可缩|)}%
\index{可缩!类型|)}%
