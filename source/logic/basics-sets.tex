\index{set|(defstyle}%

在类型论中, 有一个\emph{集合}的概念, 可以帮助解释类型论的逻辑和集合论逻辑之间的联系.
尽管类型的行为通常类似空间或者高阶广群, 但还是有子类别, 更类似传统集合论系统中的集合.
范畴学上, 我们会考虑使用\emph{离散}广群, 它只由对象的一个集合确定, 并仅将恒等映射作为高阶映射;
拓扑学上, 我们会考虑使用离散范畴所在的空间.\index{离散!空间}
更通用的, 我们会考虑\emph{等价}于此类的广群或空间; 因为我们在类型论中做的最终都是同伦, 无法说有什么区别.

直观地, 我们认为类型 ``是一个集合'' 意味着它没有高阶同伦信息: 任意两条相同的路径都相等 (直到同伦), 而且所有尺寸的高阶路径都一样.
幸运地,因为同伦类型论天然的函子性/连续性,
\index{类型论中函数的连续性@类型论中函数的连续性``连续性''}%
\index{类型论中函数的函子性@类型论中函数的``函子性''}%
只需要考虑最低的等级就足够了.

\begin{defn}
    \label{defn:set}
    当对于所有 $x,y:A$ 和所有 $p,q:x=y$, 都存在 $p=q$ 时,
    类型 $A$ 是一个 \define{集合}.
\end{defn}

更准确地, 命题 $\isset(A)$ 被定义为这个类型
\[ \isset(A) \defeq \prd{x,y:A}{p,q:x=y} (p=q). \]

就像在 \cref{sec:types-vs-sets} 中提及的,
同伦类型论中的集合不像 ZT 类型论中的集合, 也就是没有一个通用的 ``所属关系断言'' $\in$.
它们更像结构数学和范畴学中用到的集合, 其元素是 ``抽象点'', 而我们用函数和关系给定其结构.
这样就可以将它当作大多数基于集合的数学的基础系统;
在 \cref{cha:set-math} 中有一些例子.

哪些类型是集合?
在\cref{cha:hlevels}会深入地学习这个问题更通用的形式, 不过现在可以观察一些简单的例子.

\begin{eg}
    \label{eg:isset-unit}
    类型 \unit 是一个集合.
    根据\cref{thm:path-unit}, 对于任何 $x,y:\unit$ 类型 $(x=y)$ 等价于 \unit.
    因为任意两个 \unit 的元素都相等, 这意味着任意两个 $x=y$ 的元素都相等.
\end{eg}

\begin{eg}
    \label{eg:isset-empty}
    类型 $\emptyt$ 是一个集合, 根据 $\emptyt$ 的归纳原理, 给定任意 $x,y:\emptyt$ 可以演绎任何想要的东西.
\end{eg}

\begin{eg}
    \label{thm:nat-set}
    自然数的类型 \nat 也是一个集合.
    根据 \cref{thm:path-nat}, 因为所有等同类型 $\id[\nat]xy$ 等价于 \unit 或者 \emptyt, 而且 \unit 或 \emptyt 的两个居留元都相等.
    在 \cref{cha:hlevels} 中, 还有另外一种证明.
\end{eg}

大部分我们考虑过的类型都是集合.

\begin{eg}
    \label{thm:isset-prod}
    如果 $A$ 和 $B$ 是集合, 那么 $A\times B$ 也如此.
    给定 $x,y:A\times B$ 和 $p,q:x=y$, 根据 \cref{thm:path-prod} 有 $p= \pairpath(\projpath1(p),\projpath2(p))$ 和 $q= \pairpath(\projpath1(q),\projpath2(q))$.
    而 $\projpath1(p)=\projpath1(q)$,  因为 $A$ 是集合, 以及 $\projpath2(p)=\projpath2(q)$, 因为 $B$ 是集合; 所以 $p=q$.

    类似地, 如果 $A$ 是集合而且 $B:A\to\type$ 满足所有 $B(x)$ 是集合, 那么 $\sm{x:A} B(x)$ 是一个集合.
\end{eg}

\begin{eg}
    \label{thm:isset-forall}
    如果 $A$ 是\emph{任意}类型, 而 $B:A\to \type$ 满足每个 $B(x)$ 是一个集合, 那么类型 $\prd{x:A} B(x)$ 是一个集合.
    提供 $f,g:\prd{x:A} B(x)$ 和 $p,q:f=g$.
    根据函数外延性, 有
    %
    \begin{equation*}
        p = {\funext (x \mapsto \happly(p,x))}
        \quad\text{和}\quad
        q = {\funext (x \mapsto \happly(q,x))}.
    \end{equation*}
    %
    不过对于所有 $x:A$, 有
    %
    \begin{equation*}
        \happly(p,x):f(x)=g(x)
        \qquad\text{和}\qquad
        \happly(q,x):f(x)=g(x),
    \end{equation*}
    %
    因为 $B(x)$ 是一个集合, 有 $\happly(p,x) = \happly(q,x)$.
    再次使用函数的外延性, 依值函数 $(x \mapsto \happly(p,x))$ 等于 $(x \mapsto \happly(q,x))$, 因此 (应用 $\apfunc{\funext}$) 就是~$p$ 和~$q$.
\end{eg}

更多的例子, 参见 \cref{ex:isset-coprod,ex:isset-sigma}. 为了更系统地了解同伦类型论中所有集合的子系统(范畴), 参见~\cref{cha:set-math}.

\index{n-阶类型@$n$-阶类型|(}%

集合只是\emph{同伦 $n$-阶类型}中的第一层级.
下一层由 \emph{$1$-types} 组成, 和范畴论中的 $1$-阶广群类似.
定义集合的的性质 (又叫做 \emph{$0$-阶类型}) 是它没有非平凡的性质.
类似地, 定义 $1$-阶类型的性质是它的路径之间没有非平凡的路径:

\begin{defn}
    \label{defn:1type}
    类型 $A$ 是一个 \define{1-阶类型}
    \indexdef{1-阶类型}%
    对于所有 $x,y:A$ 和 $p,q:x=y$ 以及 $r,s:p=q$, 有 $r=s$.
\end{defn}

类似地, 可以定义 $2$-阶类型, $3$-阶类型, 等等.
在\cref{cha:hlevels}中, 会定义 $n$-阶类型的归纳的通用概念, 并学习不同 $n$ 值之间, $n$-阶类型的关系.

不过, 现在只需要注意两个现象.
首先, 这些等级是向上封闭的: 如果 $A$ 是一个 $n$-阶类型, 那么 $A$ 是一个 $(n+1)$-阶类型.
例如:
%%  will make precise the sense in which this
%% ``suffices for all higher levels'', but as an example, we observe that
%% it suffices for the next level up.

\begin{lem}
    \label{thm:isset-is1type}
    如果 $A$ 是一个集合 (也就是说 $\isset(A)$ 有居留元), 那么 $A$ 是 1-阶类型.
\end{lem}
\begin{proof}
    给定 $f:\isset(A)$; 那么对于任何 $x,y:A$ 和 $p,q:x=y$ 有 $f(x,y,p,q):p=q$.
    固定 $x$, $y$, 和 $p$, 并定义 $g: \prd{q:x=y} (p=q)$ 为 $g(q) \defeq f (x,y,p,q)$.
    那么对于任何 $r:q=q'$, 有 $\apdfunc{g}(r) : \trans{r}{g(q)} = g(q')$.
    根据 \cref{cor:transport-path-prepost}, 于是有 $g(q) \ct r = g(q')$.

    特殊地, 给定 $x,y,p,q$ 和 $r,s:p=q$, 按照 \cref{defn:1type}, 定义 $g$ 如上.
    然后 $g(p) \ct r = g(q)$ 而且 $g(p) \ct s = g(q)$, 约去后得到 $r=s$.
\end{proof}

第二, 这种按类型等级的分层并不退化, 也就说意味着不是所有类型都是集合:

\begin{eg}
    \label{thm:type-is-not-a-set}
    \index{类型!全集}%
    全集 \type 不是集合.
    为了证明它, 展示类型 $A$ 和路径 $p:A=A$ 且不等于 $\refl A$ 即可.
    令 $A=\bool$, 并定义 $f:A\to A$ 为 $f(\bfalse)\defeq \btrue$ 和 $f(\btrue)\defeq \bfalse$.
    然后 $f(f(x))=x$ 对于所有 $x$ (很容易分析), 所以 $f$ 是一个等价.
    因此, 按照单值性, $f$ 可以得到 $p:A=A$.

    如果 $p$ 等于 $\refl A$, 那么 (再次根据单值性) $f$ 会等于 $A$ 的恒等函数.
    不过这样会得到 $\bfalse=\btrue$, 和 \cref{rmk:true-neq-false} 矛盾.
\end{eg}

\cref{cha:hits,cha:homotopy}中, 会展示对于任意 $n$, 都有不是 $n$-阶类型的类型.

注意 $A$ 是一个 1-阶类型当且仅当对于任何 $x,y:A$, 恒等类型 $\id[A]xy$ 是一个集合.
(因此, \cref{thm:isset-is1type} 等价于说集合的恒等类型也是集合.)
这是 \cref{cha:hlevels} 中将给出的递归地定义 $n$-阶类型的基础.

还可以从集合 ``向下'' 扩展这个概念.
也就是说, 类型 $A$ 是一个集合仅当对于任意 $x,y:A$, $\id[A]xy$ 的两个元素都是相等的.
因为集合等价于 0-阶类型, 自然地, 有一个 \emph{$(-1)$-阶类型}, 如果它们有后者的性质 (任意两个元素都相等).
这些类型可以被视为 \emph{狭义上的命题}, 而它们的研究也就是常说的 ``逻辑'';
它会在本章的剩余部分出现.

\index{n-阶类型@$n$-阶类型|)}%
\index{集合|)}%
