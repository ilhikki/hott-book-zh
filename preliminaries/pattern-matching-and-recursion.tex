\index{pattern matching|(defstyle}%
\indexsee{matching}{pattern matching}%
\index{definition!by pattern matching|(}%
The natural numbers introduce an additional subtlety over the types considered up until now.
In the case of coproducts, for instance, we could define a function $f:A+B\to C$ either with the recursor:
\[ f \defeq \rec{A+B}(C, g_0, g_1) \]
or by giving the defining equations:
\begin{align*}
  f(\inl(a)) &\defeq g_0(a)\\
  f(\inr(b)) &\defeq g_1(b).
\end{align*}
To go from the former expression of $f$ to the latter, we simply use the computation rules for the recursor.
Conversely, given any defining equations
\begin{align*}
  f(\inl(a)) &\defeq \Phi_0\\
  f(\inr(b)) &\defeq \Phi_1
\end{align*}
where $\Phi_0$ and $\Phi_1$ are expressions that may involve the variables
\index{variable}%
$a$ and $b$ respectively, we can express these equations equivalently in terms of the recursor by using $\lambda$-abstraction\index{lambda abstraction@$\lambda$-abstraction}:
\[ f\defeq \rec{A+B}(C, \lam{a} \Phi_0, \lam{b} \Phi_1).\]
In the case of the natural numbers, however, the ``defining equations'' of a function such as $\dbl$:
\begin{align}
  \dbl(0) &\defeq 0 \label{eq:dbl0}\\
  \dbl(\suc(n)) &\defeq \suc(\suc(\dbl(n)))\label{eq:dblsuc}
\end{align}
involve \emph{the function $\dbl$ itself} on the right-hand side.
However, we would still like to be able to give these equations, rather than~\eqref{eq:dbl-as-rec}, as the definition of \dbl, since they are much more convenient and readable.
The solution is to read the expression ``$\dbl(n)$'' on the right-hand side of~\eqref{eq:dblsuc} as standing in for the result of the recursive call, which in a definition of the form $\dbl\defeq \rec{\nat}(\nat,c_0,c_s)$ would be the second argument of $c_s$.

More generally, if we have a ``definition'' of a function $f:\nat\to C$ such as
\begin{align*}
  f(0) &\defeq \Phi_0\\
  f(\suc(n)) &\defeq \Phi_s
\end{align*}
where $\Phi_0$ is an expression of type $C$, and $\Phi_s$ is an expression of type $C$ which may involve the variable $n$ and also the symbol ``$f(n)$'', we may translate it to a definition
\[ f \defeq \rec{\nat}(C,\,\Phi_0,\,\lam{n}{r} \Phi_s') \]
where $\Phi_s'$ is obtained from $\Phi_s$ by replacing all occurrences of ``$f(n)$'' by the new variable $r$.

This style of defining functions by recursion (or, more generally, dependent functions by induction) is so convenient that we frequently adopt it.
It is called definition by \define{pattern matching}.
Of course, it is very similar to how a computer programmer may define a recursive function with a body that literally contains recursive calls to itself.
However, unlike the programmer, we are restricted in what sort of recursive calls we can make: in order for such a definition to be re-expressible using the recursion principle, the function $f$ being defined can only appear in the body of $f(\suc(n))$ as part of the composite symbol ``$f(n)$''.
Otherwise, we could write nonsense functions such as
\begin{align*}
  f(0)&\defeq 0\\
  f(\suc(n)) &\defeq f(\suc(\suc(n))).
\end{align*}
If a programmer wrote such a function, it would simply call itself forever on any positive input, going into an infinite loop and never returning a value.
In mathematics, however, to be worthy of the name, a \emph{function} must always associate a unique output value to every input value, so this would be unacceptable.

This point will be even more important when we introduce more complicated inductive types in \cref{cha:induction,cha:hits,cha:real-numbers}.
Whenever we introduce a new kind of inductive definition, we always begin by deriving its induction principle.
Only then do we introduce an appropriate sort of ``pattern matching'' which can be justified as a shorthand for the induction principle.

\index{pattern matching|)}%
\index{definition!by pattern matching|)}%