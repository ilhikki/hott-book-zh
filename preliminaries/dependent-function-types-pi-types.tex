\index{type!dependent function|(defstyle}%
\index{function!dependent|(defstyle}%
\indexsee{dependent!function}{function, dependent}%
\indexsee{type!Pi-@$\Pi$-}{type, dependent function}%
\indexsee{Pi-type@$\Pi$-type}{type, dependent function}%
In type theory we often use a more general version of function
types, called a \define{$\Pi$-type} or \define{dependent function type}. The elements of
a $\Pi$-type are functions
whose codomain type can vary depending on the
element of the domain to which the function is applied, called \define{dependent functions}. The name ``$\Pi$-type''
is used because this type can also be regarded as the  cartesian
product over a given type.

Given a type $A:\UU$ and a family $B:A \to \UU$, we may construct
the type of dependent functions $\prd{x:A}B(x) : \UU$.
There are many alternative notations for this type, such as
\[ \tprd{x:A} B(x) \qquad \dprd{x:A}B(x) \qquad \lprd{x:A} B(x). \]
If $B$ is a constant family, then the dependent product type is the ordinary function type:
\[\tprd{x:A} B \jdeq (A \to B).\]
Indeed, all the constructions of $\Pi$-types are generalizations of the corresponding constructions on ordinary function types.

\indexdef{definition!of function, direct}%
We can introduce dependent functions by explicit definitions: to
define $f : \prd{x:A}B(x)$, where $f$ is the name of a dependent function to be
defined, we need an expression $\Phi : B(x)$ possibly involving the variable $x:A$,
\index{variable}%
and we write
\[ f(x) \defeq \Phi \qquad \mbox{for $x:A$}.\]
Alternatively, we can use \define{$\lambda$-abstraction}%
\index{lambda abstraction@$\lambda$-abstraction|defstyle}%
\begin{equation}
  \label{eq:lambda-abstraction}
  \lamu{x:A} \Phi \ :\ \prd{x:A} B(x).
\end{equation}
\indexdef{application!of dependent function}%
\indexdef{function!dependent!application}%
As with non-dependent functions, we can \define{apply} a dependent function $f : \prd{x:A}B(x)$ to an argument $a:A$ to obtain an element $f(a):B(a)$.
The equalities are the same as for the ordinary function type, i.e.\ we have the computation rule
\index{computation rule!for dependent function types}%
given $a:A$ we have $f(a) \jdeq \Phi'$ and  
$(\lamu{x:A} \Phi)(a) \jdeq \Phi'$, where $\Phi' $ is obtained by replacing all
occurrences of $x$ in $\Phi$ by $a$ (avoiding variable capture, as always).
Similarly, we have the uniqueness principle $f\jdeq (\lam{x} f(x))$ for any $f:\prd{x:A} B(x)$.
\index{uniqueness!principle!for dependent function types}%

As an example, recall from \cref{sec:universes} that there is a type family $\Fin:\nat\to\UU$ whose values are the standard finite sets, with elements $0_n,1_n,\dots,(n-1)_n : \Fin(n)$.
There is then a dependent function $\fmax : \prd{n:\nat} \Fin(n+1)$
which returns the ``largest'' element of each nonempty finite type, $\fmax(n) \defeq n_{n+1}$.
\index{finite!sets, family of}%
As was the case for $\Fin$ itself, we cannot define $\fmax$ yet, but we will be able to soon; see \cref{ex:fin}.

Another important class of dependent function types, which we can define now, are functions which are \define{polymorphic}
\indexdef{function!polymorphic}%
\indexdef{polymorphic function}%
over a given universe.
A polymorphic function is one which takes a type as one of its arguments, and then acts on elements of that type (or of other types constructed from it).
\symlabel{idfunc}%
\indexdef{function!identity}%
\indexdef{identity!function}%
An example is the polymorphic identity function $\idfunc : \prd{A:\UU} A \to A$, which we define by $\idfunc{} \defeq \lam{A:\type}{x:A} x$.
(Like $\lambda$-abstractions, $\Pi$s automatically scope\index{scope} over the rest of the expression unless delimited; thus $\idfunc : \prd{A:\UU} A \to A$ means $\idfunc : \prd{A:\UU} (A \to A)$.
This convention, though unusual in mathematics, is common in type theory.)

We sometimes write some arguments of a dependent function as subscripts.
For instance, we might equivalently define the polymorphic identity function by $\idfunc[A](x) \defeq x$.
Moreover, if an argument can be inferred from context, we may omit it altogether.
For instance, if $a:A$, then writing $\idfunc(a)$ is unambiguous, since $\idfunc$ must mean $\idfunc[A]$ in order for it to be applicable to $a$.

Another, less trivial, example of a polymorphic function is the ``swap'' operation that switches the order of the arguments of a (curried) two-argument function:
\[ \mathsf{swap} : \prd{A:\UU}{B:\UU}{C:\UU} (A\to B\to C) \to (B\to A \to C). \]
We can define this by
\[ \mathsf{swap}(A,B,C,g) \defeq \lam{b}{a} g(a)(b). \]
We might also equivalently write the type arguments as subscripts:
\[ \mathsf{swap}_{A,B,C}(g)(b,a) \defeq g(a,b). \]

Note that as we did for ordinary functions, we use currying to define dependent functions with
several arguments (such as $\mathsf{swap}$). However, in the dependent case the second domain may
depend on the first one, and the codomain may depend on both. That is,
given $A:\UU$ and type families $B : A \to \UU$ and $C : \prd{x:A}B(x) \to \UU$, we may construct
the type $\prd{x:A}{y : B(x)} C(x,y)$ of functions with two
arguments.
In the case when $B$ is constant and equal to $A$, we may condense the notation and write $\prd{x,y:A}$; for instance, the type of $\mathsf{swap}$ could also be written as
\[ \mathsf{swap} : \prd{A,B,C:\UU} (A\to B\to C) \to (B\to A \to C). \]
Finally, given $f:\prd{x:A}{y : B(x)} C(x,y)$ and arguments $a:A$ and $b:B(a)$, we have $f(a)(b) : C(a,b)$, which,
as before, we write as $f(a,b) : C(a,b)$.

\index{type!dependent function|)}%
\index{function!dependent|)}%