\chapter{集合和逻辑}
\label{cha:logic}
正式或非正式地说, 类型论的规则的汇集, 这些规则用来操作类型和类型的元素.
不过当非正式的讨论数学时, 普遍会用那些熟悉的词语, 特别是逻辑连词比如 ``而且'' 和 ``或'', 以及逻辑两次比如 ``对于所有'' 和 ``存在''.
相比集合论, 类型论提供了超过一种方法来将自然语言作为类型上的运算.
为了解决潜在的歧义, 需要设置局部或者全局的约定, 或者在非形式的数学中引入新的注解, 或者两个都要.
其中一些要求以及实现了, 因为类型论可以更好的分析逻辑, 我们可以更准确的表示数学, 其中 ``语言的滥用'' 比集合论更少.
本章我们会讲解这些相关的缺陷, 并证明我们的选择是正确的.


\section{集合和\texorpdfstring{$n$}{n}-阶类型}
\label{sec:basics-sets}
\index{set|(defstyle}%

在类型论中, 有一个\emph{集合}的概念, 可以帮助解释类型论的逻辑和集合论逻辑之间的联系.
尽管类型的行为通常类似空间或者高阶广群, 但还是有子类别, 更类似传统集合论系统中的集合.
范畴学上, 我们会考虑使用\emph{离散}广群, 它只由对象的一个集合确定, 并仅将恒等映射作为高阶映射;
拓扑学上, 我们会考虑使用离散范畴所在的空间.\index{离散!空间}
更通用的, 我们会考虑\emph{等价}于此类的广群或空间; 因为我们在类型论中做的最终都是同伦, 无法说有什么区别.

直观地, 我们认为类型 ``是一个集合'' 意味着它没有高阶同伦信息: 任意两条相同的路径都相等 (直到同伦), 而且所有尺寸的高阶路径都一样.
幸运地,因为同伦类型论天然的函子性/连续性,
\index{类型论中函数的连续性@类型论中函数的连续性``连续性''}%
\index{类型论中函数的函子性@类型论中函数的``函子性''}%
只需要考虑最低的等级就足够了.

\begin{defn}
    \label{defn:set}
    当对于所有 $x,y:A$ 和所有 $p,q:x=y$, 都存在 $p=q$ 时,
    类型 $A$ 是一个 \define{集合}.
\end{defn}

更准确地, 命题 $\isset(A)$ 被定义为这个类型
\[ \isset(A) \defeq \prd{x,y:A}{p,q:x=y} (p=q). \]

就像在 \cref{sec:types-vs-sets} 中提及的,
同伦类型论中的集合不像 ZT 类型论中的集合, 也就是没有一个通用的 ``所属关系断言'' $\in$.
它们更像结构数学和范畴学中用到的集合, 其元素是 ``抽象点'', 而我们用函数和关系给定其结构.
这样就可以将它当作大多数基于集合的数学的基础系统;
在 \cref{cha:set-math} 中有一些例子.

哪些类型是集合?
在\cref{cha:hlevels}会深入地学习这个问题更通用的形式, 不过现在可以观察一些简单的例子.

\begin{eg}
    \label{eg:isset-unit}
    类型 \unit 是一个集合.
    根据\cref{thm:path-unit}, 对于任何 $x,y:\unit$ 类型 $(x=y)$ 等价于 \unit.
    因为任意两个 \unit 的元素都相等, 这意味着任意两个 $x=y$ 的元素都相等.
\end{eg}

\begin{eg}
    \label{eg:isset-empty}
    类型 $\emptyt$ 是一个集合, 根据 $\emptyt$ 的归纳原理, 给定任意 $x,y:\emptyt$ 可以演绎任何想要的东西.
\end{eg}

\begin{eg}
    \label{thm:nat-set}
    自然数的类型 \nat 也是一个集合.
    根据 \cref{thm:path-nat}, 因为所有等同类型 $\id[\nat]xy$ 等价于 \unit 或者 \emptyt, 而且 \unit 或 \emptyt 的两个居留元都相等.
    在 \cref{cha:hlevels} 中, 还有另外一种证明.
\end{eg}

大部分我们考虑过的类型都是集合.

\begin{eg}
    \label{thm:isset-prod}
    如果 $A$ 和 $B$ 是集合, 那么 $A\times B$ 也如此.
    给定 $x,y:A\times B$ 和 $p,q:x=y$, 根据 \cref{thm:path-prod} 有 $p= \pairpath(\projpath1(p),\projpath2(p))$ 和 $q= \pairpath(\projpath1(q),\projpath2(q))$.
    而 $\projpath1(p)=\projpath1(q)$,  因为 $A$ 是集合, 以及 $\projpath2(p)=\projpath2(q)$, 因为 $B$ 是集合; 所以 $p=q$.

    类似地, 如果 $A$ 是集合而且 $B:A\to\type$ 满足所有 $B(x)$ 是集合, 那么 $\sm{x:A} B(x)$ 是一个集合.
\end{eg}

\begin{eg}
    \label{thm:isset-forall}
    如果 $A$ 是\emph{任意}类型, 而 $B:A\to \type$ 满足每个 $B(x)$ 是一个集合, 那么类型 $\prd{x:A} B(x)$ 是一个集合.
    提供 $f,g:\prd{x:A} B(x)$ 和 $p,q:f=g$.
    根据函数外延性, 有
    %
    \begin{equation*}
        p = {\funext (x \mapsto \happly(p,x))}
        \quad\text{和}\quad
        q = {\funext (x \mapsto \happly(q,x))}.
    \end{equation*}
    %
    不过对于所有 $x:A$, 有
    %
    \begin{equation*}
        \happly(p,x):f(x)=g(x)
        \qquad\text{和}\qquad
        \happly(q,x):f(x)=g(x),
    \end{equation*}
    %
    因为 $B(x)$ 是一个集合, 有 $\happly(p,x) = \happly(q,x)$.
    再次使用函数的外延性, 依值函数 $(x \mapsto \happly(p,x))$ 等于 $(x \mapsto \happly(q,x))$, 因此 (应用 $\apfunc{\funext}$) 就是~$p$ 和~$q$.
\end{eg}

更多的例子, 参见 \cref{ex:isset-coprod,ex:isset-sigma}. 为了更系统地了解同伦类型论中所有集合的子系统(范畴), 参见~\cref{cha:set-math}.

\index{n-阶类型@$n$-阶类型|(}%

集合只是\emph{同伦 $n$-阶类型}中的第一层级.
下一层由 \emph{$1$-types} 组成, 和范畴论中的 $1$-阶广群类似.
定义集合的的性质 (又叫做 \emph{$0$-阶类型}) 是它没有非平凡的性质.
类似地, 定义 $1$-阶类型的性质是它的路径之间没有非平凡的路径:

\begin{defn}
    \label{defn:1type}
    类型 $A$ 是一个 \define{1-阶类型}
    \indexdef{1-阶类型}%
    对于所有 $x,y:A$ 和 $p,q:x=y$ 以及 $r,s:p=q$, 有 $r=s$.
\end{defn}

类似地, 可以定义 $2$-阶类型, $3$-阶类型, 等等.
在\cref{cha:hlevels}中, 会定义 $n$-阶类型的归纳的通用概念, 并学习不同 $n$ 值之间, $n$-阶类型的关系.

不过, 现在只需要注意两个现象.
首先, 这些等级是向上封闭的: 如果 $A$ 是一个 $n$-阶类型, 那么 $A$ 是一个 $(n+1)$-阶类型.
例如:
%%  will make precise the sense in which this
%% ``suffices for all higher levels'', but as an example, we observe that
%% it suffices for the next level up.

\begin{lem}
    \label{thm:isset-is1type}
    如果 $A$ 是一个集合 (也就是说 $\isset(A)$ 有居留元), 那么 $A$ 是 1-阶类型.
\end{lem}
\begin{proof}
    给定 $f:\isset(A)$; 那么对于任何 $x,y:A$ 和 $p,q:x=y$ 有 $f(x,y,p,q):p=q$.
    固定 $x$, $y$, 和 $p$, 并定义 $g: \prd{q:x=y} (p=q)$ 为 $g(q) \defeq f (x,y,p,q)$.
    那么对于任何 $r:q=q'$, 有 $\apdfunc{g}(r) : \trans{r}{g(q)} = g(q')$.
    根据 \cref{cor:transport-path-prepost}, 于是有 $g(q) \ct r = g(q')$.

    特殊地, 给定 $x,y,p,q$ 和 $r,s:p=q$, 按照 \cref{defn:1type}, 定义 $g$ 如上.
    然后 $g(p) \ct r = g(q)$ 而且 $g(p) \ct s = g(q)$, 约去后得到 $r=s$.
\end{proof}

第二, 这种按类型等级的分层并不退化, 也就说意味着不是所有类型都是集合:

\begin{eg}
    \label{thm:type-is-not-a-set}
    \index{类型!全集}%
    全集 \type 不是集合.
    为了证明它, 展示类型 $A$ 和路径 $p:A=A$ 且不等于 $\refl A$ 即可.
    令 $A=\bool$, 并定义 $f:A\to A$ 为 $f(\bfalse)\defeq \btrue$ 和 $f(\btrue)\defeq \bfalse$.
    然后 $f(f(x))=x$ 对于所有 $x$ (很容易分析), 所以 $f$ 是一个等价.
    因此, 按照单值性, $f$ 可以得到 $p:A=A$.

    如果 $p$ 等于 $\refl A$, 那么 (再次根据单值性) $f$ 会等于 $A$ 的恒等函数.
    不过这样会得到 $\bfalse=\btrue$, 和 \cref{rmk:true-neq-false} 矛盾.
\end{eg}

\cref{cha:hits,cha:homotopy}中, 会展示对于任意 $n$, 都有不是 $n$-阶类型的类型.

注意 $A$ 是一个 1-阶类型当且仅当对于任何 $x,y:A$, 恒等类型 $\id[A]xy$ 是一个集合.
(因此, \cref{thm:isset-is1type} 等价于说集合的恒等类型也是集合.)
这是 \cref{cha:hlevels} 中将给出的递归地定义 $n$-阶类型的基础.

还可以从集合 ``向下'' 扩展这个概念.
也就是说, 类型 $A$ 是一个集合仅当对于任意 $x,y:A$, $\id[A]xy$ 的两个元素都是相等的.
因为集合等价于 0-阶类型, 自然地, 有一个 \emph{$(-1)$-阶类型}, 如果它们有后者的性质 (任意两个元素都相等).
这些类型可以被视为 \emph{狭义上的命题}, 而它们的研究也就是常说的 ``逻辑'';
它会在本章的剩余部分出现.

\index{n-阶类型@$n$-阶类型|)}%
\index{集合|)}%


\section{命题作为类型?}
\label{subsec:pat?}
\index{proposition!as types|(}%
\index{logic!constructive vs classical|(}%
\index{anger|(}%
Until now, we have been following the straightforward ``propositions as types'' philosophy described in \cref{sec:pat}, according to which English phrases such as ``there exists an $x:A$ such that $P(x)$'' are interpreted by corresponding types such as $\sm{x:A} P(x)$, with the proof of a statement being regarded as judging some specific element to inhabit that type.
However, we have also seen some ways in which the ``logic'' resulting from this reading seems unfamiliar to a classical mathematician.
For instance, in \cref{thm:ttac} we saw that the statement
\index{axiom!of choice!type-theoretic}%
\begin{equation}\label{eq:english-ac}
  \parbox{\textwidth-2cm}{``If for all $x:X$ there exists an $a:A(x)$ such that $P(x,a)$, then there exists a function $g:\prd{x:X} A(x)$ such that for all $x:X$ we have $P(x,g(x))$'',}
\end{equation}
which looks like the classical\index{mathematics!classical} \emph{axiom of choice}, is always true under this reading. This is a noteworthy, and often useful, feature of the propositions-as-types logic, but it also illustrates how significantly it differs from the classical interpretation of logic, under which the axiom of choice is not a logical truth, but an additional ``axiom''.

On the other hand, we can now also show that corresponding statements looking like the classical \emph{law of double negation} and \emph{law of excluded middle} are incompatible with the univalence axiom.
\index{univalence axiom}%

\begin{thm}\label{thm:not-dneg}
  \index{double negation, law of}%
  It is not the case that for all $A:\UU$ we have $\neg(\neg A) \to A$.
\end{thm}
\begin{proof}
  Recall that $\neg A \jdeq (A\to\emptyt)$.
  We also read ``it is not the case that \dots'' as the operator $\neg$.
  Thus, in order to prove this statement, it suffices to assume given some $f:\prd{A:\UU} (\neg\neg A \to A)$ and construct an element of \emptyt.

  The idea of the following proof is to observe that $f$, like any function in type theory, is ``continuous''.
  \index{continuity of functions in type theory@``continuity'' of functions in type theory}%
  \index{functoriality of functions in type theory@``functoriality'' of functions in type theory}%
  By univalence, this implies that $f$ is \emph{natural} with respect to equivalences of types.
  From this, and a fixed-point-free autoequivalence\index{automorphism!fixed-point-free}, we will be able to extract a contradiction.

  Let $e:\eqv\bool\bool$ be the equivalence defined by $e(\btrue)\defeq\bfalse$ and $e(\bfalse)\defeq\btrue$, as in \cref{thm:type-is-not-a-set}.
  Let $p:\bool=\bool$ be the path corresponding to $e$ by univalence, i.e.\ $p\defeq \ua(e)$.
  Then we have $f(\bool) : \neg\neg\bool \to\bool$ and
  \[\apd f p : \transfib{A\mapsto (\neg\neg A \to A)}{p}{f(\bool)} = f(\bool).\]
  Hence, for any $u:\neg\neg\bool$, we have
  \[\happly(\apd f p,u) : \transfib{A\mapsto (\neg\neg A \to A)}{p}{f(\bool)}(u) = f(\bool)(u).\]

  Now by~\eqref{eq:transport-arrow}, transporting $f(\bool):\neg\neg\bool\to\bool$ along $p$ in the type family ${A\mapsto (\neg\neg A \to A)}$ is equal to the function which transports its argument along $\opp p$ in the type family $A\mapsto \neg\neg A$, applies $f(\bool)$, then transports the result along $p$ in the type family $A\mapsto A$:
  \begin{narrowmultline*}
    \transfib{A\mapsto (\neg\neg A \to A)}{p}{f(\bool)}(u) =
    \narrowbreak
    \transfib{A\mapsto A}{p}{f(\bool) (\transfib{A\mapsto \neg\neg
        A}{\opp{p}}{u})}.
  \end{narrowmultline*}
  %
  However, any two points $u,v:\neg\neg\bool$ are equal by function extensionality, since for any $x:\neg\bool$ we have $u(x):\emptyt$ and thus we can derive any conclusion, in particular $u(x)=v(x)$.
  Thus, we have $\transfib{A\mapsto \neg\neg A}{\opp{p}}{u} = u$, and so from $\happly(\apd f p,u)$ we obtain an equality
  \[ \transfib{A\mapsto A}{p}{f(\bool)(u)} = f(\bool)(u).\]
  Finally, as discussed in \cref{sec:compute-universe}, transporting in the type family $A\mapsto A$ along the path $p\jdeq \ua(e)$ is equivalent to applying the equivalence $e$; thus we have
  \begin{equation}
    e(f(\bool)(u)) = f(\bool)(u).\label{eq:fpaut}
  \end{equation}
  %
  However, we can also prove that
  \begin{equation}
    \prd{x:\bool} \neg(e(x)=x).\label{eq:fpfaut}
  \end{equation}
  This follows from a case analysis on $x$: both cases are immediate from the definition of $e$ and the fact that $\bfalse\neq\btrue$ (\cref{rmk:true-neq-false}).
  Thus, applying~\eqref{eq:fpfaut} to $f(\bool)(u)$ and~\eqref{eq:fpaut}, we obtain an element of $\emptyt$.
\end{proof}

\begin{rmk}
  \index{choice operator}%
  \indexsee{operator!choice}{choice operator}%
  In particular, this implies that there can be no Hilbert-style ``choice operator'' which selects an element of every nonempty type.
  The point is that no such operator can be \emph{natural}, and under the univalence axiom, all functions acting on types must be natural with respect to equivalences.
\end{rmk}

\begin{rmk}
  It is, however, still the case that $\neg\neg\neg A \to \neg A$ for any $A$; see \cref{ex:neg-ldn}.
\end{rmk}

\begin{cor}\label{thm:not-lem}
  \index{excluded middle}%
  It is not the case that for all $A:\UU$ we have $A+(\neg A)$.
\end{cor}
\begin{proof}
  Suppose we had $g:\prd{A:\UU} (A+(\neg A))$.
  We will show that then $\prd{A:\UU} (\neg\neg A \to A)$, so that we can apply \cref{thm:not-dneg}.
  Thus, suppose $A:\UU$ and $u:\neg\neg A$; we want to construct an element of $A$.

  Now $g(A):A+(\neg A)$, so by case analysis, we may assume either $g(A)\jdeq \inl(a)$ for some $a:A$, or $g(A)\jdeq \inr(w)$ for some $w:\neg A$.
  In the first case, we have $a:A$, while in the second case we have $u(w):\emptyt$ and so we can obtain anything we wish (such as $A$).
  Thus, in both cases we have an element of $A$, as desired.
\end{proof}

Thus, if we want to assume the univalence axiom (which, of course, we do) and still leave ourselves the option of classical\index{mathematics!classical} reasoning (which is also desirable), we cannot use the unmodified propositions-as-types principle to interpret \emph{all} informal mathematical statements into type theory, since then the law of excluded middle would be false.
However, neither do we want to discard propositions-as-types entirely, because of its many good properties (such as simplicity, constructivity, and computability).
We now discuss a modification of propositions-as-types which resolves these problems; in \cref{subsec:when-trunc} we will return to the question of which logic to use when.
\index{anger|)}%
\index{proposition!as types|)}%
\index{logic!constructive vs classical|)}%


\section{纯命题}
\label{subsec:hprops}
\index{逻辑!纯命题的|(}%
\index{纯命题(defstyle}%
\indexsee{命题!纯}{纯命题}%
我们已经看到命题作为类型的逻辑的优点和缺点.
它们有这样的共性: 当类型视为命题, 它们存在更多的信息, 不止是对和错, 它们上的 ``逻辑'' 构造都遵循这些额外信息.
这表明我们可以得到更常规的逻辑, 通过只关注\emph{不}包含任何比真值外信息的类型, 并单纯视其为逻辑命题.

比如类型 $A$ 为 ``真'' 当其被居留时, 而 ``假'' 如果它的居留元可以得到矛盾 (即 $\neg A \jdeq (A\to\emptyt)$ 被居留).
\index{被居留的类型}%
为了得到更加传统的逻辑, 要避免将这样的类型视为逻辑命题, 给出它们的元素比知道这个类型被居留提供了更多的信息.
例如, 如果给出 \bool 的元素, 那么会收到根多的信息, 相比 \bool 存在元素这个单纯的事实.
确实, 得到了确切的\emph{一比特}
\index{比特}%
的额外信息: 知道了给出的是 \bool 的\emph{哪个}元素.
相反的, 如果给出 \unit 的元素, 那么不会得到更多的信息, 相比 \unit 存在元素这个单纯的事实, 因为任何两个 \unit 的元素彼此相等.
这启发了这个定义.

\begin{defn}\label{defn:isprop}
  类型 $P$ 是一个\define{纯命题}
  如果对于所有 $x,y:P$ 有 $x=y$.
\end{defn}

主义因为我们还是在类型论\emph{中}做数学, 这个定义\emph{在}类型论\emph{中}, 意味着一个类型 --- 或是一个类型族.
特殊的, 对于任意 $P:\type$, 类型 $\isprop(P)$ 被定义为
\[ \isprop(P) \defeq \prd{x,y:P} (x=y). \]
因此, 假定 ``$P$ 是纯命题'' 意味着展示 $\isprop(P)$ 的一个居留元, 即通过路径连接任意两个 $P$ 的元素的依值函数.
该函数的连续性/自然性暗示了不仅 $P$ 的任意两个元素都相等, 而且 $P$ 也不存在高阶同伦.

\begin{lem}\label{thm:inhabprop-eqvunit}
  如果 $P$ 是纯命题而且 $x_0:P$, 那么 $\eqv P \unit$.
\end{lem}
\begin{proof}
  定义 $f:P\to\unit$ 为 $f(x)\defeq \ttt$, 而 $g:\unit\to P$ 为 $g(u)\defeq x_0$.
  然后使用下面的引理, 其中根据 \cref{thm:path-unit}, \unit 是纯命题.
\end{proof}

\begin{lem}\label{lem:equiv-iff-hprop}
  如果 $P$ 和 $Q$ 是纯命题而 $P\to Q$ 且 $Q\to P$, 那么 $\eqv P Q$.
\end{lem}
\begin{proof}
  提供 $f:P\to Q$ 和 $g:Q\to P$.
  对于任何 $x:P$, 有 $g(f(x))=x$ 因为 $P$ 是纯命题.
  类似地, 对于任何 $y:Q$ 有 $f(g(y))=y$ 因为 $Q$ 是纯命题; 因此 $f$ 准逆于 $g$.
\end{proof}

也就是正如 \cref{sec:pat} 中约定的一样, 如果两个纯命题之间逻辑上等价, 那么它们等价.

同伦论中, 同伦等价于 \unit 的空间是\emph{可缩的}.
因此, 任何被居留的纯命题都是可缩的 (参见 \cref{sec:contractibility}).
另外, 不被居留的类型 \emptyt 也是 (空洞的) 纯命题.
至少在经典\index{数学!经典的}数学中, 只有这两种可能.

纯命题也被称为\emph{子终止对象} (范畴学上), \emph{单元素子集} (集合论上), 或者\emph{h-propositions}.
\indexsee{对象!子终结}{纯命题}%
\indexsee{子终结对象}{纯命题}%
\indexsee{子终结}{纯命题}%
\indexsee{h-proposition}{纯命题}
按照 \cref{sec:basics-sets} 中的讨论, 应该把它们称为 \emph{$(-1)$-类型}; 在 \cref{cha:hlevels} 会返回到它.
形容词 ``纯'' 强调尽管可以将任何类型视为命题 (通过给出居留元来证明它), 作为纯命题的类型, 只有将其视为命题, 否则没有意义: 它为真的见证不包含额外信息.

注意类型 $A$ 是一个集合当且仅当对于所有 $x,y:A$, 恒等类型 $\id[A]xy$ 是纯命题.
另一方面, 通过复制和简化 \cref{thm:isset-is1type} 的证明, 有:

\begin{lem}\label{thm:prop-set}
  每个纯命题都是集合.
\end{lem}
\begin{proof}
  假设 $f:\isprop(A)$; 因此对于所有 $x,y:A$ 有 $f(x,y):x=y$.
  固定 $x:A$ 并定义 $g(y)\defeq f(x,y)$.
  然后对于任何 $y,z:A$ 且 $p:y=z$ 有 $\apd g p : \trans{p}{g(y)}={g(z)}$.
  于是通过 \cref{cor:transport-path-prepost}, 有 $g(y)\ct p = g(z)$, 也就是说 $p=\opp{g(y)}\ct g(z)$.
  因此, 对于所有 $p,q:x=y$, 有 $p = \opp{g(x)}\ct g(y) = q$.
\end{proof}

特别地, 这意味着:

\begin{lem}\label{thm:isprop-isprop}\label{thm:isprop-isset}
  对于任何类型 $A$, 类型 $\isprop(A)$ 和 $\isset(A)$ 都是纯命题.
\end{lem}
\begin{proof}
  假设 $f,g:\isprop(A)$.
  通过函数外延性, 为了展示 $f=g$ 只需展示 $f(x,y)=g(x,y)$ 对于任何 $x,y:A$.
  不过 $f(x,y)$ 和 $g(x,y)$ 都是 $A$ 中的路径, 而它们相等, 因为根据 $f$ 或 $g$, $A$ 是纯命题, 因此根据 \cref{thm:prop-set} 它是集合.
  类似地, 假设 $f,g:\isset(A)$, 也就是说对于所有 $a,b:A$ 和 $p,q:a=b$, 有 $f(a,b,p,q):p=q$ 和 $g(a,b,p,q):p=q$.
  因为 $A$ 是集合 (根据 $f$ 或 $g$), 所以也是 1-阶类型, 于是有 $f(a,b,p,q)=g(a,b,p,q)$;
  因此根据函数外延性 $f=g$.
\end{proof}

以前的已经见过了另一个例子: \cref{sec:basics-equivalences} 中的判断~\ref{item:be3} 假定对于任何函数 $f$, 类型 $\isequiv (f)$ 应该是纯命题.

\index{逻辑!,纯命题的|)}%
\index{纯命题)}%


\section{经典逻辑和直觉主义逻辑}
\label{sec:intuitionism}
\index{逻辑!构造性与经典|(}%
\index{否定|(}%
掌握了纯命题的概念, 现在可以给同伦类型论中给出\define{排中律}的正确公式:
\indexdef{排中}%
\indexsee{公理!排中}{排中}%
\indexsee{法则!排中律}{排中法则}%
\begin{equation}
    \label{eq:lem}
    \LEM{}\;\defeq\;
    \prd{A:\UU} \Big(\isprop(A) \to (A + \neg A)\Big).
\end{equation}
类似地, \define{双重否定法则}%
\indexdef{双重否定, 法则}%
\indexdef{公理!双重否定}%
\indexdef{法则!双重否定}%
是
\begin{equation}
    \label{eq:ldn}
    % \mathsf{DN}\;\defeq\;
    \prd{A:\UU} \Big(\isprop(A) \to (\neg\neg A \to A)\Big).
\end{equation}
很容易发现它们两者之间等价---参见 \cref{ex:lem-ldn}---所以从现在开始统称为 \LEM{}.

公式 \LEM{} 避免了\cref{thm:not-dneg,thm:not-lem}的``悖论'', 因为 \bool 不是一个纯命题.
为了将其从命题作为类型公式区分它, 重新命名为:
\symlabel{lem-infty}
\begin{equation*}
    \LEM\infty \defeq \prd{A:\UU} (A + \neg A).
\end{equation*}
强调, 正确的版本~\eqref{eq:lem}
被记作 $\LEM{-1}$;
参见 \cref{ex:lemnm}.
尽管 $\LEM{}$ 不是在 \cref{cha:typetheory}讨论的基本类型论的推论, 它也可以一致地作为一个公理被假设 (不像它的 $\infty$-版本).
例如, 在 \cref{sec:wellorderings} 中会假设它.

不过, 意外的是不使用 \LEM{}, 我们能走多远.
很常见的是, 简单地重新将定义和定理重新表示可以避免使用排中律.
尽管需要一些时间来适应, 通常是值得的, 从而得到更优雅和通用的证明.
导论中讨论了它的优势.

例如, 经典\index{数学!经典}数学中, 不必要的双重否定被频繁使用.
一个非常简单的例子是假设普通集合 $A$ ``非空'', 字面意思就是, $A$ \emph{无法}\emph{没有}元素.
其实就是 $A$  存在至少一个元素, 通过移除双重否懂, 让语句不再依赖 \LEM{}.
回顾之前说 $A$ \emph{被居留}%
\index{被居留类型}%
当假定 $A$ 是命题 (即构造 $A$ 的元素, 通常无需命名).
因此, 通常转换经典的证明为构造主义逻辑, 替换 ``非空'' 为 ``被巨留'' (尽管有时必须替换为 ``被纯居留''; 参见 \cref{subsec:prop-trunc}).

类似地, 经典数学里通常不会根据矛盾做非必要的证明.
\index{证明!根据矛盾}%
当然, 经典的通过矛盾证明的方法是使用双重否定: 假设 $\neg A$ 可以推导出一个矛盾, 就能推导出 $\neg \neg A$, 并通过双重否定得到 $A$.
不过, 从 $\neg A$ 派生的矛盾通常可以改成直接产生 $A$ 的证明, 不需要使用 \LEM{}.

值得注意的是, 如果是为了证明 \emph{否定}\index{否定}, 那么 ``根据矛盾证明'' 不会调用 \LEM{}.
实际上, 因为 $\neg A$ 被定义为类型 $A\to\emptyt$, 根据这个定义, 证明 $\neg A$ 就是假设 $A$ 成立, 然后证明矛盾 (\emptyt).
类似地, 双重否定法则对于否定命题也成立: $\neg\neg\neg A \to \neg A$.
通过实践, 可以学习如何更小心的区别否定命题和非否定命题, 并注意何时该包含 \LEM{} 而何时不该.

因此, 和表面相反, ``可构造性'' 的数学 通常不会包含不会抛弃重要的定理, 而是找到最好的方式来这个定义, 让这个重要的定理构造式的证明.
也就是, 可以在初次研究一个主题是使用 \LEM{}, 不过一旦有更好的解释, 就希望能改进这个证明来避免这个公理.
% For instance, the theory of ordinal numbers, which classically makes heavy use of \LEM{}, works quite well constructively once we choose the correct definition of ``ordinal''; see \cref{sec:ordinals}.
这种观点在\emph{同伦}类型论中更加明显, 其中强大的单值和高阶归纳类型让我们能够构造式地处理很多传统上需要经典推理的问题\index{数学!经典的}.
在 \cref{part:mathematics} 中有一些例子.
% For instance, none of the ``synthetic'' homotopy theory we will develop in \cref{cha:homotopy} requires \LEM{} --- despite the fact that classical homotopy theory (formulated using topological spaces or simplicial sets) makes heavy use of them (as well as the axiom of choice).

同样值得注意的是, 构造主义数学中, 排中律对于\emph{某些}命题是成立的.
它们在传统上被命名为\emph{可判定}.

\begin{defn}
    \label{defn:decidable-equality}
    \mbox{}
    \begin{enumerate}
        \item 类型 $A$ 被称为 \define{可判定}
        \indexdef{可判定!类型}%
        \indexdef{类型!可判定}%
        如果 $A+\neg A$.
        \item 类似地, 类型族 $B:A\to \type$ \define{可判定}
        \indexdef{可判定!类型族}%
        \indexdef{类型!族!可判定}%
        如果 \narrowequation{\prd{a:A} (B(a)+\neg B(a)).} \label{item:decidable-equality2}
        \item 特别地, $A$ \define{可判定等同}
        \indexdef{可判定!等同}%
        \indexsee{等同!可判定}{可判定等同}%
        如果 \narrowequation{\prd{a,b:A} ((a=b) + \neg(a=b)).}
    \end{enumerate}
\end{defn}

因此, $\LEM{}$ 本质就是所有的纯命题可判定, 以及所有纯命题的类型族.
特别地, $\LEM{}$ 暗示了所有集合 (从 \cref{sec:basics-sets} 的意义上说) 都可判定等同.
在这个意义上, 具备可判定等同是非常强大的; 参见 \cref{thm:hedberg}.

\index{拒绝|)}%
\index{逻辑!构造主义和经典的|)}%


\section{Subsets and propositional resizing}
\label{subsec:prop-subsets}
\index{纯命题(}%

作为纯命题的另一个有用的例子, 我们来讨论子集合 (并推广到子类型).
假设 $P:A\to\type$ 是类型族, 每个类型 $P(x)$ 被视为一个命题.
然后 $P$ 本身就是 $A$ 上的一个\emph{断言}, 或者 $A$ 的元素的一个 \emph{性质} .

集合论中, 只要在集合 $A$ 上有断言 $P$ 时, 就可以构造子集合 $\setof{x\in A | P(x)}$.
正如 \cref{sec:pat} 中提起的一样, 类型论中对应的就是 $\Sigma$-类型 $\sm{x:A} P(x)$.
$\sm{x:A} P(x)$ 的居留元显然就是 $(x,p)$, 其中 $x:A$ 而 $p$ 是 $P(x)$ 的一个证明.
不过, 对于更通用的 $P$, 元素 $a:A$ 可以可以对应 $\sm{x:A} P(x)$ 的多个元素, 如果命题 $P(a)$ 有超过一个证明.
这与常规的子集合直觉相反.
不过 $P$ 是\emph{纯}命题, 这将不会发生.

\begin{lem}
    \label{thm:path-subset}
    假设 $P:A\to\type$ 是类型族, 那么对于所有 $x:A$, $P(x)$ 是纯命题.
    如果 $u,v:\sm{x:A} P(x)$ 满足 $\proj1(u) = \proj1(v)$, 那么 $u=v$.
\end{lem}
\begin{proof}
    假设 $p:\proj1(u) = \proj1(v)$.
    根据 \cref{thm:path-sigma}, 为了得到 $u=v$, 展示 $\trans{p}{\proj2(u)} = \proj2(v)$ 即可.
    而 $\trans{p}{\proj2(u)}$ 和 $\proj2(v)$ 都是 $P(\proj1(v))$ 的元素, 而它是纯命题; 因此它们相等.
\end{proof}

例如, 回顾 \cref{sec:basics-equivalences}, 定义了
\[(\eqv A B) \;\defeq\; \sm{f:A\to B} \isequiv (f),\]
其中每个类型 $\isequiv (f)$ 都被假设为了纯命题.
因此如果两个等价内部的函数都相等, 那么这两个等价彼此也相等.

\label{defn:setof}%
现在开始, 如果 $P:A\to \type$ 是纯命题的类型族 (即每个 $P(x)$ 都是纯命题), 可以将
%
\begin{equation}
    \label{eq:subset}
    \setof{x:A | P(x)}
\end{equation}
%
用另外的符号 $\sm{x:A} P(x)$ 代替.
(技术上没有理由限制对 $P$ 使用这个符号, 不过这种无意识的做法可能会带来困扰.)
如果 $A$ 是集合, 可以将\eqref{eq:subset}称为 $A$ 的\define{子集合};
\indexdef{子集合}%
\indexdef{类型!子集合}%
作为推广可以把 $A$ 称为\define{子类型}.
\indexdef{子类型}%
也可以将 $P$ 称为 $A$ 的\emph{子集合}或\emph{子类型};
实际上,这更正确,因为孤立地看,类型~\eqref{eq:subset}不记得它和 $A$ 的关系.

\symlabel{membership}
给定 $P$ 和 $a:A$, 可以写作 $a\in P$ 或 $a\in \setof{x:A | P(x)}$ 来表示纯命题 $P(a)$.
如果它成立, 那可以说 $a$ 是 $P$ 的\define{成员}.
\symlabel{subset}
类似地, 如果 $\setof{x:A | Q(x)}$ 是 $A$ 的另一个集合, 且 $\prd{x:A}(P(x)\rightarrow Q(x))$, 那么可以说 $P$ \define{包含于}%
\indexdef{包含关系!子集合}%
$Q$ 中, 并写作 $P\subseteq Q$.

进一步地, 可以在全集 \UU 中, 定义集合与纯命题的 ``子全集'':
\symlabel{setU}\symlabel{propU}
\begin{align*}
    \setU &\defeq \setof{A:\UU | \isset(A) },\\
    \propU &\defeq \setof{A:\UU | \isprop(A) }.
\end{align*}
$\setU$ 的元素是类型 $A:\UU$ 以及证明 $s:\isset(A)$, 而 $\propU$ 也类似.
\cref{thm:path-subset} 中暗示了 $\id[\setU]{(A,s)}{(B,t)}$ 等价于 $\id[\UU]AB$ (因此有 $\eqv AB$).
因此, 我们会经常滥用记号, 用 $A:\setU$ 代表 $(A,s):\setU$.
有时也会省略下标 \UU, 如果不需要具体说明问题中的全集.

回顾对于任意两个全集 $\UU_i$ 和 $\UU_{i+1}$, 如果 $A:\UU_i$ 那么 $A:\UU_{i+1}$.
因此, 对于任意 $(A,s):\set_{\UU_i}$ 同样有 $(A,s):\set_{\UU_{i+1}}$, 而 $\prop_{\UU_i}$ 也类似, 得到一个自然的映射
\begin{align}
    \set_{\UU_i} &\to \set_{\UU_{i+1}}\label{eq:set-up},\\
    \prop_{\UU_i} &\to \prop_{\UU_{i+1}}.\label{eq:prop-up}
\end{align}
该映射~\eqref{eq:set-up}不是等价的, 否则可以产生类似于 Cantorian 集合论的自指的悖论\index{悖论}.
不过, 尽管~\eqref{eq:prop-up} 在我们提出的类型论中暂时还不是天然的等价, 不过它是一致的.
也就是说, 可以考虑在类型论中添加这样的公理.

\begin{axiom}[命题缩放]
    \indexsee{公理!命题缩放}{命题缩放}%
    \indexdef{命题!缩放}%
    \indexsee{缩放!命题}{命题缩放}%
    映射 $\prop_{\UU_i} \to \prop_{\UU_{i+1}}$ 是等价.
\end{axiom}

这个公理被称为\define{命题缩放},
因为它意味着全集 $\UU_{i+1}$ 的任何纯命题可以 ``缩小'' 到等价的更小的全集 $\UU_i$.
它自动地成立, 前提是 $\UU_{i+1}$ 满足 \LEM{} (参见 \cref{ex:lem-impred}).
我们不会在所有场景假设这个公理成立, 尽管在某些地方会将它作为明确的假设.
这是纯命题\emph{非直谓}的构造, 通过避免使用它, 类型论可以保留\emph{直谓性}.
\indexsee{非直谓!纯命题的}{直谓的缩放}%
\indexdef{数学!直谓性}%

实践中, 我们最想要的是一个不同的陈述: 考虑某全集 \UU 包含类型 ``classifies all mere propositions''.
换句话说, 我们想要一个类型 $\Omega:\UU$, 以及以 $\Omega$-为索引的纯命题类型族, 其中包含所有纯命题的等价类.
这个陈述遵循之前的命题缩放, 如果 $\UU$ 不是最小的全集 $\UU_0$, 因为可以定义 $\Omega\defeq \prop_{\UU_0}$.

非直谓性的一个用处是定义幂集合.
可以自然地定义 $A$ 的\define{幂集合}\indexdef{幂集合}为 $A\to\propU$;
不过如果缺少非直谓性, 这个定义依赖 (甚至是等价于) 宇宙的算则 \UU.
不过根据命题缩放, 可以定义幂集合为
\symlabel{powerset}%
\[ \power A \defeq (A\to\Omega),\]
这样就不依赖 $\UU$.
参见 \cref{subsec:piw}.



\section{The logic of mere propositions}
\label{subsec:logic-hprop}
\index{logic!of mere propositions|(}%
We mentioned in \cref{sec:types-vs-sets} that in contrast to type theory, which has only one basic notion (types), set-theoretic foundations have two basic notions: sets and propositions.
Thus, a classical\index{mathematics!classical} mathematician is accustomed to manipulating these two kinds of objects separately.

It is possible to recover a similar dichotomy in type theory, with the role of the set-theoretic propositions being played by the types (and type families) that are \emph{mere} propositions.
In many cases, the logical connectives and quantifiers can be represented in this logic by simply restricting the corresponding type-former to the mere propositions.
Of course, this requires knowing that the type-former in question preserves mere propositions.

\begin{eg}
  If $A$ and $B$ are mere propositions, so is $A\times B$.
  This is easy to show using the characterization of paths in products, just like \cref{thm:isset-prod} but simpler.
  Thus, the connective ``and'' preserves mere propositions.
\end{eg}

\begin{eg}\label{thm:isprop-forall}
  If $A$ is any type and $B:A\to \type$ is such that for all $x:A$, the type $B(x)$ is a mere proposition, then $\prd{x:A} B(x)$ is a mere proposition.
  The proof is just like \cref{thm:isset-forall} but simpler: given $f,g:\prd{x:A} B(x)$, for any $x:A$ we have $f(x)=g(x)$ since $B(x)$ is a mere proposition.
  But then by function extensionality, we have $f=g$.

  In particular, if $B$ is a mere proposition, then so is $A\to B$ regardless of what $A$ is.
  In even more particular, since \emptyt is a mere proposition, so is $\neg A \jdeq (A\to\emptyt)$.
  \index{quantifier!universal}%
  Thus, the connectives ``implies'' and ``not'' preserve mere propositions, as does the quantifier ``for all''.
\end{eg}

On the other hand, some type formers do not preserve mere propositions.
Even if $A$ and $B$ are mere propositions, $A+B$ will not in general be.
For instance, \unit is a mere proposition, but $\bool=\unit+\unit$ is not.
Logically speaking, $A+B$ is a ``purely constructive'' sort of ``or'': a witness of it contains the additional information of \emph{which} disjunct is true.
Sometimes this is very useful, but if we want a more classical sort of ``or'' that preserves mere propositions, we need a way to ``truncate'' this type into a mere proposition by forgetting this additional information.

\index{quantifier!existential}%
The same issue arises with the $\Sigma$-type $\sm{x:A} P(x)$.
This is a purely constructive interpretation of ``there exists an $x:A$ such that $P(x)$'' which remembers the witness $x$, and hence is not generally a mere proposition even if each type $P(x)$ is.
(Recall that we observed in \cref{subsec:prop-subsets} that $\sm{x:A} P(x)$ can also be regarded as ``the subset of those $x:A$ such that $P(x)$''.)



\section{Propositional truncation}
\label{subsec:prop-trunc}
\index{截断!命题|(defstyle}%
\indexsee{type!squash}{truncation, propositional}%
\indexsee{squash type}{truncation, propositional}%
\indexsee{bracket type}{truncation, propositional}%
\indexsee{type!bracket}{truncation, propositional}%
\emph{命题截断}, 也被称为 \emph{$(-1)$-截断}, \emph{bracket type}, 或 \emph{挤压类型}, 是一种额外的类型构造器, 它把类型 ``挤压'' 或 ``截断'' 为一个纯命题, 遗忘该类型的居留元除了它们是否存在外的信息.

准确地说, 对于任何类型 $A$, 有类型 $\brck{A}$.
它有两个构造器:
\begin{itemize}
    \item 对于 $a:A$ 有 $\bproj a : \brck A$.
    \item 对于任何 $x,y:\brck A$, 有 $x=y$.
\end{itemize}
第一个构造器意味着若 $A$ 被居留, 那么 $\brck A$ 也如此.
第二个确定了 $\brck A$ 是纯命题; 我们通常不会为这个事实的证明指定名字.

\index{递归原理!截断的}%
$\brck A$ 的递归原理表示:
\begin{itemize}
    \item 如果 $B$ 是纯命题而且有 $f:A\to B$, 那么存在一个归纳函数 $g:\brck A \to B$ 对于所有 $a:A$ 满足 $g(\bproj a) \jdeq f(a)$.
\end{itemize}
换句话说, 任何纯命题, 如果可以从 $A$ (被居留) 可以推导出来, 那么也可以从 $\brck A$ 推导出来.
因此, $\brck A$, 作为一个纯命题, 除了 $A$ 被居留外的任何信息都不存在.
($\brck A$ 的归纳原理也是存在的, 但是并没有更加有用; 参见 \cref{ex:prop-trunc-ind}.)

在 \cref{ex:lem-brck,ex:impred-brck,sec:hittruncations} 会描述一些构造 $\brck{A}$ 的更一般的方法.
目前, 仅仅是假定为与 \cref{cha:typetheory} 的规则并列的额外规则.

根据命题截断, 可以将 ``纯命题的逻辑'' 扩展到析取和存在量词.
也就是说, $\brck{A+B}$ 是 ``$A$ 或 $B$'' 的纯命题版本, 它不会 ``记住'' 那个析取项为真的信息.

截断的递归原则意味着\emph{当试图证明一个纯粹命题时}, 我们仍然可以对 $\brck{A+B}$ 进行情况分析.
也就是, 假设有 $u:\brck{A+B}$ 而试图证明纯命题 $Q$.
换句话说, 试图定义 $\brck{A+B} \to Q$ 的元素.
由于 $Q$  是纯命题, 根据命题截断的递归原则, 定义元素 $A+B\to Q$ 即可.
但现在我们可以在 $A+B$ 上进行情况分析.

类似地, 对于类型族 $P:A\to\type$, 可以考虑 $\brck{\sm{x:A} P(x)}$, 它是 ``存在 $x:A$ 满足 $P(x)$'' 的纯命题版本.
关于析取, 结合截断和 $\Sigma$-类型的归纳原则, 如果有类型为 $\brck{\sm{x:A} P(x)}$ 的假设, \emph{在试图证明一个纯粹命题时} 可以引入新的假设 $x:A$ 和 $y:P(x)$ .
换句话说, 如果知道存在某个 $x:A$ 使得 $P(x)$ 成立, 但我们手头没有特定的 $x$, 可以自由使用这样的 $x$, 只要不试图构造可能依赖于特定 $x$ 的内容.
要求目标类型是一个纯粹命题表达了结果与证明的独立性, 因为这样类型的所有可能成员都必须相等.

在 \cref{cha:real-numbers,cha:set-math} 中主要处理集合和纯粹命题的集合水平数学, 因此使用传统的逻辑符号来表示 ``命题截断逻辑'' 会更加方便.

\begin{defn}
    \label{defn:logical-notation}
    使用截断定义 \define{传统逻辑符号}
    \indexdef{implication}%
    \indexdef{traditional logical notation}%
    \indexdef{logical notation, traditional}%
    \index{量词}%
    \indexsee{existential quantifier}{quantifier, existential}%
    \index{量词!存在}%
    \indexsee{universal!quantifier}{quantifier, universal}%
    \index{量词!全称}%
    \indexdef{conjunction}%
    \indexdef{disjunction}%
    \indexdef{true}%
    \indexdef{false}%
    如下, 其中 $P$ 和 $Q$ 代表纯命题 (或它们的族):
        {\allowdisplaybreaks
    \begin{align*}
        \top            &\ \defeq \ \unit \\
        \bot            &\ \defeq \ \emptyt \\
        P \land Q       &\ \defeq \ P \times Q \\
        P \Rightarrow Q &\ \defeq \ P \to Q \\
        P \Leftrightarrow Q &\ \defeq \ P = Q \\
        \neg P          &\ \defeq \ P \to \emptyt \\
        P \lor Q        &\ \defeq \ \brck{P + Q} \\
        \fall{x : A} P(x) &\ \defeq \ \prd{x : A} P(x) \\
        \exis{x : A} P(x) &\ \defeq \ \Brck{\sm{x : A} P(x)}
    \end{align*}}
\end{defn}

符号 $\land$ 和 $\lor$ 在同伦论中也用于表示点空间的粘合积和楔积, 在 \cref{cha:hits} 中会介绍它们.
技术上这可能会引发潜在的冲突, 但通常不会导致混淆.

类似地, 在讨论 \cref{subsec:prop-subsets} 时, 可以使用传统的符号表示交集, 并集, 和补集:
\indexdef{intersection!of subsets}%
\symlabel{intersection}%
\indexdef{union!of subsets}%
\symlabel{union}%
\indexdef{complement, of a subset}%
\symlabel{complement}%
\begin{align*}
    \setof{x:A | P(x)} \cap \setof{x:A | Q(x)}
    &\defeq \setof{x:A | P(x) \land Q(x)},\\
    \setof{x:A | P(x)} \cup \setof{x:A | Q(x)}
    &\defeq \setof{x:A | P(x) \lor Q(x)},\\
    A \setminus \setof{x:A | P(x)}
    &\defeq \setof{x:A | \neg P(x)}.
\end{align*}
当然, 当没有排中律 \LEM{} 时, 后者不是通常意义上的 ``补集'': 不一定对于集合 $A$ 中的每个子集 $B$ 都有 $B \cup (A\setminus B) = A$.

\index{截断!命题|)}%
\index{纯命题|)}%
\index{逻辑!纯命题的|)}%



\section{The axiom of choice}
\label{sec:axiom-choice}
\index{axiom!of choice|(defstyle}%
\index{denial|(}%
We can now properly formulate the axiom of choice in homotopy type theory.
Assume a type $X$ and type families
%
\begin{equation*}
  A:X\to\type
  \qquad\text{and}\qquad
  P:\prd{x:X} A(x)\to\type,
\end{equation*}
%
and moreover that
\begin{itemize}
\item $X$ is a set,
\item $A(x)$ is a set for all $x:X$, and
\item $P(x,a)$ is a mere proposition for all $x:X$ and $a:A(x)$.
\end{itemize}
The \define{axiom of choice}
$\choice{}$ asserts that under these assumptions,
\begin{equation}\label{eq:ac}
  \Parens{\prd{x:X} \Brck{\sm{a:A(x)} P(x,a)}}
  \to
  \Brck{\sm{g:\prd{x:X} A(x)} \prd{x:X} P(x,g(x))}.
\end{equation}
Of course, this is a direct translation of~\eqref{eq:english-ac} where we read ``there exists $x:A$ such that $B(x)$'' as $\brck{\sm{x:A}B(x)}$, so we could have written the statement in the familiar logical notation as
\begin{narrowmultline*}
  \textstyle
  \Big(\fall{x:X}\exis{a:A(x)} P(x,a)\Big)
  \Rightarrow \narrowbreak
  \Big(\exis{g : \prd{x:X} A(x)} \fall{x : X} P(x,g(x))\Big).
\end{narrowmultline*}
%
In particular, note that the propositional truncation appears twice.
The truncation in the domain means we assume that for every $x$ there exists some $a:A(x)$ such that $P(x,a)$, but that these values are not chosen or specified in any known way.
The truncation in the codomain means we conclude that there exists some function $g$, but this function is not determined or specified in any known way.

In fact, because of \cref{thm:ttac}, this axiom can also be expressed in a simpler form.

\begin{lem}\label{thm:ac-epis-split}
  The axiom of choice~\eqref{eq:ac} is equivalent to the statement that for any set $X$ and any $Y:X\to\type$ such that each $Y(x)$ is a set, we have
  \begin{equation}
    \Parens{\prd{x:X} \Brck{Y(x)}}
    \to
    \Brck{\prd{x:X} Y(x)}.\label{eq:epis-split}
  \end{equation}
\end{lem}

This corresponds to a well-known equivalent form of the classical\index{mathematics!classical} axiom of choice, namely ``the cartesian product of a family of nonempty sets is nonempty''.

\begin{proof}
  By \cref{thm:ttac}, the codomain of~\eqref{eq:ac} is equivalent to
  \[\Brck{\prd{x:X} \sm{a:A(x)} P(x,a)}.\]
  Thus,~\eqref{eq:ac} is equivalent to the instance of~\eqref{eq:epis-split} where \narrowequation{Y(x) \defeq \sm{a:A(x)} P(x,a).}
  (This is a set by \cref{thm:isset-prod,thm:prop-set}.)
  Conversely,~\eqref{eq:epis-split} is equivalent to the instance of~\eqref{eq:ac} where $A(x)\defeq Y(x)$ and $P(x,a)\defeq\unit$.
  Thus, the two are logically equivalent.
  Since both are mere propositions, by \cref{lem:equiv-iff-hprop} they are equivalent types.
\end{proof}

As with \LEM{}, the equivalent forms~\eqref{eq:ac} and~\eqref{eq:epis-split} are not a consequence of our basic type theory, but they may consistently be assumed as axioms.

\begin{rmk}
  It is easy to show that the right side of~\eqref{eq:epis-split} always implies the left.
  Since both are mere propositions, by \cref{lem:equiv-iff-hprop} the axiom of choice is also equivalent to asking for an equivalence
  \[ \eqv{\Parens{\prd{x:X} \Brck{Y(x)}}}{\Brck{\prd{x:X} Y(x)}} \]
  This illustrates a common pitfall: although dependent function types preserve mere propositions (\cref{thm:isprop-forall}), they do not commute with truncation: $\brck{\prd{x:A} P(x)}$ is not generally equivalent to $\prd{x:A} \brck{P(x)}$.
  The axiom of choice, if we assume it, says that this is true \emph{for sets}; as we will see below, it fails in general.
\end{rmk}

The restriction in the axiom of choice to types that are sets can be relaxed to a certain extent.
For instance, we may allow $A$ and $P$ in~\eqref{eq:ac}, or $Y$ in~\eqref{eq:epis-split}, to be arbitrary type families; this results in a seemingly stronger statement that is equally consistent.
We may also replace the propositional truncation by the more general $n$-truncations to be considered in \cref{cha:hlevels}, obtaining a spectrum of axioms $\choice n$ interpolating between~\eqref{eq:ac}, which we call simply \choice{} (or $\choice{-1}$ for emphasis), and \cref{thm:ttac}, which we shall call $\choice\infty$.
See also \cref{ex:acnm,ex:acconn}.
However, observe that we cannot relax the requirement that $X$ be a set.

\begin{lem}\label{thm:no-higher-ac}
  There exists a type $X$ and a family $Y:X\to \type$ such that each $Y(x)$ is a set, but such that~\eqref{eq:epis-split} is false.
\end{lem}
\begin{proof}
  Define $X\defeq \sm{A:\type} \brck{\bool = A}$, and let $x_0 \defeq (\bool, \bproj{\refl{\bool}}) : X$.
  Then by the identification of paths in $\Sigma$-types, the fact that $\brck{A=\bool}$ is a mere proposition, and univalence, for any $(A,p),(B,q):X$ we have $\eqv{(\id[X]{(A,p)}{(B,q)})}{(\eqv AB)}$.
  In particular, $\eqv{(\id[X]{x_0}{x_0})}{(\eqv \bool\bool)}$, so as in \cref{thm:type-is-not-a-set}, $X$ is not a set.

  On the other hand, if $(A,p):X$, then $A$ is a set; this follows by induction on truncation for $p:\brck{\bool=A}$ and the fact that $\bool$ is a set.
  Since $\eqv A B$ is a set whenever $A$ and $B$ are, it follows that $\id[X]{x_1}{x_2}$ is a set for any $x_1,x_2:X$, i.e.\ $X$ is a 1-type.
  In particular, if we define $Y:X\to\UU$ by $Y(x) \defeq (x_0=x)$, then each $Y(x)$ is a set.

  Now by definition, for any $(A,p):X$ we have $\brck{\bool=A}$, and hence $\brck{x_0 = (A,p)}$.
  Thus, we have $\prd{x:X} \brck{Y(x)}$.
  If~\eqref{eq:epis-split} held for this $X$ and $Y$, then we would also have $\brck{\prd{x:X} Y(x)}$.
  Since we are trying to derive a contradiction ($\emptyt$), which is a mere proposition, we may assume $\prd{x:X} Y(x)$, i.e.\ that $\prd{x:X} (x_0=x)$.
  But this implies $X$ is a mere proposition, and hence a set, which is a contradiction.
\end{proof}

\index{denial|)}%
\index{axiom!of choice|)}%


\section{The principle of unique choice}
\label{sec:unique-choice}
\index{unique!choice|(defstyle}%
\indexsee{axiom!of choice!unique}{unique choice}%

The following observation is trivial, but very useful.

\begin{lem}\label{thm:prop-equiv-trunc}
  If $P$ is a mere proposition, then $\eqv P {\brck P}$.
\end{lem}
\begin{proof}
  Of course, we have $P\to \brck{P}$ by definition.
  And since $P$ is a mere proposition, the universal property of $\brck P$ applied to $\idfunc[P] :P\to P$ yields $\brck P \to P$.
  These functions are quasi-inverses by \cref{lem:equiv-iff-hprop}.
\end{proof}

Among its important consequences is the following.

\begin{cor}[The principle of unique choice]\label{cor:UC}
  Suppose a type family $P:A\to \type$ such that
  \begin{enumerate}
  \item For each $x$, the type $P(x)$ is a mere proposition, and
  \item For each $x$ we have $\brck {P(x)}$.
  \end{enumerate}
  Then we have $\prd{x:A} P(x)$.
\end{cor}
\begin{proof}
  Immediate from the two assumptions and the previous lemma.
\end{proof}

The corollary also encapsulates a very useful technique of reasoning.
Namely, suppose we know that $\brck A$, and we want to use this to construct an element of some other type $B$.
We would like to use an element of $A$ in our construction of an element of $B$, but this is allowed only if $B$ is a mere proposition, so that we can apply the induction principle for the propositional truncation $\brck A$; the most we could hope to do in general is to show $\brck B$.
%
Instead, we can extend $B$ with additional data which characterizes \emph{uniquely} the object we wish to construct.
Specifically, we define a predicate $Q:B\to\type$ such that $\sm{x:B} Q(x)$ is a mere proposition.
Then from an element of $A$ we construct an element $b:B$ such that $Q(b)$, hence from $\brck A$ we can construct $\brck{\sm{x:B} Q(x)}$, and because $\brck{\sm{x:B} Q(x)}$ is equivalent to $\sm{x:B} Q(x)$ an element of $B$ may be projected from it.
An example can be found in \cref{ex:decidable-choice}.

A similar issue arises in set-theoretic mathematics, although it manifests slightly
differently. If we are trying to define a function $f: A \to B$, and depending on an
element $a : A$ we are able to prove mere existence of some $b : B$, we are not done yet
because we need to actually pinpoint an element of~$B$, not just prove its existence.
One option is of course to refine the argument to unique existence of $b : B$, as we did in type theory. But in set theory the problem can often be avoided more simply by an application of the axiom of choice, which picks the required elements for us.
In homotopy type theory, however, quite apart from any desire to avoid choice, the available forms of choice are simply less applicable, since they require that the domain of choice be a \emph{set}.
Thus, if $A$ is not a set (such as perhaps a universe $\UU$), there is no consistent form of choice that will allow us to simply pick an element of $B$ for each $a : A$ to use in defining $f(a)$.

\index{unique!choice|)}%



\section{When are propositions truncated?}
\label{subsec:when-trunc}
\index{逻辑!纯命题|(}%
\index{纯命题|(}%
\index{逻辑!截断}%

初看起来, $+$ 和 $\Sigma$ 的截断的版本, 比非截断的版本更接近``或''和``存在''在非正式数学中的含义.
当然, 截断版本更接近一阶逻辑\index{一阶!逻辑}中的``或''和``存在'' 的\emph{准确}含义, 因为一阶逻辑中并不试图记住命题为真的任何证据, 而一阶逻辑是形式集合论的基础.
然而, 令人惊讶的是, \emph{非正式}的数学的实践中, 往往使用非截断版本.

\index{素数}%
例如, 考虑这个陈述 ``素数要么是 $2$ 要么是奇数''.
数学家会在毫无顾忌地使用这一事实, 不仅是证明关于素数的\emph{定理}, 还用来\emph{构造}素数, 在素数是 $2$ 的情况下做一件事, 而在素数是奇数时做另一件事.
构造的结果不仅是某个陈述是对的, 还是一段可能依赖素数奇偶性的数据信息.
因此, 从类型论的视角, 这种构造自然地用余积类型 ``$(p=2)+(p\text{ 是奇数})$'' 表示, 而不是它的命题截断.

诚然, 这不是一个理想的例子, 因为 ``$p=2$'' 和 ``$p$ 是奇数'' 互不相交, 所以 $(p=2)+(p\text{ 是奇数})$ 实际上已经是纯命题, 因此它等于它的截断 (参见 \cref{ex:disjoint-or}).
更令人信服的例子来自于存在两次.
常见的做法是, 先证明一个形如 ``存在 $x$ 满足 \dots'' 的定理, 然后在后面引用它 ``该 $x$ 在定理 Y 中构建'' (注意定冠词的使用).
此外, 当进一步推导 $x$ 的性质时, 可能会使用 ``由证明 Y 时构造的 $x$'' 这样的短语.

一个非常常见的例子是 ``$A$ 同构于 $B$'', 严格意义上这仅代表存在\emph{某些} $A$ 和 $B$ 之间的同构.
但几乎在证明这样的陈述时, 总会展示一个特定的同构, 或是证明已知的映射是同构, 而且给出的这个同态往往很重要.

接受集合论训练的数学家通常对这种``语言滥用''感到一丝愧疚.\index{滥用!语言}
我们可能试图为此道歉, 从最终稿中删除这些表达, 或者用 ``典范'' 这样模糊的词来替代.
% 比如 Show that if $\UU_{i+1}$ satisfies \LEM{}, then the canonical inclusion $\prop_{\UU_i} \to \prop_{\UU_{i+1}}$ is an equivalence.
这个问题在形式化的集合论中更加严重, 因为技术上根本无法 ``构建'' 对象 --- 只能证明某个具有特定属性的对象存在.
因此, 类型论中的未截断逻辑能够更直接地反映非正式数学中的一些常见做法, 而这些做法在集合论的重构中往往被模糊掉了.
(这类似于单值公理如何使识别同构对象的常见做法合法化, 尽管这种做法缺乏正式的证明.)

一方面, 某些时候截断是必要的.
我们已经在 \LEM{} 和 \choice{} 的陈述中看到了这一点;
一些其他的例子会在本书的稍后出现.
因此, 我们面对一个问题: 当编写非正式类型论时, ``或'' 和 ``存在'' 这两个词应该意味什么 (以及 ``有'' 和 ``我们有'' 等常见同义词)?

达成普遍共识好像不太可能.
根据数学分类, 也许一种约定可能比另一种更有用 --- 也有可能无关紧要.
这种情况下, 数学论文开头的备注可能就将其使用的语言约定告知读者.
然而, 尽管选择了总体约定, 其他种类的逻辑也可能偶尔出现, 所以需要一种方法来引用它.
更通俗一些说, 可以考虑把命题截断替换为, 表现类似的其他的操作, 例如双重否定 $A\mapsto \neg\neg A$, 或者 \cref{cha:hlevels} 中出现的 $n$-截断.
作为一种表达的实验, 在接下来的内容中, 偶尔使用\emph{副词}\index{副词}来表示诸如类型截断 ``模态'' 的应用.

例如, 如果非截断逻辑是默认约定, 那么会使用副词\define{仅}%
\indexdef{仅}%
来表示截断命题.
因此短句
\begin{center}
    ``仅存在 $x:A$ 满足 $P(x)$''
\end{center}
表示类型 $\brck{\sm{x:A} P(x)}$.
类似地, 可以说类型 $A$ \define{仅被居留}%
\indexdef{仅!被居留}%
\indexdef{被居留类型!仅}%
来代表命题截断 $\brck A$ 被居留 (例如我们有该类型的未命名的元素).
注意这里对副词``仅''的\emph{定义}\index{定义!副词} 是为了在我们的非正式数学语言中使用, 就像我们定义有特定数学含义的词一样, 比如名词\index{名词} ``群'' 和 ``环'', 形容词\index{形容词} ``正规'' 和 ``正态''.
我们并没没有说字典里的 ``仅'' 的定义指的是命题截断;
选择这个词的意义只是提醒数学读者, 纯命题的包含``仅''一个真值的信息, 而没有其他(``纯命题'' 的英语是 ``mere proposition'', 而副词的英语是 ``merely'').

另一方面, 如果被截断逻辑是默认约定, 会使用副词\define{确实}%
\indexdef{确实}%
或 \define{构建性地} 来表示没有截断, 所以
\begin{center}
    ``确实存在 $x:A$ 满足 $P(x)$''
\end{center}
表示类型 $\sm{x:A} P(x)$.
即使在默认的情况下, 也可以使用 ``确实'' 来强调缺少截断.

本书会继续使用非截断逻辑作为默认约定, 有这些原因.
\begin{enumerate}[label=(\arabic*)]
    \item 我们希望鼓励新手去尝试它, 而不是仅仅因为更熟悉而坚持使用截断逻辑.
    \item 在类型论中使用截断逻辑作为默认会面临与集合论基础相同种类的 ``语言滥用''\index{滥用!语言}问题, 而非截断逻辑不会.
    例如, 定义 ``$\eqv A B$'' 作为 $A$ 和 $B$ 之间的等价类型, 而不是它的命题截断, 这代表证明定理 ``$\eqv A B$'' 字面上是构造一个具体的等价.
    这个具体的等价可以在之后引用
    \item 希望强调 ``纯命题'' 这个概念不是类型论的一部分.
    正如 \cref{cha:hlevels} 中所见的, 纯命题只是无限阶梯的第二层, 而且有许多其他模态根本不在这条阶梯上.
    \item 许多本来是纯命题的经典陈述, 在同伦类型论中, 不再是纯命题了.
    当然, 其中最重要的就是等同.
    \item 一方面, 同伦类型论最有趣的地方是, 很多类型居然\emph{自动地}就是纯命题, 或者只需稍加修改就能称为这样, 而不需要任何截断.
    (参见 \cref{thm:isprop-isprop,cha:equivalences,cha:hlevels,cha:category-theory,cha:set-math}.)
    因此, 尽管这些类型不存在超出命题为真的任何数据, 也可以用它来构造非截断的对象, 因为不需要使用命题截断的归纳原则.
    如果命题截断对于所有的语句都是默认, 会跟臃肿地表达这些有用的事实.
    \item 最后, 截断在本书的大部分数学中并不是很有用, 因此在发生时明确标记它们会更简单.
\end{enumerate}

\index{纯命题|)}%
\index{逻辑!纯命题|)}%


\section{Contractibility}
\label{sec:contractibility}
\index{类型!可缩的|(defstyle}%
\index{可缩的!类型|(defstyle}%

在 \cref{thm:inhabprop-eqvunit} 中可以观察到, 有居留元的纯命题必须等价于 $\unit$,
\index{type!unit}%
不难看到, 反过来也一样.
类型的这种性质叫做 \emph{可缩性}.
收缩性的另一个等效定义有时也很方便, 如下所示.

\begin{defn}
    \label{defn:contractible}
    对于类型 $A$, 如果有 $a:A$, ,
    \indexdef{中心!收缩}%
    对于所有 $x:A$ 满足 $a=x$.
    那么类型 $A$ 是 \define{可收缩的},
    或者 \define{单例},
    \indexdef{类型!单例}%
    \indexsee{单例类型}{类型, 单例}%
    $x:A$ 是它的 \define{收缩中心}.
    这个特定的路径 $a=x$ 记作 $\contr_x$.
\end{defn}

换句话说, 类型 $\iscontr(A)$ 的定义是
\[ \iscontr(A) \defeq \sm{a:A} \prd{x:A}(a=x). \]
注意从命题作为类型的视角来看, 可以将 $\iscontr(A)$ 读作 ``$A$ 确实存在一个元素'', 或更准确地 ``$A$ 存在一个元素, 而且所有 $A$ 的元素都彼此相等''.

\begin{rmk}
    也可以把 $\iscontr(A)$ 读作拓扑学上的 ``存在 $a:A$ 满足对于所有 $x:A$ 存在从 $a$ 到 $x$ 的路径''.
    对于熟悉经典数学概念的人来说, 这个定义听起来像是\emph{连通性}而不是可缩性.
    \index{类型论中的函数的连续性@类型论中的函数的``连续性''}%
    \index{类型论中的函数的函子性@类型论中的函数的``函子性''}%
    重点在于, 这句话中的 ``存在'' 是连续或者自然的.

    连通性可以更好地表示为 $\sm{a:A}\prd{x:A} \brck{a=x}$.
    当 $A$ 可以被视为是点时, 这是正确的 --- 参见 \cref{thm:connected-pointed} --- 但是对于更一般的类型, 没有点也可以是连通的.
    在 \cref{sec:connectivity} 将连通性定义为一般意义上$n$-连同的 $n=0$ 的情况, 而在 \cref{ex:connectivity-inductively} 中, 读者被要求证明这个定义等价于 $\brck{A}$ 且 $\prd{x,y:A} \brck{x=y}$.
\end{rmk}

\begin{lem}
    \label{thm:contr-unit}
    对于类型 $A$, 以下是逻辑上等价的.
    \begin{enumerate}
        \item $A$ 于 \cref{defn:contractible} 意义上是可缩的.\label{item:contr}
        \item $A$ 是纯命题,而且存在点 $a:A$.\label{item:contr-inhabited-prop}
        \item $A$ 等价于 \unit.\label{item:contr-eqv-unit}
    \end{enumerate}
\end{lem}
\begin{proof}
    如果 $A$ 是可缩的, 那么必然存在点 $a:A$ (即收缩的中心), 于是对于任意 $x,y:A$ 有 $x=a=y$; 因此 $A$ 是纯命题.
    相反, 如果有 $a:A$ 而且 $A$ 是纯命题, 那么多余任何 $x:A$ 有 $x=a$; 因此 $A$ 是可缩的.
    然后在 \cref{thm:inhabprop-eqvunit} 中有 ~\ref{item:contr-inhabited-prop}$\Rightarrow$\ref{item:contr-eqv-unit}, 而反方向易证 \unit 有性质~\ref{item:contr-inhabited-prop}.
\end{proof}

\begin{lem}
    \label{thm:isprop-iscontr}
    对于任何类型 $A$, 类型 $\iscontr(A)$ 是纯命题.
\end{lem}
\begin{proof}
    给定 $c,c':\iscontr(A)$.
    假设 $c\jdeq(a,p)$ 和 $c'\jdeq(a',p')$ 对于 $a,a':A$ 和 $p:\prd{x:A} (a=x)$ 和 $p':\prd{x:A} (a'=x)$.
    通过 $\Sigma$-类型中的路径中的特征, 证明 $c=c'$ 只需展示 $q:a=a'$ 满足 $\trans{q}{p}=p'$.
    %
    选择 $q\defeq p(a')$.
    现在因为 $A$ 是可缩的 (根据 $c$ 或 $c'$), 因为 \cref{thm:contr-unit} 是纯命题.
    因此, 通过 \cref{thm:prop-set,thm:isprop-forall}, 所以 $\prd{x:A}(a'=x)$; 因此 $\trans{q}{p}=p'$.
\end{proof}

\begin{cor}
    \label{thm:contr-contr}
    如果 $A$ 可缩, 那么 $\iscontr(A)$ 也是.
\end{cor}
\begin{proof}
    根据 \cref{thm:isprop-iscontr} 和 \cref{thm:contr-unit}\ref{item:contr-inhabited-prop}.
\end{proof}

就像纯命题, 可缩类型被很多类型构造器保留.
例如, 有:

\begin{lem}
    \label{thm:contr-forall}
    如果 $P:A\to\type$ 是类型族而且每个 $P(a)$ 都是可缩的, 那么 $\prd{x:A} P(x)$ 是可缩的.
\end{lem}
\begin{proof}
    根据 \cref{thm:isprop-forall}, 因为每个 $P(x)$ 都是纯命题, 所以 $\prd{x:A} P(x)$ 是纯命题.
    但是也有元素, 即函数发送每个 $x:A$ 到 $P(x)$ 收缩的中心.
    因此根据 \cref{thm:contr-unit}\ref{item:contr-inhabited-prop}, $\prd{x:A} P(x)$ 是可缩的.
\end{proof}

\index{函数外延性}%
(实际上, 语句 \cref{thm:contr-forall} 等价于函数外延性公理.
参见~\cref{sec:univalence-implies-funext}.)

当然, 如果 $A$ 等价于 $B$ 而且 $A$ 可缩, 那么 $B$ 也如此.
跟一般地, 它满足 $B$ 是 $A$ \emph{retract}.
根据定义, \define{收缩映射}
\indexdef{收缩映射}%
\indexdef{函数!收缩映射}%
是函数 $r : A \to B$ 满足存在 $s : B \to A$, 也就是 \define{截面},
\indexdef{截面}%
\indexdef{函数!截面}%
和同伦 $\epsilon:\prd{y:B} (r(s(y))=y)$; 那么 $B$ 是 $A$ 的 \define{收缩映射}%
\indexdef{收缩映射!类型}.

\begin{lem}
    \label{thm:retract-contr}
    如果 $B$ 是 $A$ 的收缩映射, 而且 $A$ 是可缩的, 那么 $B$ 也如此.
\end{lem}
\begin{proof}
    令 $a_0 : A$ 为收缩的中心.
    我们声称 $b_0 \defeq r(a_0) : B$ 是 $B$ 的收缩中心.
    令 $b : B$; 需要路径 $b = b_0$.
    而 $\epsilon_b : r(s(b)) = b$ 且 $\contr_{s(b)} : s(b) = a_0$, 所以通过组合
    \[ \opp{\epsilon_b} \ct \ap{r}{\contr_{s(b)}} : b = r(a_0) \jdeq b_0. \qedhere\]
\end{proof}

可缩类型看起来不是很有意思, 因为它们都等价于 \unit.
它们有用的一个原因是, 有时一组单个的平凡数据可以共同的构成一个可缩类型.
一个重要的例子是一些空间上的路径, 它们具有一个自由端点.
在 \cref{sec:identity-systems} 会看到, 这个事实本质上概括了恒等类型的基础路径归纳原理.


\begin{lem}
    \label{thm:contr-paths}
    对于任何 $A$ 和任何 $a:A$, 类型 $\sm{x:A} (a=x)$ 是可缩的.
\end{lem}
\begin{proof}
    选择点 $(a,\refl a)$ 作为中心.
    现在假设 $(x,p):\sm{x:A}(a=x)$;
    需要证明 $(a,\refl a) = (x,p)$.
    根据 $\Sigma$-类型中的路径的特征, 足以得到 $q:a=x$ 满足 $\trans{q}{\refl a} = p$.
    不过可以获取 $q\defeq p$, 这种情况下 $\trans{q}{\refl a} = p$ 从路径类型中的传输特征.
\end{proof}

这种情况发生时, 允许我们化简复杂的构造到等价, 非正式定理中可收缩的数据可以自由地省略.
这个定理包含很多引理, 大部分留给读者; 一下是一些例子.

\begin{lem}
    \label{thm:omit-contr}
    令 $P:A\to\type$ 为类型族.
    \begin{enumerate}
        \item 如果每个 $P(x)$ 是可缩的, 那么 $\sm{x:A} P(x)$ 等价于 $A$.\label{item:omitcontr1}
        \item 如果 $A$ 是可缩的, 而且中心是 $a$, 那么 $\sm{x:A} P(x)$ 等价于 $P(a)$.\label{item:omitcontr2}
    \end{enumerate}
\end{lem}
\begin{proof}
    在~\ref{item:omitcontr1}中, 我们证明了 $\proj1:\sm{x:A} P(x) \to A$ 是一个等价关系.
    对于准逆我们定义 $g(x)\defeq (x,c_x)$ 其中 $c_x$ 是 $P(x)$ 的中心.
    组合 $\proj1 \circ g$ 明显地 $\idfunc[A]$, 然而对应的组成同伦于很等, 通过使用每个 $P(x)$ 的可缩性.

    我们在 ~\ref{item:omitcontr2} 会将这个证明留给读者(参见 \cref{ex:omit-contr2}).
\end{proof}

另一个可缩类型有意思的理由是, 它们\cref{sec:basics-sets}中提到的 $n$-类型的阶梯向下移动了一步.

\begin{lem}
    \label{thm:prop-minusonetype}
    类型 $A$ 是纯命题当且仅当对于所有 $x,y:A$, 类型 $\id[A]xy$ 可缩.
\end{lem}
\begin{proof}
    对于 ``当'', 容易观察到任意类型都满足.
    对于 ``仅当'', 观察在 \cref{subsec:hprops} 中每个纯命题是集合, 所以每个类型 $\id[A]xy$ 是纯命题.
    但它也未被居留 (因为 $A$ 是纯命题), 因此通过 \cref{thm:contr-unit}\ref{item:contr-inhabited-prop} 它可缩.
\end{proof}

因此, 可缩类型也被叫做 \define{$(-2)$-类型}.
它们是 $n$-类型阶梯的底部, 而且在 \cref{cha:hlevels} 中会使递归定义 $n$-类型的基础情况.

\index{类型!可缩|)}%
\index{可缩!类型|)}%


\sectionNotes
VoevodskyThe 最先观察到, 类型论中可以定义集合, 纯命题, 和高阶可缩类型, 而且所有的高阶同伦都会被自动处理, 就像\cref{sec:basics-sets,subsec:hprops,sec:contractibility}中一样.
事实上, 他通过归纳定义了 $n$-类型全部层级, 参见 \cref{cha:hlevels}.\index{n-type@$n$-type!definable in type theory}%

\cref{thm:not-dneg,thm:not-lem} 本质上依赖于 Hedberg 的经典理论,
\index{Hedberg 的理论}%
\index{理论!Hedberg 的}%
我们在 \cref{sec:hedberg} 中证明了这个理论.
Mart\'\i n Escard\'o 在 \Agda 的邮件列表中发现到, 命题即类型形式中的 \LEM{} 和单值性之间存在矛盾.
这件事在 \cref{thm:not-dneg} 给出了证明, 它是由 Thierry Coquand 完成的.

命题截断在 1983 年被 Constable~\cite{Con85} 引入到 \index{证明!助手!\NuPRL}, 作为 ``子集'' 和 ``商'' 类型.
本书中的 ``命题截断'' 在 \NuPRL 类型论~\cite{constable+86nuprl-book} 中叫做 ``挤压''.
在~\cite{ab:bracket-types}中, 给出了外延类型论中, 能直接描述命题截断的规则.
Voevodsky 构造了同伦类型论中内延的版本, 使用了非直谓\index{非直谓!截断}的量词, 然后 Lumsdaine 使用高阶归纳类型 (参见 \cref{sec:hittruncations}).

\index{命题的!重调大小}%
Voevodsky~\cite{Universe-poly} 提出了 \cref{subsec:prop-subsets} 中讨论的重调大小规则 .
\index{公理!可约性}\index{重调大小}%
这显然与著名的 Russell 在他和 Whitehead 的 \emph{Principia Mathematica}~\cite{PM2} 书中的\emph{可约性公理}有关.\index{Russell, Bertrand}

副词 ``纯'' 用于指代未截断逻辑时, 用了单子模式在编程语言中用于建模的效果的用法;
参见 \cref{sec:modalities} 和 \cref{cha:hlevels} 的备注.

类型论中有很多方式来处理逻辑关系.
例如, 除了在 \cref{sec:pat} 描述的命题即类型逻辑, 以及 \cref{subsec:logic-hprop} 中仅使用纯命题的替代方案外, 还可以引入一种独立的命题 ``种类'', 它的行为与类型类似, 但和类型不完全相同.
这种方法被用于逻辑扩展类型理论~\cite{aczel2002collection}, 如拓扑斯\index{拓扑斯}及相关范畴的内部语言(例如~\cite{jacobs1999categorical,elephant}), 以及证明助手 \Coq 也有体现.\index{证明!助手!Coq@\textsc{Coq}}
这种方法更为一般化, 但也较为局限.
例如, 在 \Coq~\cite{Spiwack} 的集值范畴, 逻辑扩展类型理论~\cite{aczel2002collection}, 以及最小类型理论y~\cite{maietti2005toward}中, 唯一选择原理 (\cref{sec:unique-choice}) 不成立.
\index{集值}
因此, 单值公理让我们的类型理论更像是一个拓扑斯的内部逻辑;
参见 \cref{cha:set-math}.\index{拓扑斯}

Martin-L\"of~\cite{martin2006100} 提供了一段关于选择公理历史的讨论.
当然, 构造性和直觉主义数学有着悠久且复杂的历史, 在此不作深入探讨;
参见~\cite{TroelstraI,TroelstraII}.


\sectionExercises
\begin{ex}\label{ex:equiv-functor-set}
  Prove that if $\eqv A B$ and $A$ is a set, then so is $B$.
\end{ex}

\begin{ex}\label{ex:isset-coprod}
  Prove that if $A$ and $B$ are sets, then so is $A+B$.
\end{ex}

\begin{ex}\label{ex:isset-sigma}
  Prove that if $A$ is a set and $B:A\to \type$ is a type family such that $B(x)$ is a set for all $x:A$, then $\sm{x:A} B(x)$ is a set.
\end{ex}

\begin{ex}\label{ex:prop-endocontr}
  Show that $A$ is a mere proposition if and only if $A\to A$ is contractible.
\end{ex}

\begin{ex}\label{ex:prop-inhabcontr}
  Show that $\eqv{\isprop(A)}{(A\to\iscontr(A))}$.
\end{ex}

\begin{ex}\label{ex:lem-mereprop}
  Show that if $A$ is a mere proposition, then so is $A+(\neg A)$.
  Thus, there is no need to insert a propositional truncation in~\eqref{eq:lem}.
\end{ex}

\begin{ex}\label{ex:disjoint-or}
  More generally, show that if $A$ and $B$ are mere propositions and $\neg(A\times B)$, then $A+B$ is also a mere proposition.
\end{ex}

% \begin{ex}\label{ex:hprop-iff-equiv}
%   Show that if $A$ and $B$ are mere propositions such that $A\to B$ and $B\to A$, then $\eqv A B$.
% \end{ex}

% \begin{ex}\label{ex:isprop-isprop}
%   Show that for any type $A$, the types $\isprop(A)$ and $\isset(A)$ are mere propositions.
% \end{ex}

% \begin{ex}\label{ex:prop-eqvtrunc}
%   Show that if $A$ is already a mere proposition, then $\eqv A{\brck{A}}$.
% \end{ex}

\begin{ex}\label{ex:brck-qinv}
  Assuming that some type $\isequiv(f)$ satisfies conditions~\ref{item:be1}--\ref{item:be3} of \cref{sec:basics-equivalences}, show that the type $\brck{\qinv(f)}$ satisfies the same conditions and is equivalent to $\isequiv(f)$.
\end{ex}

\begin{ex}\label{ex:lem-impl-prop-equiv-bool}
  Show that if \LEM{} holds, then the type $\prop \defeq \sm{A:\type} \isprop(A)$ is equivalent to \bool.
\end{ex}

\begin{ex}\label{ex:lem-impred}
  Show that if $\UU_{i+1}$ satisfies \LEM{}, then the canonical inclusion $\prop_{\UU_i} \to \prop_{\UU_{i+1}}$ is an equivalence.
\end{ex}

\begin{ex}\label{ex:not-brck-A-impl-A}
  Show that it is not the case that for all $A:\type$ we have $\brck{A} \to A$.
  (However, there can be particular types for which $\brck{A}\to A$.
  \cref{ex:brck-qinv} implies that $\qinv(f)$ is such.)
\end{ex}

\begin{ex}\label{ex:lem-impl-simple-ac}
  \index{axiom!of choice}%
  Show that if \LEM{} holds, then for all $A:\type$ we have $\bbrck{(\brck A \to A)}$.
  (This property is a very simple form of the axiom of choice, which can fail in the absence of \LEM{}; see~\cite{krausgeneralizations}.)
\end{ex}

\begin{ex}\label{ex:naive-lem-impl-ac}
  We showed in \cref{thm:not-lem} that the following naive form of \LEM{} is inconsistent with univalence:
  \[ \prd{A:\type} (A+(\neg A)) \]
  In the absence of univalence, this axiom is consistent.
  However, show that it implies the axiom of choice~\eqref{eq:ac}.
\end{ex}

\begin{ex}\label{ex:lem-brck}
  Show that assuming \LEM{}, the double negation $\neg \neg A$ has the same recursion principle as the propositional truncation $\brck A$ but with a propositional computation rule rather than a judgmental one.
  In other words, prove that assuming \LEM{}, if $B$ is a mere proposition and we have $f:A\to B$, then there is an induced $g:\neg\neg A \to B$ such that $g(\bproj a) = f(a)$ for all $a:A$.
  Deduce that (assuming \LEM{}) we have $\eqv{\neg\neg A}{\brck{A}}$.
  Thus, under \LEM{}, the propositional truncation can be defined rather than taken as a separate type former.
\end{ex}

\begin{ex}\label{ex:impred-brck}
  \index{propositional!resizing}%
  Show that if we assume propositional resizing as in \cref{subsec:prop-subsets}, then the type
  \[\prd{P:\prop} \Parens{(A\to P)\to P}\]
  has the same recursion principle as $\brck A$, \emph{with} the same judgmental computation rule.
  Thus, we can also define the propositional truncation in this case.
\end{ex}

\begin{ex}\label{ex:lem-impl-dn-commutes}
  Assuming \LEM{}, show that double negation commutes with universal quantification of mere propositions over sets.
  That is, show that if $X$ is a set and each $Y(x)$ is a mere proposition, then \LEM{} implies
  \begin{equation}
    \eqv{\Parens{\prd{x:X} \neg\neg Y(x)}}{\Parens{\neg\neg \prd{x:X} Y(x)}}.\label{eq:dnshift}
  \end{equation}
  Observe that if we assume instead that each $Y(x)$ is a set, then~\eqref{eq:dnshift} becomes equivalent to the axiom of choice~\eqref{eq:epis-split}.
\end{ex}

\begin{ex}\label{ex:prop-trunc-ind}
  \index{induction principle!for truncation}%
  Show that the rules for the propositional truncation given in \cref{subsec:prop-trunc} are sufficient to imply the following induction principle: for any type family $B:\brck A \to \type$ such that each $B(x)$ is a mere proposition, if for every $a:A$ we have $B(\bproj a)$, then for every $x:\brck A$ we have $B(x)$.
\end{ex}

\begin{ex}\label{ex:lem-ldn}
  Show that the law of excluded middle~\eqref{eq:lem} and the law of double negation~\eqref{eq:ldn} are logically equivalent.
\end{ex}

\begin{ex}\label{ex:decidable-choice}
  Suppose $P:\nat\to\type$ is a decidable family (see \cref{defn:decidable-equality}\ref{item:decidable-equality2}) of mere propositions.
  Prove that
  \[ \Brck{\sm{n:\nat} P(n)} \;\to\; \sm{n:\nat}P(n).\]
\end{ex}

\begin{ex}\label{ex:omit-contr2}
  Prove \cref{thm:omit-contr}\ref{item:omitcontr2}: if $A$ is contractible with center $a$, then $\sm{x:A} P(x)$ is equivalent to $P(a)$.
\end{ex}

\begin{ex}\label{ex:isprop-equiv-equiv-bracket}
  Prove that $\isprop(P) \simeq (P \simeq \brck P)$.
\end{ex}

\begin{ex}\label{ex:finite-choice}
  As in classical set theory, the finite version of the axiom of choice is a theorem.  Prove that the axiom of choice \eqref{eq:ac} holds when $X$ is a finite type $\Fin(n)$ (as defined in \cref{ex:fin}).
\end{ex}

\begin{ex}\label{ex:decidable-choice-strong}
  Show that the conclusion of \cref{ex:decidable-choice} is true if $P:\nat\to\type$ is any decidable family.
\end{ex}

% Local Variables:
% TeX-master: "hott-online"
% End:
