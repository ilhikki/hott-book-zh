\chapter{集合和逻辑}
\label{cha:logic}
正式或非正式地说, 类型论的规则的汇集, 这些规则用来操作类型和类型的元素.
不过当非正式的讨论数学时, 普遍会用那些熟悉的词语, 特别是逻辑连词比如 ``而且'' 和 ``或'', 以及逻辑两次比如 ``对于所有'' 和 ``存在''.
相比集合论, 类型论提供了超过一种方法来将自然语言作为类型上的运算.
为了解决潜在的歧义, 需要设置局部或者全局的约定, 或者在非形式的数学中引入新的注解, 或者两个都要.
其中一些要求以及实现了, 因为类型论可以更好的分析逻辑, 我们可以更准确的表示数学, 其中 ``语言的滥用'' 比集合论更少.
本章我们会讲解这些相关的缺陷, 并证明我们的选择是正确的.


\section{集合和\texorpdfstring{$n$}{n}-阶类型}
\label{sec:basics-sets}
\index{set|(defstyle}%

在类型论中, 有一个\emph{集合}的概念, 可以帮助解释类型论的逻辑和集合论逻辑之间的联系.
尽管类型的行为通常类似空间或者高阶广群, 但还是有子类别, 更类似传统集合论系统中的集合.
范畴学上, 我们会考虑使用\emph{离散}广群, 它只由对象的一个集合确定, 并仅将恒等映射作为高阶映射;
拓扑学上, 我们会考虑使用离散范畴所在的空间.\index{离散!空间}
更通用的, 我们会考虑\emph{等价}于此类的广群或空间; 因为我们在类型论中做的最终都是同伦, 无法说有什么区别.

直观地, 我们认为类型 ``是一个集合'' 意味着它没有高阶同伦信息: 任意两条相同的路径都相等 (直到同伦), 而且所有尺寸的高阶路径都一样.
幸运地,因为同伦类型论天然的函子性/连续性,
\index{类型论中函数的连续性@类型论中函数的连续性``连续性''}%
\index{类型论中函数的函子性@类型论中函数的``函子性''}%
只需要考虑最低的等级就足够了.

\begin{defn}
    \label{defn:set}
    当对于所有 $x,y:A$ 和所有 $p,q:x=y$, 都存在 $p=q$ 时,
    类型 $A$ 是一个 \define{集合}.
\end{defn}

更准确地, 命题 $\isset(A)$ 被定义为这个类型
\[ \isset(A) \defeq \prd{x,y:A}{p,q:x=y} (p=q). \]

就像在 \cref{sec:types-vs-sets} 中提及的,
同伦类型论中的集合不像 ZT 类型论中的集合, 也就是没有一个通用的 ``所属关系断言'' $\in$.
它们更像结构数学和范畴学中用到的集合, 其元素是 ``抽象点'', 而我们用函数和关系给定其结构.
这样就可以将它当作大多数基于集合的数学的基础系统;
在 \cref{cha:set-math} 中有一些例子.

哪些类型是集合?
在\cref{cha:hlevels}会深入地学习这个问题更通用的形式, 不过现在可以观察一些简单的例子.

\begin{eg}
    \label{eg:isset-unit}
    类型 \unit 是一个集合.
    根据\cref{thm:path-unit}, 对于任何 $x,y:\unit$ 类型 $(x=y)$ 等价于 \unit.
    因为任意两个 \unit 的元素都相等, 这意味着任意两个 $x=y$ 的元素都相等.
\end{eg}

\begin{eg}
    \label{eg:isset-empty}
    类型 $\emptyt$ 是一个集合, 根据 $\emptyt$ 的归纳原理, 给定任意 $x,y:\emptyt$ 可以演绎任何想要的东西.
\end{eg}

\begin{eg}
    \label{thm:nat-set}
    自然数的类型 \nat 也是一个集合.
    根据 \cref{thm:path-nat}, 因为所有等同类型 $\id[\nat]xy$ 等价于 \unit 或者 \emptyt, 而且 \unit 或 \emptyt 的两个居留元都相等.
    在 \cref{cha:hlevels} 中, 还有另外一种证明.
\end{eg}

大部分我们考虑过的类型都是集合.

\begin{eg}
    \label{thm:isset-prod}
    如果 $A$ 和 $B$ 是集合, 那么 $A\times B$ 也如此.
    给定 $x,y:A\times B$ 和 $p,q:x=y$, 根据 \cref{thm:path-prod} 有 $p= \pairpath(\projpath1(p),\projpath2(p))$ 和 $q= \pairpath(\projpath1(q),\projpath2(q))$.
    而 $\projpath1(p)=\projpath1(q)$,  因为 $A$ 是集合, 以及 $\projpath2(p)=\projpath2(q)$, 因为 $B$ 是集合; 所以 $p=q$.

    类似地, 如果 $A$ 是集合而且 $B:A\to\type$ 满足所有 $B(x)$ 是集合, 那么 $\sm{x:A} B(x)$ 是一个集合.
\end{eg}

\begin{eg}
    \label{thm:isset-forall}
    如果 $A$ 是\emph{任意}类型, 而 $B:A\to \type$ 满足每个 $B(x)$ 是一个集合, 那么类型 $\prd{x:A} B(x)$ 是一个集合.
    提供 $f,g:\prd{x:A} B(x)$ 和 $p,q:f=g$.
    根据函数外延性, 有
    %
    \begin{equation*}
        p = {\funext (x \mapsto \happly(p,x))}
        \quad\text{和}\quad
        q = {\funext (x \mapsto \happly(q,x))}.
    \end{equation*}
    %
    不过对于所有 $x:A$, 有
    %
    \begin{equation*}
        \happly(p,x):f(x)=g(x)
        \qquad\text{和}\qquad
        \happly(q,x):f(x)=g(x),
    \end{equation*}
    %
    因为 $B(x)$ 是一个集合, 有 $\happly(p,x) = \happly(q,x)$.
    再次使用函数的外延性, 依值函数 $(x \mapsto \happly(p,x))$ 等于 $(x \mapsto \happly(q,x))$, 因此 (应用 $\apfunc{\funext}$) 就是~$p$ 和~$q$.
\end{eg}

更多的例子, 参见 \cref{ex:isset-coprod,ex:isset-sigma}. 为了更系统地了解同伦类型论中所有集合的子系统(范畴), 参见~\cref{cha:set-math}.

\index{n-阶类型@$n$-阶类型|(}%

集合只是\emph{同伦 $n$-阶类型}中的第一层级.
下一层由 \emph{$1$-types} 组成, 和范畴论中的 $1$-阶广群类似.
定义集合的的性质 (又叫做 \emph{$0$-阶类型}) 是它没有非平凡的性质.
类似地, 定义 $1$-阶类型的性质是它的路径之间没有非平凡的路径:

\begin{defn}
    \label{defn:1type}
    类型 $A$ 是一个 \define{1-阶类型}
    \indexdef{1-阶类型}%
    对于所有 $x,y:A$ 和 $p,q:x=y$ 以及 $r,s:p=q$, 有 $r=s$.
\end{defn}

类似地, 可以定义 $2$-阶类型, $3$-阶类型, 等等.
在\cref{cha:hlevels}中, 会定义 $n$-阶类型的归纳的通用概念, 并学习不同 $n$ 值之间, $n$-阶类型的关系.

不过, 现在只需要注意两个现象.
首先, 这些等级是向上封闭的: 如果 $A$ 是一个 $n$-阶类型, 那么 $A$ 是一个 $(n+1)$-阶类型.
例如:
%%  will make precise the sense in which this
%% ``suffices for all higher levels'', but as an example, we observe that
%% it suffices for the next level up.

\begin{lem}
    \label{thm:isset-is1type}
    如果 $A$ 是一个集合 (也就是说 $\isset(A)$ 有居留元), 那么 $A$ 是 1-阶类型.
\end{lem}
\begin{proof}
    给定 $f:\isset(A)$; 那么对于任何 $x,y:A$ 和 $p,q:x=y$ 有 $f(x,y,p,q):p=q$.
    固定 $x$, $y$, 和 $p$, 并定义 $g: \prd{q:x=y} (p=q)$ 为 $g(q) \defeq f (x,y,p,q)$.
    那么对于任何 $r:q=q'$, 有 $\apdfunc{g}(r) : \trans{r}{g(q)} = g(q')$.
    根据 \cref{cor:transport-path-prepost}, 于是有 $g(q) \ct r = g(q')$.

    特殊地, 给定 $x,y,p,q$ 和 $r,s:p=q$, 按照 \cref{defn:1type}, 定义 $g$ 如上.
    然后 $g(p) \ct r = g(q)$ 而且 $g(p) \ct s = g(q)$, 约去后得到 $r=s$.
\end{proof}

第二, 这种按类型等级的分层并不退化, 也就说意味着不是所有类型都是集合:

\begin{eg}
    \label{thm:type-is-not-a-set}
    \index{类型!全集}%
    全集 \type 不是集合.
    为了证明它, 展示类型 $A$ 和路径 $p:A=A$ 且不等于 $\refl A$ 即可.
    令 $A=\bool$, 并定义 $f:A\to A$ 为 $f(\bfalse)\defeq \btrue$ 和 $f(\btrue)\defeq \bfalse$.
    然后 $f(f(x))=x$ 对于所有 $x$ (很容易分析), 所以 $f$ 是一个等价.
    因此, 按照单值性, $f$ 可以得到 $p:A=A$.

    如果 $p$ 等于 $\refl A$, 那么 (再次根据单值性) $f$ 会等于 $A$ 的恒等函数.
    不过这样会得到 $\bfalse=\btrue$, 和 \cref{rmk:true-neq-false} 矛盾.
\end{eg}

\cref{cha:hits,cha:homotopy}中, 会展示对于任意 $n$, 都有不是 $n$-阶类型的类型.

注意 $A$ 是一个 1-阶类型当且仅当对于任何 $x,y:A$, 恒等类型 $\id[A]xy$ 是一个集合.
(因此, \cref{thm:isset-is1type} 等价于说集合的恒等类型也是集合.)
这是 \cref{cha:hlevels} 中将给出的递归地定义 $n$-阶类型的基础.

还可以从集合 ``向下'' 扩展这个概念.
也就是说, 类型 $A$ 是一个集合仅当对于任意 $x,y:A$, $\id[A]xy$ 的两个元素都是相等的.
因为集合等价于 0-阶类型, 自然地, 有一个 \emph{$(-1)$-阶类型}, 如果它们有后者的性质 (任意两个元素都相等).
这些类型可以被视为 \emph{狭义上的命题}, 而它们的研究也就是常说的 ``逻辑'';
它会在本章的剩余部分出现.

\index{n-阶类型@$n$-阶类型|)}%
\index{集合|)}%


\section{命题作为类型?}
\label{subsec:pat?}
\index{命题!作为类型|(}%
\index{逻辑!构建主义与经典的|(}%
\index{anger|(}%
直到现在, 都在遵循 \cref{sec:pat} 中``命题作为类型''的简单哲学, 也就是自然语言的 ``存在 $x:A$ 满足 $P(x)$'' 可以理解为对应的类型 $\sm{x:A} P(x)$, 而证明一个语句被视为判断某些特殊的元素居留于这个类型.
不过, 也有一些经典数学家不熟悉的 ``逻辑''.
例如, \cref{thm:ttac} 中的这个语句
\index{公理!选择!类型论风格}%
\begin{equation}
    \label{eq:english-ac}
    \parbox{\textwidth-2cm}{``如果对于所有 $x:X$ 存在一个 $a:A(x)$ 满足 $P(x,a)$, 那么存在函数 $g:\prd{x:X} A(x)$ 满足对于所有 $x:X$ 有 $P(x,g(x))$'',}
\end{equation}
这看起来像是经典的\index{数学!经典的}\emph{选择公理}, 这种情况下永远是对的.
这是一个常用的重点, 命题作为类型的逻辑, 不过这也说明了它和经典的逻辑演绎的显著区别, 后者的选择公理不在逻辑上成立, 而是额外的 ``公理''.

另一方面, 还会展示经典的\emph{双重否定}以及\emph{排中律}与单值公理不相容.
\index{单值公理}%

\begin{thm}
    \label{thm:not-dneg}
    \index{双重否定, 律}%
    这是错误的: 对于所有 $A:\UU$ 都有 $\neg(\neg A) \to A$.
\end{thm}
\begin{proof}
    回顾 $\neg A \jdeq (A\to\emptyt)$.
    可以将运算符 $\neg$ 看作 ``\dots 是错误的''.
    因此, 为了证明这个语句, 根据 $f:\prd{A:\UU} (\neg\neg A \to A)$ 构建 \emptyt 的元素既可.

    下面这个证明的理念是观察 $f$, 与类型论中任何的函数一样, 都是 ``连续的''.
    \index{类型论中函数的连续性@类型论中函数的``连续性''}%
    \index{类型论中函数的函子性@类型论中函数的``函子性''}%
    根据单值, 这意味 $f$ \emph{天然}地遵循类型的等价.
    由此以及一个 fixed-point-free autoequivalence\index{automorphism!fixed-point-free}, 可以提取一个结论.

    定义等价 $e:\eqv\bool\bool$, 其中 $e(\btrue)\defeq\bfalse$ 而 $e(\bfalse)\defeq\btrue$, 就像 \cref{thm:type-is-not-a-set}.
    令 $p:\bool=\bool$ 为 $e$ 由单值对应的路径, 即 $p\defeq \ua(e)$.
    然后有 $f(\bool) : \neg\neg\bool \to\bool$ 和
    \[\apd f p : \transfib{A\mapsto (\neg\neg A \to A)}{p}{f(\bool)} = f(\bool).\]
    因此, 对于任何 $u:\neg\neg\bool$, 有
    \[\happly(\apd f p,u) : \transfib{A\mapsto (\neg\neg A \to A)}{p}{f(\bool)}(u) = f(\bool)(u).\]

    现在通过~\eqref{eq:transport-arrow}, 沿着 $p$ 在类型族 ${A\mapsto (\neg\neg A \to A)}$ 中运输 $f(\bool):\neg\neg\bool\to\bool$, 它等于一个函数, 这个函数沿着 $\opp p$ 在类型族 $A\mapsto \neg\neg A$ 中运输, 应用 $f(\bool)$ 后, 沿着 $p$ 在类型族 $A\mapsto A$ 运输:
    \begin{narrowmultline*}
        \transfib{A\mapsto (\neg\neg A \to A)}{p}{f(\bool)}(u) =
        \narrowbreak
        \transfib{A\mapsto A}{p}{f(\bool) (\transfib{A\mapsto \neg\neg
        A}{\opp{p}}{u})}.
    \end{narrowmultline*}
    %
    而任意两个点 $u,v:\neg\neg\bool$ 按照函数外延性都相等, 应为对于任何 $x:\neg\bool$ 有 $u(x):\emptyt$ 因此可以得到任意结论, 比如 $u(x)=v(x)$.
    因此, 有 $\transfib{A\mapsto \neg\neg A}{\opp{p}}{u} = u$, 陨石从 $\happly(\apd f p,u)$ 得到一个等同
    \[ \transfib{A\mapsto A}{p}{f(\bool)(u)} = f(\bool)(u).\]
    最后, 就像在 \cref{sec:compute-universe} 中讨论的, 沿着路径 $p\jdeq \ua(e)$ 运输类型族 $A\mapsto A$ 的结果等价于应用这个等价 $e$;
    因此有
    \begin{equation}
        e(f(\bool)(u)) = f(\bool)(u).\label{eq:fpaut}
    \end{equation}
    %
    也可以证明
    \begin{equation}
        \prd{x:\bool} \neg(e(x)=x).\label{eq:fpfaut}
    \end{equation}
    这可以通过在 $x$ 上分析情况得到: 每个情况都直接代入 $e$ 的定义而实际上 $\bfalse\neq\btrue$ (\cref{rmk:true-neq-false}).
    因此, 应用~\eqref{eq:fpfaut} 于 $f(\bool)(u)$ 和~\eqref{eq:fpaut}, 可以得到 $\emptyt$ 的元素.
\end{proof}

\begin{rmk}
    \index{选择运算符}%
    \indexsee{运算符!选择}{选择运算符}%
    实际上, 这意味着没有 Hilbert 风格的 ``选择运算符'', 即选择每个非空类型的一个元素.
    关键在于没有\emph{天然的}这种运算, 在单值公理下, 作用于类型的所有函数必须是天然地遵循等价性.
\end{rmk}

\begin{rmk}
    不过, 对于任何 $A$ 还是有 $\neg\neg\neg A \to \neg A$; 参见 \cref{ex:neg-ldn}.
\end{rmk}

\begin{cor}
    \label{thm:not-lem}
    \index{排中}%
    这是错误的: 对于所有 $A:\UU$ 有 $A+(\neg A)$.
\end{cor}
\begin{proof}
    给定 $g:\prd{A:\UU} (A+(\neg A))$.
    只要能得到 $\prd{A:\UU} (\neg\neg A \to A)$, 就能应用 \cref{thm:not-dneg}.
    因此, 给定 $A:\UU$ 和 $u:\neg\neg A$;
    要构造 $A$ 的元素.

    现在 $g(A):A+(\neg A)$, 于是根据模式匹配, 只有 $g(A)\jdeq \inl(a)$ 其中 $a:A$, 或者 $g(A)\jdeq \inr(w)$ 其中 $w:\neg A$ 两种情况.
    第一种情况, 有 $a:A$, 而第二种情况有 $u(w):\emptyt$ 于是可以得到任何想要的 (比如 $A$).
    因此, 每种情况都能得到所需的 $A$ 的元素.
\end{proof}

因此, 如果像假定单值公理 (当然, 我们会这样), 仍然可以选择(所需的)经典\index{数学!经典的}定理, 不能使用未改动的命题作为类型原则来诠释 \emph{所有} 非形式的数学语句到类型轮中, 因为排中律可能是错的.
不过, 我们也不想完全地放弃命题作为类型, 因为它有很多好的性质 (比如简单, 可构造, 可计算).
接下来会讨论一个命题作为类型的改进, 可以解决这个问题;
在 \cref{subsec:when-trunc} 会回到什么时候使用什么逻辑的问题.
\index{anger|)}%
\index{命题!作为类型|)}%
\index{逻辑!构造主义与经典的|)}%


\section{纯命题}
\label{subsec:hprops}
\index{逻辑!纯命题的|(}%
\index{纯命题(defstyle}%
\indexsee{命题!纯}{纯命题}%
我们已经看到命题作为类型的逻辑的优点和缺点.
它们有这样的共性: 当类型视为命题, 它们存在更多的信息, 不止是对和错, 它们上的 ``逻辑'' 构造都遵循这些额外信息.
这表明我们可以得到更常规的逻辑, 通过只关注\emph{不}包含任何比真值外信息的类型, 并单纯视其为逻辑命题.

比如类型 $A$ 为 ``真'' 当其被居留时, 而 ``假'' 如果它的居留元可以得到矛盾 (即 $\neg A \jdeq (A\to\emptyt)$ 被居留).
\index{被居留的类型}%
为了得到更加传统的逻辑, 要避免将这样的类型视为逻辑命题, 给出它们的元素比知道这个类型被居留提供了更多的信息.
例如, 如果给出 \bool 的元素, 那么会收到根多的信息, 相比 \bool 存在元素这个单纯的事实.
确实, 得到了确切的\emph{一比特}
\index{比特}%
的额外信息: 知道了给出的是 \bool 的\emph{哪个}元素.
相反的, 如果给出 \unit 的元素, 那么不会得到更多的信息, 相比 \unit 存在元素这个单纯的事实, 因为任何两个 \unit 的元素彼此相等.
这启发了这个定义.

\begin{defn}\label{defn:isprop}
  类型 $P$ 是一个\define{纯命题}
  如果对于所有 $x,y:P$ 有 $x=y$.
\end{defn}

主义因为我们还是在类型论\emph{中}做数学, 这个定义\emph{在}类型论\emph{中}, 意味着一个类型 --- 或是一个类型族.
特殊的, 对于任意 $P:\type$, 类型 $\isprop(P)$ 被定义为
\[ \isprop(P) \defeq \prd{x,y:P} (x=y). \]
因此, 假定 ``$P$ 是纯命题'' 意味着展示 $\isprop(P)$ 的一个居留元, 即通过路径连接任意两个 $P$ 的元素的依值函数.
该函数的连续性/自然性暗示了不仅 $P$ 的任意两个元素都相等, 而且 $P$ 也不存在高阶同伦.

\begin{lem}\label{thm:inhabprop-eqvunit}
  如果 $P$ 是纯命题而且 $x_0:P$, 那么 $\eqv P \unit$.
\end{lem}
\begin{proof}
  定义 $f:P\to\unit$ 为 $f(x)\defeq \ttt$, 而 $g:\unit\to P$ 为 $g(u)\defeq x_0$.
  然后使用下面的引理, 其中根据 \cref{thm:path-unit}, \unit 是纯命题.
\end{proof}

\begin{lem}\label{lem:equiv-iff-hprop}
  如果 $P$ 和 $Q$ 是纯命题而 $P\to Q$ 且 $Q\to P$, 那么 $\eqv P Q$.
\end{lem}
\begin{proof}
  提供 $f:P\to Q$ 和 $g:Q\to P$.
  对于任何 $x:P$, 有 $g(f(x))=x$ 因为 $P$ 是纯命题.
  类似地, 对于任何 $y:Q$ 有 $f(g(y))=y$ 因为 $Q$ 是纯命题; 因此 $f$ 准逆于 $g$.
\end{proof}

也就是正如 \cref{sec:pat} 中约定的一样, 如果两个纯命题之间逻辑上等价, 那么它们等价.

同伦论中, 同伦等价于 \unit 的空间是\emph{可缩的}.
因此, 任何被居留的纯命题都是可缩的 (参见 \cref{sec:contractibility}).
另外, 不被居留的类型 \emptyt 也是 (空洞的) 纯命题.
至少在经典\index{数学!经典的}数学中, 只有这两种可能.

纯命题也被称为\emph{子终止对象} (范畴学上), \emph{单元素子集} (集合论上), 或者\emph{h-propositions}.
\indexsee{对象!子终结}{纯命题}%
\indexsee{子终结对象}{纯命题}%
\indexsee{子终结}{纯命题}%
\indexsee{h-proposition}{纯命题}
按照 \cref{sec:basics-sets} 中的讨论, 应该把它们称为 \emph{$(-1)$-类型}; 在 \cref{cha:hlevels} 会返回到它.
形容词 ``纯'' 强调尽管可以将任何类型视为命题 (通过给出居留元来证明它), 作为纯命题的类型, 只有将其视为命题, 否则没有意义: 它为真的见证不包含额外信息.

注意类型 $A$ 是一个集合当且仅当对于所有 $x,y:A$, 恒等类型 $\id[A]xy$ 是纯命题.
另一方面, 通过复制和简化 \cref{thm:isset-is1type} 的证明, 有:

\begin{lem}\label{thm:prop-set}
  每个纯命题都是集合.
\end{lem}
\begin{proof}
  假设 $f:\isprop(A)$; 因此对于所有 $x,y:A$ 有 $f(x,y):x=y$.
  固定 $x:A$ 并定义 $g(y)\defeq f(x,y)$.
  然后对于任何 $y,z:A$ 且 $p:y=z$ 有 $\apd g p : \trans{p}{g(y)}={g(z)}$.
  于是通过 \cref{cor:transport-path-prepost}, 有 $g(y)\ct p = g(z)$, 也就是说 $p=\opp{g(y)}\ct g(z)$.
  因此, 对于所有 $p,q:x=y$, 有 $p = \opp{g(x)}\ct g(y) = q$.
\end{proof}

特别地, 这意味着:

\begin{lem}\label{thm:isprop-isprop}\label{thm:isprop-isset}
  对于任何类型 $A$, 类型 $\isprop(A)$ 和 $\isset(A)$ 都是纯命题.
\end{lem}
\begin{proof}
  假设 $f,g:\isprop(A)$.
  通过函数外延性, 为了展示 $f=g$ 只需展示 $f(x,y)=g(x,y)$ 对于任何 $x,y:A$.
  不过 $f(x,y)$ 和 $g(x,y)$ 都是 $A$ 中的路径, 而它们相等, 因为根据 $f$ 或 $g$, $A$ 是纯命题, 因此根据 \cref{thm:prop-set} 它是集合.
  类似地, 假设 $f,g:\isset(A)$, 也就是说对于所有 $a,b:A$ 和 $p,q:a=b$, 有 $f(a,b,p,q):p=q$ 和 $g(a,b,p,q):p=q$.
  因为 $A$ 是集合 (根据 $f$ 或 $g$), 所以也是 1-阶类型, 于是有 $f(a,b,p,q)=g(a,b,p,q)$;
  因此根据函数外延性 $f=g$.
\end{proof}

之前有另一个例子: \cref{sec:basics-equivalences} 中的判断~\ref{item:be3} 假定对于任何函数 $f$, 类型 $\isequiv (f)$ 应该是纯命题.

\index{逻辑!,纯命题的|)}%
\index{纯命题)}%


\section{经典逻辑和直觉主义逻辑}
\label{sec:intuitionism}
\index{logic!constructive vs classical|(}%
\index{denial|(}%
With the notion of mere proposition in hand, we can now give the proper formulation of the \define{law of excluded middle}
\indexdef{excluded middle}%
\indexsee{axiom!excluded middle}{excluded middle}%
\indexsee{law!of excluded middle}{excluded middle}%
in homotopy type theory:
\begin{equation}
  \label{eq:lem}
  \LEM{}\;\defeq\;
  \prd{A:\UU} \Big(\isprop(A) \to (A + \neg A)\Big).
\end{equation}
Similarly, the \define{law of double negation}
\indexdef{double negation, law of}%
\indexdef{axiom!double negation}%
\indexdef{law!of double negation}%
is
\begin{equation}
  \label{eq:ldn}
  % \mathsf{DN}\;\defeq\;
  \prd{A:\UU} \Big(\isprop(A) \to (\neg\neg A \to A)\Big).
\end{equation}
The two are also easily seen to be equivalent to each other---see \cref{ex:lem-ldn}---so from now on we will generally speak only of \LEM{}.

This formulation of \LEM{} avoids the ``paradoxes'' of \cref{thm:not-dneg,thm:not-lem}, since \bool is not a mere proposition.
In order to distinguish it from the more general propositions-as-types formulation, we rename the latter:
\symlabel{lem-infty}
\begin{equation*}
  \LEM\infty \defeq \prd{A:\UU} (A + \neg A).
\end{equation*}
For emphasis, the proper version~\eqref{eq:lem}
may be denoted $\LEM{-1}$;
see also \cref{ex:lemnm}.
Although $\LEM{}$
is not a consequence of the basic type theory described in \cref{cha:typetheory}, it may be consistently assumed as an axiom (unlike its $\infty$-counterpart).
For instance, we will assume it in \cref{sec:wellorderings}.

However, it can be surprising how far we can get without using \LEM{}.
Quite often, a simple reformulation of a definition or theorem enables us to avoid invoking excluded middle.
While this takes a little getting used to sometimes, it is often worth the hassle, resulting in more elegant and more general proofs.
We discussed some of the benefits of this in the introduction.

For instance, in classical\index{mathematics!classical} mathematics, double negations are frequently used unnecessarily.
A very simple example is the common assumption that a set $A$ is ``nonempty'', which literally means it is \emph{not} the case that $A$ contains \emph{no} elements.
Almost always what is really meant is the positive assertion that $A$ \emph{does} contain at least one element, and by removing the double negation we make the statement less dependent on \LEM{}.
Recall that we say that a type $A$ is \emph{inhabited}
\index{inhabited type}%
when we assert $A$ itself as a proposition (i.e.\ we construct an element of $A$, usually unnamed).
Thus, often when translating a classical proof into constructive logic, we replace the word ``nonempty'' by ``inhabited'' (although sometimes we must replace it instead by ``merely inhabited''; see \cref{subsec:prop-trunc}).

Similarly, it is not uncommon in classical mathematics to find unnecessary proofs by contradiction.
\index{proof!by contradiction}%
Of course, the classical form of proof by contradiction proceeds by way of the law of double negation: we assume $\neg A$ and derive a contradiction, thereby deducing $\neg \neg A$, and thus by double negation we obtain $A$.
However, often the derivation of a contradiction from $\neg A$ can be rephrased slightly so as to yield a direct proof of $A$, avoiding the need for \LEM{}.

It is also important to note that if the goal is to prove a \emph{negation}\index{negation}, then ``proof by contradiction'' does not involve \LEM{}.
In fact, since $\neg A$ is by definition the type $A\to\emptyt$, by definition to prove $\neg A$ is to prove a contradiction (\emptyt) under the assumption of $A$.
Similarly, the law of double negation does hold for negated propositions: $\neg\neg\neg A \to \neg A$.
With practice, one learns to distinguish more carefully between negated and non-negated propositions and to notice when \LEM{} is being used and when it is not.

Thus, contrary to how it may appear on the surface, doing mathematics ``constructively'' does not usually involve giving up important theorems, but rather finding the best way to state the definitions so as to make the important theorems constructively provable.
That is, we may freely use the \LEM{} when first investigating a subject, but once that subject is better understood, we can hope to refine its definitions and proofs so as to avoid that axiom.
% For instance, the theory of ordinal numbers, which classically makes heavy use of \LEM{}, works quite well constructively once we choose the correct definition of ``ordinal''; see \cref{sec:ordinals}.
This sort of observation is even more pronounced in \emph{homotopy} type theory, where the powerful tools of univalence and higher inductive types allow us to constructively attack many problems that traditionally would require classical\index{mathematics!classical} reasoning.
We will see several examples of this in \cref{part:mathematics}.
% For instance, none of the ``synthetic'' homotopy theory we will develop in \cref{cha:homotopy} requires \LEM{} --- despite the fact that classical homotopy theory (formulated using topological spaces or simplicial sets) makes heavy use of them (as well as the axiom of choice).

It is also worth mentioning that even in constructive mathematics, the law of excluded middle can hold for \emph{some} propositions.
The name traditionally given to such propositions is \emph{decidable}.

\begin{defn}\label{defn:decidable-equality}
  \mbox{}
  \begin{enumerate}
  \item A type $A$ is called \define{decidable}
    \indexdef{decidable!type}%
    \indexdef{type!decidable}%
    if $A+\neg A$.
  \item Similarly, a type family $B:A\to \type$ is \define{decidable}
    \indexdef{decidable!type family}%
    \indexdef{type!family of!decidable}%
    if \narrowequation{\prd{a:A} (B(a)+\neg B(a)).} \label{item:decidable-equality2}
  \item In particular, $A$ has \define{decidable equality}
    \indexdef{decidable!equality}%
    \indexsee{equality!decidable}{decidable equality}%
    if \narrowequation{\prd{a,b:A} ((a=b) + \neg(a=b)).}
  \end{enumerate}
\end{defn}

Thus, $\LEM{}$ is exactly the statement that all mere propositions are decidable, and hence so are all families of mere propositions.
In particular, $\LEM{}$ implies that all sets (in the sense of \cref{sec:basics-sets}) have decidable equality.
Having decidable equality in this sense is very strong; see \cref{thm:hedberg}.

\index{denial|)}%
\index{logic!constructive vs classical|)}%


\section{子集合与命题重置}
\label{subsec:prop-subsets}
\index{mere proposition|(}%

As another example of the usefulness of mere propositions, we discuss subsets (and more generally subtypes).
Suppose $P:A\to\type$ is a type family, with each type $P(x)$ regarded as a proposition.
Then $P$ itself is a \emph{predicate} on $A$, or a \emph{property} of elements of $A$.

In set theory, whenever we have a predicate $P$ on a set $A$, we may form the subset $\setof{x\in A | P(x)}$.
As mentioned briefly in \cref{sec:pat}, the obvious analogue in type theory is the $\Sigma$-type $\sm{x:A} P(x)$.
An inhabitant of $\sm{x:A} P(x)$ is, of course, a pair $(x,p)$ where $x:A$ and $p$ is a proof of $P(x)$.
However, for general $P$, an element $a:A$ might give rise to more than one distinct element of $\sm{x:A} P(x)$, if the proposition $P(a)$ has more than one distinct proof.
This is counter to the usual intuition of a subset.
But if $P$ is a \emph{mere} proposition, then this cannot happen.

\begin{lem}\label{thm:path-subset}
  Suppose $P:A\to\type$ is a type family such that $P(x)$ is a mere proposition for all $x:A$.
  If $u,v:\sm{x:A} P(x)$ are such that $\proj1(u) = \proj1(v)$, then $u=v$.
\end{lem}
\begin{proof}
  Suppose $p:\proj1(u) = \proj1(v)$.
  By \cref{thm:path-sigma}, to show $u=v$ it suffices to show $\trans{p}{\proj2(u)} = \proj2(v)$.
  But $\trans{p}{\proj2(u)}$ and $\proj2(v)$ are both elements of $P(\proj1(v))$, which is a mere proposition; hence they are equal.
\end{proof}

For instance, recall that in \cref{sec:basics-equivalences} we defined
\[(\eqv A B) \;\defeq\; \sm{f:A\to B} \isequiv (f),\]
where each type $\isequiv (f)$ was supposed to be a mere proposition.
It follows that if two equivalences have equal underlying functions, then they are equal as equivalences.

\label{defn:setof}%
Henceforth, if $P:A\to \type$ is a family of mere propositions (i.e.\ each $P(x)$ is a mere proposition), we may write
%
\begin{equation}
  \label{eq:subset}
  \setof{x:A | P(x)}
\end{equation}
%
as an alternative notation for $\sm{x:A} P(x)$.
(There is no technical reason not to use this notation for arbitrary $P$ as well, but such usage could be confusing due to unintended connotations.)
If $A$ is a set, we call \eqref{eq:subset} a \define{subset}
\indexdef{subset}%
\indexdef{type!subset}%
of $A$; for general $A$ we might call it a \define{subtype}.
\indexdef{subtype}%
We may also refer to $P$ itself as a \emph{subset} or \emph{subtype} of $A$; this is actually more correct, since the type~\eqref{eq:subset} in isolation doesn't remember its relationship to $A$.

\symlabel{membership}
Given such a $P$ and $a:A$, we may write $a\in P$ or $a\in \setof{x:A | P(x)}$ to refer to the mere proposition $P(a)$.
If it holds, we may say that $a$ is a \define{member} of $P$.
\symlabel{subset}
Similarly, if $\setof{x:A | Q(x)}$ is another subset of $A$, then we say that $P$ is \define{contained}
\indexdef{containment!of subsets}%
\indexdef{inclusion!of subsets}%
in $Q$, and write $P\subseteq Q$, if we have $\prd{x:A}(P(x)\rightarrow Q(x))$.

As further examples of subtypes, we may define the ``subuniverses'' of sets and of mere propositions in a universe \UU:
\symlabel{setU}\symlabel{propU}
\begin{align*}
  \setU &\defeq \setof{A:\UU | \isset(A) },\\
  \propU &\defeq \setof{A:\UU | \isprop(A) }.
\end{align*}
An element of $\setU$ is a type $A:\UU$ together with evidence $s:\isset(A)$, and similarly for $\propU$.
\cref{thm:path-subset} implies that $\id[\setU]{(A,s)}{(B,t)}$ is equivalent to $\id[\UU]AB$ (and hence to $\eqv AB$).
Thus, we will frequently abuse notation and write simply $A:\setU$ instead of $(A,s):\setU$.
We may also drop the subscript \UU if there is no need to specify the universe in question.

Recall that for any two universes $\UU_i$ and $\UU_{i+1}$, if $A:\UU_i$ then also $A:\UU_{i+1}$.
Thus, for any $(A,s):\set_{\UU_i}$ we also have $(A,s):\set_{\UU_{i+1}}$, and similarly for $\prop_{\UU_i}$, giving natural maps
\begin{align}
  \set_{\UU_i} &\to \set_{\UU_{i+1}}\label{eq:set-up},\\
  \prop_{\UU_i} &\to \prop_{\UU_{i+1}}.\label{eq:prop-up}
\end{align}
The map~\eqref{eq:set-up} cannot be an equivalence, since then we could reproduce the paradoxes\index{paradox} of self-reference that are familiar from Cantorian set theory.
However, although~\eqref{eq:prop-up} is not automatically an equivalence in the type theory we have presented so far, it is consistent to suppose that it is.
That is, we may consider adding to type theory the following axiom.

\begin{axiom}[Propositional resizing]
  \indexsee{axiom!propositional resizing}{propositional resizing}%
  \indexdef{propositional!resizing}%
  \indexsee{resizing!propositional}{propositional resizing}%
  The map $\prop_{\UU_i} \to \prop_{\UU_{i+1}}$ is an equivalence.
\end{axiom}

We refer to this axiom as \define{propositional resizing},
since it means that any mere proposition in the universe $\UU_{i+1}$ can be ``resized'' to an equivalent one in the smaller universe $\UU_i$.
It follows automatically if $\UU_{i+1}$ satisfies \LEM{} (see \cref{ex:lem-impred}).
We will not assume this axiom in general, although in some places we will use it as an explicit hypothesis.
It is a form of \emph{impredicativity} for mere propositions, and by avoiding its use, the type theory is said to remain \emph{predicative}.
\indexsee{impredicativity!for mere propositions}{propositional resizing}%
\indexdef{mathematics!predicative}%

In practice, what we want most frequently is a slightly different statement: that a universe \UU under consideration contains a type which ``classifies all mere propositions''.
In other words, we want a type $\Omega:\UU$ together with an $\Omega$-indexed family of mere propositions, which contains every mere proposition up to equivalence.
This statement follows from propositional resizing as stated above if $\UU$ is not the smallest universe $\UU_0$, since then we can define $\Omega\defeq \prop_{\UU_0}$.

One use for impredicativity is to define power sets.
It is natural to define the \define{power set}\indexdef{power set} of a set $A$ to be $A\to\propU$; but in the absence of impredicativity, this definition depends
(even up to equivalence) on the choice of the universe \UU.
But with propositional resizing, we can define the power set to be
\symlabel{powerset}%
\[ \power A \defeq (A\to\Omega),\]
which is then independent of $\UU$.
See also \cref{subsec:piw}.



\section{纯命题的逻辑}
\label{subsec:logic-hprop}
\index{逻辑!纯命题的|(}%
注意\cref{sec:types-vs-sets}中, 类型论只有一个基础概念 (类型), 集合论有两个基础概念: 集合和类型.
因此, 经典数学家\index{数学!经典的}习惯于分开处理这两种对象.

在类型论中可能还原类似的区分, 其中集合论的命题, 在类型论中被表现为\emph{纯}命题的类型(或者类型族).
在很多时候, 逻辑的连词和量词可以被通过限制对应的类型构造器为纯命题来表示.
当然, 需要确保问题中的类型构造器保留了纯命题.

\begin{eg}
  如果 $A$ 和 $B$ 是纯命题, 那么 $A\times B$ 也是.
  使用乘积中路径的特征即可, 类似于 \cref{thm:isset-prod}, 而且更简单.
  因此, 连词 ``与'' 保留了纯命题.
\end{eg}

\begin{eg}\label{thm:isprop-forall}
  如果 $A$ 是一个类型而且 $B:A\to \type$ 满足对于所有 $x:A$ 类型 $B(x)$ 是命题, 那么 $\prd{x:A} B(x)$ 是纯命题.
  证明类似于 \cref{thm:isset-forall} 而且更佳简单: 给定 $f,g:\prd{x:A} B(x)$, 对于任何 $x:A$ 有 $f(x)=g(x)$ 因为 $B(x)$ 是纯命题.
  然后根据函数的外延性, 有 $f=g$.

  特别地, 如果 $B$ 是纯命题, $A\to B$ 也是, 无论 $A$ 是否为纯命题.
  更特殊地, 因为 \emptyt 是纯命题, 所以 $\neg A \jdeq (A\to\emptyt)$ 也是.
  \index{量词!通用的}%
  因此, 连词 ``蕴含'' 和 ``非'' 保留纯命题, 量词 ``对于所有'' 也是.
\end{eg}

另一方面, 一些类型构造不保留纯命题.
即使 $A$ 和 $B$ 都是纯命题, $A+B$ 也通常不是.
例如, \unit 是纯命题, 但 $\bool=\unit+\unit$ 不是.
逻辑上来讲, $A+B$ 是一种``纯构造性''的``或'': 它是一个证明, 而且包含了额外的信息, 即\emph{哪个}分支为真.
某些时候这很有用, 但是如果需要保留纯命题的更经典的``或'', 需要一个方法来``截取''这个类型, 通过抛弃额外的信息.

\index{量词!外延的}%
类似的问题在 $\Sigma$-类型 $\sm{x:A} P(x)$ 也存在.
这是关于 ``存在 $x:A$ 满足 $P(x)$'' 的一个纯构造性解释, 而且包含了证据 $x$, 因此它通常也不是纯命题. 即使 $P(x)$ 是纯命题.
(回顾在 \cref{subsec:prop-subsets} 观察的, $\sm{x:A} P(x)$ 可以被视为是 ``满足 $P(x)$ 的 $x:A$ 的子集''.)



\section{命题截断}
\label{subsec:prop-trunc}
\index{truncation!propositional|(defstyle}%
\indexsee{type!squash}{truncation, propositional}%
\indexsee{squash type}{truncation, propositional}%
\indexsee{bracket type}{truncation, propositional}%
\indexsee{type!bracket}{truncation, propositional}%
The \emph{propositional truncation}, also called the \emph{$(-1)$-truncation}, \emph{bracket type}, or \emph{squash type}, is an additional type former which ``squashes'' or ``truncates'' a type down to a mere proposition, forgetting all information contained in inhabitants of that type other than their existence.

More precisely, for any type $A$, there is a type $\brck{A}$.
It has two constructors:
\begin{itemize}
\item For any $a:A$ we have $\bproj a : \brck A$.
\item For any $x,y:\brck A$, we have $x=y$.
\end{itemize}
The first constructor means that if $A$ is inhabited, so is $\brck A$.
The second ensures that $\brck A$ is a mere proposition; usually we leave the witness of this fact nameless.

\index{recursion principle!for truncation}%
The recursion principle of $\brck A$ says that:
\begin{itemize}
\item If $B$ is a mere proposition and we have $f:A\to B$, then there is an induced $g:\brck A \to B$ such that $g(\bproj a) \jdeq f(a)$ for all $a:A$.
\end{itemize}
In other words, any mere proposition which follows from (the inhabitedness of) $A$ already follows from $\brck A$.
Thus, $\brck A$, as a mere proposition, contains no more information than the inhabitedness of $A$.
(There is also an induction principle for $\brck A$, but it is not especially useful; see \cref{ex:prop-trunc-ind}.)

In \cref{ex:lem-brck,ex:impred-brck,sec:hittruncations} we will describe some ways to construct $\brck{A}$ in terms of more general things.
For now, we simply assume it as an additional rule alongside those of \cref{cha:typetheory}.

With the propositional truncation, we can extend the ``logic of mere propositions'' to cover disjunction and the existential quantifier.
Specifically, $\brck{A+B}$ is a mere propositional version of ``$A$ or $B$'', which does not ``remember'' the information of which disjunct is true.

The recursion principle of truncation implies that we can still do a case analysis on $\brck{A+B}$ \emph{when attempting to prove a mere proposition}.
That is, suppose we have an assumption $u:\brck{A+B}$ and we are trying to prove a mere proposition $Q$.
In other words, we are trying to define an element of $\brck{A+B} \to Q$.
Since $Q$ is a mere proposition, by the recursion principle for propositional truncation, it suffices to construct a function $A+B\to Q$.
But now we can use case analysis on $A+B$.

Similarly, for a type family $P:A\to\type$, we can consider $\brck{\sm{x:A} P(x)}$, which is a mere propositional version of ``there exists an $x:A$ such that $P(x)$''.
As for disjunction, by combining the induction principles of truncation and $\Sigma$-types, if we have an assumption of type $\brck{\sm{x:A} P(x)}$, we may introduce new assumptions $x:A$ and $y:P(x)$ \emph{when attempting to prove a mere proposition}.
In other words, if we know that there exists some $x:A$ such that $P(x)$, but we don't have a particular such $x$ in hand, then we are free to make use of such an $x$ as long as we aren't trying to construct anything which might depend on the particular value of $x$.
Requiring the codomain to be a mere proposition expresses this independence of the result on the witness, since all possible inhabitants of such a type must be equal.

For the purposes of set-level mathematics in \cref{cha:real-numbers,cha:set-math},
where we deal mostly with sets and mere propositions, it is convenient to use the
traditional logical notations to refer only to ``propositionally truncated logic''.

\begin{defn} \label{defn:logical-notation}
  We define \define{traditional logical notation}
  \indexdef{implication}%
  \indexdef{traditional logical notation}%
  \indexdef{logical notation, traditional}%
  \index{quantifier}%
  \indexsee{existential quantifier}{quantifier, existential}%
  \index{quantifier!existential}%
  \indexsee{universal!quantifier}{quantifier, universal}%
  \index{quantifier!universal}%
  \indexdef{conjunction}%
  \indexdef{disjunction}%
  \indexdef{true}%
  \indexdef{false}%
  using truncation as follows, where $P$ and $Q$ denote mere propositions (or families thereof):
  {\allowdisplaybreaks
  \begin{align*}
    \top            &\ \defeq \ \unit \\
    \bot            &\ \defeq \ \emptyt \\
    P \land Q       &\ \defeq \ P \times Q \\
    P \Rightarrow Q &\ \defeq \ P \to Q \\
    P \Leftrightarrow Q &\ \defeq \ P = Q \\
    \neg P          &\ \defeq \ P \to \emptyt \\
    P \lor Q        &\ \defeq \ \brck{P + Q} \\
    \fall{x : A} P(x) &\ \defeq \ \prd{x : A} P(x) \\
    \exis{x : A} P(x) &\ \defeq \ \Brck{\sm{x : A} P(x)}
  \end{align*}}
\end{defn}

The notations $\land$ and $\lor$ are also used in homotopy theory for the smash product and the wedge of pointed spaces, which we will introduce in \cref{cha:hits}.
This technically creates a potential for conflict, but no confusion will generally arise.

Similarly, when discussing subsets as in \cref{subsec:prop-subsets}, we may use the traditional notation for intersections, unions, and complements:
\indexdef{intersection!of subsets}%
\symlabel{intersection}%
\indexdef{union!of subsets}%
\symlabel{union}%
\indexdef{complement, of a subset}%
\symlabel{complement}%
\begin{align*}
  \setof{x:A | P(x)} \cap \setof{x:A | Q(x)}
  &\defeq \setof{x:A | P(x) \land Q(x)},\\
  \setof{x:A | P(x)} \cup \setof{x:A | Q(x)}
  &\defeq \setof{x:A | P(x) \lor Q(x)},\\
  A \setminus \setof{x:A | P(x)}
  &\defeq \setof{x:A | \neg P(x)}.
\end{align*}
Of course, in the absence of \LEM{}, the latter are not ``complements'' in the usual sense: we may not have $B \cup (A\setminus B) = A$ for every subset $B$ of $A$.

\index{truncation!propositional|)}%
\index{mere proposition|)}%
\index{logic!of mere propositions|)}%



\section{选择公理}
\label{sec:axiom-choice}
\index{公理!选择|(defstyle}%
\index{否定|(}%
现在可以在同伦类型论中正式表述选择公理.
给定一个类型 $X$ 和类型族
%
\begin{equation*}
    A:X\to\type
    \qquad\text{和}\qquad
    P:\prd{x:X} A(x)\to\type,
\end{equation*}
%
以及
\begin{itemize}
    \item $X$ 是一个集合,
    \item 对于所有 $x:X$, $A(x)$ 是一个集合
    \item 对于所有 $x:X$ 和 $a:A(x)$, $P(x,a)$ 是一个纯命题.
\end{itemize}
\define{选择公理}
$\choice{}$ 可以表示为,
\begin{equation}
    \label{eq:ac}
    \Parens{\prd{x:X} \Brck{\sm{a:A(x)} P(x,a)}}
    \to
    \Brck{\sm{g:\prd{x:X} A(x)} \prd{x:X} P(x,g(x))}.
\end{equation}
当然, 这对~\eqref{eq:english-ac} 的直接翻译,  可以将 ``存在 $x:A$ 满足 $B(x)$'' 视作 $\brck{\sm{x:A}B(x)}$, 于是可以将这个命题用熟悉的符号表示
\begin{narrowmultline*}
    \textstyle
    \Big(\fall{x:X}\exis{a:A(x)} P(x,a)\Big)
    \Rightarrow \narrowbreak
    \Big(\exis{g : \prd{x:X} A(x)} \fall{x : X} P(x,g(x))\Big).
\end{narrowmultline*}
%
特别地, 注意命题截断出现了两次.
定义域中出现的截断代表着, 对于每个 $x$ 存在一些 $a:A(x)$ 满足 $P(x,a)$, 但是它们的值没有被任何方式选择或指定.
值域中出现的截断代表着, 可以推断出存在一些函数 $g$, 但是这些函数无法被确定或指定.

实际上, 因为 \cref{thm:ttac}, 这个公理可以被简化表示.

\begin{lem}
    \label{thm:ac-epis-split}
    选择公理~\eqref{eq:ac} 等价于这个语句: 对于任意集合 $X$ 和任意 $Y:X\to\type$ 使得每个 $Y(x)$ 都是集合, 有
    \begin{equation}
        \Parens{\prd{x:X} \Brck{Y(x)}}
        \to
        \Brck{\prd{x:X} Y(x)}.\label{eq:epis-split}
    \end{equation}
\end{lem}

这对应于经典\index{数学的!经典的}选择公理的等价形式, 即 ``非空集合族的笛卡尔积是非空的''.

\begin{proof}
    通过 \cref{thm:ttac}, ~\eqref{eq:ac} 的值域等价于
    \[\Brck{\prd{x:X} \sm{a:A(x)} P(x,a)}.\]
    因此,~\eqref{eq:ac} 等价于~\eqref{eq:epis-split} 的实例, 其中 \narrowequation{Y(x) \defeq \sm{a:A(x)} P(x,a).}
    (根据 \cref{thm:isset-prod,thm:prop-set}, 这是一个集合.)
    反过来,~\eqref{eq:epis-split} 等价于~\eqref{eq:ac} 的实例, 其中 $A(x)\defeq Y(x)$ 而 $P(x,a)\defeq\unit$.
    因此, 两边都逻辑地等价.
    因为它们都是纯命题, 根据 \cref{lem:equiv-iff-hprop} 它们是等价的类型.
\end{proof}

类似于 \LEM{}, ~\eqref{eq:ac} 和~\eqref{eq:epis-split} 的等价形式不是我们的基本类型论中的推论, 但是可以作为公理假设, 且具备一致性.

\begin{rmk}
    很容易证明~\eqref{eq:epis-split} 的右边可以推出左边.
    因为它们都是纯命题, 根据 \cref{lem:equiv-iff-hprop} 选择公理也等价于要求这个等价关系
    \[ \eqv{\Parens{\prd{x:X} \Brck{Y(x)}}}{\Brck{\prd{x:X} Y(x)}} \]
    这说明了一个常见的陷阱: 尽管依赖函数类型保持纯命题 (\cref{thm:isprop-forall}), 但它们与截断不是交换的: $\brck{\prd{x:A} P(x)}$ 通常不等价于 $\prd{x:A} \brck{P(x)}$.
    选择公理, 如果假定了它, 代表\emph{对于集合}是成立的; 正如我们将在下文看到的, 一般是不成立的.
\end{rmk}

对于选择公理的限制, 可以在一定程度上放宽为任意类型.
例如, 在 在~\eqref{eq:ac} 中可以放宽 $A$ 和 $P$, 或放宽 ~\eqref{eq:epis-split} 中的 $Y$, 为任意的类型族;
这会导致一个看似更强的语句, 但同样是一致的.
也可以用\cref{cha:hlevels} 中, 更一般的 $n$-截断替换命题截断, 得到一系列 $\choice n$, 插值于在~\eqref{eq:ac}, 简称 \choice{} (或强调为 $\choice{-1}$ ), 和 \cref{thm:ttac}, 又叫做 $\choice\infty$ 之间.
参见 \cref{ex:acnm,ex:acconn}.
不过, 需要注意的是我们无法放宽 $X$ 是一个集合的要求.

\begin{lem}
    \label{thm:no-higher-ac}
    存在类型 $X$ 和族 $Y:X\to \type$ 满足每个 $Y(x)$ 都是集合, 但是~\eqref{eq:epis-split} 不成立.
\end{lem}
\begin{proof}
    定义 $X\defeq \sm{A:\type} \brck{\bool = A}$, 并令 $x_0 \defeq (\bool, \bproj{\refl{\bool}}) : X$.
    然后通过 $\Sigma$-类型的路径的恒等性, 事实上 $\brck{A=\bool}$ 是纯命题, 以及单值性, 对于任意 $(A,p),(B,q):X$ 有 $\eqv{(\id[X]{(A,p)}{(B,q)})}{(\eqv AB)}$.
    特别的, $\eqv{(\id[X]{x_0}{x_0})}{(\eqv \bool\bool)}$, 所以正如 \cref{thm:type-is-not-a-set}, $X$ 不是集合.

    另一边, 如果 $(A,p):X$, 那么 $A$ 是一个集合;
    可以这样得到: 对截断 $p:\brck{\bool=A}$ 归纳以及 $\bool$ 是一个集合.
    因为 $\eqv A B$ 是一个集合无论 $A$ 和 $B$ 是不是, 这代表 $\id[X]{x_1}{x_2}$ 对于任意 $x_1,x_2:X$ 是集合, 即 $X$ 是 1-类型.
    特别的, 如果通过 $Y(x) \defeq (x_0=x)$ 定义 $Y:X\to\UU$, 那么每个 $Y(x)$ 都是一个集合.

    现在通过定义, 对于任何 $(A,p):X$ 有 $\brck{\bool=A}$, 因此 $\brck{x_0 = (A,p)}$.
    于是, 有 $\prd{x:X} \brck{Y(x)}$.
    如果~\eqref{eq:epis-split} 保留这样的 $X$ 和 $Y$, 那么同样有 $\brck{\prd{x:X} Y(x)}$.
    因为我们正在尝试推导出矛盾 ($\emptyt$), 是一个纯命题, 假设 $\prd{x:X} Y(x)$, 即 $\prd{x:X} (x_0=x)$.
    而 $X$ 是一个纯命题, 因此也是一个集合, 所以这是一个矛盾.
\end{proof}

\index{否定|)}%
\index{公理!选择|)}%


\section{The principle of unique choice}
\label{sec:unique-choice}
\index{唯一!选择|(defstyle}%
\indexsee{公理!选择!唯一}{唯一选择}%

接下来的内容是平凡的, 但又是很有用的.

\begin{lem}
    \label{thm:prop-equiv-trunc}
    如果 $P$ 是纯命题, 那么 $\eqv P {\brck P}$.
\end{lem}
\begin{proof}
    显然, 根据定义, 有 $P\to \brck{P}$.
    然后因为 $P$ 是纯命题,  将 $\brck P$ 的泛性质应用于 $\idfunc[P] :P\to P$ 得到 $\brck P \to P$.
    这些函数互为准逆, 因为 \cref{lem:equiv-iff-hprop}.
\end{proof}

其中重要的结论如下.

\begin{cor}[The principle of unique choice]
    \label{cor:UC}
    假设类型族 $P:A\to \type$ 满足
    \begin{enumerate}
        \item 对于每个 $x$, 类型 $P(x)$ 是纯命题, 并且
        \item 对于每个 $x$ 有 $\brck {P(x)}$.
    \end{enumerate}
    然后有 $\prd{x:A} P(x)$.
\end{cor}
\begin{proof}
    从这两个假设和上一个引理直接可得.
\end{proof}

这个推论还包含了一种非常有用的推理技巧.
具体来说, 假设知道 $\brck A$, 并且想利用它来构建某个其他类型 $B$ 的元素.
我们希望使用 $A$ 的一个元素, 用于构建 $B$ 的元素, 但是这只有在 $B$ 是纯命题时才可以, 这样才能应用命题截断 $\brck A$ 的归纳原理;
通常我们期望的最好的结果就是证明 $\brck B$.
%
作为替代, 可以扩展 $B$, 通过添加额外的数据, 来\emph{唯一地}表征希望构建的对象.
具体来说, 定义一个性质 $Q:B\to\type$ 使得 $\sm{x:B} Q(x)$ 是纯命题.
然后从 $A$ 的一个元素, 构造出 $b:B$ 并满足 $Q(b)$, 从而可以从 $\brck A$ 构建 $\brck{\sm{x:B} Q(x)}$, 并且因为 $\brck{\sm{x:B} Q(x)}$ 等价于 $\sm{x:B} Q(x)$, 可以从中投影出 $B$ 的一个元素.
在 \cref{ex:decidable-choice} 中有一个例子.

集合论中, 也存在类似的问题, 尽管表现方式不同.
如果尝试定义函数 $f: A \to B$, 并依赖于 $a : A$ 的一个元素, 我们只能够证明某个 $b : B$ 的存在, 这并没有完成,因为需要实际的给出 $B$ 的一个元素, 而不是仅仅证明它存在.
一种选择是将论证细化到证明 $b : B$ 的唯一性, 就像在我们类型论做的那样.
但是在集合论中, 这个问题通常可以被避免, 只需应用选择公理, 它可以挑选我们需要的元素.
但是在同伦类型论中, 除了避免使用选择公理的想法以外, 可用的选择公理的形式也更少适用, 因为这要求选择的定义域是\emph{集合}.
因此, 如果 $A$ 不是集合 (比如可能是一个 $\UU$), 就没有一致的选择形式允许为每个 $a : A$ 直接挑选 $B$ 中的一个元素, 用来定义 $f(a)$.

\index{唯一!选择|)}%



\section{When are propositions truncated?}
\label{subsec:when-trunc}
\index{logic!of mere propositions|(}%
\index{mere proposition|(}%
\index{logic!truncated}%

At first glance, it may seem that the truncated versions of $+$ and $\Sigma$ are actually closer to the informal mathematical meaning of ``or'' and ``there exists'' than the untruncated ones.
Certainly, they are closer to the \emph{precise} meaning of ``or'' and ``there exists'' in the first-order logic \index{first-order!logic} which underlies formal set theory, since the latter makes no attempt to remember any witnesses to the truth of propositions.
However, it may come as a surprise to realize that the practice of \emph{informal} mathematics is often more accurately described by the untruncated forms.

\index{prime number}%
For example, consider a statement like ``every prime number is either $2$ or odd''.
The working mathematician feels no compunction about using this fact not only to prove \emph{theorems} about prime numbers, but also to perform \emph{constructions} on prime numbers, perhaps doing one thing in the case of $2$ and another in the case of an odd prime.
The end result of the construction is not merely the truth of some statement, but a piece of data which may depend on the parity of the prime number.
Thus, from a type-theoretic perspective, such a construction is naturally phrased using the induction principle for the coproduct type ``$(p=2)+(p\text{ is odd})$'', not its propositional truncation.

Admittedly, this is not an ideal example, since ``$p=2$'' and ``$p$ is odd'' are mutually exclusive, so that $(p=2)+(p\text{ is odd})$ is in fact already a mere proposition and hence equivalent to its truncation (see \cref{ex:disjoint-or}).
More compelling examples come from the existential quantifier.
It is not uncommon to prove a theorem of the form ``there exists an $x$ such that \dots'' and then refer later on to ``the $x$ constructed in Theorem Y'' (note the definite article).
Moreover, when deriving further properties of this $x$, one may use phrases such as ``by the construction of $x$ in the proof of Theorem Y''.

A very common example is ``$A$ is isomorphic to $B$'', which strictly speaking means only that there exists \emph{some} isomorphism between $A$ and~$B$.
But almost invariably, when proving such a statement, one exhibits a specific isomorphism or proves that some previously known map is an isomorphism, and it often matters later on what particular isomorphism was given.

Set-theoretically trained mathematicians often feel a twinge of guilt at such ``abuses of language''.\index{abuse!of language}
We may attempt to apologize for them, expunge them from final drafts, or weasel out of them with vague words like ``canonical''.
The problem is exacerbated by the fact that in formalized set theory, there is technically no way to ``construct'' objects at all --- we can only prove that an object with certain properties exists.
Untruncated logic in type theory thus captures some common practices of informal mathematics that the set theoretic reconstruction obscures.
(This is similar to how the univalence axiom validates the common, but formally unjustified, practice of identifying isomorphic objects.)

On the other hand, sometimes truncated logic is essential.
We have seen this in the statements of \LEM{} and \choice{}; some other examples will appear later on in the book.
Thus, we are faced with the problem: when writing informal type theory, what should we mean by the words ``or'' and ``there exists'' (along with common synonyms such as ``there is'' and ``we have'')?

A universal consensus may not be possible.
Perhaps depending on the sort of mathematics being done, one convention or the other may be more useful --- or, perhaps, the choice of convention may be irrelevant.
In this case, a remark at the beginning of a mathematical paper may suffice to inform the reader of the linguistic conventions in use therein.
However, even after one overall convention is chosen, the other sort of logic will usually arise at least occasionally, so we need a way to refer to it.
More generally, one may consider replacing the propositional truncation with another operation on types that behaves similarly, such as the double negation operation $A\mapsto \neg\neg A$, or the $n$-truncations to be considered in \cref{cha:hlevels}.
As an experiment in exposition,  in what follows we will occasionally use \emph{adverbs}\index{adverb} to denote the application of such ``modalities'' as propositional truncation.

For instance, if untruncated logic is the default convention, we may use the adverb \define{merely}
\indexdef{merely}%
to denote propositional truncation.
Thus the phrase
\begin{center}
  ``there merely exists an $x:A$ such that $P(x)$''
\end{center}
indicates the type $\brck{\sm{x:A} P(x)}$.
Similarly, we will say that a type $A$ is \define{merely inhabited}
\indexdef{merely!inhabited}%
\indexdef{inhabited type!merely}%
to mean that its propositional truncation $\brck A$ is inhabited (i.e.\ that we have an unnamed element of it).
Note that this is a \emph{definition}\index{definition!of adverbs} of the adverb ``merely'' as it is to be used in our informal mathematical English, in the same way that we define nouns\index{noun} like ``group'' and ``ring'', and adjectives\index{adjective} like ``regular'' and ``normal'', to have precise mathematical meanings.
We are not claiming that the dictionary definition of ``merely'' refers to propositional truncation; the choice of word is meant only to remind the mathematician reader that a mere proposition contains ``merely'' the information of a truth value and nothing more.

On the other hand, if truncated logic is the current default convention, we may use an adverb such as \define{purely}
\indexdef{purely}%
or \define{constructively} to indicate its absence, so that
\begin{center}
``there purely exists an $x:A$ such that $P(x)$''
\end{center}
would denote the type $\sm{x:A} P(x)$.
We may also use ``purely'' or ``actually'' just to emphasize the absence of truncation, even when that is the default convention.

In this book we will continue using untruncated logic as the default convention, for a number of reasons.
\begin{enumerate}[label=(\arabic*)]
\item We want to encourage the newcomer to experiment with it, rather than sticking to truncated logic simply because it is more familiar.
\item Using truncated logic as the default in type theory suffers from the same sort of ``abuse of language''\index{abuse!of language} problems as set-theoretic foundations, which untruncated logic avoids.
  For instance, our definition of ``$\eqv A B$'' as the type of equivalences between $A$ and $B$, rather than its propositional truncation, means that to prove a theorem of the form ``$\eqv A B$'' is literally to construct a particular such equivalence.
  This specific equivalence can then be referred to later on.
\item We want to emphasize that the notion of ``mere proposition'' is not a fundamental part of type theory.
  As we will see in \cref{cha:hlevels}, mere propositions are just the second rung on an infinite ladder, and there are also many other modalities not lying on this ladder at all.
\item Many statements that classically are mere propositions are no longer so in homotopy type theory.
  Of course, foremost among these is equality.
\item On the other hand, one of the most interesting observations of homotopy type theory is that a surprising number of types are \emph{automatically} mere propositions, or can be slightly modified to become so, without the need for any truncation.
  (See \cref{thm:isprop-isprop,cha:equivalences,cha:hlevels,cha:category-theory,cha:set-math}.)
  Thus, although these types contain no data beyond a truth value, we can nevertheless use them to construct untruncated objects, since there is no need to use the induction principle of propositional truncation.
  This useful fact is more clumsy to express if propositional truncation is applied to all statements by default.
\item Finally, truncations are not very useful for most of the mathematics we will be doing in this book, so it is simpler to notate them explicitly when they occur.
\end{enumerate}

\index{mere proposition|)}%
\index{logic!of mere propositions|)}%


\section{Contractibility}
\label{sec:contractibility}
\index{type!contractible|(defstyle}%
\index{contractible!type|(defstyle}%

In \cref{thm:inhabprop-eqvunit} we observed that a mere proposition which is inhabited must be equivalent to $\unit$,
\index{type!unit}%
and it is not hard to see that the converse also holds.
A type with this property is called \emph{contractible}.
Another equivalent definition of contractibility, which is also sometimes convenient, is the following.

\begin{defn}\label{defn:contractible}
  A type $A$ is \define{contractible},
  or a \define{singleton},
  \indexdef{type!singleton}%
  \indexsee{singleton type}{type, singleton}%
  if there is $a:A$, called the \define{center of contraction},
  \indexdef{center!of contraction}%
  such that $a=x$ for all $x:A$.
  We denote the specified path $a=x$ by $\contr_x$.
\end{defn}

In other words, the type $\iscontr(A)$ is defined to be
\[ \iscontr(A) \defeq \sm{a:A} \prd{x:A}(a=x). \]
Note that under the usual propositions-as-types reading, we can pronounce $\iscontr(A)$ as ``$A$ contains exactly one element'', or more precisely ``$A$ contains an element, and every element of $A$ is equal to that element''.

\begin{rmk}
  We can also pronounce $\iscontr(A)$ more topologically as ``there is a point $a:A$ such that for all $x:A$ there exists a path from $a$ to $x$''.
  Note that to a classical ear, this sounds like a definition of \emph{connectedness} rather than contractibility.
  \index{continuity of functions in type theory@``continuity'' of functions in type theory}%
  \index{functoriality of functions in type theory@``functoriality'' of functions in type theory}%
  The point is that the meaning of ``there exists'' in this sentence is a continuous/natural one.

  A better way to express connectedness would be $\sm{a:A}\prd{x:A} \brck{a=x}$.
  This is indeed correct if $A$ is assumed to be pointed --- see the remark after \cref{thm:connected-pointed} --- but in general a type can be connected without being pointed.
  In \cref{sec:connectivity} we will define connectedness as the $n=0$ case of a general notion of $n$-connectedness, and in \cref{ex:connectivity-inductively} the reader is asked to show that this definition is equivalent to having both $\brck{A}$ and $\prd{x,y:A} \brck{x=y}$.
\end{rmk}

\begin{lem}\label{thm:contr-unit}
  For a type $A$, the following are logically equivalent.
  \begin{enumerate}
  \item $A$ is contractible in the sense of \cref{defn:contractible}.\label{item:contr}
  \item $A$ is a mere proposition, and there is a point $a:A$.\label{item:contr-inhabited-prop}
  \item $A$ is equivalent to \unit.\label{item:contr-eqv-unit}
  \end{enumerate}
\end{lem}
\begin{proof}
  If $A$ is contractible, then it certainly has a point $a:A$ (the center of contraction), while for any $x,y:A$ we have $x=a=y$; thus $A$ is a mere proposition.
  Conversely, if we have $a:A$ and $A$ is a mere proposition, then for any $x:A$ we have $x=a$; thus $A$ is contractible.
  And we showed~\ref{item:contr-inhabited-prop}$\Rightarrow$\ref{item:contr-eqv-unit} in \cref{thm:inhabprop-eqvunit}, while the converse follows since \unit easily has property~\ref{item:contr-inhabited-prop}.
\end{proof}

\begin{lem}\label{thm:isprop-iscontr}
  For any type $A$, the type $\iscontr(A)$ is a mere proposition.
\end{lem}
\begin{proof}
  Suppose given $c,c':\iscontr(A)$.
  We may assume $c\jdeq(a,p)$ and $c'\jdeq(a',p')$ for $a,a':A$ and $p:\prd{x:A} (a=x)$ and $p':\prd{x:A} (a'=x)$.
  By the characterization of paths in $\Sigma$-types, to show $c=c'$ it suffices to exhibit $q:a=a'$ such that $\trans{q}{p}=p'$.
  %
  We choose $q\defeq p(a')$.
  Now since $A$ is contractible (by $c$ or $c'$), by \cref{thm:contr-unit} it is a mere proposition.
  Hence, by \cref{thm:prop-set,thm:isprop-forall}, so is $\prd{x:A}(a'=x)$; thus $\trans{q}{p}=p'$ is automatic.
\end{proof}

\begin{cor}\label{thm:contr-contr}
  If $A$ is contractible, then so is $\iscontr(A)$.
\end{cor}
\begin{proof}
  By \cref{thm:isprop-iscontr} and \cref{thm:contr-unit}\ref{item:contr-inhabited-prop}.
\end{proof}

Like mere propositions, contractible types are preserved by many type constructors.
For instance, we have:

\begin{lem}\label{thm:contr-forall}
  If $P:A\to\type$ is a type family such that each $P(a)$ is contractible, then $\prd{x:A} P(x)$ is contractible.
\end{lem}
\begin{proof}
  By \cref{thm:isprop-forall}, $\prd{x:A} P(x)$ is a mere proposition since each $P(x)$ is.
  But it also has an element, namely the function sending each $x:A$ to the center of contraction of $P(x)$.
  Thus by \cref{thm:contr-unit}\ref{item:contr-inhabited-prop}, $\prd{x:A} P(x)$ is contractible.
\end{proof}

\index{function extensionality}%
(In fact, the statement of \cref{thm:contr-forall} is equivalent to the function extensionality axiom.
See~\cref{sec:univalence-implies-funext}.)

Of course, if $A$ is equivalent to $B$ and $A$ is contractible, then so is $B$.
More generally, it suffices for $B$ to be a \emph{retract} of $A$.
By definition, a \define{retraction}
\indexdef{retraction}%
\indexdef{function!retraction}%
is a function $r : A \to B$ such that there exists a function $s : B \to A$, called its \define{section},
\indexdef{section}%
\indexdef{function!section}%
and a homotopy $\epsilon:\prd{y:B} (r(s(y))=y)$; then we say that $B$ is a \define{retract}%
\indexdef{retract!of a type}
of $A$.

\begin{lem}\label{thm:retract-contr}
  If $B$ is a retract of $A$, and $A$ is contractible, then so is $B$.
\end{lem}
\begin{proof}
  Let $a_0 : A$ be the center of contraction.
  We claim that $b_0 \defeq r(a_0) : B$ is a center of contraction for $B$.
  Let $b : B$; we need a path $b = b_0$.
  But we have $\epsilon_b : r(s(b)) = b$ and $\contr_{s(b)} : s(b) = a_0$, so by composition
  \[ \opp{\epsilon_b} \ct \ap{r}{\contr_{s(b)}} : b = r(a_0) \jdeq b_0. \qedhere\]
\end{proof}

Contractible types may not seem very interesting, since they are all equivalent to \unit.
One reason the notion is useful is that sometimes a collection of individually nontrivial data will collectively form a contractible type.
An important example is the space of paths with one free endpoint.
As we will see in \cref{sec:identity-systems}, this fact essentially
encapsulates the based path induction principle for identity types.


\begin{lem}\label{thm:contr-paths}
  For any $A$ and any $a:A$, the type $\sm{x:A} (a=x)$ is contractible.
\end{lem}
\begin{proof}
  We choose as center the point $(a,\refl a)$.
  Now suppose $(x,p):\sm{x:A}(a=x)$; we must show $(a,\refl a) = (x,p)$.
  By the characterization of paths in $\Sigma$-types, it suffices to exhibit $q:a=x$ such that $\trans{q}{\refl a} = p$.
  But we can take $q\defeq p$, in which case $\trans{q}{\refl a} = p$ follows from the characterization of transport in path types.
\end{proof}

When this happens, it can allow us to simplify a complicated construction up to equivalence, using the informal principle that contractible data can be freely ignored.
This principle consists of many lemmas, most of which we leave to the reader; the following is an example.

\begin{lem}\label{thm:omit-contr}
  Let $P:A\to\type$ be a type family.
  \begin{enumerate}
  \item If each $P(x)$ is contractible, then $\sm{x:A} P(x)$ is equivalent to $A$.\label{item:omitcontr1}
  \item If $A$ is contractible with center $a$, then $\sm{x:A} P(x)$ is equivalent to $P(a)$.\label{item:omitcontr2}
  \end{enumerate}
\end{lem}
\begin{proof}
  In the situation of~\ref{item:omitcontr1}, we show that $\proj1:\sm{x:A} P(x) \to A$ is an equivalence.
  For quasi-inverse we define $g(x)\defeq (x,c_x)$ where $c_x$ is the center of $P(x)$.
  The composite $\proj1 \circ g$ is obviously $\idfunc[A]$, whereas the opposite composite is homotopic to the identity by using the contractions of each $P(x)$.

  We leave the proof of~\ref{item:omitcontr2} to the reader (see \cref{ex:omit-contr2}).
\end{proof}

Another reason contractible types are interesting is that they extend the ladder of $n$-types mentioned in \cref{sec:basics-sets} downwards one more step.

\begin{lem}\label{thm:prop-minusonetype}
  A type $A$ is a mere proposition if and only if for all $x,y:A$, the type $\id[A]xy$ is contractible.
\end{lem}
\begin{proof}
  For ``if'', we simply observe that any contractible type is inhabited.
  For ``only if'', we observed in \cref{subsec:hprops} that every mere proposition is a set, so that each type $\id[A]xy$ is a mere proposition.
  But it is also inhabited (since $A$ is a mere proposition), and hence by \cref{thm:contr-unit}\ref{item:contr-inhabited-prop} it is contractible.
\end{proof}

Thus, contractible types may also be called \define{$(-2)$-types}.
They are the bottom rung of the ladder of $n$-types, and will be the base case of the recursive definition of $n$-types in \cref{cha:hlevels}.

\index{type!contractible|)}%
\index{contractible!type|)}%


\sectionNotes
The fact that it is possible to define sets, mere propositions, and contractible types in type theory, with all higher homotopies automatically taken care of as in \cref{sec:basics-sets,subsec:hprops,sec:contractibility}, was first observed by Voevodsky.
In fact, he defined the entire hierarchy of $n$-types by induction, as we will do in \cref{cha:hlevels}.\index{n-type@$n$-type!definable in type theory}%

\cref{thm:not-dneg,thm:not-lem} rely in essence on a classical theorem of Hedberg,
\index{Hedberg's theorem}%
\index{theorem!Hedberg's}%
which we will prove in \cref{sec:hedberg}.
The implication that the propositions-as-types form of \LEM{} contradicts univalence was observed by Mart\'\i n Escard\'o on the \Agda mailing list.
The proof we have given of \cref{thm:not-dneg} is due to Thierry Coquand.

The propositional truncation was introduced in the extensional type theory of
\index{proof!assistant!\NuPRL}%
\NuPRL in 1983 by Constable~\cite{Con85} as an
application of ``subset'' and ``quotient'' types.  What is here called the
``propositional truncation'' was called ``squashing'' in the \NuPRL type theory~\cite{constable+86nuprl-book}.
Rules characterizing the propositional truncation directly, still in extensional type theory, were given in~\cite{ab:bracket-types}.
The intensional version in homotopy type theory was constructed by Voevodsky using an impredicative\index{impredicative!truncation} quantification, and later by Lumsdaine using higher inductive types (see \cref{sec:hittruncations}).

\index{propositional!resizing}%
Voevodsky~\cite{Universe-poly} has proposed resizing rules of the kind considered in \cref{subsec:prop-subsets}.
\index{axiom!of reducibility}\index{resizing}%
These are clearly related to the notorious \emph{axiom of reducibility} proposed by Russell in his and Whitehead's \emph{Principia Mathematica}~\cite{PM2}.\index{Russell, Bertrand}

The adverb ``purely'' as used to refer to untruncated logic is a reference to the use of monadic modalities to model effects in programming languages; see \cref{sec:modalities} and the Notes to \cref{cha:hlevels}.

There are many different ways in which logic can be treated relative to type theory.
For instance, in addition to the plain propositions-as-types logic described in \cref{sec:pat}, and the alternative which uses mere propositions only as described in \cref{subsec:logic-hprop}, one may introduce a separate ``sort'' of propositions, which behave somewhat like types but are not identified with them.
This is the approach taken in logic enriched type theory~\cite{aczel2002collection} and in some presentations of the internal languages of toposes\index{topos} and related categories (e.g.~\cite{jacobs1999categorical,elephant}), as well as in the proof assistant \Coq.\index{proof!assistant!Coq@\textsc{Coq}}
Such an approach is more general, but less powerful.
For instance, the principle of unique choice (\cref{sec:unique-choice}) fails in the category of so-called setoids in \Coq~\cite{Spiwack}, in logic enriched type theory~\cite{aczel2002collection}, and in minimal type theory~\cite{maietti2005toward}.\index{setoid}
Thus, the univalence axiom makes our type theory behave more like the internal logic of a topos;~see also \cref{cha:set-math}.\index{topos}

Martin-L\"of~\cite{martin2006100} provides a discussion on the history of axioms of choice.
Of course, constructive and intuitionistic mathematics has a long and complicated history, which we will not delve into here; see for instance~\cite{TroelstraI,TroelstraII}.


\sectionExercises
\begin{ex}
    \label{ex:equiv-functor-set}
    证明 $\eqv A B$ 且 $A$ 是集合, 那么 $B$ 也是集合.
\end{ex}

\begin{ex}
    \label{ex:isset-coprod}
    证明如果 $A$ 和 $B$ 都是集合, 那么 $A+B$ 也是集合.
\end{ex}

\begin{ex}
    \label{ex:isset-sigma}
    证明如果 $A$ 是集合, $B:A\to \type$ 是类型族, 而且对于所有 $x:A$, $B(x)$ 是集合, 那么 $\sm{x:A} B(x)$ 是集合.
\end{ex}

\begin{ex}
    \label{ex:prop-endocontr}
    证明 $A$ 是纯命题, 当且仅当 $A\to A$ 可缩.
\end{ex}

\begin{ex}
    \label{ex:prop-inhabcontr}
    证明 $\eqv{\isprop(A)}{(A\to\iscontr(A))}$.
\end{ex}

\begin{ex}
    \label{ex:lem-mereprop}
    证明如果 $A$ 是纯命题, 那么 $A+(\neg A)$ 也是纯命题.
    因此, 在 \eqref{eq:lem} 中不需要插入命题截断.
\end{ex}

\begin{ex}
    \label{ex:disjoint-or}
    更概括地, 证明 $A$ 和 $B$ 是纯命题而且 $\neg(A\times B)$, 那么 $A+B$ 也是纯命题.
\end{ex}

% \begin{ex}\label{ex:hprop-iff-equiv}
%   Show that if $A$ and $B$ are mere propositions such that $A\to B$ and $B\to A$, then $\eqv A B$.
% \end{ex}

% \begin{ex}\label{ex:isprop-isprop}
%   Show that for any type $A$, the types $\isprop(A)$ and $\isset(A)$ are mere propositions.
% \end{ex}

% \begin{ex}\label{ex:prop-eqvtrunc}
%   Show that if $A$ is already a mere proposition, then $\eqv A{\brck{A}}$.
% \end{ex}

\begin{ex}
    \label{ex:brck-qinv}
    假定一些类型 $\isequiv(f)$ 满足 \cref{sec:basics-equivalences} 中的条件 \ref{item:be1}--\ref{item:be3}, 证明类型 $\brck{\qinv(f)}$ 同样的条件而且等价于 $\isequiv(f)$.
\end{ex}

\begin{ex}
    \label{ex:lem-impl-prop-equiv-bool}
    证明如果 \LEM{} 成立, 那么类型 $\prop \defeq \sm{A:\type} \isprop(A)$ 等价于 \bool.
\end{ex}

\begin{ex}
    \label{ex:lem-impred}
    证明 $\UU_{i+1}$ 满足 \LEM{}, 典范包含 $\prop_{\UU_i} \to \prop_{\UU_{i+1}}$ 是一个等价.
\end{ex}

\begin{ex}
    \label{ex:not-brck-A-impl-A}
    证明并非对于所有类型 $A:\type$ 有 $\brck{A} \to A$.
    (然而, 纯在特定的类型有 $\brck{A}\to A$.
    \cref{ex:brck-qinv} 表明 $\qinv(f)$ 就是这样的类型.)
\end{ex}

\begin{ex}
    \label{ex:lem-impl-simple-ac}
    \index{公理!选择}%
    注意如果 \LEM{} 成立, 那么对于所有 $A:\type$ 有 $\bbrck{(\brck A \to A)}$.
    (从选择公理很容易得到这个性质, 没有 \LEM{} 的情况可能失败;
    参见 \cite{krausgeneralizations}.)
\end{ex}

\begin{ex}
    \label{ex:naive-lem-impl-ac}
    在 \cref{thm:not-lem} 下面的 \LEM{} 的朴素形式和单值是不一致的:
    \[ \prd{A:\type} (A+(\neg A)) \]
    没有单值的情况, 这个公理是一致的.
    然而, 证明这个公理暗含了 \eqref{eq:ac}.
\end{ex}

\begin{ex}
    \label{ex:lem-brck}
    假定 \LEM{} 成立, 证明双重否定 $\neg \neg A$ 有和命题截断 $\brck A$ 相同的递归原则, 但是使用的是命题计算规则而不是判断计算规则.
    换句话说, 假设 \LEM{} 成立, 证明如果 $B$ 是纯命题而且有 $f:A\to B$, 那么有一个诱导的 $g:\neg\neg A \to B$ 满足 $g(\bproj a) = f(a)$ 对于所有 $a:A$.
    推导 (假设 \LEM{} 成立) $\eqv{\neg\neg A}{\brck{A}}$.
    因此, 在 \LEM{} 成立的情况, 命题截断可以被定义, 而不是作为一个单独的类型构造器.
\end{ex}

\begin{ex}
    \label{ex:impred-brck}
    \index{命题的!重调大小}%
    证明如果假设命题重调, 就像 \cref{subsec:prop-subsets} 中的一样, 那么类型
    \[\prd{P:\prop} \Parens{(A\to P)\to P}\]
    有相同的归纳命题 $\brck A$, \emph{以及}相同的判断计算规则.
    因此, 也可以这样定义命题截断.
\end{ex}

\begin{ex}
    \label{ex:lem-impl-dn-commutes}
    假设 \LEM{} 成立, 证明双重否定交换于集合上的纯命题的全称量词.
    也就是说, 证明如果 $X$ 是集合而且每个 $Y(x)$ 是纯命题, 那么 \LEM{} 意味
    \begin{equation}
        \eqv{\Parens{\prd{x:X} \neg\neg Y(x)}}{\Parens{\neg\neg \prd{x:X} Y(x)}}.\label{eq:dnshift}
    \end{equation}
    观察到如果假设每个 $Y(x)$ 是集合, 那么 \eqref{eq:dnshift} 等价于选择公理 \eqref{eq:epis-split}.
\end{ex}

\begin{ex}
    \label{ex:prop-trunc-ind}
    \index{induction principle!for truncation}%
    证明 \cref{subsec:prop-trunc} 给定的命题截断规则可以推断出下面的归纳原理:
    对于任何类型族 $B:\brck A \to \type$, 其中 $B(x)$ 是纯命题, 如果对于每个 $a:A$ 有 $B(\bproj a)$, 那么对于每个 $x:\brck A$ 有 $B(x)$.
\end{ex}

\begin{ex}
    \label{ex:lem-ldn}
    证明排中律\eqref{eq:lem}和双重否定\eqref{eq:ldn}是逻辑等价的.
\end{ex}

\begin{ex}
    \label{ex:decidable-choice}
    假设 $P:\nat\to\type$ 是纯命题的一个可判定的族 (参见 \cref{defn:decidable-equality}\ref{item:decidable-equality2}).
    证明
    \[ \Brck{\sm{n:\nat} P(n)} \;\to\; \sm{n:\nat}P(n).\]
\end{ex}

\begin{ex}
    \label{ex:omit-contr2}
    证明 \cref{thm:omit-contr}\ref{item:omitcontr2}: 如果 $A$ 是可错的, 而且中心是 $a$, 那么 $\sm{x:A} P(x)$ 等价于 $P(a)$.
\end{ex}

\begin{ex}
    \label{ex:isprop-equiv-equiv-bracket}
    证明 $\isprop(P) \simeq (P \simeq \brck P)$.
\end{ex}

\begin{ex}
    \label{ex:finite-choice}
    在经典集合论中, 选择公理的有限版本是一个定理. 证明选择公理 \eqref{eq:ac} 成立, 当 $X$ 是有限类型 $\Fin(n)$ (定义于 \cref{ex:fin}).
\end{ex}

\begin{ex}
    \label{ex:decidable-choice-strong}
    如果 $P:\nat\to\type$ 是可判定的族, 证明 \cref{ex:decidable-choice} 的结论为真.
\end{ex}

% Local Variables:
% TeX-master: "hott-online"
% End:
